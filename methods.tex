\section{Methods}

We quantified the impact of numerical variability on structural MRI findings in
Parkinson's disease (PD) using (1) stochastic perturbation
experiments and (2) analytical uncertainty propagation. Empirically, we processed a
longitudinal PPMI cohort using FreeSurfer instrumented with stochastic numerical
noise to isolate run-to-run variability. Analytically, we derived closed-form
approximations linking this pipeline instability (\navr) to uncertainty in
effect sizes and test statistics, enabling retrospective quality control using
only summary statistics.

The experimental workflow proceeded in four stages: (1) curation of a
longitudinal dataset with two visits per participant; (2) repeated processing
under Monte Carlo Arithmetic (MCA) perturbations via Fuzzy-libm; (3) rigorous
quality control to distinguish numerical artifacts from technical failures; and
(4) statistical evaluation of inference instability across repetitions.

\subsection{Participants}

Structural MRI data were obtained from the Parkinson's Progression Markers
Initiative (PPMI; \url{www.ppmi-info.org}). The
study included 201 participants: 112 individuals diagnosed with Parkinson's
disease without mild cognitive impairment (PD-non-MCI) and 89 healthy controls
(HC). All participants had two usable T1-weighted MRI scans acquired
approximately $1.4 \pm 0.5$ years apart ($0.9$-$2.0$ years). Patients with mild
cognitive impairment were excluded to minimize confounding effects of cognitive
decline.

Inclusion criteria were: (i) diagnosis of idiopathic Parkinson's disease
(PD-non-MCI) or healthy control status; (ii) availability of two high-quality
T1-weighted scans at distinct visits; and (iii) absence of other neurological or
psychiatric conditions. PD severity was quantified using the Unified Parkinson's
Disease Rating Scale part III (UPDRS-III) in the OFF medication state at both
baseline and follow-up visits.

All procedures were approved by the research ethics boards of participating PPMI
sites, and written informed consent was obtained from all participants in
accordance with the Declaration of Helsinki and was exempt from Concordia University’s
Research Ethics Unit. The PD and HC groups did not differ
significantly in age, education, or sex distribution ($p > 0.05$;
Table~\ref{tab:cohort_stat}).

\begin{table}[h!]
    \centering
    \begin{tabular}{lcc}
        \toprule
        \textbf{Cohort}              & \textbf{HC}     & \textbf{PD-non-MCI} \\
        \hline
        $n$                          & $89$            & $112$               \\
        Age (years)                  & $60.7 \pm 9.7$  & $60.6 \pm 8.9$      \\
        Age range                    & $30.6$ - $79.8$ & $39.2$ - $78.3$     \\
        Gender (male,\%)             & $47$ (52.8\%)   & $74$ (66.1\%)       \\
        Education (years)            & $16.7 \pm 3.4$  & $16.0 \pm 3.1$      \\
        UPDRS-III OFF baseline       & $-$             & $23.3 \pm 10.0$     \\
        UPDRS-III OFF follow-up      & $-$             & $25.6 \pm 11.2$     \\
        Inter-visit interval (years) & $1.4 \pm 0.5$   & $1.4 \pm 0.6$       \\
        \bottomrule
    \end{tabular}
    \vspace{1em}
    \caption{\textbf{Participant characteristics.} Values represent mean $\pm$
        standard deviation. PD = Parkinson's disease; MCI = mild cognitive impairment;
        UPDRS = Unified Parkinson's Disease Rating Scale. The PD-non-MCI longitudinal
        subset corresponds to participants with available follow-up MRI and  and disease
        severity scores available. \label{tab:cohort_stat}}
\end{table}

\subsection{Image acquisition and preprocessing}

High-resolution T1-weighted MRI scans were obtained from the Parkinson's
Progression Markers Initiative (PPMI). Data were acquired using standardized 3D
MPRAGE protocols (TR = 2.3 s, TE = 2.98 ms, TI = 0.9 s, voxel size = 1 mm
isotropic) across multiple sites. While the protocol was standardized, we
account for minor acquisition variations inherent to the PPMI multisite design.

Structural images were processed using FreeSurfer 7.3.1 instrumented with
Fuzzy-libm, a modified mathematical library that injects stochastic
perturbations into floating-point operations to probe numerical stability
(Section~\ref{sec:numerical_variability_assessment}). To estimate numerical
variability, each scan was processed 34 times. We filtered the resulting outputs
to exclude both technical failures (e.g., incomplete execution and FreeSurfer
failures) and quality control (QC) failures.

For QC, we visually inspected nine representative slices (3 along each axis) per
run to identify gross artifacts, such as missing brain tissue, blurring, or
biologically implausible segmentation. Appendix Table~\ref{tab:qc} details the exclusion
rates. To ensure statistical consistency across the cohort, we randomly
subsampled the remaining runs to retain exactly 26 valid realisations per
participant.

\subsection{Numerical variability assessment}
\label{sec:numerical_variability_assessment}

To quantify the numerical instability in the FreeSurfer pipeline, we
employed Monte Carlo Arithmetic (MCA)~\cite{parker1997monte}. MCA is a
stochastic technique that simulates the propagation of rounding errors by
introducing controlled random perturbations into floating-point operations. We
utilized the Random Rounding (RR) mode, where the result of an arithmetic
operation is perturbed by a zero-mean random noise scaled to the magnitude of
the least significant bit. For any operation producing an exact result $x$, the
perturbed result $\tilde{x}$ is modeled as $\tilde{x} = x + 2^{e_x - t}\xi$
where $e_x = \lfloor \log_2 |x| \rfloor$ is the exponent of $x$, $t$ is the
\textit{virtual precision} parameter, and $\xi$ is a random variable drawn
uniformly from the interval $(-\frac{1}{2}, \frac{1}{2})$.

While comprehensive MCA instrumentation provides a rigorous bound on numerical
error, it incurs a prohibitive computational cost (typically $100\times$ to
$1000\times$ slowdown), rendering it intractable for large-scale neuroimaging
cohorts. To address this, we used \textit{Fuzzy-libm}~\cite{salari2021accurate},
a lightweight implementation that restricts MCA instrumentation to elementary
mathematical library functions (e.g., \texttt{exp}, \texttt{log}, \texttt{sin},
\texttt{cos}). By targeting these elementary functions, which are sources of
divergence across operating systems~\cite{glatard2015reproducibility},
Fuzzy-libm significantly reduces computational overhead while effectively
capturing the numerical variability relevant to cross-platform
reproducibility~\cite{salari2021accurate,vila2024impact,des2023reproducibility}.

Fuzzy-libm is deployed via a Docker container and uses the \texttt{LD\_PRELOAD}
mechanism to dynamically intercept calls to the standard system math library and
redirect them to the instrumented version. The library is compiled using
Verificarlo~\cite{denis2016verificarlo}, an LLVM-based compiler that injects the
MCA logic at compile time. Virtual precision parameters are set to match
standard hardware precision ($t=53$ bits for double precision and $t=24$ bits
for single precision), ensuring that the simulated variability remains
representative of realistic machine-level precision errors.

\subsubsection{Numerical-Population Variability Ratio (\navr)}

To quantify computational stability relative to population variation, we
introduce the Numerical-Population Variability Ratio (\navr). For each brain
region, \navr measures the ratio of measurement uncertainty arising from
computational processes to natural inter-subject variation:

\[
    \text{\navr} = \frac{\sigma_{\mathrm{num}}}{\sigma_{\mathrm{pop}}}
\]

where $\sigma_{\mathrm{num}}$ represents numerical variability (measurement
precision across MCA repetitions for individual subjects) and
$\sigma_{\mathrm{pop}}$ represents population variability (inter-subject
differences within each repetition).
For each region of interest, measurements from $k$ MCA repetitions across $n$
subject-visit pairs form a data matrix $\mathcal{M}_{k \times n}$ with entries
$x_i^{(r)}$, where $i=1,\dots,n$ indexes subject-visits and $r=1,\dots,k$ indexes repetitions.
Let
\[
    \bar x_i=\frac{1}{k}\sum_{r=1}^k x_i^{(r)}, \qquad
    \bar x^{(r)}=\frac{1}{n}\sum_{i=1}^n x_i^{(r)}.
\]

\textbf{Numerical variability (within-subject, across repetitions):}
\begin{equation}
    \sigma^2_{\mathrm{num}}
    = \frac{1}{n}\sum_{i=1}^{n}\left[
    \frac{1}{k-1}\sum_{r=1}^{k}\bigl(x_i^{(r)}-\bar x_i\bigr)^2
    \right].
    \label{eq:sigma_num}
\end{equation}

\textbf{Population variability (within-repetition, across subjects):}
\begin{equation}
    \sigma^2_{\mathrm{pop}}
    = \frac{1}{k}\sum_{r=1}^{k}\left[
    \frac{1}{n-1}\sum_{i=1}^{n}\bigl(x_i^{(r)}-\bar x^{(r)}\bigr)^2
    \right].
    \label{eq:sigma_anat}
\end{equation}
where $\bar{x}_i$ and $\bar{x}^{(r)}$ denote column and row means,
respectively. Higher \navr values indicate regions where computational
uncertainty approaches population variation.

\subsubsection{Relationship between \navr~and downstream statistical test uncertainty}
\label{sec:theoretical_derivations}

To establish a quantitative link between a method's computational reproducibility
and the reliability of group-level statistical inferences, we derived analytical
expressions connecting numerical variability to the uncertainty of commonly used
statistical tests (Cohen's~$d$, $t$-tests, partial correlation, and ANCOVA).
Our goal is to characterize how numerical noise propagates through the analytical
pipeline to produce uncertainty in the reported statistics.

For each Monte Carlo Arithmetic (MCA) repetition $r$, we denote by
$\tilde{\mathbf{x}}^{(r)} = (\tilde{x}_1^{(r)}, \dots, \tilde{x}_N^{(r)})^\top$
the vector of perturbed measurements across $N$ subjects.
Each perturbed observation is modeled as the sum of the subject's
biological value and a numerical error term:
\begin{equation*}
    \tilde{x}_i^{(r)} = x_i + \varepsilon_i^{(r)},
    \qquad
    \mathbb{E}[\varepsilon_i^{(r)}] = 0,
    \qquad
    \mathrm{Cov}[\boldsymbol{\varepsilon}^{(r)}] = \Sigma_{\mathrm{num}}.
    \label{eq:model_noise}
\end{equation*}
Here, $\mathbf{x} = (x_1, \dots, x_N)^\top$ represents the fixed,
biological measurements, while $\boldsymbol{\varepsilon}^{(r)}$
captures the random numerical perturbations introduced during computation.

\vspace{0.4em}
\noindent\textbf{Assumptions.}
To isolate the contribution of numerical variability, we make three simplifying assumptions:
\begin{enumerate}

    \item \textbf{Numerical error model.} The numerical perturbations are
          modeled as independent, zero-mean Gaussian random variables:
          $\varepsilon_i^{(r)} \sim \mathcal{N}(0,\,\sigma_{\mathrm{num},i}^2),
              \boldsymbol{\varepsilon}^{(r)} \sim
              \mathcal{N}\!\left(\mathbf{0},\,\Sigma_{\mathrm{num}}\right)$, where,
          under homoscedasticity, $\Sigma_{\mathrm{num}} =
              \sigma_{\mathrm{num}}^2 I_N$.

    \item \textbf{Population variability.} The between-subject variance
          $\sigma_{\mathrm{pop}}^2 = \Var(\{x_i\})$ dominates the numerical
          noise, i.e.\ $\sigma_{\mathrm{num},i} \ll \sigma_{\mathrm{pop}}$. This
          implies that the pooled empirical standard deviation of the observed
          data can be approximated by the biological one, $s_p \approx
              \sigma_{\mathrm{pop}}$.

    \item \textbf{Null-hypothesis scenario.} We condition on the
          observed biological measurements $\mathbf{x}$ and quantify variability
          only from numerical perturbations $\boldsymbol{\varepsilon}^{(r)}$.
          This isolates numerical variability effects from sampling variation;
          the resulting expressions remain accurate near the null and for small
          effect sizes. In this setting, the biological values $\mathbf{x} =
              (x_1, \dots, x_N)^\top$ are treated as fixed, and all randomness
          arises from numerical perturbations $\boldsymbol{\varepsilon}^{(r)}$.

\end{enumerate}

Under these assumptions, each downstream statistic
$ds = f(\tilde{\mathbf{x}})$ can be linearized around the
baseline $\mathbf{x}$ as
$ds(\tilde{\mathbf{x}}) \approx ds(\mathbf{x}) +
    \nabla_{\!\mathbf{x}}f(\mathbf{x})^\top \boldsymbol{\varepsilon}$,
allowing the numerical variance to be expressed through the delta method as
\begin{equation}
    \Var_{\mathrm{num}}[ds]
    \approx
    \nabla_{\!\mathbf{x}} f(\mathbf{x})^\top
    \Sigma_{\mathrm{num}}
    \nabla_{\!\mathbf{x}} f(\mathbf{x})
    = \sigma_{\mathrm{num}}^2 \|\nabla_{\!\mathbf{x}} f(\mathbf{x})\|_2^2.
    \label{eq:var-ds}
\end{equation}



Table~\ref{tab:stat_uncertainty} summarizes the derived expressions for the
numerical standard deviation of several common statistics.

\paragraph{Cohen's \textit{d}}

Cohen's effect size quantifies the standardized difference between two sample means.
For two independent groups $G_1$ and $G_2$ with sample sizes $n_1$ and $n_2$
($df = n_1 + n_2 - 2$), we define:
\begin{equation}
    d = \frac{\Delta}{s_p}
    = \frac{\bar{x}_1 - \bar{x}_2}{s_p},
    \qquad
    s_p = \sqrt{\frac{(n_1 - 1)s_1^2 + (n_2 - 1)s_2^2}{df}}.
    \label{eq:cohen-d-def}
\end{equation}

The variance of $d$ across Monte Carlo Arithmetic (MCA) repetitions, conditional
on the fixed dataset $\mathbf{x}$, is given by:
\begin{equation*}
    \Var_{\mathrm{num}}[d]
    = \Var\!\left[d(\tilde{\mathbf{x}})\,\big|\,\mathbf{x}\right].
    \label{eq:var-rep-def}
\end{equation*}
Applying the multivariate delta method (Eq.~\ref{eq:var-ds}) around the baseline $\mathbf{x}_0 = \mathbf{x}$, we obtain:
\begin{equation}
    \Var_{\mathrm{num}}[d]
    \approx
    \sigma_{\mathrm{num}}^2
    \sum_{i=1}^{n}
    \left(\frac{\partial d}{\partial x_i}\right)^{\!2}.
    \label{eq:var-rep-approx}
\end{equation}
For an observation $x_i \in G_g$ ($g \in \{1,2\}$), the chain rule gives:
\begin{equation*}
    \frac{\partial d}{\partial x_i} = \frac{1}{s_p}\frac{\partial \Delta}{\partial x_i} - \frac{\Delta}{s_p^2}\frac{\partial s_p}{\partial x_i}, \quad
    \frac{\partial \Delta}{\partial x_i} = \pm \frac{1}{n_g}, \quad
    \frac{\partial s_p}{\partial x_i} = \frac{x_i - \bar{x}_g}{df s_p}
    \label{eq:d-deriv-chain}
\end{equation*}
where the sign in $\partial \Delta/\partial x_i$ is positive for $g=1$ and negative for $g=2$.
Substituting back into the expression for $\partial d/\partial x_i$, we have:
\begin{equation*}
    \frac{\partial d}{\partial x_i} = \pm\frac{1}{n_g s_p} - \frac{\Delta(x_i - \bar{x}_g)}{df s_p^3}
    \Rightarrow {\left(\frac{\partial d}{\partial x_i}\right)}^2 = \frac{1}{n_g^2 s_p^2} \pm \frac{2\Delta(x_i - \bar{x}_g)}{n_g df s_p^4} + \frac{\Delta^2(x_i - \bar{x}_g)^2}{df^2 s_p^6}
\end{equation*}
and summing over all $i$ in group $G_g$:
\begin{equation*}
    \sum_{i\in G_g}\left(\frac{\partial d}{\partial x_i}\right)^2 =
    \frac{1}{n_g s_p^2} \pm \frac{2\Delta}{n_g df s_p^4}\sum_{i\in G_g}(x_i - \bar{x}_g) + \frac{\Delta^2}{df^2 s_p^6}\sum_{i\in G_g}(x_i - \bar{x}_g)^2 = \frac{1}{n_g s_p^2} + \frac{\Delta^2}{df^2 s_p^6}\left((n_g - 1)s_g^2 \right)
\end{equation*}
since $\sum_{i\in G_g}(X_i - \bar{X}_g) = 0$, so finally summing over both groups:
\begin{align*}
    \sum_{i = 1}^{n}\left(\frac{\partial d}{\partial X_i}\right)^2
     & = \frac{1}{s_p^2}\left(\frac{1}{n_1} + \frac{1}{n_2}\right) +
    \frac{\Delta^2}{df^2 s_p^6}\left((n_1 - 1)s_1^2 + (n_2 - 1)s_2^2 \right)                                         \\
     & = \frac{1}{s_p^2}\left(\frac{1}{n_1} + \frac{1}{n_2} +
    \frac{1}{df}\frac{\Delta^2}{s_p^2}\frac{1}{s_p^2}\frac{\left((n_1 - 1)s_1^2 + (n_2 - 1)s_2^2 \right)}{df}\right) \\
     & = \frac{1}{s_p^2}\left(\frac{1}{n_1} + \frac{1}{n_2} + \frac{d^2}{df} \right).
\end{align*}
Finally, assuming $s_p \approx \sigma_{\mathrm{pop}}$, the population (between-subject)
variance, Eq.~\eqref{eq:var-rep-approx} becomes:
\begin{equation}
    \Var_{\mathrm{num}}[d]
    \approx
    \frac{\sigma_{\mathrm{num}}^2}{\sigma_{\mathrm{pop}}^2}
    \left(\frac{1}{n_1}+\frac{1}{n_2}+\frac{d^2}{df}\right).
    \label{eq:var-rep-simplified}
\end{equation}
For balanced groups ($n_1 = n_2 = n/2$) and large $n$, the $d^2/df$ term is negligible:
\begin{align*}
    \Var_{\mathrm{num}}[d]   & \approx \frac{4}{n} \frac{\sigma_{\mathrm{num}}^2}{\sigma_{\mathrm{pop}}^2}   = \frac{4}{n} \nu_{\mathrm{npv}}^2 \\
    \sigma_{\mathrm{num}}[d] & \approx \frac{2}{\sqrt{n}}\,\nu_{\mathrm{npv}}
    \label{eq:var-rep-final-simplified}
\end{align*}
where $\nu_{\mathrm{npv}} = \sigma_{\mathrm{num}} / \sigma_{\mathrm{pop}}$
is the numerical-population variability ratio.
This expression quantitatively links the numerical uncertainty captured by
$\nu_{\mathrm{npv}}$ to the variability of Cohen's~$d$ effect size,
providing a practical measure of the stability of statistical inferences
under finite-precision arithmetic.

\paragraph{Two-sample \textit{t}-test statistic}

The pooled two-sample $t$ statistic quantifies the standardized difference between
two group means:
\begin{equation*}
    t =
    \frac{\bar{x}_1 - \bar{x}_2}{s_p\sqrt{\tfrac{1}{n_1} + \tfrac{1}{n_2}}}
    =
    \frac{d}{\sqrt{\tfrac{1}{n_1} + \tfrac{1}{n_2}}},
    \label{eq:t-stat-def}
\end{equation*}
where $d$ is Cohen's~$d$ defined in Eq.~\eqref{eq:cohen-d-def}.
Defining
\(
\omega_n = \tfrac{1}{n_1} + \tfrac{1}{n_2},
\)
the $t$ statistic can be expressed as
\(t = d / \sqrt{\omega_n}\).
From Eq.~\eqref{eq:var-rep-simplified}, the variance of $d$ due to numerical
perturbations propagates to the variance of $t$ as:
\begin{equation}
    \begin{aligned}
        \Var_{\mathrm{num}}[t]
         & =
        \Var\!\left[\frac{d}{\sqrt{\omega_n}}\right]
        \approx
        \frac{1}{\omega_n}
        \nu_{\mathrm{npv}}^{2}
        \left(\omega_n + \frac{d^{2}}{df}\right) \\[3pt]
         & =
        \nu_{\mathrm{npv}}^{2}
        \left(1 + \frac{d^{2}}{df\,\omega_n}\right).
        \label{eq:t-var-approx}
    \end{aligned}
\end{equation}

\vspace{0.5em}
\noindent
To analyze the correction term, consider
\begin{equation*}
    df\,\omega_n = (n_1 + n_2 - 2)
    \left(\frac{1}{n_1} + \frac{1}{n_2}\right)
    = (n_1 + n_2 - 2)\frac{n_1 + n_2}{n_1 n_2}
    \label{eq:df-omega-analysis}
\end{equation*}
When $n_1 \gg n_2$,
\(
df\,\omega_n \approx \tfrac{n_1 - 2}{n_2};
\)
symmetrically, when $n_2 \gg n_1$,
\(
df\,\omega_n \approx \tfrac{n_2 - 2}{n_1}.
\)
In both cases, $1/(df\,\omega_n) \to 0$, so the correction term
\(\tfrac{d^2}{df\,\omega_n}\) vanishes for unbalanced groups.
When the groups are balanced ($n_1 = n_2 = n/2$),
$df \omega_n = 4\!\left(1 - \tfrac{2}{n}\right) \to 4$ as $n \to \infty$,
so that $\tfrac{d^2}{df\,\omega_n} \to d^2 / 4$.
Substituting these results into Eq.~\eqref{eq:t-var-approx} gives:
\begin{equation*}
    \Var_{\mathrm{num}}[t]
    \approx
    \nu_{\mathrm{npv}}^{2}\,(1 + \epsilon),
    \label{eq:t-var-final}
\end{equation*}
where $\epsilon$ tends to $0$ for strongly unbalanced groups and to $d^{2}/4$
for large balanced samples.
For small effect sizes ($d^2 \ll 4$) and unbalanced groups, the correction term
$\epsilon$ is negligible, yielding the simplified expression:
\begin{align*}
    \Var_{\mathrm{num}}[t]   & \approx \nu_{\mathrm{npv}}^{2} \notag \\
    \sigma_{\mathrm{num}}[t] & \approx \nu_{\mathrm{npv}}.
    \label{eq:t-var-simplified}
\end{align*}

\vspace{0.5em}
\noindent
The uncertainty of the corresponding $p$-values can then be derived. Let $X$ be the
random variable with $\mathbb{E}_{\mathrm{num}}[X]=t_0$ and
$\Var_{\mathrm{num}}[X]=\nu_{\mathrm{npv}}^{2}$. Let $f_t$, $F_t$ be the
probability density and cumulative distribution functions of the Student
$t$-distribution with $df$ degrees of freedom. Applying the delta method to the
two-sided $p$-value $p(X) = 2\left(1 - F_t(|X|)\right)$ gives:
\begin{align*}
    \Var_{\mathrm{num}}[p(X)] & = \Var_{\mathrm{num}}\left[2(1 - F_t(|X|))\right]                             \notag     \\
                              & \approx {\left(-2f_t(|t_0|)\,\mathrm{sign}(t_0)\right)}^{2}\Var_{\mathrm{num}}[X] \notag \\
                              & \approx 4{\left(f_t(|t_0|)\right)}^{2}\nu_{\mathrm{npv}}^{2},
\end{align*}
which gives the standard deviation of the numerical uncertainty in the
$p$-value:
\begin{equation*}
    \sigma_{\mathrm{num}}[p(X)] = 2f_t(|t_0|)\,\nu_{\mathrm{npv}}.
    \label{eq:p-std}
\end{equation*}

\paragraph{ANCOVA group effect.}

Analysis of covariance (ANCOVA) evaluates group differences using a general
linear model:
\begin{equation*}
    \mathbf{y} = \mathbf{X}\boldsymbol{\beta} + \boldsymbol{\varepsilon},
    \qquad
    \boldsymbol{\varepsilon} \sim \mathcal{N}(\mathbf{0},\, \sigma^2 I),
    \label{eq:ancova-model}
\end{equation*}
where $\mathbf{y}$ is the vector of measurements across subjects, and
$\mathbf{X}$ includes an intercept, diagnostic group (PD vs.\ HC), and
covariates (e.g., age and sex). The adjusted group difference is expressed as
the one-degree-of-freedom contrast $c^{\top}\boldsymbol{\beta}$, with contrast
vector $c = [0,\, 1,\, 0,\, 0]^{\top}$. Let
$\widehat{\boldsymbol{\beta}} = (\mathbf{X}^{\top}\mathbf{X})^{-1}
    \mathbf{X}^{\top}\mathbf{y}$ denote the ordinary least-squares (OLS) estimator
and $\widehat{\sigma}_{\mathrm{res}}^{2} = SS_{\mathrm{res}} / df_2$ the
residual mean square, where $df_2 = n - \mathrm{rank}(\mathbf{X})$.
The sum of squares associated with the group effect is:
\begin{equation*}
    SS_{\mathrm{group}}
    =
    \frac{(c^{\top}\widehat{\boldsymbol{\beta}})^{2}}
    {c^{\top}(\mathbf{X}^{\top}\mathbf{X})^{-1}c},
    \qquad df_1 = 1.
\end{equation*}
The corresponding ANCOVA $F$ statistic is given by:
\begin{equation*}
    F
    =
    \frac{MS_{\mathrm{group}}}{MS_{\mathrm{res}}}
    =
    \frac{(c^{\top}\widehat{\boldsymbol{\beta}})^{2}}
    {\widehat{\sigma}_{\mathrm{res}}^{2}\;
    c^{\top}(\mathbf{X}^{\top}\mathbf{X})^{-1}c},
    \qquad
    F \sim \mathcal{F}(df_1 = 1,\, df_2).
    \label{eq:ancova-F}
\end{equation*}
Significance is evaluated using the upper-tail of the central $\mathcal{F}$ distribution under
the null hypothesis:
\[
    p(X) = 1 - F_{\mathcal{F}}\left(X;df_1,df_2\right),
\]
with $X \sim \mathcal{F}(df_1, df_2)$ and $F_{\mathcal{F}}$ the cumulative distribution
function of the $F$ distribution with $(df_1, df_2)$ degrees of freedom. For
$df_1 = 1$, the ANCOVA $F$ statistic is equivalent to the two-sample $t$-test
through $F = t^{2}$ (see~\cite[p.~403]{johnson1995continuous}).
Then the uncertainty in the $F$ statistic due to numerical noise follows directly from the uncertainty of $t$:
\begin{align*}
    \Var_{\mathrm{num}}[F]
     & = \Var_{\mathrm{num}}[t^{2}]
    \approx (2t)^{2}\Var_{\mathrm{num}}[t]
    = 4t^{2}\nu_{\mathrm{npv}}^{2}
    = 4F\nu_{\mathrm{npv}}^{2},
\end{align*}
yielding
\begin{equation}
    \sigma_{\mathrm{num}}[F] = 2\sqrt{F}\,\nu_{\mathrm{npv}}.
    \label{eq:ancova-F-sigma}
\end{equation}

\vspace{0.4em}
\noindent
The uncertainty in the corresponding $p$-values can be obtained by the delta
method. Let $X$ be a random variable with $\mathbb{E}_{\mathrm{num}}[X]=F_0$ and
$\Var_{\mathrm{num}}[X]=4F_0\nu_{\mathrm{npv}}^{2}$  and $f_{\mathcal{F}}$, $F_{\mathcal{F}}$ the probability
density and cumulative distribution functions. Applying the delta-method to the
upper-tail $p$-value is $p(X) = 1 - F_{\mathcal{F}}(X)$ yields:
\begin{align*}
    \Var_{\mathrm{num}}[p(X)] & = \Var_{\mathrm{num}}[1 - F_{\mathcal{F}}(X)]           \\
                              & \approx f_{\mathcal{F}}(F_0)^{2}\Var_{\mathrm{num}}[X]  \\
                              & = 4F_0{f_{\mathcal{F}}(F_0)}^{2}\nu_{\mathrm{npv}}^{2},
\end{align*}
so that
\begin{equation}
    \sigma_{\mathrm{num}}[p(X)] = 2\sqrt{F_0}\,f_{\mathcal{F}}(F_0)\,\nu_{\mathrm{npv}}.
    \label{eq:ancova-p-sigma}
\end{equation}

Equations~\eqref{eq:ancova-F-sigma} and~\eqref{eq:ancova-p-sigma} show that
numerical imprecision introduces a variance in the estimated $F$ statistic and
its corresponding $p$-value that scales linearly with the numerical-population
variability ratio $\nu_{\mathrm{npv}}$, and proportionally to $\sqrt{F}$ for
the group effect magnitude.

\paragraph{Partial correlation.}

Partial correlation measures the association between two variables $(x,y)$ while
controlling for the influence of one or more additional variables $z$.
In our analysis, this corresponds to quantifying the relationship between
regional brain measurements and UPDRS-III motor scores, controlling for age and sex.
The sample partial correlation is defined as:
\begin{equation*}
    r_{xy,z}
    =
    \frac{r_{xy} - r_{xz}r_{yz}}
    {\sqrt{(1 - r_{xz}^{2})(1 - r_{yz}^{2})}},
    \label{eq:partial-r-def}
\end{equation*}
where $r_{xy}$ denotes the Pearson correlation between variables $x$ and $y$,
\[
    r_{xy} = \frac{s_{xy}}{s_x s_y}.
\]
To simplify notation, we set
$a = r_{xy}$, $b = r_{xz}$, and $c = r_{yz}$ so that
\[
    R(a,b,c)
    = \frac{a - bc}{\sqrt{(1 - b^{2})(1 - c^{2})}}
    = \frac{a - bc}{D},
    \qquad
    D = \sqrt{(1 - b^{2})(1 - c^{2})}.
\]
Applying the delta method (Eq.~\ref{eq:var-ds}) to the partial correlation, we have:
\begin{equation}
    \Var_{\mathrm{num}}[R] \approx \sigma_{\mathrm{num}}^{2} \sum_{i=1}^{n} \left(\frac{\partial R}{\partial x_i}\right)^{2},
    \label{eq:delta-partial}
\end{equation}

Assuming only $x$ is affected by numerical perturbations while $y$ and $z$ are
fixed, the gradient with respect to each observation $x_i$ is:
\[
    \frac{\partial R}{\partial x_i}
    =
    \frac{\partial R}{\partial a}\frac{\partial a}{\partial x_i}
    +
    \frac{\partial R}{\partial b}\frac{\partial b}{\partial x_i}.
\]
The first-order partial derivatives are:
\[
    \frac{\partial R}{\partial a} = \frac{1}{D}, \qquad
    \frac{\partial R}{\partial b} = \frac{(1 - c^{2})(ab - c)}{D^{3}}.
\]
and the derivatives of the correlations with respect to $x_i$ are (see
Eq~\ref{eq:pearson_derivative}):
\[
    \frac{\partial a}{\partial x_i}
    =
    \frac{(v_i - a\,u_i)}{(n-1)s_x} = \frac{\alpha_i}{(n-1)s_x},
    \qquad
    \frac{\partial b}{\partial x_i}
    = \frac{(w_i - b\,u_i)}{(n-1)s_x} = \frac{\beta_i}{(n-1)s_x}.
\]
where $u_i = (x_i - \bar{x})/s_x$, $v_i = (y_i - \bar{y})/s_y$, and $w_i = (z_i - \bar{z})/s_z$ are standardized and centered observations of
$x$, $y$, and $z$ respectively.
Then $\partial R / \partial x_i$ becomes:
\begin{align*}
    \frac{\partial R}{\partial x_i}
     & =
    \frac{1}{(n-1) s_x}
    \left[
        \frac{\alpha_i}{D} + \beta_i
        \frac{(1 - c^{2})(ab - c)}{D^{3}}
        \right]
\end{align*}
thus $(\partial R / \partial x_i)^{2}$ is:
\begin{align*}
    \left(\frac{\partial R}{\partial x_i}\right)^{2}
     & =
    \frac{1}{(n-1)^{2}s_{x}^{2}}
    \left[
        \frac{\alpha_i^{2}}{D^{2}}
        +
        2\frac{\alpha_i\beta_i(1 - c^{2})(ab - c)}{D^{4}}
        +
        \frac{\beta_i^{2}(1 - c^{2})^{2}(ab - c)^{2}}{D^{6}}
    \right]                           \\
     & =
    \frac{1}{(n-1)^{2}s_{x}^{2}}
    \left[
        \frac{\alpha_i^{2}}{(1 - b^{2})(1 - c^{2})}
        +
        2\frac{\alpha_i\beta_i(1 - c^{2})(ab - c)}{(1 - b^{2})^{2}(1 - c^{2})^{2}}
        +
        \frac{\beta_i^{2}(1 - c^{2})^{2}(ab - c)^{2}}{(1 - b^{2})^{3}(1 - c^{2})^{3}}
    \right]                           \\
     & = \frac{1}{(n-1)^{2}s_{x}^{2}}
    \left[
        \frac{\alpha_i^{2}}{(1 - b^{2})(1 - c^{2})}
        +
        2\frac{\alpha_i\beta_i(ab - c)}{(1 - b^{2})^{2}(1 - c^{2})}
        +
        \frac{\beta_i^{2}(ab - c)^{2}}{(1 - b^{2})^{3}(1 - c^{2})}
        \right].
\end{align*}
Using the correlation identities
$\sum\alpha_i^2 = (n-1)(1 - a^{2})$,
$\sum\beta_i^2 = (n-1)(1 - b^{2})$, and
$\sum\alpha_i\beta_i = (n-1)(ab - c)$ (see proof in
Appendix~\ref{sec:correlation_identities}),
we sum over all $i$ to obtain:
\begin{align*}
    \sum_{i=1}^{n}
    \left(\frac{\partial R}{\partial x_i}\right)^{2}
     & =
    \frac{1}{(n-1)s_{x}^{2}}
    \left[
        \frac{(1 - a^{2})}{(1 - b^{2})(1 - c^{2})}
        +
        2\frac{(ab - c)^{2}}{(1 - b^{2})^{2}(1 - c^{2})}
        +
        \frac{(ab - c)^{2}}{(1 - b^{2})^{2}(1 - c^{2})}
    \right] \\
     & =
    \frac{1}{(n-1)s_{x}^{2}}
    \left[
        \frac{(1 - a^{2})}{(1 - b^{2})(1 - c^{2})}
        +
        \frac{3(ab - c)^{2}}{(1 - b^{2})^{2}(1 - c^{2})}
    \right] \\
     & =
    \frac{1}{(n-1)s_{x}^{2}}
    \left[
    \frac{(1 - a^{2})(1+3r_{yz,x}^{2})}{(1 - b^{2})(1 - c^{2})}
    \right].
\end{align*}
Substituting back into Eq.~\eqref{eq:delta-partial} with $s_x^2 \simeq \sigma_{\mathrm{pop}}^2$ gives:
\begin{equation}
    \Var_{\mathrm{num}}[R]
    \approx
    \frac{\nu_{\mathrm{npv}}^{2}}{(n-1)}
    \left[
    \frac{(1 - a^{2})(1+3r_{yz,x}^{2})}{(1 - b^{2})(1 - c^{2})}
    \right].
    \label{eq:partial-var}
\end{equation}
Since $a,b,c$ are rarely not reported in practice, we further simplify
this expression by deriving the lower bound:
\begin{equation}
    (1 - R^{2})^{3} \le \frac{(1 - a^{2})}{(1 - b^{2})(1 - c^{2})},
    \label{eq:partial-corr-bound}
\end{equation}
First, note that the squared partial correlation is:
\[
    1 - R^{2}
    =
    \frac{(1-b^{2})(1-c^{2})-(a-bc)^{2}}{(1-b^{2})(1-c^{2})}
    =
    \frac{\Delta}{(1-b^{2})(1-c^{2})},
\]
where
\[
    \Delta
    =
    (1-b^{2})(1-c^{2})-(a-bc)^{2}
    =
    (1-a^{2})(1-b^{2})-(ac-b)^{2}
    =
    (1-a^{2})(1-c^{2})-(ab-c)^{2}.
\]
Each equality above follows from expanding both sides.
Because every squared term is nonnegative, we obtain the three inequalities
\begin{equation}
    \label{eq:three-ineq}
    \Delta \le (1-a^{2})(1-b^{2}),
    \qquad
    \Delta \le (1-b^{2})(1-c^{2}),
    \qquad
    \Delta \le (1-c^{2})(1-a^{2}).
\end{equation}
Multiplying the first and third inequalities in~\eqref{eq:three-ineq} gives
\[
    \Delta^{2} \le (1-a^{2})^{2}(1-b^{2})(1-c^{2}),
\]
and multiplying also by the middle one yields
\[
    \Delta^{3} \le (1-a^{2})^{2}(1-b^{2})^{2}(1-c^{2})^{2}.
\]
Since $0\le 1-a^{2}\le 1$, we have $(1-a^{2})^{2}\le (1-a^{2})$, so that
\[
    \Delta^{3} \le (1-a^{2})(1-b^{2})^{2}(1-c^{2})^{2}.
\]
Dividing both sides by $(1-b^{2})^{3}(1-c^{2})^{3}$ and substituting
$\Delta=(1-b^{2})(1-c^{2})(1-R^{2})$ yields
\[
    (1-R^{2})^{3}
    =
    \frac{\Delta^{3}}{(1-b^{2})^{3}(1-c^{2})^{3}}
    \le
    \frac{1-a^{2}}{(1-b^{2})(1-c^{2})}.
\]
This establishes the claimed bound~\eqref{eq:partial-corr-bound}.
Since $3r_{yz,x}^2+1 \geq 1$ it follows immediately that
\[
    (1 - R^{2})^{3} \leq
    \frac{(1-a^{2})}{(1-b^{2})(1-c^{2})} \leq
    \frac{(1-a^{2})(1+3r_{yz,x}^{2})}{(1-b^{2})(1-c^{2})}
\]
So, substituting into Eq.~\eqref{eq:partial-var} gives the lower bound:
\begin{equation}
    \nu_{\mathrm{npv}}^{2}\frac{(1 - R^{2})^{3}}{n-1}
    \lesssim
    \Var_{\mathrm{num}}[R].
    \label{eq:partial-var-bound}
\end{equation}
Taking the square root yields the standard deviation:
\[
    \nu_{\mathrm{npv}}
    \sqrt{\frac{(1 - r_{xy,z}^{2})^{3}}{n - 1}}
    \lesssim
    \sigma_{\mathrm{num}}[R].
\]
The two-sided significance of a partial correlation is computed from the
$t$-statistic
\[
    t
    =
    R\sqrt{\frac{df}{1-R^{2}}},
    \qquad
    df = n-k-2,
\]
where $k$ is the number of controlling variables. Let $X$ be the random variable
with $\mathbb{E}_{\mathrm{num}}[X]=R_0$, $t_0^2=R_0(df/(1-R_0^2))$ with
$\Var_{\mathrm{num}}[X]$ bounded by~\Cref{eq:partial-var-bound}. Let $f_t$,
$F_t$ be the probability density and cumulative distribution functions of the
Student $t$-distribution with $df$ degrees of freedom. Applying the delta method
to the two-sided $p$-value $p(X) = 2\left(1 - F_t(|X|)\right)$ gives:
\begin{equation}
    \Var_{\mathrm{num}}[p(X)]
    \approx
    \left(\frac{\partial p}{\partial t}\frac{\partial t}{\partial R}\right)^{2}
    \Var_{\mathrm{num}}[X],
    \label{eq:partial-p-var}
\end{equation}
with the partial derivatives given by:
\begin{equation}
    \frac{\partial p}{\partial t} = -2 f_t(|t|)\,\mathrm{sign}(t),\qquad
    \frac{\partial t}{\partial R} = \sqrt{\frac{df}{(1-R^{2})^{3}}}.
    \label{eq:partial-t-deriv}
\end{equation}
Combining equations \eqref{eq:partial-p-var} and \eqref{eq:partial-t-deriv} yields
\[
    \Var_{\mathrm{num}}[p(X)]
    \geq
    4f_t(|t_0|)^{2}\frac{df}{(1-R_0^{2})^{3}}\Var_{\mathrm{num}}[X].
\]
Using \Cref{eq:partial-var-bound} to bound $\Var_{\mathrm{num}}[R]$,
the dependence on $(1-R^{2})^{3}$ cancels, leading to:
\[
    \sigma_{\mathrm{num}}[p(X)]
    \geq
    2f_t(|t_0|)\sqrt{\frac{df}{n-1}}\nu_{\mathrm{npv}}.
\]

\subsection{Probability of false positives induced by numerical uncertainty}
\label{sec:false-positives-modeling}

To quantify the probability of false positive findings arising from numerical
variability, we model the computed $p$-value as a random variable following a
Beta distribution, $p \sim \text{Beta}(a, b)$, which is suitable for modeling
probabilities and proportions bounded on $[0,1]$. The distribution is
parameterized by shape parameters $(a,b)$ determined from a target mean $\mu_p$
and variance $\sigma_p^2$ estimated using the uncertainty propagation formulae
reported in Table~\ref{tab:stat_uncertainty}. The parameters are given by:

\begin{equation*}
    a = \mu_p\left(\frac{\mu_p(1 - \mu_p)}{\sigma_p^2} - 1\right), \quad
    b = (1 - \mu_p)\left(\frac{\mu_p(1 - \mu_p)}{\sigma_p^2} - 1\right).
    \label{eq:beta-params}
\end{equation*}

This formulation defines a distribution of plausible $p$-values centered around
the nominal value $p_0=\mu_p$, with dispersion determined by numerical
instability. Importantly, this model enables direct estimation of error rates
caused by numerical perturbations relative to a fixed significance threshold
$\alpha$. Two distinct regimes are considered, depending on whether the nominal
$p$-value $p_0$ lies above or below $\alpha$
(Fig.~\ref{fig:false-positives-model}):

\begin{itemize}
    \item Negative Case ($p_0 > \alpha$): The primary finding is
          non-significant, i.e., a true negative (TN). Here, we calculate the
          probability that numerical noise shifts the $p$-value below $\alpha$,
          resulting in a false negative (FN) (instability leading to Type I-like
          errors).
    \item Positive Case ($p_0 \leq \alpha$): The primary finding is significant,
          i.e., a true positive (TP). We calculate the probability that
          numerical noise shifts the $p$-value above $\alpha$, resulting in a
          false positive (FP) (instability leading to Type II-like errors).
\end{itemize}

The probabilities for these outcomes are obtained by computing the cumulative
distribution function (CDF) of the Beta distribution up to the threshold
$\alpha$, as illustrated in Figure~\ref{fig:false-positives-model}. This
probabilistic modelisation provides an explicit link between numerical
variability and classical Type~I and Type~II error rates, allowing us to isolate
the contribution of numerical variability to spurious statistical significance.
Numerical validation of this model is presented in
Appendix Section~\ref{appendix:numerical_validation_flip_significance}.

\vspace{1em}

% --- Figure Code ---
\begin{figure}[h]
    \centering
    % --- H1 Figure (Negative Case) ---
    \begin{minipage}[t]{0.48\textwidth}
        \centering
        \subcaption{Negative case ($p_0 > \alpha$)\label{fig:fp-negative-case}}
        \begin{tikzpicture}
            \begin{axis}[
                    axis lines=left,
                    xmin=0, xmax=1.05,
                    ymin=0, ymax=3.5,
                    xlabel={$p$-value},
                    ytick=\empty,
                    xtick={0,0.25,1},
                    xticklabels={$0$,$\alpha$,$1$},
                    width=\linewidth, height=5cm,
                    clip=false
                ]
                % Beta Distribution Curve for H1 (Skewed Left approximation)
                \addplot[thick, samples=200, domain=0.001:1, name path=H1Curve]
                {30 * x^(2.5) * (1-x)^1.5};

                \path[name path=xaxis] (axis cs:0,0) -- (axis cs:1,0);

                % Alpha line
                \draw[dashed, axisGrey] (axis cs:0.25,0) -- (axis cs:0.25,3.2);

                % --- Shading ---
                % Area below alpha (Spurious Significance / "False Positive")
                \addplot[falseColor!30] fill between[of=H1Curve and xaxis, soft clip={domain=0:0.25}];
                % Area above alpha (Consistent Negative)
                \addplot[trueColor!30] fill between[of=H1Curve and xaxis, soft clip={domain=0.25:1}];

                % Labels
                % Note: Check definitions of FN/TN in your preamble. 
                % Standard stats: p0 > alpha is Negative. If noise makes p < alpha, it's a False Positive.
                \node[falseColor!80!black] at (axis cs:0.1, 0.55) {\textbf{\FN}};
                \node[trueColor!80!black] at (axis cs:0.45, 0.55) {\textbf{\TN}};

                % Sample p0
                \draw[-{Latex}, thick] (axis cs:0.62, 0) -- (axis cs:0.62, 2.5) node[above, right] {$p_0$};
            \end{axis}
        \end{tikzpicture}
        \begin{align*}
            \mathbb{P}[\FN] & = \int_0^\alpha \text{Beta}(x; a, b) dx \\
            \mathbb{P}[\TN] & = 1 - \mathbb{P}[\FN]
        \end{align*}
    \end{minipage}%
    \hfill
    % --- H0 Figure (Positive Case) ---
    \begin{minipage}[t]{0.48\textwidth}
        \centering
        \subcaption{Positive case ($p_0 \leq \alpha$)\label{fig:fp-positive-case}}
        \begin{tikzpicture}
            \begin{axis}[
                    axis lines=left,
                    xmin=0, xmax=1.05,
                    ymin=0, ymax=2.5,
                    xlabel={$p$-value},
                    ytick=\empty,
                    xtick={0,0.25,1},
                    xticklabels={$0$,$\alpha$,$1$},
                    width=\linewidth, height=5cm,
                    clip=false
                ]
                % Beta Distribution Curve for H0 (Skewed Right approximation)
                \addplot[thick, samples=200, domain=0:1, name path=H0Curve]
                {7.5 * x^(0.5) * (1-x)^(5)};

                \path[name path=xaxis] (axis cs:0,0) -- (axis cs:1,0);

                % Alpha line
                \draw[dashed, axisGrey] (axis cs:0.25,0) -- (axis cs:0.25,2.2);

                % --- Shading ---
                % Area below alpha (Consistent Positive)
                \addplot[trueColor!30] fill between[of=H0Curve and xaxis, soft clip={domain=0:0.25}];
                % Area above alpha (Missed Significance / "False Negative")
                \addplot[falseColor!30] fill between[of=H0Curve and xaxis, soft clip={domain=0.25:1}];

                % Labels
                \node[trueColor!80!black] at (axis cs:0.17, 0.5) {\textbf{\TP}};
                \node[falseColor!80!black] at (axis cs:0.5, 0.5) {\textbf{\FP}};

                % Sample p0
                \draw[-{Latex}, thick] (axis cs:0.085, 0) -- (axis cs:0.085, 1.75) node[above, right] {$p_0$};
            \end{axis}
        \end{tikzpicture}
        \begin{align*}
            \mathbb{P}[\TP] & = \int_0^\alpha \text{Beta}(x; a, b) dx \\
            \mathbb{P}[\FP] & = 1 - \mathbb{P}[\TP]
        \end{align*}
    \end{minipage}
    \caption{\textbf{Modeling inference stability using Beta distributions.}
        The panels illustrate the probability of significance flipping due to
        numerical uncertainty. The negative case ($p_0 > \alpha$, Fig.~\ref{fig:fp-negative-case}),
        where the tail of the distribution crossing $\alpha$ represents the
        probability of a false positive finding (\FN). The positive
        case ($p_0 \leq \alpha$, Fig.~\ref{fig:fp-positive-case}), where the tail extending beyond $\alpha$
        represents the probability of a false negative finding (\FP). Shaded
        regions indicate the integration of the Beta PDF.}
    \label{fig:false-positives-model}
\end{figure}




