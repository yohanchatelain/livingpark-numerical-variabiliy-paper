\documentclass{article}

% Language setting Replace `english' with e.g. `spanish' to change the document
% language
\usepackage[english]{babel}

% Set page size and margins Replace `letterpaper' with `a4paper' for UK/EU
% standard size
\usepackage[letterpaper,top=2cm,bottom=2cm,left=3cm,right=3cm,marginparwidth=1.75cm]{geometry}

% Useful packages
\usepackage{amsmath}
\usepackage{graphicx}
\usepackage[colorlinks=true, allcolors=blue]{hyperref}
\usepackage{todonotes}
\usepackage{longtable}

\title{Numerical Variability in FreeSurfer 7.3.1: Implications for Parkinson's
    Disease Research}
\author{You}

\begin{document}
\maketitle

\begin{abstract}
    Neuroimaging studies, particularly those involving structural MRI analysis,
    have faced challenges in reproducibility due to software variability. This
    study investigates the impact of numerical instability in FreeSurfer 7.3.1 on
    structural MRI measures and their relationships with clinical outcomes in
    Parkinson's disease (PD). Utilizing Monte Carlo Arithmetic (MCA), we evaluate
    the extent of numerical variability in FreeSurfer's estimation of brain
    structure metrics. Our findings suggest that software variability ...
    \todo[inline]{complete with results}
\end{abstract}

\section{Introduction}

FreeSurfer is a widely used, publicly available software for analyzing and
visualizing structural neuroimaging data. Version 7.3.1 represents one of the
latest major releases, offering advanced capabilities for estimating various
brain structure metrics such as gray matter volume, cortical thickness, and
surface area.

Recent studies have highlighted significant variability in neuroimaging
results due to differences in software and processing pipelines. For instance,
Botvinik-Nezer et al. (2020) demonstrated low agreement levels (21\% to 37\%)
among research teams analyzing the same fMRI data with different pipelines.
This variability extends to structural MRI measures, with studies showing
significant differences in brain volume and cortical thickness estimations
between software versions and toolboxes.

Parkinson's disease (PD) is a progressive neurodegenerative disorder affecting
millions worldwide. Its prevalence and the potential of neuroimaging as a
diagnostic and prognostic tool make it a critical area for investigating the
impact of numerical variability in brain structure analysis.

\subsection{Motivation}

The reproducibility crisis in neuroimaging, particularly in the context of
clinical research, underscores the importance of studying numerical
variability. In Parkinson's disease research, MRI-derived measures have shown
associations with disease severity, progression, and differentiation between
various syndromes. However, the lack of established MRI measures for PD
diagnosis or tracking may partly stem from measurement variability across
studies. Understanding the extent and impact of this variability is crucial
for:

\begin{enumerate}
    \item  Enhancing the reliability of neuroimaging studies in PD research.
    \item  Improving the potential of MRI as a diagnostic and prognostic tool
          for PD.
    \item  Advancing our understanding of the relationship between brain
          structure and PD clinical outcomes.
\end{enumerate}

\subsection{Objectives}

\emph{Cross-sectional Analysis}: To assess the difference in numerical
variability between Healthy Control (HC) and Parkinson's Disease (PD)
populations using FreeSurfer 7.3.1. This involves comparing the estimation of
structural MRI measures (gray matter volume, cortical thickness, and surface
area) between the two groups and evaluating how numerical instability affects
these estimations.

\emph{Longitudinal Analysis}: To evaluate the impact of numerical variability
on clinical research outcomes in PD. This includes:
\begin{itemize}
    \item Investigating how numerical variability affects the detection of group
          differences between HC and PD in subcortical volumes and cortical thickness
          over time.
    \item Assessing the influence of numerical instability on the observed
          correlations between disease severity and brain structure measures in PD
          patients, both at baseline and longitudinally.
\end{itemize}

By addressing these objectives, we aim to provide insights into the reliability
of FreeSurfer 7.3.1 in PD research and offer recommendations for mitigating the
effects of numerical variability in clinical neuroimaging studies.

\section{Methods}

\subsection{Numerical variability assessment}

The floating-point arithmetic, ruled by the IEEE-754 norm, system approximates
real numbers using finite precision. Given this limited precision, rounding off
becomes essential, which can lead to cumulative errors, rendering some
calculations imprecise or incorrect. The challenge of accurately quantifying
the
error in extensive scientific computations arises from the large number of
calculations involved. Stochastic arithmetic, with its reliance on randomness,
simplifies this challenge by converting it into a matter of statistical
sampling.

\subsubsection{Monte Carlo Arithmetic}

Among the stochastic arithmetic techniques, Monte Carlo
Arithmetic~\cite{parker1997monte} (MCA) extends floating-point arithmetic by
exploiting randomness to assess numerical instability that arises from finite
precision calculations. MCA introduces random noise into basic arithmetic
operations $(+,-,\times,\div)$ to emulate unbiased, random rounding that is
independent of prior operations. Consequently, MCA facilitates assessments of
changes in Operating Systems or architectures, enabling the evaluation of their
impact on the significance of results. By running the same code multiple times,
the stability of outcomes can be measured by calculating the number of
significant digits in the outputs.

\subsubsection{Significant digits formula}

We compute the number of significant bits \(\hat{s}\) with probability
\(p_s=0.95\) and confidence \(1-\alpha_s=0.95\) using the \emph{Significant
    Digits}
package\footnote{\url{https://github.com/verificarlo/significantdigits}}
(version 0.2.0). \emph{Significant Digits} implements the Centered Normality
Hypothesis approach described in~\cite{sohier2021confidence}:
\[
    \hat{s_i} = -\log_2 \left| \frac{\hat{\sigma_i}}{\hat{\mu_i}} \right| -
    \delta(n, \alpha_s, p_s),
\]
where \(\hat{\sigma_i}\) and \(\hat{\mu_i}\) are the average and standard
deviation over the repetitions, and
\begin{equation}
    \delta(n, \alpha_s, p_s) = \log_2 \left(
    \sqrt{\frac{n-1}{\chi^2_{1-\alpha_s/2}}} \Phi^{-1} \left( \frac{p_s+1}{2}
    \right) \right)
\end{equation}
is a penalty term for estimating \(\hat{s_i}\) with probability \(p_s\) and
confidence level \(1-\alpha_s\) for a sample size \(n\). \(\Phi^{-1}\) is the
inverse cumulative distribution of the standard normal distribution and
\(\chi^2\) is the Chi-2 distribution with \(n\)-1 degrees of freedom.

\subsubsection{Fuzzy-libm}

MCA has been extended to several libraries including the \texttt{libm} which
comprises elementary mathematical functions \texttt{(exp, log, cos, sin, ...)}.
The Fuzzy-libm~\cite{salari2021accurate} project presents an ecosystem of
Docker images, offering a \texttt{libm} that is recompiled with
Verificarlo~\cite{denis2015verificarlo}. This LLVM-based compiler replaces
floating-point arithmetic instructions for their MCA equivalents. The
Fuzzy-libm applies a random perturbation on the results of \texttt{libm}'
functions to simulate different implementations. It has been demonstrated to be
a good proxy to simulate OS changes~\cite{salari2021accurate}.

\subsection{Participants}

\begin{itemize}
    \item Details on the execution of 10 Freesurfer runs with Fuzzy-libm for
          each of the 315 subjects (106 HC and 209 PD).
\end{itemize}

We executed \emph{FreeSurfer} \texttt{recon-all} command 10 times for each
subject with the Fuzzy-libm. We set the virtual precision to 53 for binary64
(double precision) and 24 for binary32 (single precision) to simulate
error-machine precision.

\todo{Detail failing execution + image QC}

In total, 210 patients diagnosed with Parkinson's Disease (PD) participated in
the study: 181 with PD without mild cognitive impairment (PD-non-MCI) and 29
with PD with mild cognitive impairment (PD-MCI). Additionally, 106 healthy
controls (HC) were included. All participants were sourced from the Parkinson’s
Progression Markers Initiative (PPMI; www.ppmi-info.org). The inclusion
criteria for PD patients comprised a primary diagnosis of PD, the availability
of a T1-weighted scan, and the absence of other neurological diagnoses. For the
clinical analysis, 125 PD-non-MCI patients and all 106 HC were considered.
Every participant in this subset had two study visits with T1-weighted images
available. Additionally, PD patients were evaluated using the Unified
Parkinson’s Disease Rating Scale (UPDRS) scores. PD-MCI patients were excluded
from the clinical analysis to prevent potential confounding effects of MCI on
clinical evaluations. The data collection was sanctioned by the local ethics
committees of PPMI's participating institutions, and all participants provided
written informed consent. Descriptive statistics for each group can be found in
Table 1. The study adhered to the Declaration of Helsinki and received an
exemption from Concordia University’s Research Ethics Unit.

\subsection{Image acquisition}

T1-weighted MRI images were obtained from PPMI that uses standardized
acquisition parameters: repetition time = 2.3 s, echo time = 2.98 s, inversion
time = 0.9 s, slice thickness = 1 mm, number of slices = 192, field of view =
256 mm, and matrix size = 256 $\times$ 256. However, since PPMI is a multisite
project there may be slight differences in the sites’ setup.

Brain images were processed using FreeSurfer 7.3.1~\cite{fischl2012freesurfer}.
FreeSurfer’s \texttt{recon-all} function was used for cortical reconstruction.
Volumes, cortical thickness, and surface area were extracted for each
participant. The longitudinal preprocessing stream was used to analyze images
in cohort II~\cite{reuter2012within}. The two-time points were processed
cross-sectionally with the default pipeline, an unbiased template from the two
images was created, and data were processed longitudinally. Specifically, an
unbiased within-subject template space and image \cite{reuter2011avoiding} is
created using robust, inverse-consistent registration~\cite{reuter2010highly}.
We did analyses on the Virtual Imaging Platform~\cite{glatard2012virtual} that
utilizes resources offered by the Biomed virtual organization within the
European Grid Infrastructure (EGI). Specifically, the authors wish to extend
gratitude to Sorina Pop from CREATIS, Lyon, France. Quality control of raw
images was performed with MRIQC. Preprocessed images were visually inspected
for quality.

\subsection{Data Analysis}

We executed two distinct statistical analyses: computational and clinical. We
delved into the variations present in FreeSurfer outputs, specifically focusing
on the estimation of volume, cortical thickness, and surface area. For each
image, we extracted the cortical and subcortical volumes, as well as the
cortical surface areas and thicknesses from all FreeSurfer iterations. To
quantify the numerical variability across these measurements, we calculated the
number of significant digits. For a more targeted analysis, we evaluated the
Parkinson's Disease (PD) and Healthy Control (HC) groups separately. We
employed
the Whitney-Mann test to find out potential variances in software stability
between these subgroups.

\begin{itemize}
    \item Description of the measurement of numerical variability in terms of
          the number of significant digits.
    \item Explanation of statistical analysis, including the Whitney-Mann test
          and Bonferroni correction.
\end{itemize}

\section{Results}

\subsection{Descriptive Statistics}

\begin{figure}
    \includegraphics*[width=\linewidth]{figures/group-thickness-sig.png}
    \caption{Number of significant digits of cortical thickness for each ROI
        and each group.}
\end{figure}

\begin{figure}
    \includegraphics*[width=\linewidth]{figures/group-area-sig.png}
    \caption{Number of significant digits of cortical surface area for each ROI
        and each group.}
\end{figure}

\begin{figure}
    \includegraphics*[width=\linewidth]{figures/group-volume-sig.png}
    \caption{Number of significant digits of cortical volume for each ROI and
        each group.}
\end{figure}

\begin{figure}

    \includegraphics*[width=\linewidth]{figures/group-subcortical-volume-sig.png}
    \caption{Number of significant digits of subcortical volume for each ROI
        and each group.}
\end{figure}

\begin{longtable}[h]{|l|c|c|c|}
    \hline
                                                   & HC               &
    PD-non-MCI                                     & All
    \\
    \hline
    Cortical thickness                             & $1.37 \pm 0.27$  & $1.39
    \pm 0.29$                                      & $1.38 \pm 0.28$
    \\
    \hline
    \quad G\_Ins\_lg\_and\_S\_cent\_ins\_thickness & $1.02 \pm	0.21$   & $1.04
    \pm  0.21 $                                    & $1.03 \pm 0.21$
    \\
    \quad G\_and\_S\_cingul-Ant\_thickness         & $1.50 \pm	0.19$   & $1.47
    \pm  0.19 $                                    & $1.48 \pm 0.19$
    \\
    \quad G\_and\_S\_cingul-Mid-Ant\_thickness     & $1.44 \pm	0.18$   & $1.47
    \pm  0.21 $                                    & $1.45 \pm 0.20$
    \\
    \quad G\_and\_S\_cingul-Mid-Post\_thickness    & $1.52 \pm	0.17$   & $1.51
    \pm  0.22 $                                    & $1.51 \pm 0.20$
    \\
    \quad G\_and\_S\_frontomargin\_thickness       & $1.24 \pm	0.22$   & $1.26
    \pm  0.19 $                                    & $1.25 \pm 0.20$
    \\
    \quad G\_and\_S\_occipital\_inf\_thickness     & $1.37 \pm	0.19$   & $1.40
    \pm  0.20 $                                    & $1.38 \pm 0.20$
    \\
    \quad G\_and\_S\_paracentral\_thickness        & $1.30 \pm	0.26$   & $1.37
    \pm  0.25 $                                    & $1.34 \pm 0.25$
    \\
    \quad G\_and\_S\_subcentral\_thickness         & $1.50 \pm	0.18$   & $1.48
    \pm  0.23 $                                    & $1.49 \pm 0.21$
    \\
    \quad G\_and\_S\_transv\_frontopol\_thickness  & $1.18 \pm	0.24$   & $1.24
    \pm  0.24 $                                    & $1.22 \pm 0.24$
    \\
    \quad G\_cingul-Post-dorsal\_thickness         & $1.38 \pm	0.16$   & $1.38
    \pm  0.21 $                                    & $1.38 \pm 0.19$
    \\
    \quad G\_cingul-Post-ventral\_thickness        & $1.14 \pm	0.18$   & $1.12
    \pm  0.20 $                                    & $1.13 \pm 0.19$
    \\
    \quad G\_cuneus\_thickness                     & $1.27 \pm	0.24$   & $1.34
    \pm  0.24 $                                    & $1.31 \pm 0.24$
    \\
    \quad G\_front\_inf-Opercular\_thickness       & $1.54 \pm	0.18$   & $1.56
    \pm  0.24 $                                    & $1.55 \pm 0.21$
    \\
    \quad G\_front\_inf-Orbital\_thickness         & $1.25 \pm	0.26$   & $1.28
    \pm  0.24 $                                    & $1.26 \pm 0.25$
    \\
    \quad G\_front\_inf-Triangul\_thickness        & $1.36 \pm	0.25$   & $1.41
    \pm  0.23 $                                    & $1.39 \pm 0.24$
    \\
    \quad G\_front\_middle\_thickness              & $1.53 \pm	0.24$   & $1.56
    \pm  0.23 $                                    & $1.55 \pm 0.23$
    \\
    \quad G\_front\_sup\_thickness                 & $1.63 \pm	0.23$   & $1.67
    \pm  0.25 $                                    & $1.65 \pm 0.24$
    \\
    \quad G\_insular\_short\_thickness             & $1.01 \pm	0.23$   & $1.03
    \pm  0.24 $                                    & $1.02 \pm 0.23$
    \\
    \quad G\_oc-temp\_lat-fusifor\_thickness       & $1.45 \pm	0.19$   & $1.48
    \pm  0.23 $                                    & $1.47 \pm 0.21$
    \\
    \quad G\_oc-temp\_med-Lingual\_thickness       & $1.21 \pm	0.23$   & $1.28
    \pm  0.24 $                                    & $1.25 \pm 0.24$
    \\
    \quad G\_oc-temp\_med-Parahip\_thickness       & $1.25 \pm	0.23$   & $1.29
    \pm  0.24 $                                    & $1.27 \pm 0.23$
    \\
    \quad G\_occipital\_middle\_thickness          & $1.52 \pm	0.17$   & $1.52
    \pm  0.22 $                                    & $1.52 \pm 0.20$
    \\
    \quad G\_occipital\_sup\_thickness             & $1.44 \pm	0.21$   & $1.46
    \pm  0.24 $                                    & $1.45 \pm 0.22$
    \\
    \quad G\_orbital\_thickness                    & $1.37 \pm	0.22$   & $1.40
    \pm  0.24 $                                    & $1.38 \pm 0.23$
    \\
    \quad G\_pariet\_inf-Angular\_thickness        & $1.58 \pm	0.18$   & $1.60
    \pm  0.20 $                                    & $1.59 \pm 0.19$
    \\
    \quad G\_pariet\_inf-Supramar\_thickness       & $1.61 \pm	0.20$   & $1.60
    \pm  0.22 $                                    & $1.61 \pm 0.21$
    \\
    \quad G\_parietal\_sup\_thickness              & $1.59 \pm	0.23$   & $1.62
    \pm  0.24 $                                    & $1.61 \pm 0.24$
    \\
    \quad G\_postcentral\_thickness                & $1.49 \pm	0.26$   & $1.52
    \pm  0.27 $                                    & $1.51 \pm 0.27$
    \\
    \quad G\_precentral\_thickness                 & $1.43 \pm	0.29$   & $1.48
    \pm  0.32 $                                    & $1.45 \pm 0.31$
    \\
    \quad G\_precuneus\_thickness                  & $1.58 \pm	0.16$   & $1.59
    \pm  0.21 $                                    & $1.59 \pm 0.19$
    \\
    \quad G\_rectus\_thickness                     & $1.22 \pm	0.21$   & $1.23
    \pm  0.23 $                                    & $1.23 \pm 0.22$
    \\
    \quad G\_subcallosal\_thickness                & $0.93 \pm	0.18$   & $0.95
    \pm  0.17 $                                    & $0.94 \pm 0.17$
    \\
    \quad G\_temp\_sup-G\_T\_transv\_thickness     & $1.28 \pm	0.26$   & $1.30
    \pm  0.26 $                                    & $1.29 \pm 0.26$
    \\
    \quad G\_temp\_sup-Lateral\_thickness          & $1.52 \pm	0.22$   & $1.54
    \pm  0.23 $                                    & $1.53 \pm 0.22$
    \\
    \quad G\_temp\_sup-Plan\_polar\_thickness      & $1.14 \pm	0.23$   & $1.18
    \pm  0.26 $                                    & $1.16 \pm 0.24$
    \\
    \quad G\_temp\_sup-Plan\_tempo\_thickness      & $1.50 \pm	0.21$   & $1.51
    \pm  0.24 $                                    & $1.50 \pm 0.23$
    \\
    \quad G\_temporal\_inf\_thickness              & $1.45 \pm	0.19$   & $1.46
    \pm  0.20 $                                    & $1.45 \pm 0.20$
    \\
    \quad G\_temporal\_middle\_thickness           & $1.56 \pm	0.21$   & $1.58
    \pm  0.21 $                                    & $1.57 \pm 0.21$
    \\
    \quad Lat\_Fis-ant-Horizont\_thickness         & $1.24 \pm	0.22$   & $1.25
    \pm  0.22 $                                    & $1.24 \pm 0.22$
    \\
    \quad Lat\_Fis-ant-Vertical\_thickness         & $1.15 \pm	0.21$   & $1.17
    \pm  0.24 $                                    & $1.16 \pm 0.23$
    \\
    \quad Lat\_Fis-post\_thickness                 & $1.52 \pm	0.18$   & $1.52
    \pm  0.22 $                                    & $1.52 \pm 0.20$
    \\
    \quad MeanThickness\_thickness                 & $1.92 \pm	0.19$   & $1.95
    \pm  0.22 $                                    & $1.93 \pm 0.21$
    \\
    \quad Pole\_occipital\_thickness               & $1.32 \pm	0.21$   & $1.34
    \pm  0.24 $                                    & $1.33 \pm 0.22$
    \\
    \quad Pole\_temporal\_thickness                & $1.24 \pm	0.25$   & $1.32
    \pm  0.27 $                                    & $1.28 \pm 0.27$
    \\
    \quad S\_calcarine\_thickness                  & $1.30 \pm	0.21$   & $1.34
    \pm  0.25 $                                    & $1.32 \pm 0.23$
    \\
    \quad S\_central\_thickness                    & $1.51 \pm	0.24$   & $1.49
    \pm  0.27 $                                    & $1.50 \pm 0.26$
    \\
    \quad S\_cingul-Marginalis\_thickness          & $1.55 \pm	0.18$   & $1.53
    \pm  0.23 $                                    & $1.54 \pm 0.21$
    \\
    \quad S\_circular\_insula\_ant\_thickness      & $1.24 \pm	0.19$   & $1.30
    \pm  0.22 $                                    & $1.27 \pm 0.21$
    \\
    \quad S\_circular\_insula\_inf\_thickness      & $1.34 \pm	0.18$   & $1.36
    \pm  0.23 $                                    & $1.35 \pm 0.21$
    \\
    \quad S\_circular\_insula\_sup\_thickness      & $1.43 \pm	0.17$   & $1.45
    \pm  0.20 $                                    & $1.44 \pm 0.19$
    \\
    \quad S\_collat\_transv\_ant\_thickness        & $1.23 \pm	0.21$   & $1.26
    \pm  0.22 $                                    & $1.25 \pm 0.21$
    \\
    \quad S\_collat\_transv\_post\_thickness       & $1.19 \pm	0.19$   & $1.19
    \pm  0.22 $                                    & $1.19 \pm 0.21$
    \\
    \quad S\_front\_inf\_thickness                 & $1.48 \pm	0.20$   & $1.50
    \pm  0.22 $                                    & $1.49 \pm 0.21$
    \\
    \quad S\_front\_middle\_thickness              & $1.33 \pm	0.20$   & $1.35
    \pm  0.22 $                                    & $1.34 \pm 0.21$
    \\
    \quad S\_front\_sup\_thickness                 & $1.53 \pm	0.19$   & $1.55
    \pm  0.21 $                                    & $1.54 \pm 0.20$
    \\
    \quad S\_interm\_prim-Jensen\_thickness        & $1.13 \pm	0.27$   & $1.16
    \pm  0.28 $                                    & $1.15 \pm 0.28$
    \\
    \quad S\_intrapariet\_and\_P\_trans\_thickness & $1.59 \pm	0.18$   & $1.59
    \pm  0.24 $                                    & $1.59 \pm 0.22$
    \\
    \quad S\_oc-temp\_lat\_thickness               & $1.30 \pm	0.20$   & $1.34
    \pm  0.22 $                                    & $1.32 \pm 0.21$
    \\
    \quad S\_oc-temp\_med\_and\_Lingual\_thickness & $1.46 \pm	0.18$   & $1.49
    \pm  0.18 $                                    & $1.48 \pm 0.18$
    \\
    \quad S\_oc\_middle\_and\_Lunatus\_thickness   & $1.35 \pm	0.18$   & $1.36
    \pm  0.21 $                                    & $1.36 \pm 0.20$
    \\
    \quad S\_oc\_sup\_and\_transversal\_thickness  & $1.49 \pm	0.17$   & $1.46
    \pm  0.22 $                                    & $1.47 \pm 0.20$
    \\
    \quad S\_occipital\_ant\_thickness             & $1.31 \pm	0.19$   & $1.30
    \pm  0.23 $                                    & $1.31 \pm 0.21$
    \\
    \quad S\_orbital-H\_Shaped\_thickness          & $1.37 \pm	0.22$   & $1.38
    \pm  0.18 $                                    & $1.37 \pm 0.20$
    \\
    \quad S\_orbital\_lateral\_thickness           & $1.10 \pm	0.21$   & $1.11
    \pm  0.19 $                                    & $1.10 \pm 0.20$
    \\
    \quad S\_orbital\_med-olfact\_thickness        & $1.11 \pm	0.18$   & $1.17
    \pm  0.18 $                                    & $1.14 \pm 0.19$
    \\
    \quad S\_parieto\_occipital\_thickness         & $1.59 \pm	0.18$   & $1.58
    \pm  0.23 $                                    & $1.58 \pm 0.21$
    \\
    \quad S\_pericallosal\_thickness               & $1.08 \pm	0.16$   & $1.13
    \pm  0.18 $                                    & $1.11 \pm 0.18$
    \\
    \quad S\_postcentral\_thickness                & $1.60 \pm	0.20$   & $1.61
    \pm  0.24 $                                    & $1.61 \pm 0.22$
    \\
    \quad S\_precentral-inf-part\_thickness        & $1.53 \pm	0.23$   & $1.52
    \pm  0.25 $                                    & $1.52 \pm 0.24$
    \\
    \quad S\_precentral-sup-part\_thickness        & $1.46 \pm	0.23$   & $1.44
    \pm  0.28 $                                    & $1.45 \pm 0.26$
    \\
    \quad S\_suborbital\_thickness                 & $1.06 \pm	0.20$   & $1.07
    \pm  0.23 $                                    & $1.06 \pm 0.22$
    \\
    \quad S\_subparietal\_thickness                & $1.40 \pm	0.16$   & $1.38
    \pm  0.20 $                                    & $1.39 \pm 0.18$
    \\
    \quad S\_temporal\_inf\_thickness              & $1.32 \pm	0.19$   & $1.33
    \pm  0.18 $                                    & $1.33 \pm 0.18$
    \\
    \quad S\_temporal\_sup\_thickness              & $1.61 \pm	0.16$   & $1.62
    \pm  0.20 $                                    & $1.61 \pm 0.19$
    \\
    \quad S\_temporal\_transverse\_thickness       & $1.17 \pm	0.22$   & $1.18
    \pm  0.23 $                                    & $1.17 \pm 0.22$
    \\

    \hline
    Cortical surface area                          & $0.93 \pm	0.30$   & $0.94
    \pm  0.31$                                     & $0.93 \pm 0.31$
    \\
    \hline
    Cortical volume                                & $0.88 \pm	0.29$   & $
        0.89
    \pm  0.30 $                                    & $0.89  \pm 0.29$
    \\
    \hline
    Subcortical volume                             & $1.44 \pm	0.76$   & $1.45
    \pm 0.75$                                      & $1.45 \pm 0.75$
    \\
    \hline
    \caption{Significant digits average across all subjects and ROI.}
\end{longtable}

\begin{itemize}
    \item Presentation of the basic numerical summary of the collected data for
          both HC and PD groups.
\end{itemize}

\subsection{Inferential Statistics}

\begin{itemize}
    \item Detailed presentation of the results of the Whitney-Mann test and
          Bonferroni correction.
    \item Comparative analysis between HC and PD populations based on the
          different ROIs for cortical and subcortical volumes, surface area,
          and
          cortical thickness.
\end{itemize}

\section{Discussion}

\subsection{Interpretation of Results}

\begin{itemize}
    \item Detailed interpretation of the observed numerical variability in HC
          and PD populations and its implications.
\end{itemize}

\subsection{Impact on Clinical Research}

\begin{itemize}
    \item Discussion on how the observed numerical variability can affect
          clinical research.
    \item Exploration of potential ramifications in neurological studies,
          particularly those involving Parkinson's disease.
\end{itemize}

\subsection{Limitations and Future Work}

\section{Conclusions}

Summary of the findings regarding the numerical variability in Freesurfer 7.3.1
between HC and PD populations and its prospective impact on clinical research.
Provisional conclusion pending the exploration of goal 2.

\section{Acknowledgements}

ReproVIP team.

\bibliographystyle{alpha}
\bibliography{biblio}

\end{document}