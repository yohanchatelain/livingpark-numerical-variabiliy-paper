\documentclass{article}

% Language setting Replace `english' with e.g. `spanish' to change the document
% language
\usepackage[english]{babel}

% Set page size and margins Replace `letterpaper' with `a4paper' for UK/EU
% standard size
\usepackage[letterpaper,top=2cm,bottom=2cm,left=3cm,right=3cm,marginparwidth=1.75cm]{geometry}

% Useful packages
\usepackage{amsmath}
\usepackage{graphicx}
\usepackage[colorlinks=true, allcolors=blue]{hyperref}
\usepackage{todonotes}
\usepackage{longtable}
\usepackage{booktabs}
\usepackage{multirow}
\usepackage{subcaption}
\usepackage{xcolor}
\usepackage{tabularray}
\usepackage{booktabs} % For \toprule, \midrule, \bottomrule
\usepackage{xspace} % For \xspace command

\newcommand{\YC}[1]{\textcolor{red}{YC: #1}}

\newcommand{\0}{\mspace{9mu}}
\newcommand{\navr}{$\nu_{\text{nav}}$\xspace}

\title{Numerical vs. Anatomical Variability: Impact on Numerical Reliability on MRI measures of Parkinson's Disease}
% let's see if we put back Freesurfer

\author{Yohan Chatelain, Andrzej Sokołowski, Madeleine Sharp, Jean-Baptiste Poline, Tristan Glatard}

\begin{document}

\maketitle

\begin{abstract}
    Reproducibility in neuroimaging is critically hampered by computational
    variability within analysis software. Here, we quantify the numerical
    instability of FreeSurfer, a ubiquitous tool, using structural MRI data from the
    Parkinson's Progression Markers Initiative. By introducing controlled
    perturbations with Monte Carlo Arithmetic, we assessed the variability of
    cortical and subcortical measurements. We found that numerical variability in
    FreeSurfer is substantial, with key structural metrics often precise to only a
    single significant digit. This computational noise was frequently comparable to
    or greater than the biological variation between Parkinson's disease patients
    and healthy controls. Consequently, the statistical significance of group
    differences and correlations with clinical severity fluctuated dramatically
    across identical analyses. Our results demonstrate that subtle computational
    errors can produce unreliable findings in clinical neuroimaging. Addressing such
    numerical instability is essential for developing robust biomarkers for
    neurological disorders like Parkinson's disease.
\end{abstract}

\section{Introduction}

Neuroimaging reproducibility has emerged as a critical challenge in neuroscience
research. While inter-software variability is
well-documented~\cite{botvinik2020variability,gronenschild2012effects,bhagwat2021understanding},
within-version numerical variability—small output variations from identical
software runs—remains underexplored despite potentially significant clinical
implications. Numerical variability arises from computational factors including
floating-point precision, parallel processing, and random initializations. In
Parkinson's disease (PD) research, where MRI-derived metrics like cortical
thickness and subcortical volumes serve as potential biomarkers, such
variability could obscure subtle disease-related changes and compromise
statistical reliability.

Previous studies have demonstrated substantial between-version differences in
FreeSurfer outputs~\cite{haddad2023multisite}, but the impact of computational
uncertainty within single software versions on clinical associations remains
unclear. This gap is particularly concerning for PD research, where establishing
reliable brain-behavior relationships is essential for developing neuroimaging
biomarkers. Despite promising associations between MRI-derived metrics and PD
severity, no neuroimaging biomarkers are widely accepted for clinical diagnosis
or monitoring. Measurement variability across studies undermines reliability and
generalizability, hindering translation to clinical practice. This computational
uncertainty could significantly impact PD research by: (1) masking subtle
disease-related changes essential for early detection, (2) compromising
statistical power for detecting group differences and clinical correlations, and
(3) reducing reproducibility across studies using identical analysis pipelines.

Here, we investigate numerical variability in FreeSurfer 7.3.1 using Monte Carlo
Arithmetic to simulate realistic computational perturbations. We introduce the
Numerical-Anatomical Variability Ratio (\navr) to quantify computational
uncertainty relative to biological variation and derive its theoretical
relationship to statistical effect sizes. Using longitudinal data from the
Parkinson's Progression Markers Initiative, we assess how numerical precision
affects group comparisons and clinical correlations in PD research.
Specifically, our aims include: (1) quantifying how computational uncertainty
affects group difference detection between PD patients and healthy controls, (2)
assessing numerical variability effects on brain-behavior correlations with
clinical measures (UPDRS scores), and (3) developing the \navr framework to
predict statistical reliability from computational precision. Our findings will
inform strategies for mitigating numerical variability effects and enhancing
reproducibility in clinical neuroimaging studies.

\section{Results}

PD and HC groups showed no significant age differences ($p > 0.05$) but differed
in education ($t = -2.05$, $p = 0.04$) and sex distribution ($\chi^2 = 4.15$, $p
    = 0.04$). The longitudinal cohort showed no significant demographic differences
between groups (Table~\ref{tab:cohort_stat}).

\subsection{Numerical variability impacts MRI derived findings}

% takes Greg paper example. Summarize the methods, so people understand the method
% look at example fmriprep or experimental paper of Nature Communications
% W: We simulate 26 numerical states, ...

We assessed numerical variability in FreeSurfer 7.3.1 using Monte Carlo
Arithmetic (MCA)~\cite{parker1997monte}, more particularly the Fuzzy-libm
extension~\cite{salari2021accurate} that applies random perturbations to
mathematical library functions's output. We executed 26 \texttt{recon-all} runs
for each subject's MRI data, simulating 26 numerical states. Using \texttt{recon-all} we collected
cortical thickness, surface area, and subcortical volumes for each subject.
We present results cortical thickness and subcortical volumes, surface area and cortical volume are reported in the appendix.
We selected 103 healthy controls (HC) and 121 Parkinson's disease (PD) patients from the
Parkinson's Progression Markers Initiative (PPMI) database.

We tested the statistical significance of group differences and correlations with clinical measures (UPDRS scores) across the 26 MCA repetitions.
We reported the fraction of statistically significant tests ($p < 0.05$) for each metric and region.
Statistical significance proportions varied substantially for
cortical thickness (Figure~\ref{fig:significance_correlation_thickness}) and
subcortical volumes (Figure~\ref{fig:significance_correlation_subcortical_volume}). Ratios near 0.5
indicated maximal uncertainty, while values approaching 0 or 1 suggested
consistent results across computational variations.

This variability reflects in the partial correlation coefficients and
F-statistics from ANCOVA analyses, which showed substantial spread around
unperturbed IEEE-754 results used as reference (red markers). This indicates
that numerical variability affects both statistical significance and effect size
estimation. We note that IEEE results are most of the time included in the
distribution of coefficients \YC{assert with t-test?} but not necessarily at the
center of the distribution. We note that the spread of coefficients varies
between regions but stay within the same order of magnitude, but some regions
show a large spread of coefficients across MCA repetitions.



\begin{figure}
    \includegraphics[width=\linewidth]{figures/consistency/subcortical_volume_significance_correlation.pdf}
    \caption{ Proportion of statistically significant tests ($p < 0.05$) across the
        26 numerical states for subcortical volume
        measures.\label{fig:significance_correlation_subcortical_volume}}
\end{figure}

\begin{figure}
    \centering
    \includegraphics[width=\textwidth]{figures/consistency/cortical_thickness_significance_correlation.pdf}
    \caption{Proportion of statistically significant tests ($p < 0.05$) across the
        26 numerical states for cortical thickness
        measures.\label{fig:significance_correlation_thickness}}
    \label{fig:navr_consistency_thickness_plot}
\end{figure}

\begin{figure}
    \includegraphics[width=\linewidth]{figures/consistency/subcortical_volume_coefficients_distribution.pdf}
    \caption{ Distribution of partial correlation coefficients (r-values) and
        F-statistics from ANCOVA across MCA repetitions for subcortical volume
        measures. Red dots represent the IEEE results. The top row shows r-values,
        while the bottom row shows F-values. The left column represents baseline
        analysis, and the right column represents longitudinal
        analysis.\label{fig:statstest_coefficients_distribution}}
\end{figure}

\begin{figure}
    \centering
    \begin{subfigure}[b]{\textwidth}
        \includegraphics[width=\textwidth]{figures/consistency/cortical_thickness_coefficients_distribution-Left.pdf}
        \caption{Left hemisphere}
        \label{fig:navr_consistency_thickness_left}
    \end{subfigure}

    \begin{subfigure}[b]{\textwidth}
        \includegraphics[width=\textwidth]{figures/consistency/cortical_thickness_coefficients_distribution-Right.pdf}
        \caption{Right hemisphere}
        \label{fig:navr_consistency_thickness_right}
    \end{subfigure}
    \caption{Distribution of partial correlation coefficients for cortical
        thickness across all subjects and regions. Red triangles indicate the
        IEEE-754 run for reference. The distribution shows the variability in
        the coefficients, with some regions exhibiting higher consistency than
        others. }
    \label{fig:navr_consistency_thickness}
\end{figure}


\subsection{\navr reveals region-specific numerical instabilities}
% Keep cortical thickness and subcortical volumes.
% Add NARPS figures for both.

A common issue with neuroimaging analysis is the difficulty in assessing
numerical precision and its impact on downstream statistical tests like effect sizes. To address this,
we introduce the Numerical-Anatomical Variability Ratio (\navr), a metric designed
to quantify the relationship between numerical variability and anatomical
variability, computed as the ratio between the standard deviation of numerical variability and the standard deviation of anatomical variability.
\[
    \nu_{\text{nav}} = \frac{\sigma_{\text{num}}}{\sigma_{\text{anat}}}
\]

Figures~\ref{fig:navr_thickness} and \ref{fig:navr_subcortical} present the \navr for
cortical thickness and subcortical volumes across all brain regions.
Regions with high \navr values indicate areas where numerical variability
may compromise the detection of true anatomical differences.

One common measure used in clinical neuroimaging studies is the Cohen's d
coefficient that quantifies the effect size of group differences. The \navr can
be used to predict the uncertainty in Cohen's d due to numerical variability.
The theoretical relationship between \navr and Cohen's d uncertainty is given by:
\[
    \sigma_d = \frac{2}{\sqrt{N}} \text{\navr}
\]
where $\sigma_d$ is the standard deviation of Cohen's d and $N$ is the
population sample size. We showed in section~\ref{sec:navr_effect_size} that
this relationship holds true for our data, allowing us to quantify the impact of
numerical variability on effect size estimation.

The strength of this relationship is that one can then predict the uncertainty
in effect size estimation based on the \navr value and thus threshold the Cohen's
d value based on the \navr value. Hence, given a paper with Cohen's d values, we
can remove values $|\sigma_d|\leq \frac{2}{\sqrt{N}} \text{\navr}$ as they
will be lower than the numerical variability threshold. This allows us to assess
the reliability of the reported effect sizes in the paper. The advantage of this
approach is that it can be applied to any neuroimaging study, regardless of the
software used, and has a specific value per region, which allows for a more
fine-grained analysis of the numerical variability impact on effect size
estimation.

\begin{figure}
    \includegraphics[width=\linewidth]{figures/NAVR_map/NAVR_subcortical_volume_all.png}
    \caption{ Numerical-Anatomical Variability Ratio (\navr) for subcortical
        volumes across regions and groups. Higher \navr values indicate
        greater computational uncertainty relative to biological
        variation.\label{fig:navr_subcortical}}
\end{figure}


\begin{figure}[h]
    \includegraphics[width=\linewidth]{figures/NAVR_map/NAVR_thickness_all.png}
    \caption{ Numerical-Anatomical Variability Ratio (\navr) for cortical
        thickness across regions and groups. Higher \navr values indicate
        greater computational uncertainty relative to biological
        variation. The color scale indicates the \navr value, with warmer colors
        indicating higher \navr values.\label{fig:navr_thickness}}
\end{figure}

\subsection{\navr helps thresholding effect size uncertainty}

We applied the \navr framework to assess the reliability of reported effect
on ENIGMA studies~\cite{thompson2014enigma}. We extracted the Cohen's d
using the \texttt{enigmatoolbox} Python package~\cite{lariviere2021enigma}
colored in black regions where the Cohen's d is below the numerical variability threshold.
Figure~\ref{fig:navr_enigma} shows the results of this analysis.

\YC{Remove color bar and  add it to the figure caption.}
\begin{figure}[h]
    \centering
    \vspace{0.2cm}

    % 22q11.2 deletion syndrome row
    \begin{minipage}[b]{\textwidth}
        \begin{minipage}[c]{0.05\textwidth}
            \centering\rotatebox{90}{\textbf{22q11.2}}
        \end{minipage}%
        \begin{minipage}[c]{0.95\textwidth}
            \begin{subfigure}[c]{0.48\textwidth}
                \includegraphics[width=\textwidth]{figures/cohen_d_map/enigma/22q_thickness_all.png}
                \label{fig:enigma_22q_unthresholded}
            \end{subfigure}
            \hfill
            \begin{subfigure}[c]{0.48\textwidth}
                \includegraphics[width=\textwidth]{figures/cohen_d_map/enigma/22q_thickness_all_thresholded.png}
                \label{fig:enigma_22q_thresholded}
            \end{subfigure}
        \end{minipage}
    \end{minipage}

    \vspace{-2cm}

    % ADHD row
    \begin{minipage}[b]{\textwidth}
        \begin{minipage}[c]{0.05\textwidth}
            \centering\rotatebox{90}{\textbf{ADHD}}
        \end{minipage}%
        \begin{minipage}[c]{0.95\textwidth}
            \begin{subfigure}[c]{0.48\textwidth}
                \includegraphics[width=\textwidth]{figures/cohen_d_map/enigma/adhd_thickness_adult.png}
                \label{fig:enigma_adhd_unthresholded}
            \end{subfigure}
            \hfill
            \begin{subfigure}[c]{0.48\textwidth}
                \includegraphics[width=\textwidth]{figures/cohen_d_map/enigma/adhd_thickness_adult_thresholded.png}
                \label{fig:enigma_adhd_thresholded}
            \end{subfigure}
        \end{minipage}
    \end{minipage}

    \vspace{-2cm}

    % Autism spectrum disorder row
    \begin{minipage}[b]{\textwidth}
        \begin{minipage}[c]{0.05\textwidth}
            \centering\rotatebox{90}{\textbf{ASD}}
        \end{minipage}%
        \begin{minipage}[c]{0.95\textwidth}
            \begin{subfigure}[c]{0.48\textwidth}
                \includegraphics[width=\textwidth]{figures/cohen_d_map/enigma/asd_thickness_meta_analysis.png}
                \label{fig:enigma_asd_unthresholded}
            \end{subfigure}
            \hfill
            \begin{subfigure}[c]{0.48\textwidth}
                \includegraphics[width=\textwidth]{figures/cohen_d_map/enigma/asd_thickness_meta_analysis_thresholded.png}
                \label{fig:enigma_asd_thresholded}
            \end{subfigure}
        \end{minipage}
    \end{minipage}
    \vspace{-2cm}

    % bipolar disorder row
    \begin{minipage}[b]{\textwidth}
        \begin{minipage}[c]{0.05\textwidth}
            \centering\rotatebox{90}{\textbf{Bipolar}}
        \end{minipage}%
        \begin{minipage}[c]{0.95\textwidth}
            \begin{subfigure}[c]{0.48\textwidth}
                \includegraphics[width=\textwidth]{figures/cohen_d_map/enigma/bipolar_thickness_adult.png}
                \label{fig:enigma_bipolar_unthresholded}
            \end{subfigure}
            \hfill
            \begin{subfigure}[c]{0.48\textwidth}
                \includegraphics[width=\textwidth]{figures/cohen_d_map/enigma/bipolar_thickness_adult_thresholded.png}
                \label{fig:enigma_bipolar_thresholded}
            \end{subfigure}
        \end{minipage}
    \end{minipage}

    \vspace{-2cm}
    % depression
    \begin{minipage}[b]{\textwidth}
        \begin{minipage}[c]{0.05\textwidth}
            \centering\rotatebox{90}{\textbf{Depression}}
        \end{minipage}%
        \begin{minipage}[c]{0.95\textwidth}
            \begin{subfigure}[c]{0.48\textwidth}
                \includegraphics[width=\textwidth]{figures/cohen_d_map/enigma/depression_area_adult.png}
                \label{fig:enigma_depression_unthresholded}
            \end{subfigure}
            \hfill
            \begin{subfigure}[c]{0.48\textwidth}
                \includegraphics[width=\textwidth]{figures/cohen_d_map/enigma/depression_area_adult_thresholded.png}
                \label{fig:enigma_depression_thresholded}
            \end{subfigure}
        \end{minipage}
    \end{minipage}

    \vspace{-2cm}
    % epilepsy
    \begin{minipage}[b]{\textwidth}
        \begin{minipage}[c]{0.05\textwidth}
            \centering\rotatebox{90}{\textbf{Epilepsy}}
        \end{minipage}%
        \begin{minipage}[c]{0.95\textwidth}
            \begin{subfigure}[c]{0.48\textwidth}
                \includegraphics[width=\textwidth]{figures/cohen_d_map/enigma/epilepsy_thickness_allepilepsy.png}
                \label{fig:enigma_epilepsy_unthresholded}
            \end{subfigure}
            \hfill
            \begin{subfigure}[c]{0.48\textwidth}
                \includegraphics[width=\textwidth]{figures/cohen_d_map/enigma/epilepsy_thickness_allepilepsy_thresholded.png}
                \label{fig:enigma_epilepsy_thresholded}
            \end{subfigure}
        \end{minipage}
    \end{minipage}

    \vspace{-2cm}
    % ocd 
    \begin{minipage}[b]{\textwidth}
        \begin{minipage}[c]{0.05\textwidth}
            \centering\rotatebox{90}{\textbf{OCD}}
        \end{minipage}%
        \begin{minipage}[c]{0.95\textwidth}
            \begin{subfigure}[c]{0.48\textwidth}
                \includegraphics[width=\textwidth]{figures/cohen_d_map/enigma/ocd_thickness_adult.png}
                \label{fig:enigma_ocd_unthresholded}
            \end{subfigure}
            \hfill
            \begin{subfigure}[c]{0.48\textwidth}
                \includegraphics[width=\textwidth]{figures/cohen_d_map/enigma/ocd_thickness_adult_thresholded.png}
                \label{fig:enigma_ocd_thresholded}
            \end{subfigure}
        \end{minipage}
    \end{minipage}

    \vspace{-2cm}
    % schizophrenia
    \begin{minipage}[b]{\textwidth}
        \begin{minipage}[c]{0.05\textwidth}
            \centering\rotatebox{90}{\textbf{Schizophrenia}}
        \end{minipage}%
        \begin{minipage}[c]{0.95\textwidth}
            \begin{subfigure}[c]{0.48\textwidth}
                \includegraphics[width=\textwidth]{figures/cohen_d_map/enigma/schizophrenia_thickness_all.png}
                \label{fig:enigma_schizophrenia_unthresholded}
            \end{subfigure}
            \hfill
            \begin{subfigure}[c]{0.48\textwidth}
                \includegraphics[width=\textwidth]{figures/cohen_d_map/enigma/schizophrenia_thickness_all_thresholded.png}
                \label{fig:enigma_schizophrenia_thresholded}
            \end{subfigure}
        \end{minipage}
    \end{minipage}

    \caption{ENIGMA cortical thickness Cohen's d maps showing unthresholded effect sizes (left) and effect sizes thresholded by the \(\nu_{\mathrm{NA}}\) framework (right) for different disorders. Black regions indicate areas where Cohen's d values fall below the numerical variability threshold, demonstrating regions where reported effect sizes may be unreliable due to computational uncertainty.}
    \label{fig:navr_enigma}
\end{figure}

\begin{figure}[h]
    \centering
    \vspace{0.2cm}

    % 22q11.2 deletion syndrome row
    \begin{minipage}[b]{\textwidth}
        \begin{minipage}[c]{0.05\textwidth}
            \centering\rotatebox{90}{\textbf{22q11.2}}
        \end{minipage}%
        \begin{minipage}[c]{0.95\textwidth}
            \begin{subfigure}[c]{0.48\textwidth}
                \includegraphics[width=\textwidth]{figures/cohen_d_map/enigma/22q_subcortical_volume_all.png}
                \label{fig:enigma_22q_unthresholded_subcortical}
            \end{subfigure}
            \hfill
            \begin{subfigure}[c]{0.48\textwidth}
                \includegraphics[width=\textwidth]{figures/cohen_d_map/enigma/22q_subcortical_volume_all_thresholded.png}
                \label{fig:enigma_22q_thresholded_subcortical}
            \end{subfigure}
        \end{minipage}
    \end{minipage}

    \vspace{-2cm}

    % ADHD row
    \begin{minipage}[b]{\textwidth}
        \begin{minipage}[c]{0.05\textwidth}
            \centering\rotatebox{90}{\textbf{ADHD}}
        \end{minipage}%
        \begin{minipage}[c]{0.95\textwidth}
            \begin{subfigure}[c]{0.48\textwidth}
                \includegraphics[width=\textwidth]{figures/cohen_d_map/enigma/adhd_subcortical_volume_adult.png}
                \label{fig:enigma_adhd_unthresholded_subcortical}
            \end{subfigure}
            \hfill
            \begin{subfigure}[c]{0.48\textwidth}
                \includegraphics[width=\textwidth]{figures/cohen_d_map/enigma/adhd_subcortical_volume_adult_thresholded.png}
                \label{fig:enigma_adhd_thresholded_subcortical}
            \end{subfigure}
        \end{minipage}
    \end{minipage}

    \vspace{-2cm}

    % Autism spectrum disorder row
    \begin{minipage}[b]{\textwidth}
        \begin{minipage}[c]{0.05\textwidth}
            \centering\rotatebox{90}{\textbf{ASD}}
        \end{minipage}%
        \begin{minipage}[c]{0.95\textwidth}
            \begin{subfigure}[c]{0.48\textwidth}
                \includegraphics[width=\textwidth]{figures/cohen_d_map/enigma/asd_subcortical_volume_meta_analysis.png}
                \label{fig:enigma_asd_unthresholded_subcortical}
            \end{subfigure}
            \hfill
            \begin{subfigure}[c]{0.48\textwidth}
                \includegraphics[width=\textwidth]{figures/cohen_d_map/enigma/asd_subcortical_volume_meta_analysis_thresholded.png}
                \label{fig:enigma_asd_thresholded_subcortical}
            \end{subfigure}
        \end{minipage}
    \end{minipage}
    \vspace{-2cm}

    % bipolar disorder row
    \begin{minipage}[b]{\textwidth}
        \begin{minipage}[c]{0.05\textwidth}
            \centering\rotatebox{90}{\textbf{Bipolar}}
        \end{minipage}%
        \begin{minipage}[c]{0.95\textwidth}
            \begin{subfigure}[c]{0.48\textwidth}
                \includegraphics[width=\textwidth]{figures/cohen_d_map/enigma/bipolar_subcortical_volume_typeII.png}
                \label{fig:enigma_bipolar_unthresholded_subcortical}
            \end{subfigure}
            \hfill
            \begin{subfigure}[c]{0.48\textwidth}
                \includegraphics[width=\textwidth]{figures/cohen_d_map/enigma/bipolar_subcortical_volume_typeII_thresholded.png}
                \label{fig:enigma_bipolar_thresholded_subcortical}
            \end{subfigure}
        \end{minipage}
    \end{minipage}

    \vspace{-2cm}
    % depression
    \begin{minipage}[b]{\textwidth}
        \begin{minipage}[c]{0.05\textwidth}
            \centering\rotatebox{90}{\textbf{Depression}}
        \end{minipage}%
        \begin{minipage}[c]{0.95\textwidth}
            \begin{subfigure}[c]{0.48\textwidth}
                \includegraphics[width=\textwidth]{figures/cohen_d_map/enigma/depression_subcortical_volume_all.png}
                \label{fig:enigma_depression_unthresholded_subcortical}
            \end{subfigure}
            \hfill
            \begin{subfigure}[c]{0.48\textwidth}
                \includegraphics[width=\textwidth]{figures/cohen_d_map/enigma/depression_subcortical_volume_all_thresholded.png}
                \label{fig:enigma_depression_thresholded_subcortical}
            \end{subfigure}
        \end{minipage}
    \end{minipage}

    \vspace{-2cm}
    % epilepsy
    \begin{minipage}[b]{\textwidth}
        \begin{minipage}[c]{0.05\textwidth}
            \centering\rotatebox{90}{\textbf{Epilepsy}}
        \end{minipage}%
        \begin{minipage}[c]{0.95\textwidth}
            \begin{subfigure}[c]{0.48\textwidth}
                \includegraphics[width=\textwidth]{figures/cohen_d_map/enigma/epilepsy_subcortical_volume_allepilepsy.png}
                \label{fig:enigma_epilepsy_unthresholded_subcortical}
            \end{subfigure}
            \hfill
            \begin{subfigure}[c]{0.48\textwidth}
                \includegraphics[width=\textwidth]{figures/cohen_d_map/enigma/epilepsy_subcortical_volume_allepilepsy_thresholded.png}
                \label{fig:enigma_epilepsy_thresholded_subcortical}
            \end{subfigure}
        \end{minipage}
    \end{minipage}

    \vspace{-2cm}
    % ocd 
    \begin{minipage}[b]{\textwidth}
        \begin{minipage}[c]{0.05\textwidth}
            \centering\rotatebox{90}{\textbf{OCD}}
        \end{minipage}%
        \begin{minipage}[c]{0.95\textwidth}
            \begin{subfigure}[c]{0.48\textwidth}
                \includegraphics[width=\textwidth]{figures/cohen_d_map/enigma/ocd_subcortical_volume_adult.png}
                \label{fig:enigma_ocd_unthresholded_subcortical}
            \end{subfigure}
            \hfill
            \begin{subfigure}[c]{0.48\textwidth}
                \includegraphics[width=\textwidth]{figures/cohen_d_map/enigma/ocd_subcortical_volume_adult_thresholded.png}
                \label{fig:enigma_ocd_thresholded_subcortical}
            \end{subfigure}
        \end{minipage}
    \end{minipage}

    \vspace{-2cm}
    % schizophrenia
    \begin{minipage}[b]{\textwidth}
        \begin{minipage}[c]{0.05\textwidth}
            \centering\rotatebox{90}{\textbf{Schizophrenia}}
        \end{minipage}%
        \begin{minipage}[c]{0.95\textwidth}
            \begin{subfigure}[c]{0.48\textwidth}
                \includegraphics[width=\textwidth]{figures/cohen_d_map/enigma/schizophrenia_subcortical_volume_all.png}
                \label{fig:enigma_schizophrenia_unthresholded_subcortical}
            \end{subfigure}
            \hfill
            \begin{subfigure}[c]{0.48\textwidth}
                \includegraphics[width=\textwidth]{figures/cohen_d_map/enigma/schizophrenia_subcortical_volume_all_thresholded.png}
                \label{fig:enigma_schizophrenia_thresholded_subcortical}
            \end{subfigure}
        \end{minipage}
    \end{minipage}

    \caption{ENIGMA subcortical volume Cohen's d maps showing unthresholded effect sizes (left) and effect sizes thresholded by the \(\nu_{\mathrm{NA}}\) framework (right) for different disorders. Black regions indicate areas where Cohen's d values fall below the numerical variability threshold, demonstrating regions where reported effect sizes may be unreliable due to computational uncertainty.}
    \label{fig:navr_enigma}
\end{figure}



% design a tool that measure numerical variability on already published paper.
% -> we can check the consistency of the results.
% -> we have applied to ENIGMA
% -> when you have a large N, the effect of numerical variability is average by this large N
% -> but when N is small, the effect of numerical variability is more pronounced.

% Add the formula. \navr (Numerical Anatomical Variability Ratio)

% ENIGMA

\begin{figure}
    \includegraphics[width=\textwidth]{figures/sigma_d_contour.pdf}
    \caption{ Relationship between \navr and population sample size  \(N\) for
        predicting the uncertainty in Cohen's d effect size estimation. The contour
        lines represent different \navr values, showing how numerical variability
        scales with sample size. With a typical \navr value of 0.2,
        to maintain reliable effect size estimates $\sigma_d \leq 0.01$,
        the plot suggests to use $N \geq 1500$.}
\end{figure}

\section{Discussion}

% if N is small, be careful about numerical variability
% if N is large, numerical variability is averaged out
% rough estimate of what N is good to have 
% our cohort is homogeneous in age so sigma_anat is low but .. 


Our analysis reveals significant numerical instability in FreeSurfer 7.3.1, with
cortical measurements showing limited precision (1-1.5 significant digits) that
substantially impacts statistical reliability in neuroimaging studies. These
precision limitations pose particular challenges for detecting subtle
disease-related changes in conditions like Parkinson's disease.

The absence of significant baseline differences between PD and HC groups,
combined with inconsistent cluster detection (only 1/26 clusters reproduced
across repetitions), demonstrates how numerical variability can compromise
reproducibility. The \navr framework quantifies this relationship, showing that
computational uncertainty approaches or exceeds biological variation in many
brain regions.

Statistical test consistency varied markedly across MCA repetitions, with
methodological choices (Z-test vs. permutation test) further influencing outcome
reliability. Effect size distributions showed substantial spread around standard
IEEE-754 results, indicating that numerical precision directly affects both
significance testing and effect size estimation.

Importantly, inter-subject variability exceeded intra-subject variability,
suggesting that FreeSurfer maintains relative consistency across different
individuals despite numerical limitations. This supports continued use while
highlighting the need for improved computational precision in future
neuroimaging software development. This study demonstrates significant numerical
limitations in FreeSurfer 7.3.1, with cortical measurements exhibiting only
1-1.5 significant digits of precision. These computational constraints
substantially impact statistical reliability and reproducibility in neuroimaging
research, particularly for detecting subtle disease-related changes.

Our \navr framework quantifies the relationship between computational uncertainty
and biological variation, revealing that numerical instability approaches or
exceeds anatomical variability in many brain regions. This finding has direct
implications for statistical power, as demonstrated by inconsistent cluster
detection (only 1/26 clusters reproduced) and variable effect sizes across
identical analyses.

While inter-subject variability exceeded intra-subject variability—supporting
relative consistency across individuals—the absence of significant PD-HC
differences and weak clinical correlations highlight how numerical limitations
can mask true biological signals. The theoretical relationship between \navr and
Cohen's d uncertainty provides a framework for predicting statistical
reliability based on computational precision.

These findings emphasize the critical need for improved numerical precision in
neuroimaging software. Future developments should prioritize computational
stability to enhance the detection of subtle neurological changes and improve
reproducibility across studies. The \navr framework offers a practical tool for
assessing and comparing the numerical reliability of neuroimaging methodologies.

\section{Methods}

\subsection{Numerical variability assessment}

We employed Monte Carlo Arithmetic (MCA)~\cite{parker1997monte} to quantify
numerical instability in FreeSurfer computations. MCA introduces controlled
random perturbations into floating-point operations, simulating rounding errors
that occur across different computational environments. This stochastic approach
enables systematic assessment of result stability by measuring variation across
multiple runs of identical analyses.

We used Fuzzy-libm~\cite{salari2021accurate}, which extends MCA to mathematical
library functions (\texttt{exp}, \texttt{log}, \texttt{sin}, \texttt{cos})
through Verificarlo~\cite{denis2016verificarlo}, an LLVM-based compiler. Virtual
precision parameters were set to 53 bits for double precision and 24 bits for
single precision to simulate realistic machine-level precision errors.

\subsection{Participants}

We analyzed data from the Parkinson's Progression Markers Initiative (PPMI), a
multi-site longitudinal study. From 316 initial participants, we selected 125
Parkinson's disease patients without mild cognitive impairment (PD-non-MCI) and
106 healthy controls (HC) with complete longitudinal T1-weighted MRI data.
PD-MCI patients were excluded to avoid confounding effects of cognitive
impairment.

Inclusion criteria required: (1) primary PD diagnosis or healthy control status,
(2) availability of two visits with T1-weighted scans, and (3) absence of other
neurological diagnoses. PD severity was assessed using the Unified Parkinson's
Disease Rating Scale (UPDRS). The study received ethics approval from
participating institutions, and all participants provided written informed
consent (Table~\ref{tab:cohort_stat}).

\begin{table}[h!]
    \centering
    \begin{tabular}{lcc}
        \toprule
        \textbf{Cohort}         & \textbf{HC}        & \textbf{PD-non-MCI} \\
        \hline
        n                       & $103 $             & $121 $              \\
        Age (y)                 & $60.7 \pm 10.3 $   & $60.7 \pm \09.1 $   \\
        Age
        range                   & $30.6 - 84.3 $     & $39.2 - 78.3 $      \\
        Gender (male, \%)       & $57 \; (55.3\%)
        $                       & $80 \; (66.1\%) $                        \\
        Education (y)           & $16.6 \pm \03.3 $  & $16.1 \pm \03.0 $   \\
        UPDRS III OFF baseline  & $- $               & $23.4 \pm 10.1 $    \\
        UPDRS III OFF follow-up & $- $               & $25.8 \pm 11.1 $    \\
        Duration T2 - T1 (y)    & $\01.4 \pm \00.5 $ & $\01.4 \pm \00.7 $  \\
        \bottomrule
    \end{tabular}
    \vspace{1em}

    \caption{\textbf{Abbreviations:} MCI = Mild Cognitive Impairment; UPDRS = Unified Parkinson's Disease Rating Scale; PD = Parkinson's disease. Values are expressed as mean $\pm$ standard deviation. PD-non-MCI longitudinal sample is a subsample of the PD-non-MCI original sample that had longitudinal data and disease severity scores available.
        \label{tab:cohort_stat}}
\end{table}

\subsection{Image acquisition and preprocessing}

T1-weighted MRI images were obtained from PPMI that uses standardized
acquisition parameters: repetition time = 2.3 s, echo time = 2.98 s, inversion
time = 0.9 s, slice thickness = 1 mm, number of slices = 192, field of view =
256 mm, and matrix size = 256 $\times$ 256. However, since PPMI is a multisite
project there may be slight differences in the sites' setup.

We processed images using FreeSurfer 7.3.1 instrumented with Fuzzy-libm to
introduce controlled numerical perturbations. Each participant underwent 34
\texttt{recon-all} executions, extracting cortical thickness, surface area, and
volumes. After quality control and exclusion of failed runs, we randomly
selected 26 successful repetitions per subject to ensure balanced datasets for
statistical analysis.

Longitudinal processing followed the standard FreeSurfer
stream~\cite{reuter2012within}: cross-sectional processing of both timepoints,
followed by creation of an unbiased within-subject
template~\cite{reuter2011avoiding} using robust
registration~\cite{reuter2010highly}. Downstream analyses used unperturbed
FreeSurfer to prevent additional numerical perturbations.

\subsection{Numerical Variability Assessment}

We assessed FreeSurfer 7.3.1 numerical stability in cross-sectional and
longitudinal contexts using the Numerical-Anatomical Variability Ratio (\navr)
and its relationship to statistical effect sizes.

\subsubsection{Numerical-Anatomical Variability Ratio (\navr)}

To quantify computational stability relative to biological variation, we
developed the Numerical-Anatomical Variability Ratio (\navr). For each brain
region, \navr measures the ratio of measurement uncertainty arising from
computational processes to natural inter-subject anatomical variation:

\[
    \text{\navr} = \frac{\sigma_{\text{num}}}{\sigma_{\text{anat}}}
\]

where $\sigma_{\text{num}}$ represents numerical variability (measurement
precision across MCA repetitions for individual subjects) and
$\sigma_{\text{anat}}$ represents anatomical variability (inter-subject
differences within each repetition).

For each region of interest, measurements from $n$ MCA repetitions across $m$
subject-visit pairs form a data matrix $\mathcal{M}_{n \times m}$, where element
$x_{i,j}$ represents the measurement for subject $j$ in repetition $i$.

Numerical variability quantifies intra-subject measurement consistency:
\begin{equation}
    \sigma^2_{\text{num}} = \frac{1}{m} \sum_{j=1}^{m} \left[ \frac{1}{n-1} \sum_{i=1}^{n} (x_{i,j} - \bar{x}_{\cdot,j})^2 \right]
\end{equation}

Anatomical variability captures inter-subject differences:
\begin{equation}
    \sigma^2_{\text{anat}} = \frac{1}{n} \sum_{i=1}^{n} \left[ \frac{1}{m-1} \sum_{j=1}^{m} (x_{i,j} - \bar{x}_{i,\cdot})^2 \right]
\end{equation}

where $\bar{x}_{\cdot,j}$ and $\bar{x}_{i,\cdot}$ denote column and row means, respectively. Higher \navr values indicate regions where computational uncertainty approaches or exceeds biological variation, potentially compromising the detection of true anatomical differences.

\subsubsection{Relationship between \navr~and Effect Size Uncertainty}

We derived the theoretical relationship between \navr~and Cohen's d variability
to quantify how measurement uncertainty affects statistical effect sizes in
group comparisons.

For a balanced two-group design with total sample size $N$, each observation
decomposes as $X_{ij} = \mu_i + \varepsilon_{ij}^{(\text{anat})} +
    \varepsilon_{ij}^{(\text{num})}$, where $\mu_i$ represents the true group mean,
$\varepsilon_{ij}^{(\text{anat})} \sim \mathcal{N}(0, \sigma_{\text{anat}}^2)$
captures anatomical variation, and $\varepsilon_{ij}^{(\text{num})} \sim
    \mathcal{N}(0, \sigma_{\text{num}}^2)$ represents numerical uncertainty.

The standard deviation of Cohen's d attributable to measurement error is:
\begin{equation}
    \sigma_d = \frac{2}{\sqrt{N}} \cdot \text{\navr}
\end{equation}

This relationship emerges from error propagation analysis. The difference in
group means has variance $\text{Var}(\bar{X}_1 - \bar{X}_2) =
    4(\sigma_{\text{anat}}^2 + \sigma_{\text{num}}^2)/N$, with the numerical
component contributing $4\sigma_{\text{num}}^2/N$. Since Cohen's d normalizes by
the pooled standard deviation $\sqrt{\sigma_{\text{anat}}^2 +
        \sigma_{\text{num}}^2}$, the measurement error contribution becomes $\sigma_d =
    (2\sigma_{\text{num}}/\sqrt{N})/\sigma_{\text{anat}} = (2/\sqrt{N}) \cdot
    \text{\navr}$.

This formula indicates that regions with \navr = 0.1 contribute approximately
$0.2/\sqrt{N}$ uncertainty to Cohen's d, while regions with \navr = 1.0
contribute $2/\sqrt{N}$ uncertainty. The relationship provides a direct link
between computational stability (\navr) and statistical reliability in
neuroimaging studies.




\section{Data Availability}
The data that support the findings of this study are available from the
Parkinson's Progression Markers Initiative (PPMI) database
(www.ppmi-info.org/access-data-specimens/download-data), but restrictions apply
to the availability of these data, which were used under license for the current
study, and so are not publicly available. Data are however available from the
authors upon reasonable request and with permission of the PPMI.

\section{Code Availability}
The code used to conduct the analyses is available at [URL to be added upon
        publication].

\section{Acknowledgements}

The analyses were conducted on the Virtual Imaging
Platform~\cite{glatard2012virtual}, which utilizes resources provided by the
Biomed virtual organization within the European Grid Infrastructure (EGI). We
extend our gratitude to Sorina Pop from CREATIS, Lyon, France, for her support.

\bibliographystyle{alpha}
\bibliography{main}

\clearpage

\appendix

\section{Formula}

\subsection{Significant digits formula}

We compute the number of significant bits \(\hat{s}\) with probability
\(p_s=0.95\) and confidence \(1-\alpha_s=0.95\) using the \texttt{significantdigits}
package\footnote{\url{https://github.com/verificarlo/significantdigits}}
(version 0.4.0). \texttt{significantdigits} implements the Centered Normality
Hypothesis approach described in~\cite{sohier2021confidence}:
\[
    \hat{s_i} = -\log_2 \left| \frac{\hat{\sigma_i}}{\hat{\mu_i}} \right| -
    \delta(n, \alpha_s, p_s),
\]
where \(\hat{\sigma_i}\) and \(\hat{\mu_i}\) are the average and standard
deviation over the repetitions, and
\begin{equation}
    \delta(n, \alpha_s, p_s) = \log_2 \left(
    \sqrt{\frac{n-1}{\chi^2_{1-\alpha_s/2}}} \Phi^{-1} \left( \frac{p_s+1}{2}
    \right) \right)
\end{equation}
is a penalty term for estimating \(\hat{s_i}\) with probability \(p_s\) and
confidence level \(1-\alpha_s\) for a sample size \(n\). \(\Phi^{-1}\) is the
inverse cumulative distribution of the standard normal distribution and
\(\chi^2\) is the Chi-2 distribution with \(n\)-1 degrees of freedom.

\subsection{Extended Sørensen-Dice coefficient}

The extended Sørensen-Dice coefficient is a measure of overlap between multiple
sets, defined as follows:
\[
    \text{Dice}(A_1, A_2, \dots, A_n) = \frac{n \left| \bigcap_{i=1}^{n} A_i \right|}{\sum_{i=1}^{n} \left| A_i \right|}
\].

\section{Cross-sectional Analysis}

As a side result, the cross-sectional analysis measures the impact of numerical
variability in FreeSurfer version 7.3.1 on the PPMI (Parkinson's Progression
Markers Initiative) cohort. This involves comparing the estimation of
structural MRI measures, including cortical and subcortical volumes, cortical
thickness, and surface area. The goal is to assess the stability of these key
metrics and quantify the numerical variability.

FreeSurfer 7.3.1 showed limited numerical precision across all cortical
measures: $1.61 \pm 0.20$ significant digits for cortical thickness, $1.33 \pm 0.23$ for
surface area, and $1.33 \pm 0.23$ for  cortical volume
(Figures~\ref{fig:sig_digits_cortical}). Subcortical volumes have a similar precision
with $1.33 \pm 0.22$ significant digits on average (Figure~\ref{fig:sig_digits_subcortical}). These
values indicate measurements are typically precise to only one decimal place,
with some instances showing complete precision loss. Regional consistency was
observed within each metric type, with cortical thickness showing the highest precision
(range: $1.22-1.93$ digits) compared to surface area ($0.82 - 1.72$ digits) and cortical volume
($0.80 - 1.72$ digits). Subcortical volumes exhibited the highest precision
(range: $0.88 - 1.57$ digits), with a mean of $1.33 \pm 0.22$ significant digits.

To measure the structural overlap, we evaluated using the extended Sørensen-Dice coefficient:
Dice coefficients revealed substantial inter-subject variability, particularly
in temporal pole regions (Figure~\ref{fig:dice}). We also observed that the Dice
coefficient varies across regions, with some regions showing higher variability
than others with cortical volume ($0.00 - 0.91$) with a mean of $0.75 \pm 0.11$
and subcortical volume ($0.18 - 0.94$) with a mean of $0.82 \pm 0.08$. Finally,
we noticed that subcortical volume measurements are more stable than cortical
volume.



\begin{figure}
    \includegraphics*[width=\linewidth]{figures/dice.pdf}
    \caption{Dice coefficient.\label{fig:dice}}

\end{figure}

\begin{figure}
    \includegraphics*[width=\linewidth]{figures/sig_digits.pdf}
    \caption{Number of significant digits for each cortical region and metric.\label{fig:sig_digits_cortical}}
\end{figure}

\begin{figure}
    \includegraphics*[width=\linewidth]{figures/sig_digits_subcortical_volume.pdf}
    \caption{Number of significant digits of subcortical volume for each subcortical region.\label{fig:sig_digits_subcortical}}
\end{figure}

\subsection{Significant digits average across all subjects}

\begin{longtblr}[
        caption={Significant digits average across all subjects.},
        label={tab:sig-cortical},
    ]{
        colspec={lcc|cc|cc}, width=0.25\linewidth,
        row{even}={white,font=\footnotesize},
        row{odd}={gray9,font=\footnotesize},
        rows = {rowsep=0pt},
        rowhead=2,
        row{1}={white,font=\bfseries},
        row{2}={gray9}
    }
    \SetCell[c=1]{c}Region   & \SetCell[c=2]{c}{cortical thickness } &                 & \SetCell[c=2]{c}{surface area} &                 & \SetCell[c=2]{c}{cortical volume} &                 \\
                             & lh                                    & rh              & lh                             & rh              & lh                                & rh              \\
    \hline
    bankssts                 & $1.65 \pm 0.16$                       & $1.69 \pm 0.13$ & $1.15 \pm 0.18$                & $1.21 \pm 0.13$ & $1.08 \pm 0.17$                   & $1.14 \pm 0.13$ \\
    caudalanteriorcingulate  & $1.38 \pm 0.14$                       & $1.40 \pm 0.14$ & $1.14 \pm 0.22$                & $1.19 \pm 0.18$ & $1.14 \pm 0.24$                   & $1.21 \pm 0.20$ \\
    caudalmiddlefrontal      & $1.77 \pm 0.18$                       & $1.77 \pm 0.19$ & $1.40 \pm 0.21$                & $1.31 \pm 0.23$ & $1.40 \pm 0.22$                   & $1.30 \pm 0.23$ \\
    cuneus                   & $1.52 \pm 0.19$                       & $1.54 \pm 0.19$ & $1.34 \pm 0.14$                & $1.33 \pm 0.14$ & $1.32 \pm 0.14$                   & $1.28 \pm 0.15$ \\
    entorhinal               & $1.22 \pm 0.23$                       & $1.22 \pm 0.23$ & $0.82 \pm 0.19$                & $0.87 \pm 0.18$ & $0.80 \pm 0.19$                   & $0.81 \pm 0.18$ \\
    fusiform                 & $1.66 \pm 0.17$                       & $1.71 \pm 0.16$ & $1.41 \pm 0.18$                & $1.43 \pm 0.19$ & $1.33 \pm 0.18$                   & $1.37 \pm 0.20$ \\
    inferiorparietal         & $1.81 \pm 0.15$                       & $1.82 \pm 0.13$ & $1.53 \pm 0.18$                & $1.59 \pm 0.20$ & $1.50 \pm 0.17$                   & $1.56 \pm 0.17$ \\
    inferiortemporal         & $1.66 \pm 0.17$                       & $1.70 \pm 0.16$ & $1.37 \pm 0.25$                & $1.38 \pm 0.21$ & $1.37 \pm 0.23$                   & $1.41 \pm 0.19$ \\
    isthmuscingulate         & $1.46 \pm 0.12$                       & $1.43 \pm 0.13$ & $1.27 \pm 0.15$                & $1.24 \pm 0.15$ & $1.27 \pm 0.14$                   & $1.27 \pm 0.15$ \\
    lateraloccipital         & $1.75 \pm 0.18$                       & $1.77 \pm 0.17$ & $1.58 \pm 0.15$                & $1.57 \pm 0.16$ & $1.49 \pm 0.16$                   & $1.50 \pm 0.15$ \\
    lateralorbitofrontal     & $1.65 \pm 0.17$                       & $1.51 \pm 0.15$ & $1.44 \pm 0.23$                & $0.95 \pm 0.13$ & $1.51 \pm 0.16$                   & $1.12 \pm 0.14$ \\
    lingual                  & $1.54 \pm 0.22$                       & $1.52 \pm 0.21$ & $1.47 \pm 0.18$                & $1.46 \pm 0.17$ & $1.50 \pm 0.18$                   & $1.49 \pm 0.18$ \\
    medialorbitofrontal      & $1.50 \pm 0.15$                       & $1.53 \pm 0.15$ & $1.09 \pm 0.16$                & $1.15 \pm 0.14$ & $1.15 \pm 0.17$                   & $1.21 \pm 0.13$ \\
    middletemporal           & $1.74 \pm 0.16$                       & $1.81 \pm 0.14$ & $1.42 \pm 0.23$                & $1.52 \pm 0.19$ & $1.44 \pm 0.21$                   & $1.55 \pm 0.18$ \\
    parahippocampal          & $1.54 \pm 0.14$                       & $1.56 \pm 0.12$ & $1.13 \pm 0.13$                & $1.09 \pm 0.13$ & $1.11 \pm 0.13$                   & $1.07 \pm 0.13$ \\
    paracentral              & $1.59 \pm 0.22$                       & $1.60 \pm 0.22$ & $1.40 \pm 0.17$                & $1.40 \pm 0.19$ & $1.36 \pm 0.18$                   & $1.36 \pm 0.20$ \\
    parsopercularis          & $1.74 \pm 0.17$                       & $1.71 \pm 0.16$ & $1.38 \pm 0.19$                & $1.30 \pm 0.18$ & $1.38 \pm 0.19$                   & $1.30 \pm 0.20$ \\
    parsorbitalis            & $1.53 \pm 0.20$                       & $1.51 \pm 0.20$ & $1.21 \pm 0.14$                & $1.21 \pm 0.18$ & $1.19 \pm 0.16$                   & $1.22 \pm 0.18$ \\
    parstriangularis         & $1.68 \pm 0.17$                       & $1.63 \pm 0.19$ & $1.33 \pm 0.16$                & $1.30 \pm 0.22$ & $1.30 \pm 0.16$                   & $1.28 \pm 0.21$ \\
    pericalcarine            & $1.33 \pm 0.21$                       & $1.30 \pm 0.22$ & $1.23 \pm 0.20$                & $1.21 \pm 0.22$ & $1.18 \pm 0.17$                   & $1.18 \pm 0.17$ \\
    postcentral              & $1.84 \pm 0.24$                       & $1.81 \pm 0.26$ & $1.68 \pm 0.23$                & $1.69 \pm 0.28$ & $1.64 \pm 0.20$                   & $1.63 \pm 0.24$ \\
    posteriorcingulate       & $1.57 \pm 0.13$                       & $1.56 \pm 0.14$ & $1.37 \pm 0.20$                & $1.35 \pm 0.21$ & $1.39 \pm 0.19$                   & $1.39 \pm 0.22$ \\
    precentral               & $1.79 \pm 0.26$                       & $1.76 \pm 0.28$ & $1.71 \pm 0.24$                & $1.64 \pm 0.27$ & $1.72 \pm 0.22$                   & $1.66 \pm 0.28$ \\
    precuneus                & $1.83 \pm 0.13$                       & $1.84 \pm 0.13$ & $1.65 \pm 0.21$                & $1.66 \pm 0.21$ & $1.61 \pm 0.18$                   & $1.62 \pm 0.19$ \\
    rostralanteriorcingulate & $1.34 \pm 0.14$                       & $1.39 \pm 0.15$ & $1.00 \pm 0.16$                & $1.07 \pm 0.17$ & $1.11 \pm 0.19$                   & $1.11 \pm 0.18$ \\
    rostralmiddlefrontal     & $1.77 \pm 0.19$                       & $1.74 \pm 0.19$ & $1.44 \pm 0.24$                & $1.41 \pm 0.28$ & $1.49 \pm 0.21$                   & $1.48 \pm 0.25$ \\
    superiorfrontal          & $1.87 \pm 0.17$                       & $1.85 \pm 0.18$ & $1.61 \pm 0.23$                & $1.56 \pm 0.27$ & $1.64 \pm 0.21$                   & $1.62 \pm 0.25$ \\
    superiorparietal         & $1.92 \pm 0.18$                       & $1.93 \pm 0.17$ & $1.72 \pm 0.24$                & $1.65 \pm 0.28$ & $1.66 \pm 0.22$                   & $1.60 \pm 0.26$ \\
    superiortemporal         & $1.83 \pm 0.17$                       & $1.85 \pm 0.15$ & $1.57 \pm 0.22$                & $1.58 \pm 0.18$ & $1.52 \pm 0.21$                   & $1.57 \pm 0.18$ \\
    supramarginal            & $1.83 \pm 0.16$                       & $1.85 \pm 0.15$ & $1.57 \pm 0.22$                & $1.59 \pm 0.26$ & $1.56 \pm 0.20$                   & $1.56 \pm 0.24$ \\
    frontalpole              & $1.26 \pm 0.23$                       & $1.23 \pm 0.20$ & $0.94 \pm 0.11$                & $0.91 \pm 0.11$ & $0.88 \pm 0.17$                   & $0.87 \pm 0.14$ \\
    temporalpole             & $1.24 \pm 0.26$                       & $1.28 \pm 0.25$ & $0.94 \pm 0.16$                & $0.99 \pm 0.19$ & $0.86 \pm 0.20$                   & $0.91 \pm 0.22$ \\
    transversetemporal       & $1.47 \pm 0.20$                       & $1.46 \pm 0.18$ & $1.17 \pm 0.13$                & $1.13 \pm 0.11$ & $1.20 \pm 0.15$                   & $1.15 \pm 0.13$ \\
    insula                   & $1.47 \pm 0.16$                       & $1.42 \pm 0.14$ & $1.13 \pm 0.18$                & $1.00 \pm 0.18$ & $1.29 \pm 0.16$                   & $1.19 \pm 0.19$ \\
\end{longtblr}

\begin{longtblr}[
        caption={Standard-deviation average across all subjects for cortical metrics.},
        label={tab:std-cortical},
    ]{
        colspec={lcc|cc|cc}, width=\linewidth,
        row{even}={white,font=\footnotesize},
        row{odd}={gray9,font=\footnotesize},
        rows = {rowsep=0pt},
        rowhead=2,
        row{1}={white,font=\bfseries},
        row{2}={gray9}
    }
    \SetCell[c=1]{c}Region   & \SetCell[c=2]{c}{cortical thickness                                                                                                                       \\ (mm)} &                 & \SetCell[c=2]{c}{surface area \\ ($\text{mm}^2$)} &                    & \SetCell[c=2]{c}{cortical volume \\ ($\text{mm}^3$)} &                     \\
                             & lh                                  & rh              & lh                     & rh                      & lh                    & rh                     \\
    \hline
    bankssts                 & $0.02 \pm 0.01$                     & $0.02 \pm 0.01$ & $\028.65 \pm \015.97$  & $\021.73 \pm \0\08.68$  & $\077.25 \pm \037.44$ & $\059.87 \pm \020.45$  \\
    caudalanteriorcingulate  & $0.04 \pm 0.01$                     & $0.04 \pm 0.01$ & $\019.98 \pm \013.83$  & $\021.01 \pm \014.96$   & $\051.33 \pm \037.32$ & $\051.67 \pm \041.74$  \\
    caudalmiddlefrontal      & $0.02 \pm 0.01$                     & $0.02 \pm 0.01$ & $\038.58 \pm \036.77$  & $\046.65 \pm \044.68$   & $104.41 \pm 108.02$   & $124.11 \pm 112.10$    \\
    cuneus                   & $0.02 \pm 0.01$                     & $0.02 \pm 0.01$ & $\028.45 \pm \011.50$  & $\031.25 \pm \015.67$   & $\060.72 \pm \025.52$ & $\074.77 \pm \034.16$  \\
    entorhinal               & $0.08 \pm 0.05$                     & $0.08 \pm 0.05$ & $\027.41 \pm \016.67$  & $\022.37 \pm \011.70$   & $125.48 \pm \071.07$  & $115.94 \pm \057.21$   \\
    fusiform                 & $0.02 \pm 0.01$                     & $0.02 \pm 0.01$ & $\050.70 \pm \025.16$  & $\047.86 \pm \028.19$   & $182.92 \pm \092.31$  & $170.22 \pm 103.05$    \\
    inferiorparietal         & $0.01 \pm 0.01$                     & $0.01 \pm 0.01$ & $\053.01 \pm \029.19$  & $\059.90 \pm \050.62$   & $145.66 \pm \072.95$  & $159.55 \pm 110.14$    \\
    inferiortemporal         & $0.02 \pm 0.01$                     & $0.02 \pm 0.01$ & $\064.73 \pm \042.27$  & $\058.75 \pm \034.04$   & $198.15 \pm 127.44$   & $168.38 \pm \084.67$   \\
    isthmuscingulate         & $0.03 \pm 0.01$                     & $0.03 \pm 0.01$ & $\023.74 \pm \011.07$  & $\023.35 \pm \013.99$   & $\057.43 \pm \029.59$ & $\053.05 \pm \034.34$  \\
    lateraloccipital         & $0.02 \pm 0.01$                     & $0.02 \pm 0.01$ & $\053.82 \pm \024.63$  & $\056.35 \pm \028.61$   & $156.83 \pm \066.16$  & $160.98 \pm \076.00$   \\
    lateralorbitofrontal     & $0.02 \pm 0.01$                     & $0.03 \pm 0.01$ & $\043.31 \pm \030.16$  & $117.14 \pm \033.75$    & $\092.60 \pm \056.29$ & $217.89 \pm \069.06$   \\
    lingual                  & $0.03 \pm 0.01$                     & $0.03 \pm 0.01$ & $\044.26 \pm \022.65$  & $\046.73 \pm \023.96$   & $\089.19 \pm \046.24$ & $\095.82 \pm \049.65$  \\
    medialorbitofrontal      & $0.03 \pm 0.01$                     & $0.03 \pm 0.01$ & $\066.04 \pm \024.11$  & $\058.06 \pm \019.00$   & $147.37 \pm \057.84$  & $134.52 \pm \042.26$   \\
    middletemporal           & $0.02 \pm 0.01$                     & $0.02 \pm 0.01$ & $\053.01 \pm \034.97$  & $\044.87 \pm \028.36$   & $165.49 \pm 108.52$   & $135.26 \pm \077.98$   \\
    parahippocampal          & $0.03 \pm 0.01$                     & $0.03 \pm 0.01$ & $\019.55 \pm \0\08.42$ & $\020.45 \pm \0\07.81$  & $\064.22 \pm \025.29$ & $\065.43 \pm \024.59$  \\
    paracentral              & $0.03 \pm 0.02$                     & $0.03 \pm 0.01$ & $\022.94 \pm \012.98$  & $\026.94 \pm \019.80$   & $\063.71 \pm \040.74$ & $\073.88 \pm \056.66$  \\
    parsopercularis          & $0.02 \pm 0.01$                     & $0.02 \pm 0.01$ & $\028.65 \pm \028.77$  & $\029.46 \pm \026.82$   & $\080.67 \pm \092.87$ & $\082.38 \pm \089.16$  \\
    parsorbitalis            & $0.03 \pm 0.02$                     & $0.03 \pm 0.02$ & $\017.82 \pm \0\09.77$ & $\021.41 \pm \010.66$   & $\060.63 \pm \045.20$ & $\068.18 \pm \036.64$  \\
    parstriangularis         & $0.02 \pm 0.01$                     & $0.02 \pm 0.01$ & $\025.67 \pm \014.65$  & $\034.86 \pm \037.79$   & $\071.73 \pm \045.49$ & $\096.87 \pm 102.22$   \\
    pericalcarine            & $0.03 \pm 0.02$                     & $0.04 \pm 0.02$ & $\036.04 \pm \020.18$  & $\042.02 \pm \024.82$   & $\059.64 \pm \029.98$ & $\068.61 \pm \034.89$  \\
    postcentral              & $0.01 \pm 0.02$                     & $0.02 \pm 0.02$ & $\043.47 \pm \067.12$  & $\045.98 \pm \083.10$   & $100.26 \pm 121.35$   & $104.53 \pm 156.51$    \\
    posteriorcingulate       & $0.02 \pm 0.01$                     & $0.02 \pm 0.01$ & $\021.93 \pm \013.05$  & $\024.39 \pm \019.52$   & $\052.42 \pm \033.33$ & $\056.27 \pm \052.59$  \\
    precentral               & $0.02 \pm 0.02$                     & $0.02 \pm 0.02$ & $\046.92 \pm \053.54$  & $\057.46 \pm \070.35$   & $118.04 \pm 157.21$   & $148.21 \pm 233.10$    \\
    precuneus                & $0.01 \pm 0.01$                     & $0.01 \pm 0.00$ & $\038.04 \pm \042.87$  & $\038.95 \pm \040.96$   & $100.91 \pm 111.15$   & $102.24 \pm \096.62$   \\
    rostralanteriorcingulate & $0.05 \pm 0.02$                     & $0.04 \pm 0.02$ & $\034.80 \pm \015.03$  & $\022.00 \pm \010.59$   & $\081.04 \pm \041.59$ & $\061.95 \pm \033.93$  \\
    rostralmiddlefrontal     & $0.02 \pm 0.01$                     & $0.02 \pm 0.01$ & $\092.87 \pm \096.23$  & $108.40 \pm 132.97$     & $213.81 \pm 259.58$   & $252.00 \pm 358.20$    \\
    superiorfrontal          & $0.01 \pm 0.01$                     & $0.01 \pm 0.01$ & $\085.23 \pm \086.47$  & $\098.14 \pm 120.75$    & $223.91 \pm 234.89$   & $243.75 \pm 304.56$    \\
    superiorparietal         & $0.01 \pm 0.01$                     & $0.01 \pm 0.01$ & $\049.49 \pm \080.81$  & $\062.89 \pm \096.86$   & $132.77 \pm 207.97$   & $161.39 \pm 235.01$    \\
    superiortemporal         & $0.02 \pm 0.01$                     & $0.01 \pm 0.01$ & $\047.70 \pm \033.64$  & $\041.38 \pm \023.84$   & $156.30 \pm 101.85$   & $129.01 \pm \078.70$   \\
    supramarginal            & $0.01 \pm 0.01$                     & $0.01 \pm 0.01$ & $\050.87 \pm \058.82$  & $\050.06 \pm \083.24$   & $136.23 \pm 168.28$   & $133.99 \pm 207.69$    \\
    frontalpole              & $0.07 \pm 0.04$                     & $0.07 \pm 0.04$ & $\012.99 \pm \0\04.02$ & $\016.42 \pm \0\04.47$  & $\056.49 \pm \032.17$ & $\067.84 \pm \028.93$  \\
    temporalpole             & $0.09 \pm 0.05$                     & $0.08 \pm 0.05$ & $\025.08 \pm \010.71$  & $\022.16 \pm \011.78$   & $154.60 \pm \079.32$  & $138.28 \pm \078.33$   \\
    transversetemporal       & $0.03 \pm 0.02$                     & $0.03 \pm 0.02$ & $\012.73 \pm \0\05.33$ & $\0\09.98 \pm \0\03.33$ & $\029.55 \pm \012.34$ & $\024.91 \pm \0\08.79$ \\
    insula                   & $0.04 \pm 0.02$                     & $0.04 \pm 0.01$ & $\073.45 \pm \030.66$  & $\095.70 \pm \037.63$   & $146.49 \pm \064.11$  & $183.39 \pm \081.47$   \\
\end{longtblr}

\begin{longtblr}[
        caption={Significant digits and standard-deviation average across all subjects for subcortical volumes.},
        label={tab:sig-std-subcortical-volume},
    ]{
        colspec={lc|c},
        row{even}={gray9,font=\footnotesize},
        row{odd}={white,font=\footnotesize},
        rows = {rowsep=0pt},
        row{Z}={font=\small},
        rowhead=1,
        row{1}={font=\bfseries}
    }
    Region               & Significant digits & {Standard deviation  \\ ($\text{mm}^3$)} \\
    \hline
    Left-Thalamus        & $1.42 \pm 0.21$    & $120.08  \pm 69.61$  \\
    Left-Caudate         & $1.57 \pm 0.20$    & $\038.83 \pm 25.11$  \\
    Left-Putamen         & $1.49 \pm 0.22$    & $\065.88 \pm 46.39$  \\
    Left-Pallidum        & $1.25 \pm 0.19$    & $\047.81 \pm 25.09$  \\
    Left-Hippocampus     & $1.48 \pm 0.17$    & $\056.23 \pm 41.03$  \\
    Left-Amygdala        & $1.13 \pm 0.16$    & $\048.71 \pm 20.04$  \\
    Left-Accumbens-area  & $0.88 \pm 0.16$    & $\024.20 \pm \08.80$ \\
    Right-Thalamus       & $1.42 \pm 0.20$    & $118.92  \pm 68.76$  \\
    Right-Caudate        & $1.51 \pm 0.24$    & $\049.37 \pm 42.71$  \\
    Right-Putamen        & $1.51 \pm 0.25$    & $\068.07 \pm 70.23$  \\
    Right-Pallidum       & $1.22 \pm 0.19$    & $\049.11 \pm 30.50$  \\
    Right-Hippocampus    & $1.55 \pm 0.18$    & $\048.59 \pm 28.98$  \\
    Right-Amygdala       & $1.23 \pm 0.17$    & $\042.21 \pm 18.68$  \\
    Right-Accumbens-area & $0.99 \pm 0.15$    & $\020.50 \pm \07.72$ \\
\end{longtblr}


\begin{table}[h]
    \centering
    \caption{Summary of executions failure and excluded subjects. To standardize the sample, we keep 26 repetitions per subject/visits pair.
        Subject/visit pairs with less than 26 repetitions were excluded which is 12 subjects.}
    \begin{tabular}{l c c}
        \toprule
        \textbf{Stage}     & \textbf{Number of rejected repetitions} & \textbf{Total number of repetitions} \\
        \midrule
        Cluster failure    & 1246 (5.80\%)                           & 21488                                \\
        FreeSurfer failure & 68 (0.33\%)                             & 21488                                \\
        QC failure         & 319 (1.48\%)                            & 21488                                \\
        Total              & 1633 (7.60\%)                           & 21488                                \\
        \bottomrule
    \end{tabular}
\end{table}

\begin{table}[h!]
    \centering
    \begin{tabular}{c|lccc}
        \toprule
        \textbf{Status} & \textbf{Cohort}         & \textbf{HC}         & \textbf{PD-non-MCI}         & \textbf{PD-MCI}   \\
        \hline
        \multirow{5}{*}{\textbf{\shortstack{Before                                                                        \\QC}}}
                        & n                       & 106                 & 181                         & 29                \\
                        & Age (y)                 & $60.6 \pm 10.2   $  & $61.7 \pm \09.6$            & $67.7 \pm \07.7$  \\
                        & Age range               & $30.6 - 84.3  $     & $36.3 - 83.3$               & $49.9 - 80.5$     \\
                        & Gender (male, \%)       & $58 \; (54.7\%)   $ & $119 \; (65.7\%)          $ & $-          $     \\
                        & Education (y)           & $16.6 \pm \03.3  $  & $15.9 \pm \02.9$            & $-          $     \\
        \hline
        \multirow{5}{*}{\textbf{\shortstack{After                                                                         \\QC}}}
                        & n                       & 103                 & 175                         & 27                \\
                        & Age (y)                 & $60.7 \pm 10.3   $  & $61.4 \pm \09.5          $  & $67.8 \pm \07.9$  \\
                        & Age range               & $30.6 - 84.3  $     & $36.3 - 79.9           $    & $49.9 - 80.5$     \\
                        & Gender (male, \%)       & $57 \; (55.3\%)   $ & $114 \; (65.1\%)       $    & $20 \; (74.1\%) $ \\
                        & Education (y)           & $16.6 \pm \03.3  $  & $15.9 \pm \02.9        $    & $15.0 \pm \03.5$  \\
        \hline
        \multirow{8}{*}{\textbf{\shortstack{After                                                                         \\MCI\\exclusion}}}
                        & n                       & $103           $    & $121                   $    & --                \\
                        & Age (y)                 & $60.7 \pm 10.3   $  & $60.7 \pm \09.1        $    & --                \\
                        & Age range               & $30.6 - 84.3  $     & $39.2 - 78.3           $    & --                \\
                        & Gender (male, \%)       & $57 \; (55.3\%)   $ & $80 \; (66.1\%)        $    & --                \\
                        & Education (y)           & $16.6 \pm \03.3  $  & $16.1 \pm \03.0        $    & --                \\
                        & UPDRS III OFF baseline  & $-            $     & $23.4 \pm 10.1         $    & --                \\
                        & UPDRS III OFF follow-up & $-            $     & $25.8 \pm 11.1         $    & --                \\
                        & Duration T2 - T1 (y)    & $\01.4 \pm \00.5 $  & $\01.4 \pm \00.7       $    & --                \\
        \bottomrule
    \end{tabular}
    \vspace{1em}

    \textbf{Abbreviations:} MCI = Mild Cognitive Impairment; UPDRS = Unified
    Parkinson's Disease Rating Scale; PD = Parkinson's disease. Descriptive
    statistics before and after quality control (QC). Values are expressed as
    mean $\pm$ standard deviation. PD-non-MCI longitudinal sample is a subsample
    of the PD-non-MCI original sample that had longitudinal data and disease
    severity scores available.
    \label{tab:cohort_stat_vertical}
\end{table}

\section{Numerical-Anatomical Variability Ratio (\navr)}

\subsection{\navr maps}

Figures \ref{fig:navr_map_area} and \ref{fig:navr_map_volume} show the
\navr maps for cortical surface area and volume, respectively. The maps show the
average \navr values across all subjects for each cortical region. The color
scale indicates the \navr value, with warmer colors indicating higher \navr
values. The maps provide a visual representation of the variability in the
\navr values across different cortical regions, highlighting regions with
higher or lower \navr values.

\begin{figure}[h]
    \includegraphics[width=\linewidth]{figures/NAVR_map/NAVR_area_all.png}
    \caption{ \navr maps for cortical surface area. The maps show the average
        \navr values across all subjects for each cortical region. The color
        scale indicates the \navr value, with warmer colors indicating higher
        \navr values.}
    \label{fig:navr_map_area}
\end{figure}

\begin{figure}[h]
    \includegraphics[width=\linewidth]{figures/NAVR_map/NAVR_volume_all.png}
    \caption{ \navr maps for cortical volume. The maps show the average \navr
        values across all subjects for each cortical region. The color scale
        indicates the \navr value, with warmer colors indicating higher \navr
        values.}
    \label{fig:navr_map_volume}
\end{figure}

\subsection{Consistency results}

\subsubsection{Consistency of statistical tests}

Figures \ref{fig:navr_consistency_area_plot} and \ref{fig:navr_consistency_volume_plot} show the
consistency of statistical tests for cortical area and volume, respectively,
across all subjects and regions. The plots show the percentage of subjects for
which the statistical test was significant ($\alpha = 0.05$) for each region. The
consistency varies across regions, with some regions showing higher consistency
than others. The red triangles indicate the IEEE-754 run for reference.

\begin{figure}[h]
    \centering
    \includegraphics[width=\textwidth]{figures/consistency/cortical_area_significance_correlation.pdf}
    \caption{Consistency of statistical tests for cortical area across all
        subjects and regions. The plot shows the percentage of subjects for which
        the statistical test was significant ($\alpha = 0.05$) for each region. The consistency varies
        across regions, with some regions showing higher consistency than others.}
    \label{fig:navr_consistency_area_plot}
\end{figure}

\begin{figure}[h]
    \centering
    \includegraphics[width=\textwidth]{figures/consistency/cortical_volume_significance_correlation.pdf}
    \caption{Consistency of statistical tests for cortical volume across all
        subjects and regions. The plot shows the percentage of subjects for which
        the statistical test was significant ($\alpha = 0.05$) for each region. The consistency varies
        across regions, with some regions showing higher consistency than others.}
    \label{fig:navr_consistency_volume_plot}
\end{figure}

\subsubsection{Distribution of statistical tests coefficients}

Figures \ref{fig:navr_consistency_area} and \ref{fig:navr_consistency_volume} show the distribution of
partial correlation coefficients for cortical area and volume, respectively,
across all subjects and regions. The red triangles indicate the IEEE-754 run
for reference. The distribution shows the variability in the coefficients, with
some regions exhibiting higher consistency than others.

\begin{figure}[h]
    \centering
    \begin{subfigure}[b]{\textwidth}
        \includegraphics[width=\textwidth]{figures/consistency/cortical_area_coefficients_distribution-Left.pdf}
        \caption{Left hemisphere}
        \label{fig:navr_consistency_area_left}
    \end{subfigure}
    \hfill
    \begin{subfigure}[b]{\textwidth}
        \includegraphics[width=\textwidth]{figures/consistency/cortical_area_coefficients_distribution-Right.pdf}
        \caption{Right hemisphere}
        \label{fig:navr_consistency_area_right}
    \end{subfigure}
    \caption{ Distribution of partial correlation coefficients for cortical area
        across all subjects and regions. Red triangles indicate the IEEE-754 run
        for reference. The distribution shows the variability in the
        coefficients, with some regions exhibiting higher consistency than
        others.}
    \label{fig:navr_consistency_area}
\end{figure}

\begin{figure}[h]
    \centering
    \begin{subfigure}[b]{\textwidth}
        \includegraphics[width=\textwidth]{figures/consistency/cortical_volume_coefficients_distribution-Left.pdf}
        \caption{Left hemisphere}
        \label{fig:navr_consistency_volume_left}
    \end{subfigure}
    \hfill
    \begin{subfigure}[b]{\textwidth}
        \includegraphics[width=\textwidth]{figures/consistency/cortical_volume_coefficients_distribution-Right.pdf}
        \caption{Right hemisphere}
        \label{fig:navr_consistency_volume_right}
    \end{subfigure}
    \caption{ Distribution of partial correlation coefficients for cortical
        volume across all subjects and regions. Red triangles indicate the
        IEEE-754 run for reference. The distribution shows the variability in
        the coefficients, with some regions exhibiting higher consistency than
        others.}
    \label{fig:navr_consistency_volume}
\end{figure}



\end{document}

% 2nd table describing the raw -> removed subjects 
% main table describing 
% 
% which part of the code is responsible for the instabilities
% include that in the discussion 
%