\documentclass{article}

% Language setting Replace `english' with e.g. `spanish' to change the document
% language
\usepackage[english]{babel}

% Set page size and margins Replace `letterpaper' with `a4paper' for UK/EU
% standard size
\usepackage[letterpaper,top=2cm,bottom=2cm,left=3cm,right=3cm,marginparwidth=1.75cm]{geometry}

% Useful packages
\usepackage{amsmath}
\usepackage{mathtools}
\usepackage{amssymb}
\usepackage[pdftex]{graphicx}
\usepackage[colorlinks=true, allcolors=blue]{hyperref}
\usepackage{todonotes}
\usepackage{longtable}
\usepackage{booktabs}
\usepackage{multirow}
\usepackage{subcaption}
\usepackage{xcolor}
\usepackage{tabularray}
\usepackage{xspace} % For \xspace command
\usepackage{tikz}
\usetikzlibrary{shadings,positioning,overlay-beamer-styles}
\usepackage{xifthen}
\usepackage{ulem}

\newcommand{\colorbar}[6]% width,height,colors,label min,label max,label step
{   \begin{tikzpicture} \foreach \x [count=\c] in {#3}{ \xdef\numcolo{\c}}
    \pgfmathsetmacro{\piecewidth}{#1/(\numcolo-1)}
    \xdef\lowcolo{} \foreach \x [count=\c] in {#3} {   \ifthenelse{\c = 1} {} {
    \fill[left color=\lowcolo,right color=\x] ({(\c-2)*\piecewidth},0) rectangle
    ({(\c-1)*\piecewidth},#2); } \xdef\lowcolo{\x} } \draw (0,0) rectangle
    (#1,#2);
    \pgfmathsetmacro{\secondlabel}{#4+#6}
    \pgfmathsetmacro{\lastlabel}{#5+0.01}
    \pgfkeys{/pgf/number format/.cd,fixed,precision=2}
    \foreach \x in {#4,\secondlabel,...,\lastlabel} { \draw
    ({(\x-#4)/(#5-#4)*#1},0) -- ++ (0,-0.1) node[below=0.05,font=\tiny]
    {\pgfmathprintnumber{\x}}; }
    \end{tikzpicture}
}

\newcommand{\YC}[1]{\textcolor{red}{YC: #1}}
\newcommand{\TG}[2][]{\ifthenelse{\equal{#1}{done}}{[\textcolor{teal}{$\blacksquare$}]~\textcolor{blue!60}{\textsc{From
Tristan:} #2}}{\textcolor{blue}{\textsc{From Tristan:} #2}}}


\newcommand{\0}{\mspace{9mu}}
\newcommand{\navr}[0]{$\nu_{\text{nav}}$\xspace}
\newcommand{\Var}[0]{\mathrm{Var}}

\title{Numerical variability in structural MRI measures of Parkinson's disease}

\author{Yohan Chatelain, Andrzej Sokołowski, Madeleine Sharp, Jean-Baptiste Poline, Tristan Glatard}

\begin{document}

\maketitle

\begin{abstract}
    Numerical variability is rarely quantified in neuroimaging despite many
    biomarkers relying on subtle morphometric differences across individuals. We
    instrumented FreeSurfer, a widely used neuroimaging pipeline, to simulate
    numerical differences across computational environments, and used it to
    measure numerical variability in MRI analyses of Parkinson's disease
    patients and controls. In multiple cortical and subcortical regions,
    numerical variation reached nearly one-third of the anatomical signal,
    altering statistical conclusions about group differences and clinical
    associations. To address this, we developed a practical tool that estimates
    the Numerical-Anatomical Variability Ratio (NAVR), enabling researchers to
    assess the impact of numerical noise in existing studies. Applying
    NAVR-based thresholding to ENIGMA brain maps showed that many small regional
    effects fall below typical numerical noise thresholds \TG{Note to self:
        re-evaluate this (strong) claim after reading the results} for standard
    study sizes \TG{not sure why study sizes are mentioned here since ENIGMA
        maps were obtained for specific sample sizes}. Our tool provides a practical
    approach to quantify the effect of numerical variability, improving the
    robustness and reproducibility of neuroimaging analyses.
\end{abstract}

\section{Introduction}

\TG[done]{Rewrite using the following structure:
    \begin{itemize}
        \item The reliability of structural MRI measures critically depends on the estimation
              of sources of analytical variability.
        \item In particular when effect sizes are moderate, such as in Parkinson's disease.
        \item Among these sources, numerical variability has been shown to have a measurable
              impact on analyses across multiple pipeline tools.
        \item Numerical variability arises from rounding and truncation errors associated
              with the use of limited-precision numerical formats.
        \item Due to the complexity of MRI analyses, and in particular their reliance on
              high-dimensional optimization, such errors propagate and sometimes lead to
              measurable differences.
        \item However, numerical variability is understudied, mainly due to practical
              challenges.
        \item In particular, the implications of numerical variability on clinical findings
              is unknown.
        \item What we have done
    \end{itemize}
}

The reliability of structural MRI measures critically depends on the estimation
of all sources of analytical variability. This is particularly important when
effect sizes are moderate, as it is the case in studies of Parkinson's disease
(PD)~\cite{he2020progressive}. Among these sources, numerical
variability—subtle differences in computational results arising from factors
like hardware, operating systems, or software library versions—has been shown
to have a measurable impact on analyses across multiple pipeline
tools~\cite{glatard2012virtual,gronenschild2012effects,sokolowski2024impact,des2023reproducibility,vila2024impact}.

Numerical variability arises from rounding and truncation errors associated
with the use of limited-precision numerical formats, such as the IEEE-754
standard for floating-point arithmetic~\cite{markstein2008new}. Due to the
complexity of MRI analyses, particularly their reliance on high-dimensional
optimization, such errors can propagate and accumulate, sometimes leading to
measurable differences in final
outputs~\cite{salari2021accurate,kiar2021numerical,chatelain2024numerical,mirhakimi2025numerical}.
However, numerical variability remains understudied, mainly due to the
practical challenges of quantifying its effects. Consequently, the implications
of numerical variability for clinical findings are largely unknown.

In this study, we systematically investigated the impact of numerical
variability on structural MRI analyses in a clinical context. We instrumented
FreeSurfer 7.3.1~\cite{fischl2012freesurfer} with Monte Carlo
Arithmetic~\cite{parker1997monte} to simulate a range of realistic
computational environments. We then processed a dataset of Parkinson's disease
(PD) patients and healthy controls (HC) from the Parkinson's Progression
Markers Initiative (PPMI) to quantify how numerical noise affects derived
biomarkers, such as cortical thickness and subcortical volume. We introduce the
Numerical-Anatomical Variability Ratio (NAVR) to compare the magnitude of
numerical noise to that of anatomical variability and demonstrate its use in
evaluating the robustness of statistical findings, both in our dataset and in
large-scale meta-analyses from the ENIGMA consortium~\cite{thompson2014enigma}.

% Complex neuroimaging pipelines transform raw MRI data into putative biomarkers
% such as cortical thickness and subcortical volume. Because disease-related
% effects may be subtle, the reliability of these metrics depends critically on
% pipeline chosen~\cite{bhagwat2021understanding,botvinik2020variability},
% software versions~\cite{sokolowski2024impact,des2023reproducibility}, and
% computational
% environments~\cite{glatard2012virtual,gronenschild2012effects,des2023reproducibility,vila2024impact}.
% However, numerical variability arising from software computations is rarely
% quantified.

% Floating-point arithmetic ruled by IEEE-754 norm~\cite{markstein2008new} can
% cause ostensibly deterministic algorithms to yield subtly different outputs.
% Whether such numerical noise significantly impacts biological conclusions
% remains unclear. Preliminary studies suggest substantial
% effects~\cite{kiar2020comparing,salari2021accurate,des2023reproducibility,chatelain2024numerical},
% yet systematic quantification within clinical datasets is lacking.

% Parkinson's disease (PD) represents an ideal context to investigate numerical
% variability, as the identification of reliable structural biomarkers is highly
% desirable for diagnosis, tracking disease progression, and developing targeted
% therapies. Structural MRI is particularly attractive for biomarker discovery
% due to its non-invasive nature, but subtle anatomical changes associated with
% PD pose significant analytical challenges. Consequently, ensuring that
% MRI-derived measures are robust to numerical variability is essential to the
% successful translation of these biomarkers into clinical practice.

% We instrumented FreeSurfer 7.3.1~\cite{fischl2012freesurfer} with Monte Carlo
% arithmetic~\cite{parker1997monte}, reprocessing identical MRI scans of PD
% patients and healthy controls across multiple perturbed numerical states. To
% systematically quantify the impact, we introduce the Numerical-Anatomical
% Variability Ratio (NAVR), directly comparing numerical variation within
% individuals to anatomical differences between individuals.

% We find that in several cortical and subcortical regions, NAVR approaches 0.3,
% indicating that numerical noise can represent nearly one-third of the measured
% biological variation. Consequently, statistical inferences regarding group
% differences and clinical correlations can fluctuate significantly. Applying
% NAVR-based thresholding to published ENIGMA~\cite{thompson2014enigma}
% consortium brain maps reveals that numerous previously reported small regional
% effects likely fall below computational noise thresholds.

% Our findings demonstrate that numerical variability is a significant,
% quantifiable source of uncertainty in neuroimaging studies. The NAVR framework
% we propose offers researchers a practical approach to quantify and report
% computational uncertainty, enhancing reproducibility and reliability in
% clinical neuroscience.

\section{Impact of numerical variability on MRI measures of Parkinson's disease~\label{sec:results}}

Numerical instability arises from floating-point arithmetic, where subtle
rounding errors accumulate over complex calculations. Floating-point rounding
errors, although individually small, can collectively introduce substantial
numerical variability. By repeatedly processing the same MRI scans under random
numerical perturbations representative of differences across hardware systems,
software libraries, or other computational parameters, we found significant
variations directly impacting statistical conclusions. We introduced
perturbations using Monte Carlo Arithmetic
(MCA)~\cite{parker1997monte,salari2021accurate} \TG[done]{add ref to Parker and
    Ali's fuzzy libmath}, a technique to perturb floating-point operations with
random, zero-mean offsets that preserve mathematical expectations.
\TG[done]{operations?} \YC{$E[x \circ y] = x \circ y$. MCA preserves the
    expectation for a single arithmetic operation.} Repeatedly executing an
analysis pipeline with MCA enables us to estimate the variability of results
that could arise from different computational environments.

%These fluctuations pose a major concern for
%interpreting subtle anatomical changes in PD, underscoring the need for robust
%numerical precision in biomarker discovery.

To test the impact of numerical variability on statistical inference, we
performed group comparisons and partial correlations between regional brain
volumes and clinical measures of Parkinson's disease (UPDRS scores),
controlling for age and sex. For both cortical and subcortical regions, we
found substantial discrepancies in statistical outcomes across the 26
numerically perturbed analyses.

We analyzed T1-weighted MRI from Parkinson Progression Marker Initiative
(PPMI)~\cite{marek2011parkinson}, including 118 PD without MCI (PD-non-MCI) and
90 HC with two visits each (final cohort after QC and exclusions;
Table~\ref{tab:cohort_stat}). \TG[done]{Rewrite the remainder of this paragraph
    to explain the type of analyses that was done on the data, and why these
    analyses were selected.} We tested (a) PD-HC group differences using ANCOVA
test and (b) associations with clinical severity using partial correlations
between regional measures and UPDRS scores, adjusting for age and sex. These
analyses were selected to evaluate the impact of numerical variability on both
group-level differences and individual correlations, providing a comprehensive
view of how computational noise may influence neuroimaging findings.

% Parkinson's disease provides an ideal scenario for testing numerical stability
% \TG{this sentence comes back a few times in the writing but it's not the right
% logic: our goal is not to find a use case to evaluate numerical stability;
% instead, we are genuinely interested in PD}, 
% as structural MRI-derived
% measurements such as cortical thickness and subcortical volume hold significant
% promise as potential biomarkers for clinical progression due to their
% non-invasive nature. Furthermore, the availability of large multi-site datasets
% (e.g., PPMI) and established region-of-interest analyses ensure PD is a
% thoroughly characterized context for evaluating the impacts of numerical
% variability. Thus, PD allows us to directly assess whether numerical noise is
% sufficiently large to obscure clinically meaningful anatomical signals.

We instrumented FreeSurfer 7.3.1 with MCA (virtual precision set to mimic
realistic perturbations), and re-analyzed each T1-weighted MRI scan 26 times,
each time with a different random state. \TG[done]{The remainder of this
    paragraph refers to supplementary data, which doesn't read well as we're
    expecting to read the key (main) results first. It should be reworked.} Across
MCA replicates ($n=26$), statistical conclusions fluctuated for a non-trivial
fraction of regions. For subcortical volumes, 27\% of PD-HC comparisons changed
significance across numerical states; for cortical thickness, 19\% did so
(Figs.~\ref{fig:significance_correlation_subcortical_volume}-\ref{fig:significance_correlation_thickness}).
Eight subcortical regions alternated between significant and non-significant
outcomes at $\alpha=0.05$, illustrating that numerical variability can directly
affect clinical interpretation. Distributions of partial correlation
coefficients and ANCOVA F-statistics from MCA runs enveloped the unperturbed
IEEE-754 results, supporting the validity of the instrumentation rather than
indicating methodological artifacts
(Fig.~\ref{fig:statstest_coefficients_distribution}). The spread tended to be
larger in longitudinal than baseline analyses, indicating greater sensitivity
of longitudinal effects to numerical noise.

% NAVR stats
% metric    | mean | std  | median | min  | max
% thickness | 0.21 | 0.06 | 0.20   | 0.11 | 0.37
% area      | 0.18 | 0.08 | 0.16   | 0.09 | 0.42 
% volume    | 0.17 | 0.08 | 0.15   | 0.09 | 0.42
% cortical  | 0.19 | 0.08 | 0.17   | 0.09 | 0.42
% subvolume | 0.15 | 0.07 | 0.16   | 0.01 | 0.27

\subsection{Numerical variability alters statistical inference for MRI measures of Parkinson's disease}
% Comments
% - Extend the results section
% - First result on a page
%   - Writing for clinician audience with no expertise in computer science
%   - Writing for computer science audience with no expertise in neuroimaging  
% - Explain why MCA is important
% - Explain why figure significance_correlation_thickness is important
% - Explain why we choose Parkinson's disease MRI
%   - Explain the context of Parkinson's disease and its impact on neuroimaging research
%   - Cite Andrezj paper on Parkinson's disease MRI
%   - Use paper to replicate a representative of existing results and methodologie used for structural MRI analysis Parkinson's disease
% - refers to NARPS paper for the figures
% - Add more interpretation of the results
% - Get inspiration from the NARPS paper for the writing results section

To test the impact of numerical variability on statistical inference, we
performed group comparisons and partial \TG[done]{why partial?} \YC{Because we
    controlling for age and sex. } correlations between regional cortical thickness
and subcortical volumes with clinical measures of Parkinson's disease (UPDRS
scores).

\TG[done]{Rewrite the remainder of this paragraph with the following outline:
    (1) key message: "for both cortical and subcortical regions, substantial
    discrepancies..., (2) focus on subcortical volumes specifically, discussing the
    different analyses, (3) focus on cortical volumes, (4) conclusion and reference
    to NARPS.}

For both cortical and subcortical regions, statistical outcomes varied
substantially across the 26 Monte Carlo Arithmetic (MCA) replicates: $20\%$ of
comparisons (66/328) changed significance at the $p<0.05$ threshold. In
subcortical volumes (14 regions;
Figure~\ref{fig:significance_correlation_subcortical_volume}), $21\%$ of group
comparisons and $25\%$ of partial correlations were inconsistent across MCA
replicates, and 8 of 14 regions ($57\%$) exhibited at least one flip in
significance. Cortical thickness showed a similar pattern (68 regions;
Figure~\ref{fig:significance_correlation_thickness}), with $10\%$ of group
comparisons and $25\%$ of partial correlations inconsistent, affecting 40 of 68
regions ($59\%$) in at least one test. Expanding to all cortical measures
(thickness, area, and volume) and subcortical metrics, the proportion of
inconsistent results increased to $28\%$ (241/872; Appendix
Table~\ref{tab:fluctuating_regions}). These fluctuations indicate that
candidate biomarker relationships may appear or vanish due to numerical noise,
echoing multi-analyst variability reported in
NARPS~\cite{botvinik2020variability}.

\begin{figure}
    \includegraphics[width=\linewidth]{figures/consistency/subcortical_volume_significance_correlation.pdf}
    \caption{ Proportion of significant tests ($p<0.05$) for subcortical volumes across 26 numerical perturbations.
        measures.\label{fig:significance_correlation_subcortical_volume}}
\end{figure}

\begin{figure}
    \centering
    \includegraphics[width=\linewidth]{figures/consistency/cortical_thickness_significance_correlation.pdf}
    \caption{Proportion of significant tests ($p<0.05$) for cortical thickness across 26 numerical perturbations.
        measures.\label{fig:significance_correlation_thickness}}
    \label{fig:navr_consistency_thickness_plot}
\end{figure}

For subcortical volumes, the distributions of partial-correlation coefficients
(r) and ANCOVA F-statistics
(Figure~\ref{fig:statstest_coefficients_distribution}) show that the
unperturbed IEEE-754 estimates (red markers) consistently fall within the
MCA-sampled range. This supports the validity of the instrumentation rather
than indicating a methodological artifact \TG[done]{not really, the variability
    could be larger than the real one, reformulate to say that it ``supports the
    validity of the instrumentation''}. The spread of the r coefficients is larger
in longitudinal than in baseline analyses (Ansari-Bradley one-sided test at
$p<0.05$ corrected with Bonferroni; detailed results in Appendix
Table~\ref{tab:stats-coef-var-subcortical}), whereas no significant difference
in spread is detected for F-statistics using the same test
(Bonferroni-corrected, $p<0.05$). \TG[done]{it's quite apparent for r
    coefficients but not so much for F-statistics. A statistical test would help.}.
This indicates that numerical variability has a more pronounced impact on
longitudinal studies. \TG[done]{reviewers will wonder about a similar graph for
    cortical regions} An analogous analysis for cortical thickness leads to similar
conclusions (Appendix Figure~\ref{fig:consistency_thickness_coefficients},
Table~\ref{tab:stats-coef-var-cortical}).

\begin{figure}
    \includegraphics[width=\linewidth]{figures/consistency/subcortical_volume_coefficients_distribution.pdf}
    \caption{ Distribution of partial correlation coefficients (r-values) and
        F-statistics from ANCOVA across MCA repetitions for subcortical volume
        measures. Red dots represent the IEEE-754 unperturbed results. The top
        row shows r-values, while the bottom row shows F-values. The left column
        represents baseline analysis, and the right column represents
        longitudinal analysis.\label{fig:statstest_coefficients_distribution}}
\end{figure}

% We first tested the statistical significance of group differences between
% Parkinson's disease (PD) patients and healthy controls (HC), and their
% correlations with clinical measures (UPDRS scores), across the 26 MCA
% repetitions. Similar to the NARPS study~\cite{botvinik2020variability} that
% highlighted the impact of analytical variability in neuroimaging, we observed
% substantial fluctuations in the results due to numerical variability. We used a
% similar layout to NARPS paper to report the results, with the proportion of
% statistically significant tests ($p < 0.05$) varied substantially for both
% cortical thickness (Figure~\ref{fig:significance_correlation_thickness}) and
% subcortical volumes
% (Figure~\ref{fig:significance_correlation_subcortical_volume}). For many brain
% regions, the significance of a finding was inconsistent, appearing in some
% computational runs but not others. Ratios near 0.5 indicate maximal uncertainty,
% where a reported result could be attributed to a lucky roll of the computational
% dice. This variability in statistical significance suggests that the conclusions
% drawn from these analyses are not robust and may be influenced by the specific
% numerical state of the computation. This is specific to brain regions, some
% regions showing high consistency across numerical states, while others exhibit
% substantial variability. This instability is particularly concerning when
% finding biomarkers for the Parkinson's disease progression, as it suggests that
% the biological signals of interest may be obscured by computational noise.
% Sokołowski et al.~\cite{sokolowski2024impact} highlighted similar issues when
% playing with different FreeSurfer versions, showing that the results can vary
% significantly between versions 5, 6 and 7. Biomarkers are crucial for
% understanding disease mechanisms and tracking progression. Having reliable
% biomarkers is essential to better understand the nature of the disease which
% remains not well understood~\cite{bloem2021parkinson}.

% This instability is also reflected in the effect sizes themselves. Partial
% correlation coefficients and F-statistics from ANCOVA analyses showed
% substantial spread around the unperturbed IEEE-754 results (red markers in
% Figure~\ref{fig:statstest_coefficients_distribution} and
% \ref{fig:navr_consistency_thickness}). This demonstrates that numerical
% variability affects not only the binary outcome of statistical significance but
% also the magnitude of the estimated effect, further complicating the
% interpretation of results. We can notice that the unperturbed results (red
% dots) are contained within the distribution of coefficients and F-statistics,
% which indicates that our methodology is sound, but the variability in the
% results highlights the need for caution in interpretation. In particular, when
% interpreting results for possible clinical applications, it is important to
% consider the potential impact of numerical variability on the robustness of
% findings. Yellow dots around red ones can be seen as many potential results
% that could be obtained by running the same analysis under computational
% environments.

\subsection{A framework to quantify the impact of numerical variability}
% Comments:
% - Build a tool that can be broadly applied to any neuroimaging
% - Analytical modeling of sigma_d
% - Explain why having a tool is important
% - Why having a tool fast is important
%   - Measuring numerical variability is a time-consuming process
%   - Analytical modeling of sigma_d allows applying the tool to any
%     neuroimaging papers, existing results.
% - Quality Control impact findings => a tool to find potentially unreliable
%   results
% - To be general, we developped an analytical model
% - Refers to the online tool on yohanchatelain.github.io/brain-render

\TG[done]{Instead of re-stating the importance of numerical variability in this
    paragraph, which supposedly should be understood from the end of the
    introduction, here you could explain the need for a tool to quickly and
    practically evaluate its impact in a given study, explain and justify your
    assumptions (e.g., numerical variability is a feature of the pipeline rather
    than the population---refer to previous works). In doing so, you could explain
    how running MCA is currently not realistic due to computational requirements
    (although new architectures may enable it in the coming years), and explain the
    need for a statistical correction.}

While the importance of numerical variability is established, its practical
impact on individual studies remains difficult to quantify. Methods like Monte
Carlo Arithmetic (MCA) can measure this variability, but their computational
cost makes them hard for routine use with complex neuroimaging pipelines,
although new architectures may democratize this in the
future~\cite{elarar2025ms2a}. This creates a critical need for a fast,
analytical framework to estimate numerical uncertainty. Our approach is built
on the assumption that numerical variability is a property of the computational
pipeline rather than the underlying anatomical population
specificity~\cite{bhagwat2021understanding,vila2024impact,chatelain2024numerical,sokolowski2024impact}.
This allows us to develop a generalizable statistical model for widespread
application. To be meaningful, any assessment of numerical variability must be
contextualized by the statistical effect size under investigation and the
study's sample size. We, therefore, first developed a model to estimate the
uncertainty ($\sigma_d$) that numerical variability introduces into a standard
measure of group difference, Cohen's d (Eq.~\ref{eq:cohen_d}). To model
$\sigma_d$, we defined a standardized metric to characterize a pipeline's
intrinsic numerical instability: the Numerical-Anatomical Variability Ratio
(NAVR). NAVR is the ratio between the standard deviation of numerical
variability ($\sigma_{\text{num}}$; Eq.\ref{eq:sigma_num}) and the
between-subject anatomical variability ($\sigma_{\text{anat}}$;
Eq.\ref{eq:sigma_anat}):

$$\nu_{\text{nav}} = \frac{\sigma_{\text{num}}}{\sigma_{\text{anat}}}$$
\TG[done]{I think I would rather present the $\sigma_d$ result first, then explain
    NAV. When introducing sigma d, you could better explain that numerical
    variability has to be evaluated in the context of a particular effect size, and its impact will be dependent on sample size.}

This metric allowed us to establish a direct, closed-form relationship between
the pipeline's instability (NAVR) and the resulting uncertainty in a study's
effect size ($\sigma_d$):

$$\sigma_d = \frac{2}{\sqrt{N}} \nu_{\text{nav}}$$

where $N$ is the total sample size. This relationship allows researchers to
estimate the numerical uncertainty in effect sizes from summary statistics
alone, without requiring expensive recomputation or access to the original
data.

NAVR provides a general, interpretable metric to evaluate the robustness of
neuroimaging measurements. Across cortical and subcortical regions, we found
numerical variability accounted for up to 37-40\% of the observed anatomical
variance (Figures~\ref{fig:navr_subcortical},~\ref{fig:navr_thickness}). Mean
NAVR values reached 0.21 for cortical thickness and 0.15 for subcortical
volumes, indicating that numerical imprecision constitutes a portion of the
anatomical signal. Our model provides immediate, practical insights from these
values. For instance, with a typical NAVR of 0.2, a study would require over
1,500 participants to reduce the numerical uncertainty in its effect size,
$\sigma_d$, to a negligible level of 0.01 (Fig.~\ref{fig:sigma_d_contour}).
\TG[done]{not sure what this means, reformulate}

\begin{figure}[h]
    \centering
    \begin{subfigure}[b]{\linewidth}
        \centering
        \includegraphics[width=\linewidth]{figures/NAVR_map/NAVR_subcortical_volume_all.png}
        \subcaption{Subcortical volumes}
        \label{fig:navr_subcortical}
    \end{subfigure}

    \begin{subfigure}[b]{\linewidth}
        \centering
        \includegraphics[width=\linewidth]{figures/NAVR_map/NAVR_thickness_all.png}
        \subcaption{Cortical thickness}
        \label{fig:navr_thickness}
    \end{subfigure}

    \begin{subfigure}[b]{\linewidth}
        \centering
        \includegraphics[width=.8\linewidth]{figures/NAVR_map/jet_colorbar.pdf}
    \end{subfigure}

    \caption{Numerical-Anatomical Variability Ratio (\navr) for subcortical
        volumes~\ref{fig:navr_subcortical} and cortical
        thickness~\ref{fig:navr_thickness} across regions and groups. Higher
        \navr values indicate greater computational uncertainty relative to
        biological variation. The color scale indicates the \navr value, with
        warmer colors indicating higher \navr values.}
\end{figure}

\begin{figure}
    \includegraphics[width=\linewidth]{figures/sigma_d_contour.pdf}
    \caption{\TG{could you show some of the papers in this figure? } Relationship between \navr and population sample size  \(N\) for
        predicting the uncertainty in Cohen's d effect size estimation. The
        contour lines represent different \navr values, showing how numerical
        variability scales with sample size. With a typical \navr value of 0.2,
        to maintain reliable effect size estimates $\sigma_d \leq 0.01$, the
        plot suggests to use $N \geq 1500$.\label{fig:sigma_d_contour}}
\end{figure}

To facilitate the widespread adoption of this framework, we have implemented it
as an interactive, public web tool available at
\href{https://yohanchatelain.github.io/brain\_render/}{yohanchatelain.github.io/brain\_render}
(Appendix Figure~\ref{fig:brain_render_tool}). \TG[done]{you could add a
    screenshot of the tool} This tool enables researchers to perform rapid
reproducibility audits, evaluate findings from published literature
retroactively, and assess numerical robustness at a fraction of the cost of
traditional methods. Furthermore, we observed that regions failing standard
quality control (QC) checks consistently exhibited higher NAVR, suggesting the
metric can also serve as a useful proxy for data quality. By making these
assessments accessible, our framework helps improve transparency, support peer
review, and ultimately foster more trustworthy scientific inference in
neuroimaging.

\subsection{Re-evaluating landmark studies reveals widespread potential for unreliable effect sizes \TG{This title is over-emphasized given the results that are currently presented}}

To assess the broader implications of numerical variability, we applied NAVR to
re-evaluate influential findings from the ENIGMA consortium, which has
substantially shaped our understanding of psychiatric and neurological
disorders. Applying NAVR-based thresholding to ENIGMA's published brain maps
identified multiple regions where reported effect sizes fell below the
computational noise floor, potentially calling into question their reliability.
\TG{Assuming that you're referring to figure 6, this is not so clear. All the
    masked areas are areas with low effect sizes compared to the other ones. On the
    contrary, it seems to me that this is a case where numerical variability
    wouldn't impact the findings much, most likely because N was high. I would
    suggest to include another example, possibly with a lower N, showing a larger
    impact of numerical variability.}

Figures~\ref{fig:navr_enigma_thickness} and~\ref{fig:navr_enigma_subcortical}
illustrate the impact of applying NAVR thresholds to cortical thickness and
subcortical volume maps from ENIGMA. Regions rendered in black indicate areas
where reported effect sizes were smaller than numerical variability, suggesting
these findings should be interpreted with caution. \TG{same comment as in the
    previous figure}

This observation highlights potential risks of overestimating small effects
\TG{I don't think that the results presented show that clearly} and underscores
the importance of systematically accounting for numerical uncertainty in
neuroimaging research. While ENIGMA's primary findings generally remained
robust due to large sample sizes, our analysis indicates that numerous
secondary, smaller-scale effects reported in the literature could be
compromised by numerical instability \TG{I think you should show examples of
    such studies}.

\begin{figure}[h]
    \centering
    \vspace{0.2cm}

    % Header row with column labels
    \begin{minipage}[b]{\linewidth}
        \begin{minipage}[c]{0.05\linewidth}
            % Empty space for alignment with condition labels
        \end{minipage}%
        \begin{minipage}[c]{0.95\linewidth}
            \begin{minipage}[c]{0.47\linewidth}
                \centering\textbf{Unthresholded}
            \end{minipage}%
            \hfill
            \begin{minipage}[c]{0.005\linewidth}
                % Vertical line separator
            \end{minipage}%
            \hfill
            \begin{minipage}[c]{0.47\linewidth}
                \centering\textbf{Thresholded}
            \end{minipage}
        \end{minipage}
    \end{minipage}

    % Horizontal line
    \noindent\rule{\linewidth}{0.5pt}
    \vspace{-1.5cm}

    \begin{minipage}[b]{\linewidth}
        \begin{minipage}[c]{0.05\linewidth}
            \centering\rotatebox{90}{\textbf{L1}}
        \end{minipage}%
        \begin{minipage}[c]{0.95\linewidth}
            \begin{subfigure}[c]{0.47\linewidth}
                \includegraphics[width=\linewidth]{figures/cohen_d_map/Laansma_2021/parkinson/png/parkinson_area_HY3-vs-HY4+5.png}
                \label{fig:Laansma_2021_parkinson_area_HY3_vs_HY45_unthresholded}
            \end{subfigure}%
            \hfill
            \begin{minipage}[c]{0.005\linewidth}
                \centering\rule{0.5pt}{4cm}
            \end{minipage}%
            \hfill
            \begin{subfigure}[c]{0.47\linewidth}
                \includegraphics[width=\linewidth]{figures/cohen_d_map/Laansma_2021/parkinson/png/parkinson_area_HY3-vs-HY4+5_thresholded.png}
                \label{fig:Laansma_2021_parkinson_area_HY3_vs_HY45_thresholded}
            \end{subfigure}
        \end{minipage}
    \end{minipage}

    \vspace{-2cm}

    \begin{minipage}[b]{\linewidth}
        \begin{minipage}[c]{0.05\linewidth}
            \centering\rotatebox{90}{\textbf{L2}} \end{minipage}%
        \begin{minipage}[c]{0.95\linewidth}
            \begin{subfigure}[c]{0.47\linewidth}
                \includegraphics[width=\linewidth]{figures/cohen_d_map/Laansma_2021/parkinson/png/parkinson_area_HY1-vs-HY2.png}
                \label{fig:Laansma_2021_parkinson_area_HY1_vs_HY2_unthresholded}
            \end{subfigure}%
            \hfill
            \begin{minipage}[c]{0.005\linewidth}
                \centering\rule{0.5pt}{4cm}
            \end{minipage}%
            \hfill
            \begin{subfigure}[c]{0.47\linewidth}
                \includegraphics[width=\linewidth]{figures/cohen_d_map/Laansma_2021/parkinson/png/parkinson_area_HY1-vs-HY2_thresholded.png}
                \label{fig:Laansma_2021_parkinson_area_HY1_vs_HY2_thresholded}
            \end{subfigure}
        \end{minipage}
    \end{minipage}

    \vspace{-2cm}

    \begin{minipage}[b]{\linewidth}
        \begin{minipage}[c]{0.05\linewidth}
            \centering\rotatebox{90}{\textbf{L3}} \end{minipage}%
        \begin{minipage}[c]{0.95\linewidth}
            \begin{subfigure}[c]{0.47\linewidth}
                \includegraphics[width=\linewidth]{figures/cohen_d_map/Laansma_2021/parkinson/png/parkinson_area_HY1-vs-PD.png}
                \label{fig:Laansma_2021_parkinson_area_HY1_vs_PD_unthresholded}
            \end{subfigure}%
            \hfill
            \begin{minipage}[c]{0.005\linewidth}
                \centering\rule{0.5pt}{4cm}
            \end{minipage}%
            \hfill
            \begin{subfigure}[c]{0.47\linewidth}
                \includegraphics[width=\linewidth]{figures/cohen_d_map/Laansma_2021/parkinson/png/parkinson_area_HY1-vs-PD_thresholded.png}
                \label{fig:Laansma_2021_parkinson_area_HY1_vs_PD_thresholded}
            \end{subfigure}
        \end{minipage}
    \end{minipage}
    % \vspace{-2cm}

    \caption{\TG{can you add a color bar to the figure?} ENIGMA Parkinson group \TG{wdym?}. LSA* \TG{what's LSA*?} Laansma et
        al.~\cite{laansma2021international}, Cortical Surface Area. L1: HY 3 vs
        HY 4\%5; L2: HY 1 vs HY 2; L3: HY 1 vs PD. Cohen's d maps showing
        unthresholded effect sizes (left) and effect sizes thresholded by the
        \navr framework (right). Black regions indicate areas where Cohen's d
        values fall below the numerical variability threshold, demonstrating
        regions where reported effect sizes may be unreliable due to
        computational uncertainty.\label{fig:navr_laansma}}
\end{figure}

% ---

\begin{figure}[h]
    \centering
    \vspace{0.2cm}

    % Header row with column labels
    \begin{minipage}[b]{\linewidth}
        \begin{minipage}[c]{0.05\linewidth}
            % Empty space for alignment with condition labels
        \end{minipage}%
        \begin{minipage}[c]{0.95\linewidth}
            \begin{minipage}[c]{0.47\linewidth}
                \centering\textbf{Unthresholded}
            \end{minipage}%
            \hfill
            \begin{minipage}[c]{0.005\linewidth}
                % Vertical line separator
            \end{minipage}%
            \hfill
            \begin{minipage}[c]{0.47\linewidth}
                \centering\textbf{Thresholded}
            \end{minipage}
        \end{minipage}
    \end{minipage}

    % Horizontal line
    \noindent\rule{\linewidth}{0.5pt}
    \vspace{-1.5cm}

    % 22q11.2 deletion syndrome row
    \begin{minipage}[b]{\linewidth}
        \begin{minipage}[c]{0.05\linewidth}
            \centering\rotatebox{90}{\textbf{22q11.2}} \end{minipage}%
        \begin{minipage}[c]{0.95\linewidth}
            \begin{subfigure}[c]{0.47\linewidth}
                \includegraphics[width=\linewidth]{figures/cohen_d_map/enigma/22q_thickness_all.png}
                \label{fig:enigma_22q_thickness_unthresholded}
            \end{subfigure}%
            \hfill
            \begin{minipage}[c]{0.005\linewidth}
                \centering\rule{0.5pt}{4cm}
            \end{minipage}%
            \hfill
            \begin{subfigure}[c]{0.47\linewidth}
                \includegraphics[width=\linewidth]{figures/cohen_d_map/enigma/22q_thickness_all_thresholded.png}
                \label{fig:enigma_22q_thickness_thresholded}
            \end{subfigure}
        \end{minipage}
    \end{minipage}

    \vspace{-2cm}

    % ADHD row
    \begin{minipage}[b]{\linewidth}
        \begin{minipage}[c]{0.05\linewidth}
            \centering\rotatebox{90}{\textbf{ADHD}} \end{minipage}%
        \begin{minipage}[c]{0.95\linewidth}
            \begin{subfigure}[c]{0.47\linewidth}
                \includegraphics[width=\linewidth]{figures/cohen_d_map/enigma/adhd_thickness_adult.png}
                \label{fig:enigma_adhd_thickness_unthresholded}
            \end{subfigure}%
            \hfill
            \begin{minipage}[c]{0.005\linewidth}
                \centering\rule{0.5pt}{4cm}
            \end{minipage}%
            \hfill
            \begin{subfigure}[c]{0.47\linewidth}
                \includegraphics[width=\linewidth]{figures/cohen_d_map/enigma/adhd_thickness_adult_thresholded.png}
                \label{fig:enigma_adhd_thickness_thresholded}
            \end{subfigure}
        \end{minipage}
    \end{minipage}

    \vspace{-2cm}

    % Autism spectrum disorder row
    \begin{minipage}[b]{\linewidth}
        \begin{minipage}[c]{0.05\linewidth}
            \centering\rotatebox{90}{\textbf{ASD}} \end{minipage}%
        \begin{minipage}[c]{0.95\linewidth}
            \begin{subfigure}[c]{0.47\linewidth}
                \includegraphics[width=\linewidth]{figures/cohen_d_map/enigma/asd_thickness_meta_analysis.png}
                \label{fig:enigma_asd_thickness_unthresholded}
            \end{subfigure}%
            \hfill
            \begin{minipage}[c]{0.005\linewidth}
                \centering\rule{0.5pt}{4cm}
            \end{minipage}%
            \hfill
            \begin{subfigure}[c]{0.47\linewidth}
                \includegraphics[width=\linewidth]{figures/cohen_d_map/enigma/asd_thickness_meta_analysis_thresholded.png}
                \label{fig:enigma_asd_thickness_thresholded}
            \end{subfigure}
        \end{minipage}
    \end{minipage}
    \vspace{-2cm}

    % bipolar disorder row
    \begin{minipage}[b]{\linewidth}
        \begin{minipage}[c]{0.05\linewidth}
            \centering\rotatebox{90}{\textbf{Bipolar}} \end{minipage}%
        \begin{minipage}[c]{0.95\linewidth}
            \begin{subfigure}[c]{0.47\linewidth}
                \includegraphics[width=\linewidth]{figures/cohen_d_map/enigma/bipolar_thickness_adult.png}
                \label{fig:enigma_bipolar_thickness_unthresholded}
            \end{subfigure}%
            \hfill
            \begin{minipage}[c]{0.005\linewidth}
                \centering\rule{0.5pt}{4cm}
            \end{minipage}%
            \hfill
            \begin{subfigure}[c]{0.47\linewidth}
                \includegraphics[width=\linewidth]{figures/cohen_d_map/enigma/bipolar_thickness_adult_thresholded.png}
                \label{fig:enigma_bipolar_thickness_thresholded}
            \end{subfigure}
        \end{minipage}
    \end{minipage}

    \vspace{-2cm}
    % depression
    \begin{minipage}[b]{\linewidth}
        \begin{minipage}[c]{0.05\linewidth}
            \centering\rotatebox{90}{\textbf{Depression}} \end{minipage}%
        \begin{minipage}[c]{0.95\linewidth}
            \begin{subfigure}[c]{0.47\linewidth}
                \includegraphics[width=\linewidth]{figures/cohen_d_map/enigma/depression_thickness_adult.png}
                \label{fig:enigma_depression_thickness_unthresholded}
            \end{subfigure}%
            \hfill
            \begin{minipage}[c]{0.005\linewidth}
                \centering\rule{0.5pt}{4cm}
            \end{minipage}%
            \hfill
            \begin{subfigure}[c]{0.47\linewidth}
                \includegraphics[width=\linewidth]{figures/cohen_d_map/enigma/depression_thickness_adult_thresholded.png}
                \label{fig:enigma_depression_thickness_thresholded}
            \end{subfigure}
        \end{minipage}
    \end{minipage}

    \vspace{-2cm}
    % epilepsy
    \begin{minipage}[b]{\linewidth}
        \begin{minipage}[c]{0.05\linewidth}
            \centering\rotatebox{90}{\textbf{Epilepsy}} \end{minipage}%
        \begin{minipage}[c]{0.95\linewidth}
            \begin{subfigure}[c]{0.47\linewidth}
                \includegraphics[width=\linewidth]{figures/cohen_d_map/enigma/epilepsy_thickness_allepilepsy.png}
                \label{fig:enigma_epilepsy_thickness_unthresholded}
            \end{subfigure}%
            \hfill
            \begin{minipage}[c]{0.005\linewidth}
                \centering\rule{0.5pt}{4cm}
            \end{minipage}%
            \hfill
            \begin{subfigure}[c]{0.47\linewidth}
                \includegraphics[width=\linewidth]{figures/cohen_d_map/enigma/epilepsy_thickness_allepilepsy_thresholded.png}
                \label{fig:enigma_epilepsy_thickness_thresholded}
            \end{subfigure}
        \end{minipage}
    \end{minipage}

    \vspace{-2cm}
    % ocd 
    \begin{minipage}[b]{\linewidth}
        \begin{minipage}[c]{0.05\linewidth}
            \centering\rotatebox{90}{\textbf{OCD}} \end{minipage}%
        \begin{minipage}[c]{0.95\linewidth}
            \begin{subfigure}[c]{0.47\linewidth}
                \includegraphics[width=\linewidth]{figures/cohen_d_map/enigma/ocd_thickness_adult.png}
                \label{fig:enigma_ocd_thickness_unthresholded}
            \end{subfigure}%
            \hfill
            \begin{minipage}[c]{0.005\linewidth}
                \centering\rule{0.5pt}{4cm}
            \end{minipage}%
            \hfill
            \begin{subfigure}[c]{0.47\linewidth}
                \includegraphics[width=\linewidth]{figures/cohen_d_map/enigma/ocd_thickness_adult_thresholded.png}
                \label{fig:enigma_ocd_thickness_thresholded}
            \end{subfigure}
        \end{minipage}
    \end{minipage}

    \vspace{-2cm}
    % schizophrenia
    \begin{minipage}[b]{\linewidth}
        \begin{minipage}[c]{0.05\linewidth}
            \centering\rotatebox{90}{\textbf{Schizophrenia}} \end{minipage}%
        \begin{minipage}[c]{0.95\linewidth}
            \begin{subfigure}[c]{0.47\linewidth}
                \includegraphics[width=\linewidth]{figures/cohen_d_map/enigma/schizophrenia_thickness_all.png}
                \label{fig:enigma_schizophrenia_thickness_unthresholded}
            \end{subfigure}%
            \hfill
            \begin{minipage}[c]{0.005\linewidth}
                \centering\rule{0.5pt}{4cm}
            \end{minipage}%
            \hfill
            \begin{subfigure}[c]{0.47\linewidth}
                \includegraphics[width=\linewidth]{figures/cohen_d_map/enigma/schizophrenia_thickness_all_thresholded.png}
                \label{fig:enigma_schizophrenia_thickness_thresholded}
            \end{subfigure}
        \end{minipage}
    \end{minipage}

    \caption{ENIGMA cortical thickness Cohen's d maps showing unthresholded
        effect sizes (left) and effect sizes thresholded by the \navr framework
        (right) for different disorders. Black regions indicate areas where
        Cohen's d values fall below the numerical variability threshold,
        demonstrating regions where reported effect sizes may be unreliable due
        to computational uncertainty.\label{fig:navr_enigma_thickness}}
\end{figure}

\begin{figure}[h]
    \centering
    \vspace{0.2cm}
    % Header row with column labels
    \begin{minipage}[b]{\linewidth}
        \begin{minipage}[c]{0.05\linewidth}
            % Empty space for alignment with condition labels
        \end{minipage}%
        \begin{minipage}[c]{0.95\linewidth}
            \begin{minipage}[c]{0.47\linewidth}
                \centering\textbf{Unthresholded}
            \end{minipage}%
            \hfill
            \begin{minipage}[c]{0.005\linewidth}
                % Vertical line separator
            \end{minipage}%
            \hfill
            \begin{minipage}[c]{0.47\linewidth}
                \centering\textbf{Thresholded}
            \end{minipage}
        \end{minipage}
    \end{minipage}

    % Horizontal line
    \noindent\rule{\linewidth}{0.5pt}
    \vspace{-1.5cm}

    % 22q11.2 deletion syndrome row
    \begin{minipage}[b]{\linewidth}
        \begin{minipage}[c]{0.05\linewidth}
            \centering\rotatebox{90}{\textbf{22q11.2}} \end{minipage}%
        \begin{minipage}[c]{0.95\linewidth}
            \begin{subfigure}[c]{0.47\linewidth}
                \includegraphics[width=\linewidth]{figures/cohen_d_map/enigma/22q_subcortical_volume_all.png}
                \label{fig:enigma_22q_unthresholded_subcortical}
            \end{subfigure}%
            \hfill
            \begin{minipage}[c]{0.005\linewidth}
                \centering\rule{0.5pt}{4cm}
            \end{minipage}%
            \hfill
            \begin{subfigure}[c]{0.47\linewidth}
                \includegraphics[width=\linewidth]{figures/cohen_d_map/enigma/22q_subcortical_volume_all_thresholded.png}
                \label{fig:enigma_22q_thresholded_subcortical}
            \end{subfigure}
        \end{minipage}
    \end{minipage}

    \vspace{-2cm}

    % ADHD row
    \begin{minipage}[b]{\linewidth}
        \begin{minipage}[c]{0.05\linewidth}
            \centering\rotatebox{90}{\textbf{ADHD}} \end{minipage}%
        \begin{minipage}[c]{0.95\linewidth}
            \begin{subfigure}[c]{0.47\linewidth}
                \includegraphics[width=\linewidth]{figures/cohen_d_map/enigma/adhd_subcortical_volume_adult.png}
                \label{fig:enigma_adhd_unthresholded_subcortical}
            \end{subfigure}%
            \hfill
            \begin{minipage}[c]{0.005\linewidth}
                \centering\rule{0.5pt}{4cm}
            \end{minipage}%
            \hfill
            \begin{subfigure}[c]{0.47\linewidth}
                \includegraphics[width=\linewidth]{figures/cohen_d_map/enigma/adhd_subcortical_volume_adult_thresholded.png}
                \label{fig:enigma_adhd_thresholded_subcortical}
            \end{subfigure}
        \end{minipage}
    \end{minipage}

    \vspace{-2cm}

    % Autism spectrum disorder row
    \begin{minipage}[b]{\linewidth}
        \begin{minipage}[c]{0.05\linewidth}
            \centering\rotatebox{90}{\textbf{ASD}} \end{minipage}%
        \begin{minipage}[c]{0.95\linewidth}
            \begin{subfigure}[c]{0.47\linewidth}
                \includegraphics[width=\linewidth]{figures/cohen_d_map/enigma/asd_subcortical_volume_meta_analysis.png}
                \label{fig:enigma_asd_unthresholded_subcortical}
            \end{subfigure}%
            \hfill
            \begin{minipage}[c]{0.005\linewidth}
                \centering\rule{0.5pt}{4cm}
            \end{minipage}%
            \hfill
            \begin{subfigure}[c]{0.47\linewidth}
                \includegraphics[width=\linewidth]{figures/cohen_d_map/enigma/asd_subcortical_volume_meta_analysis_thresholded.png}
                \label{fig:enigma_asd_thresholded_subcortical}
            \end{subfigure}
        \end{minipage}
    \end{minipage}
    \vspace{-2cm}

    % bipolar disorder row
    \begin{minipage}[b]{\linewidth}
        \begin{minipage}[c]{0.05\linewidth}
            \centering\rotatebox{90}{\textbf{Bipolar}} \end{minipage}%
        \begin{minipage}[c]{0.95\linewidth}
            \begin{subfigure}[c]{0.47\linewidth}
                \includegraphics[width=\linewidth]{figures/cohen_d_map/enigma/bipolar_subcortical_volume_typeII.png}
                \label{fig:enigma_bipolar_unthresholded_subcortical}
            \end{subfigure}%
            \hfill
            \begin{minipage}[c]{0.005\linewidth}
                \centering\rule{0.5pt}{4cm}
            \end{minipage}%
            \hfill
            \begin{subfigure}[c]{0.47\linewidth}
                \includegraphics[width=\linewidth]{figures/cohen_d_map/enigma/bipolar_subcortical_volume_typeII_thresholded.png}
                \label{fig:enigma_bipolar_thresholded_subcortical}
            \end{subfigure}
        \end{minipage}
    \end{minipage}

    \vspace{-2cm}
    % depression
    \begin{minipage}[b]{\linewidth}
        \begin{minipage}[c]{0.05\linewidth}
            \centering\rotatebox{90}{\textbf{Depression}} \end{minipage}%
        \begin{minipage}[c]{0.95\linewidth}
            \begin{subfigure}[c]{0.47\linewidth}
                \includegraphics[width=\linewidth]{figures/cohen_d_map/enigma/depression_subcortical_volume_all.png}
                \label{fig:enigma_depression_unthresholded_subcortical}
            \end{subfigure}%
            \hfill
            \begin{minipage}[c]{0.005\linewidth}
                \centering\rule{0.5pt}{4cm}
            \end{minipage}%
            \hfill
            \begin{subfigure}[c]{0.47\linewidth}
                \includegraphics[width=\linewidth]{figures/cohen_d_map/enigma/depression_subcortical_volume_all_thresholded.png}
                \label{fig:enigma_depression_thresholded_subcortical}
            \end{subfigure}
        \end{minipage}
    \end{minipage}

    \vspace{-2cm}
    % epilepsy
    \begin{minipage}[b]{\linewidth}
        \begin{minipage}[c]{0.05\linewidth}
            \centering\rotatebox{90}{\textbf{Epilepsy}} \end{minipage}%
        \begin{minipage}[c]{0.95\linewidth}
            \begin{subfigure}[c]{0.47\linewidth}
                \includegraphics[width=\linewidth]{figures/cohen_d_map/enigma/epilepsy_subcortical_volume_allepilepsy.png}
                \label{fig:enigma_epilepsy_unthresholded_subcortical}
            \end{subfigure}%
            \hfill
            \begin{minipage}[c]{0.005\linewidth}
                \centering\rule{0.5pt}{4cm}
            \end{minipage}%
            \hfill
            \begin{subfigure}[c]{0.47\linewidth}
                \includegraphics[width=\linewidth]{figures/cohen_d_map/enigma/epilepsy_subcortical_volume_allepilepsy_thresholded.png}
                \label{fig:enigma_epilepsy_thresholded_subcortical}
            \end{subfigure}
        \end{minipage}
    \end{minipage}

    \vspace{-2cm}
    % ocd 
    \begin{minipage}[b]{\linewidth}
        \begin{minipage}[c]{0.05\linewidth}
            \centering\rotatebox{90}{\textbf{OCD}} \end{minipage}%
        \begin{minipage}[c]{0.95\linewidth}
            \begin{subfigure}[c]{0.47\linewidth}
                \includegraphics[width=\linewidth]{figures/cohen_d_map/enigma/ocd_subcortical_volume_adult.png}
                \label{fig:enigma_ocd_unthresholded_subcortical}
            \end{subfigure}%
            \hfill
            \begin{minipage}[c]{0.005\linewidth}
                \centering\rule{0.5pt}{4cm}
            \end{minipage}%
            \hfill
            \begin{subfigure}[c]{0.47\linewidth}
                \includegraphics[width=\linewidth]{figures/cohen_d_map/enigma/ocd_subcortical_volume_adult_thresholded.png}
                \label{fig:enigma_ocd_thresholded_subcortical}
            \end{subfigure}
        \end{minipage}
    \end{minipage}

    \vspace{-2cm}
    % schizophrenia
    \begin{minipage}[b]{\linewidth}
        \begin{minipage}[c]{0.05\linewidth}
            \centering\rotatebox{90}{\textbf{Schizophrenia}} \end{minipage}%
        \begin{minipage}[c]{0.95\linewidth}
            \begin{subfigure}[c]{0.47\linewidth}
                \includegraphics[width=\linewidth]{figures/cohen_d_map/enigma/schizophrenia_subcortical_volume_all.png}
                \label{fig:enigma_schizophrenia_unthresholded_subcortical}
            \end{subfigure}%
            \hfill
            \begin{minipage}[c]{0.005\linewidth}
                \centering\rule{0.5pt}{4cm}
            \end{minipage}%
            \hfill
            \begin{subfigure}[c]{0.47\linewidth}
                \includegraphics[width=\linewidth]{figures/cohen_d_map/enigma/schizophrenia_subcortical_volume_all_thresholded.png}
                \label{fig:enigma_schizophrenia_thresholded_subcortical}
            \end{subfigure}
        \end{minipage}
    \end{minipage}

    \caption{ ENIGMA subcortical volume Cohen's d maps showing unthresholded
        effect sizes (left) and effect sizes thresholded by the \navr framework
        (right) for different disorders. Black regions indicate areas where
        Cohen's d values fall below the numerical variability threshold,
        demonstrating regions where reported effect sizes may be unreliable due
        to computational uncertainty.}
    \label{fig:navr_enigma_subcortical}
\end{figure}

\section{Discussion}

% Comments:
% 1. Summarize the main results and what do they mean
% 2. Extension beyond FreeSurfer and expectation to generalize the findings to other neuroimaging software
% 3. Discuss the potential sources of numerical variability (minimal local, minimal precision, etc.)

% [] Mention using neuroimaging as biomarker (n=1 scenario), personalized medicine,

Our systematic perturbation of FreeSurfer revealed that numerical variability
alone can account for up to 30\% of the anatomical variability observed in
structural MRI measurements. This level of uncertainty can significantly impact
statistical outcomes in Parkinson's disease analyses \TG[done]{in Parkinson's
    disease analyses}, leading to the appearance or disappearance of clinically
relevant group differences or correlations depending solely on computational
conditions. These findings offer a mechanistic explanation for some of the
reproducibility challenges reported in clinical neuroimaging.

To move beyond identifying the problem, we introduced the Numerical-Anatomical
Variability Ratio (NAVR), a quantitative framework for assessing the relative
magnitude of computational noise. By establishing a theoretical link between
NAVR and the uncertainty in Cohen's d effect sizes, we provide a practical tool
to assess the robustness of findings. Our re-analysis of published ENIGMA
results illustrates this utility: while large sample sizes confer robustness to
core findings, many secondary effects fall below the computational noise floor.
This suggests that in exploratory studies with lower sample size, numerical
instability may undermine the reliability of reported effects. \TG{see previous
    comments: it would be nice to include analyses for some of such smaller
    studies}

Although our primary analysis focused on FreeSurfer 7.3.1 and Parkinson's
disease, the underlying numerical issues are general. Floating-point arithmetic
\TG[done]{what does this mean? Additions of floating-point numbers is not
    associative? If so, this is out of context} \YC{Yes, FP operations are not
    associative but it is an implication of the finite precision. I just remove it
    to simplify.} is sensitive to compiler behavior, hardware architecture, and
thread scheduling. As a result, neuroimaging pipelines—though deterministic in
design—can produce divergent results across computational environments.
Previous analyses of FSL~\cite{mirhakimi2025numerical} and ANTs~\YC{cite
    Mathieu} indicate that such instability is not unique to FreeSurfer but likely
pervades the field. SPM however seems to be less impacted by numerical
variability~\cite{mirhakimi2025numerical} \TG[done]{cite Niusha}.

Traditional image processing workflows rely on nonlinear optimization
procedures that can converge to different local minima under small
perturbations, resulting in substantive changes to derived measures. To address
this problem and decrease the execution time, the neuroimaging field is
increasingly shifting toward deep learning models. The latest FreeSurfer
release (v8), for example, now incorporates neural networks such as
FastSurfer~\cite{henschel2020fastsurfer} and
Synthmorph~\cite{hoffmann2021synthmorph} to replace its classical segmentation
and registration steps. However, this shift does not eliminate the problem of
instability but rather reframes it. While these DL models have been shown
stable during the inference stage~\cite{pepe2023numerical}, their training
process is subject to its own sources of variability. Factors like weight
initialization and floating-point precision can cause different training runs
to yield distinct models with varying performance, and their quantification
remains an open question in neuroimaging. This phenomenon is analogous to the
local minima issue in classical optimization. Ultimately, whether arising from
classical optimization or DL training, such instability means that even
identical inputs can lead to divergent interpretations, raising critical
concerns for research reproducibility and clinical translation.

\TG[done]{start a new paragraph here, on the generalizability across data cohorts. You
    could include a mention that numerical variability didn't differ between PD and
    HC, and possibly link to a supplementary figure to support the claim.}
To address the generalizability of our findings, we considered the
characteristics of our data cohorts. A potential limitation is that our
Parkinson's Disease cohort was relatively homogeneous in age and phenotype,
which could reduce anatomical variance and consequently inflate NAVR values.
However, direct statistical analyses revealed no significant differences in
numerical variability between the PD and healthy control groups (see
Supplementary Fig. X). This key finding supports the idea that the pipeline's
instability is a consistent factor and that our results are likely to generalize
across these populations.\TG[done]{I would moderate this statement a little bit.
    I think it would be interesting the measure the NAV in different pipelines and
    datasets, but I don't think there's a very strong need for that.} Measuring the
NAVR across more diverse datasets, software packages, and disease contexts would
nevertheless be a valuable step. Such work would help build a more comprehensive
map of computational reliability across the entire neuroimaging landscape.

NAVR provides a scalable, interpretable metric to quantify hidden numerical
variability. Although floating-point rounding is a dominant source of
instability, future work should broaden this analysis to other contributors,
including algorithmic choices, preprocessing decisions, and data handling
practices. A comprehensive understanding of these factors is essential for
developing numerically robust software. Our results show that computational
uncertainty is as critical as statistical uncertainty in neuroimaging;
systematic assessments of numerical variability, exemplified by NAVR, are
therefore necessary to ensure the reproducibility and reliability of
neuroimaging-based biomarkers. Extending this quantification to the
deep-learning training stage is equally important, given the field's central
role in modern neuroimaging, and would support more robust and interpretable
models. Likewise, evaluating numerical variability in the classical
optimization schemes used in non-linear registration is a key milestone, as
these traditional tools often provide the reference for training deep-learning
models. Advancing both lines of analysis will benefit conventional and
learning-based approaches alike. \TG[done]{I would merge this paragraph with
    the previous one, and highlight specific next steps regarding numerical
    variability in neuroimaging.}

\section{Methods}

\TG{include a summary of your methods here.}

\TG{Overall the methods are quite brief, you should add more details so that
    people get a better sense of what you did, see detailed suggestions in the text}

\subsection{Participants}

We used structural MRI data from the Parkinson's Progression Markers Initiative
(PPMI). Participants included 125 Parkinson's disease patients without mild
cognitive impairment (PD-non-MCI) and 106 healthy controls, each providing
longitudinal T1-weighted MRI data across two visits. Patients with mild
cognitive impairment were excluded to reduce confounding influences.

T1-weighted images were drawn from the Parkinson's Progression Markers
Initiative (PPMI) database (www.ppmi-info.org). Inclusion required (i)
diagnosis of idiopathic Parkinson's disease (PD-non-MCI) or healthy control
(HC); (ii) two usable visits separated by $0.9-2.0$ years; and (iii) absence of
other neurological disorders. The final dataset comprised 90 healthy controls
and 118 PD-non-MCI participants (Extended Data \TG{why extended data?} Table
1). The study was approved by the local research ethics boards of all
contributing centres, and written informed consent was obtained from every
participant.

Inclusion criteria required \TG{this is redundant and not fully consistent with
    the previous paragraph}: (1) primary PD diagnosis or healthy control status,
(2) availability of two visits with T1-weighted scans, and (3) absence of other
neurological diagnoses. PD severity was assessed using the Unified Parkinson's
Disease Rating Scale (UPDRS). The study received ethics approval from
participating institutions, and all participants provided written informed
consent (Table~\ref{tab:cohort_stat}).

PD and HC groups showed no significant age differences ($p > 0.05$) but
differed in education ($t = -2.05$, $p = 0.04$) and sex distribution ($\chi^2 =
    4.15$, $p = 0.04$). The longitudinal cohort \TG{not sure what "longitudinal
    cohort" means here.} showed no significant demographic differences between
groups (Table~\ref{tab:cohort_stat}).

\begin{table}[h!]
    \centering
    \begin{tabular}{lcc}
        \toprule
        \textbf{Cohort}         & \textbf{HC}        & \textbf{PD-non-MCI} \\
        \hline
        n                       & $90 $              & $118 $              \\
        Age (y)                 & $60.7 \pm 9.7 $    & $61.1 \pm \09.2 $   \\
        Age range               & $30.6 - 79.8 $     & $39.2 - 78.3 $      \\
        Gender (male, \%)       & $48 \; (53.3\%) $  & $77 \; (65.3\%) $
        \\
        Education (y)           & $16.7 \pm \03.3 $  & $16.2 \pm \02.9 $   \\
        UPDRS III OFF baseline  & $- $               & $23.6 \pm 10.3 $    \\
        UPDRS III OFF follow-up & $- $               & $25.6 \pm 11.2 $    \\
        Duration T2 - T1 (y)    & $\01.4 \pm \00.5 $ & $\01.4 \pm \00.6 $  \\
        \bottomrule
    \end{tabular}
    \vspace{1em}

    \caption{\textbf{Abbreviations:} MCI = Mild Cognitive Impairment; UPDRS =
        Unified Parkinson's Disease Rating Scale; PD = Parkinson's disease.
        Values are expressed as mean $\pm$ standard deviation. PD-non-MCI
        longitudinal sample is a subsample of the PD-non-MCI original sample
        that had longitudinal data and disease severity scores available.
        \label{tab:cohort_stat}}
\end{table}

\subsection{Image acquisition and preprocessing}

T1-weighted MRI scans from PPMI were acquired using standardized protocols
(repetition time=2.3 s, echo time=2.98 ms, inversion time=0.9 s, 1 mm isotropic
resolution, number of slices = 192, field of view = 256 mm, and matrix size =
256 $\times$ 256). However, since PPMI is a multisite project there may be
slight differences in the sites' setup. Images underwent standard preprocessing
using FreeSurfer 7.3.1 instrumented with Fuzzy-libm (see next section). Each
participant's MRI data were processed 26 times under different numerical
perturbations to quantify numerical variability. Failed runs \TG{defined
    failed: failed due to technical reasons or failed QC?} were discarded, ensuring
exactly 26 successful repetitions per subject. \TG{you should mention QC too}

Longitudinal processing followed the standard FreeSurfer
stream~\cite{reuter2012within}: cross-sectional processing of both timepoints,
followed by creation of an unbiased within-subject
template~\cite{reuter2011avoiding} using robust
registration~\cite{reuter2010highly}. Downstream analyses used unperturbed
FreeSurfer to prevent additional numerical perturbations.

\subsection{Numerical Variability Assessment}

We employed Monte Carlo Arithmetic (MCA)~\cite{parker1997monte} to quantify
numerical instability in FreeSurfer computations. MCA introduces controlled
random perturbations into floating-point operations, simulating rounding errors
that occur across different computational environments. This stochastic
approach enables systematic assessment of result stability by measuring
variation across multiple runs of identical analyses.

\TG{I think you should give more details about MCA, including the equations and also your improvements to fuzzy-libm}

We used Fuzzy-libm~\cite{salari2021accurate}, which extends MCA to mathematical
library functions (\texttt{exp}, \texttt{log}, \texttt{sin}, \texttt{cos})
through Verificarlo~\cite{denis2016verificarlo}, an LLVM-based compiler.
Virtual precision parameters were set to 53 bits for double precision and 24
bits for single precision to simulate realistic machine-level precision errors.

We processed each visit with FreeSurfer 7.3.1. To sample numerical variability
we compiled FreeSurfer \TG{did you really compile Freesurfer?} with Fuzzy-libm
an implementation of Monte Carlo arithmetic (MCA) that injects zero-mean
rounding noise into every elementary function call. Virtual precision was set
to 53 bits for operations promoted to double and 24 bits for single precision,
\TG{there's quite some redundancy with the previous paragraph} thereby
preserving IEEE-754 expectations but exposing the variance of alternative
execution paths. Each subject-visit pair was processed 26 times; failed or
quality-control-flagged runs were discarded, and exactly 26 successful runs per
pair were retained for analysis.

\subsubsection{Numerical-Anatomical Variability Ratio (\navr)}

To quantify computational stability relative to anatomical variation, we
introduce the Numerical-Anatomical Variability Ratio (\navr). For each brain
region, \navr measures the ratio of measurement uncertainty arising from
computational processes to natural inter-subject anatomical variation:

\[
    \text{\navr} = \frac{\sigma_{\text{num}}}{\sigma_{\text{anat}}}
\]

where $\sigma_{\text{num}}$ represents numerical variability (measurement
precision across MCA repetitions for individual subjects) and
$\sigma_{\text{anat}}$ represents anatomical variability (inter-subject
differences within each repetition).

For each region of interest, measurements from $n$ MCA repetitions across $m$
subject-visit pairs form a data matrix $\mathcal{M}_{n \times m}$, where
element $x_{i,j}$ represents the measurement for subject $j$ in MCA repetition
$i$.

Numerical variability quantifies intra-subject measurement consistency:
\begin{equation}
    \sigma^2_{\text{num}} = \frac{1}{m} \sum_{j=1}^{m} \left[ \frac{1}{n-1} \sum_{i=1}^{n} (x_{i,j} - \bar{x}_{\cdot,j})^2 \right]
    \label{eq:sigma_num}
\end{equation}

Anatomical variability captures inter-subject differences:
\begin{equation}
    \sigma^2_{\text{anat}} = \frac{1}{n} \sum_{i=1}^{n} \left[ \frac{1}{m-1} \sum_{j=1}^{m} (x_{i,j} - \bar{x}_{i,\cdot})^2 \right]
    \label{eq:sigma_anat}
\end{equation}

where $\bar{x}_{\cdot,j}$ and $\bar{x}_{i,\cdot}$ denote column and row means,
respectively. Higher \navr values indicate regions where computational
uncertainty approaches or exceeds biological variation, potentially
compromising the detection of true anatomical differences.\TG{didn't you pool
    sigma anat across PD and HC?}

\subsubsection{Relationship between \navr~and Effect Size Uncertainty}

\TG[done]{The remainder of this paragraph is quite informal, could you tidy up the
    equations to start from the definition of Cohen's d and clearly derive its
    standard deviation? You should also clarify the assumptions that are made.}

To establish a direct mathematical link between a method's computational
reproducibility and the reliability of group-level statistical inferences, we
derived an analytical expression that connects numerical variability to the
uncertainty of the estimated effect size Cohen's $d$. The goal is to quantify
how numerical noise propagates to create uncertainty in the calculated $d$
value, which we denote by its standard deviation, $\sigma_d$.

We model an observed measurement $X_i$ for a subject $i$ as the sum of their
true, underlying biological value $\mu_i$ and a numerical error term
$\varepsilon_i$:
\[
    X_i = \mu_i + \varepsilon_i
\]
Our derivation proceeds from a specific framework designed to isolate the
impact of numerical variability. The key assumptions are: (1) Null Hypothesis
Scenario: we analyze the computational uncertainty for a fixed sample of $N$
subjects, partitioned into two balanced groups ($n_1 = n_2 = N/2$). We assume
the null hypothesis is true for this sample, meaning the mean of the true
biological values is identical between the groups ($\bar{\mu}_1 =
    \bar{\mu}_2$). In this context, the true values $\{\mu_i\}$ are treated as
constants, and the only source of randomness is the numerical error, which
models the variability across repeated computations on the same data. (2)
Numerical Error Model: the numerical error for each subject is modeled as an
independent, zero-mean Gaussian random variable $\varepsilon_i \sim
    \mathcal{N}(0, \sigma_{\text{num},i}^2)$, with $\bar\varepsilon \sim
    \mathcal{N}(0, \sigma_{\text{num}}^2/n)$. (3) Anatomical variability: we assume
the numerical variability ($\sigma_{\text{anat}}^2 = \mathrm{Var}(\{\mu_i\})$)
is small compared to the biological variability within the sample. This
condition ($\sigma_{\text{num},i} \ll \sigma_{\text{anat}}$) allows the pooled
standard deviation of the observed data, $s_p$, to be approximated by the
standard deviation of the true biological values: $s_p \approx
    \sigma_{\text{anat}}$.

\paragraph{Cohen's $d$} is defined as the difference in sample means normalized by the pooled standard
deviation:
\[
    d = \frac{\bar{X}_1 - \bar{X}_2}{s_p}
\]
Based on assumption (3), we can approximate $d$ as:
\[
    d \approx \frac{\bar{X}_1 - \bar{X}_2}{\sigma_{\text{anat}}}
\]
The uncertainty in $d$ arising from numerical noise, $\sigma_d$, is the
standard deviation of this quantity. Its variance, $\sigma_d^2$, is therefore:
\[
    \sigma_d^2 = \mathrm{Var}(d) \approx \mathrm{Var}\left(\frac{\bar{X}_1 -
        \bar{X}_2}{\sigma_{\text{anat}}}\right) = \frac{1}{\sigma_{\text{anat}}^2}
    \mathrm{Var}(\bar{X}_1 - \bar{X}_2)
\]
The difference in observed sample means is $\bar{X}_1 - \bar{X}_2 =
    (\bar{\mu}_1 + \bar{\varepsilon}_1) - (\bar{\mu}_2 + \bar{\varepsilon}_2)$.
Since the true values $\{\mu_i\}$ are constant for our fixed sample and we have
assumed $\bar{\mu}_1 = \bar{\mu}_2$, the variance of this difference depends
only on the variance of the mean numerical errors:
\[
    \mathrm{Var}(\bar{X}_1 - \bar{X}_2) = \mathrm{Var}(\bar{\varepsilon}_1 -
    \bar{\varepsilon}_2)
\]
As the numerical errors $\varepsilon_i$ are independent, the variance of the
difference between the mean errors is the sum of their respective variances:
\[
    \mathrm{Var}(\bar{\varepsilon}_1 - \bar{\varepsilon}_2) =
    \mathrm{Var}(\bar{\varepsilon}_1) + \mathrm{Var}(\bar{\varepsilon}_2) =
    \frac{\sigma_{\text{num}}^2}{n/2} + \frac{\sigma_{\text{num}}^2}{n/2} =
    \frac{4}{n}\sigma_{\text{num}}^2
\]
Substituting this result back into the expression for $\sigma_d^2$, we find:
\begin{equation*}
    \begin{split}
        \sigma_d^2 & \approx \frac{1}{\sigma_{\text{anat}}^2}
        \left(\frac{4}{n}\sigma_{\text{num}}^2\right) = \frac{4}{n}
        \left(\frac{\sigma_{\text{num}}}{\sigma_{\text{anat}}}\right)^2 \\
        \sigma_d^2 & \approx \frac{4}{n} \nu_{\text{nav}}^2
    \end{split}
\end{equation*}
Taking the square root yields
the final relationship for the standard deviation of the effect size:
\[
    \sigma_d = \frac{2}{\sqrt{n}} \nu_{\text{nav}}
\]

This provides a direct, quantitative link between the numerical variability
(captured by $\nu_{\text{nav}}$) and the the uncertainty of the effect size,
$\sigma_d$.

\paragraph{The two-sample t-test statistic} is defined as:
\[
    t = \frac{\bar{X}_1 - \bar{X}_2}{s_p \sqrt{\frac{1}{n_1} +
            \frac{1}{n_2}}}
\]
Assuming balanced groups ($n_1 = n_2 = n/2$) and using assumption (3) from the
previous section, we can approximate the t-statistic as:
\[
    t \approx \frac{\bar{X}_1 - \bar{X}_2}{\sigma_{\text{anat}}}\sqrt{\frac{n}{4}}
\]
The uncertainty in $t$ arising from numerical noise, $\sigma_t$, is the
standard deviation of this quantity. Its variance, $\sigma_t^2$, is therefore:
\begin{equation*}
    \begin{split}
        \sigma_t^2 & = \mathrm{Var}(t)
        \approx \mathrm{Var}\!\left(\frac{\bar{X}_1 - \bar{X}_2}{\sigma_{\text{anat}}}\sqrt{\frac{4}{n}}\right)
        = \mathrm{Var}\!\left(\frac{\bar{X}_1 - \bar{X}_2}{\sigma_{\text{anat}}}\right) \frac{n}{4} = \left(\frac{4}{n} \nu_{\text{nav}}^2 \right)\frac{n}{4} \\[4pt]
        \sigma_t^2 & = \nu_{\text{nav}}^2.
    \end{split}
\end{equation*}

Taking the square root yields the final relationship for the standard deviation
of the t-statistic:
\[
    \sigma_t = \nu_{\text{nav}}
\]

We can now apply this result to derive the uncertainty in p-values derived from
the t-statistic. Let $Z \sim t(df)$ be a random variable following the
t-distribution with $df$ degrees of freedom, $f_Z$ its probability density
function, and $F_Z$ its cumulative distribution function. The two-sided p-value
is a function of $Z$ defined as $p(z) = 2(1 - F_Z(|z|))$. Applying the delta
method yields:
\begin{equation*}
    \begin{aligned}
        \mathrm{Var}(p(z)) & = \mathrm{Var}(2(1 - F_Z(|z|)))                                            \\
                           & \approx \left(-2 f_Z(|z|) \cdot \mathrm{sign}(z)\right)^2 \mathrm{ Var}(z) \\
                           & = 4 (f_Z(|z|))^2 \nu_{\text{nav}}^2,
    \end{aligned}
\end{equation*}
And thus
\[
    \sigma_p = 2 f_Z(|z|) \nu_{\text{nav}}.
\]

\paragraph{ANCOVA group effect.}
Analysis of covariance (ANCOVA) evaluates group using a general linear model,
\[
    y = X\beta + \varepsilon, \qquad \varepsilon \sim \mathcal{N}(0, \sigma^2 I),
\]
where $y$ is the vector of regional measurements across subjects and $X$
contains an intercept, diagnostic group (PD vs.\ HC) and covariates (age and
sex). The adjusted group difference corresponds to the 1-degree-of-freedom
contrast $c^\top\beta$ with $c = [0,\, 1,\, 0,\, 0]^\top$. Let
$\widehat{\beta}$ be the ordinary least-squares estimator and
$\widehat{\sigma}_{\mathrm{res}}^{2} = SS_{\mathrm{res}}/df_2$ the residual
mean square, where $df_2 = n - \mathrm{rank}(X)$. The sum of squares associated
with the group effect is
\[
    SS_{\text{group}} =
    \frac{(c^\top \widehat{\beta})^2}{c^\top (X^\top X)^{-1} c}, \qquad df_1 = 1,
\]
yielding the ANCOVA $F$-statistic
\[
    F = \frac{MS_{\text{group}}}{MS_{\text{res}}}
    = \frac{(c^\top \widehat{\beta})^2}{
    \widehat{\sigma}_{\mathrm{res}}^{2}\; c^\top (X^\top X)^{-1}c}
    \sim F(df_1 = 1, df_2).
\]
Two-sided significance is evaluated using the central $F$-distribution under
the null hypothesis,
\[
    p_{\sc{F}} = 1 - F_{\sc{F}}\!\left(F; df_1, df_2\right),
\]
where $F_{\sc{F}}$ is the cumulative distribution function of the
$F$-distribution. This is equivalent to the classical two-sided $t$-test
through $F = t^2$ when $df_1$=1 (cf~\cite{johnson1995continuous} p.403).

Since $F=t^2$, the uncertainty in the $F$-statistic arising from numerical
noise, $\sigma_F$, is related to the uncertainty in the $t$-statistic,
$\sigma_t$, by
\[
    \mathrm{Var}(F)
    = \mathrm{Var}(t^2) \approx (2t)^2 \mathrm{Var}(t)
    = 4t^2 \nu_{\text{nav}}^2
    = 4F \nu_{\text{nav}}^2,
\]
so that
\[
    \sigma_F = 2\sqrt{F} \nu_{\text{nav}}.
\]

For p-values derived from the $F$-statistic, we use the delta method to
approximate the variance of $p$. Let $Z \sim F(df_1, df_2)$ be a random
variable following the $F$-distribution with $df_1$ and $df_2$ degrees of
freedom, $f_Z$ its probability density function, and $F_Z$ its cumulative
distribution function. The p-value is a function of $Z$ defined as $p(z) = 1 -
    F_Z(z)$. Applying the delta method yields:
\begin{equation*}
    \begin{aligned}
        \mathrm{Var}(p(z)) & = \mathrm{Var}(1 - F_Z(z))                                            \\
                           & \approx \left(-f_Z(z) \cdot \mathrm{sign}(z)\right)^2 \mathrm{Var}(z) \\
                           & = {f_Z(z)}^2 4z\nu_{\text{nav}}^2,
    \end{aligned}
\end{equation*}
And thus
\[
    \sigma_p = 2 \sqrt{z} f_Z(z) \nu_{\text{nav}}.
\]
\paragraph{Partial correlation under numerical noise.}
Partial correlation measures the relationship between two variables while
controlling for the influence of one or more additional variables. For
instance, in our case we compute the partial correlation between regional brain
measures and UPDRS-III scores, controlling for age and sex. Let three centered
variables $X$, $Y$, and $Z$ have pairwise correlations
\[
    a = r_{XY}, \qquad b = r_{XZ}, \qquad c = r_{YZ}.
\]
The partial correlation between $X$ and $Y$ controlling for $Z$ is
\[
    R \;\equiv\; r_{XY\cdot Z} \;=\;
    \frac{a - bc}{\sqrt{(1 - b^2)(1 - c^2)}}.
\]

We assume that only $X$ is affected by small, independent numerical
perturbations:
\[
    X_i = x_i + \varepsilon_i, \qquad
    \mathbb{E}[\varepsilon_i] = 0, \qquad
    \Var(\varepsilon_i) = \sigma_{\mathrm{num}}^2.
\]
Variables $Y$ and $Z$ are treated as fixed. The characteristic anatomical scale
of $X$ is denoted $\sigma_{\mathrm{anat}}$.

\paragraph{Step 1: First-order differential of the partial correlation.}
Applying the chain rule to $R(a,b,c)$ gives
\[
    \frac{\partial R}{\partial a} = \frac{1}{D}, \qquad
    \frac{\partial R}{\partial b} = \frac{(1 - c^2)(ab - c)}{D^3}, \qquad
    \frac{\partial R}{\partial c} = \frac{(1 - b^2)(ac - b)}{D^3},
\]
where $D = \sqrt{(1 - b^2)(1 - c^2)}$.

Because $Y$ and $Z$ are not perturbed, the only random terms are $r_{XY}$ and
$r_{XZ}$, both depending on $X$. Therefore
\[
    \frac{\partial R}{\partial X_i}
    = \frac{\partial R}{\partial a}\frac{\partial a}{\partial X_i}
    + \frac{\partial R}{\partial b}\frac{\partial b}{\partial X_i}.
\]

The first-order sensitivities of the pairwise correlations with respect to
$X_i$ are (using the standard correlation influence function)
\[
    \frac{\partial a}{\partial X_i}
    = \frac{1}{n\,\sigma_X}\,(v_i - a\,u_i), \qquad
    \frac{\partial b}{\partial X_i}
    = \frac{1}{n\,\sigma_X}\,(w_i - b\,u_i),
\]
where $u_i$, $v_i$, and $w_i$ are the standardized, centered values of $X$,
$Y$, and $Z$ respectively.

Assuming independent numerical perturbations $\varepsilon_i$ with variance
$\sigma_{\mathrm{num}}^2$, the delta method gives
\[
    \Var(R)
    = \sum_{i=1}^n \left(\frac{\partial R}{\partial X_i}\right)^2 \Var(\varepsilon_i)
    = \frac{\sigma_{\mathrm{num}}^2}{n^2 \sigma_X^2}
    \sum_{i=1}^n \left[
        \frac{(v_i - a u_i)}{D}
        + (w_i - b u_i)\frac{(1 - c^2)(ab - c)}{D^3}
        \right]^2.
\]

Using the baseline identities $\sum (v_i - a u_i)^2 = n(1 - a^2)$, $\sum (w_i -
    b u_i)^2 = n(1 - b^2)$, and $\sum (v_i - a u_i)(w_i - b u_i) = n(ab - c)$, we
obtain, after simplification,
\begin{equation}
    \label{eq:var_R}
    \Var(R)
    \;\approx\;
    \frac{\sigma_{\mathrm{num}}^2}{n\,\sigma_X^2}
    \left[
        \frac{1}{(1 - b^2)(1 - c^2)}
        + \frac{(ab - c)^2}{(1 - b^2)^2(1 - c^2)}
        \right].
\end{equation}

Using $(ab-c)^2 = (c-ab)^2 = r_{ZY,X}^2(1 - a^2)(1 - b^2)$, we can rewrite the
term in brackets as
\[
    \frac{1}{(1 - b^2)(1 - c^2)}
    + \frac{(ab - c)^2}{(1 - b^2)^2(1 - c^2)}
    = \frac{1 + r_{ZY,X}^2(1 - a^2)}{(1 - b^2)(1 - c^2)}.
\]

So equation~\eqref{eq:var_R} becomes
\[
    \Var(R)
    \;\approx\;
    \frac{\sigma_{\mathrm{num}}^2}{n\,\sigma_X^2}
    \left[
        \frac{1 + r_{ZY,X}^2(1 - a^2)}{(1 - b^2)(1 - c^2)}
        \right].
\]

Given that $a, b, c$ are unknown, we have a lower bound on the variance:
\[
    \Var(R)
    \;\geq\;
    \frac{\sigma_{\mathrm{num}}^2}{n\,\sigma_X^2}.
\]
since $0 \leq r_{ZY,X}^2(1 - a^2) \leq 1$.

so the lower bound becomes
\[
    \Var(R)
    \;\geq\;
    \frac{\sigma_{\mathrm{num}}^2}{n\,\sigma_X^2}
    (1 - R^2)^{3}.
\]

Replacing $\sigma_X$ by $\sigma_{\mathrm{anat}}$, we have the final result:
\[
    \operatorname{sd}(r_{XY\cdot Z})
    \;\leq\;
    \frac{\sigma_{\mathrm{num}}}{\sigma_{\mathrm{anat}}}
    \sqrt{\frac{(1 - r_{XY\cdot Z}^2)}{n}}
\]

\paragraph{Remarks.}
1. When $Z$ is uncorrelated with either $X$ or $Y$
($r_{XZ}=r_{YZ}=0$), the expression simplifies to the
bivariate result
\[
    \operatorname{sd}(r_{XY})
    \;\approx\;
    \frac{\sigma_{\mathrm{num}}}{\sigma_{\mathrm{anat}}}
    \sqrt{\frac{(1 - r_{XY}^2)}{n}}.
\]

\section{Data Availability}
The data that support the findings of this study are available from the
Parkinson's Progression Markers Initiative (PPMI) database
(www.ppmi-info.org/access-data-specimens/download-data), but restrictions apply
to the availability of these data, which were used under license for the
current study, and so are not publicly available. Data are however available
from the authors upon reasonable request and with permission of the PPMI.

\section{Code Availability}

All MCA instrumentation scripts, FreeSurfer build instructions and analysis
notebooks are available at [GitHub URL to be inserted]. Exact commit hashes are
archived on Zenodo (DOI [to be added]) to ensure bit-level reproducibility.

\section{Acknowledgements}

The analyses were conducted on the Virtual Imaging
Platform~\cite{glatard2012virtual}, which utilizes resources provided by the
Biomed virtual organization within the European Grid Infrastructure (EGI). We
extend our gratitude to Sorina Pop from CREATIS, Lyon, France, for her support.
\TG{acknowledge MJFF project LivingPark}

\bibliographystyle{plain}
\bibliography{main}

\clearpage

\appendix

\section{Formula}

\subsection{Significant digits formula}
\label{eq:significant_digits}

We compute the number of significant bits \(\hat{s}\) with probability
\(p_s=0.95\) and confidence \(1-\alpha_s=0.95\) using the
\texttt{significantdigits}
package\footnote{\url{https://github.com/verificarlo/significantdigits}}
(version 0.4.0). \texttt{significantdigits} implements the Centered Normality
Hypothesis approach described in~\cite{sohier2021confidence}:
\[
    \hat{s_i} = -\log_2 \left| \frac{\hat{\sigma_i}}{\hat{\mu_i}} \right| -
    \delta(n, \alpha_s, p_s),
\]
where \(\hat{\sigma_i}\) and \(\hat{\mu_i}\) are the average and standard
deviation over the repetitions, and
\begin{equation}
    \delta(n, \alpha_s, p_s) = \log_2 \left(
    \sqrt{\frac{n-1}{\chi^2_{1-\alpha_s/2}}} \Phi^{-1} \left( \frac{p_s+1}{2}
    \right) \right)
\end{equation}
is a penalty term for estimating \(\hat{s_i}\) with probability \(p_s\) and
confidence level \(1-\alpha_s\) for a sample size \(n\). \(\Phi^{-1}\) is the
inverse cumulative distribution of the standard normal distribution and
\(\chi^2\) is the Chi-2 distribution with \(n\)-1 degrees of freedom.

\subsection{Extended Sørensen-Dice coefficient}
\label{eq:extended_dice}

The extended Sørensen-Dice coefficient is a measure of overlap between multiple
sets, defined as follows:
\[
    \text{Dice}(A_1, A_2, \dots, A_n) = \frac{n \left| \bigcap_{i=1}^{n} A_i \right|}{\sum_{i=1}^{n} \left| A_i \right|}
\].

\subsection{Delta Method}
\label{eq:delta_method}

The delta method~\cite{cox2005delta} is a method of deriving the asymptotic
distribution of a random variable. It is applicable when the random variable
being considered can be defined as a differentiable function of a random
variable which is asymptotically Gaussian. The delta method is a first order
approximation for a ratio $X/Y$,
\[
    \text{Var}\!\left(\frac{X}{Y}\right)\approx
    \left(\frac{\mathbb{E}(X)}{\mathbb{E}(Y)}\right)^2\!
    \left(\frac{\text{Var}(X)}{[\mathbb{E}(X)]^2}
    -\frac{2\,\text{Cov}(X,Y)}{\mathbb{E}(X)\mathbb{E}(Y)}
    +\frac{\text{Var}(Y)}{[\mathbb{E}(Y)]^2}\right),
\]

\section{Cross-sectional Analysis}

As a side result, the cross-sectional analysis measures the impact of numerical
variability in FreeSurfer version 7.3.1 on the PPMI (Parkinson's Progression
Markers Initiative) cohort. This involves comparing the estimation of
structural MRI measures, including cortical and subcortical volumes, cortical
thickness, and surface area. The goal is to assess the stability of these key
metrics and quantify the numerical variability.

FreeSurfer 7.3.1 showed limited numerical precision across all cortical
measures: $1.61 \pm 0.20$ significant digits for cortical thickness, $1.33 \pm
    0.23$ for surface area, and $1.33 \pm 0.23$ for cortical volume
(Figures~\ref{fig:sig_digits_cortical}). Subcortical volumes have a similar
precision with $1.33 \pm 0.22$ significant digits on average
(Figure~\ref{fig:sig_digits_subcortical}). These values indicate measurements
are typically precise to only one decimal place, with some instances showing
complete precision loss. Regional consistency was observed within each metric
type, with cortical thickness showing the highest precision (range: $1.22-1.93$
digits) compared to surface area ($0.82 - 1.72$ digits) and cortical volume
($0.80 - 1.72$ digits). Subcortical volumes exhibited the highest precision
(range: $0.88 - 1.57$ digits), with a mean of $1.33 \pm 0.22$ significant
digits.

To measure the structural overlap, we evaluated using the extended
Sørensen-Dice coefficient: Dice coefficients revealed substantial inter-subject
variability, particularly in temporal pole regions (Figure~\ref{fig:dice}). We
also observed that the Dice coefficient varies across regions, with some
regions showing higher variability than others with cortical volume ($0.00 -
    0.91$) with a mean of $0.75 \pm 0.11$ and subcortical volume ($0.18 - 0.94$)
with a mean of $0.82 \pm 0.08$. Finally, we noticed that subcortical volume
measurements are more stable than cortical volume.

\begin{figure}
    \includegraphics*[width=\linewidth]{figures/dice.pdf}
    \caption{Dice coefficient.\label{fig:dice}}

\end{figure}

\begin{figure}
    \includegraphics*[width=\linewidth]{figures/sig_digits.pdf}
    \caption{Number of significant digits for each cortical region and
        metric.\label{fig:sig_digits_cortical}}
\end{figure}

\begin{figure}
    \includegraphics*[width=\linewidth]{figures/sig_digits_subcortical_volume.pdf}
    \caption{Number of significant digits of subcortical volume for each
        subcortical region.\label{fig:sig_digits_subcortical}}
\end{figure}

\subsection{Within-subject significant digits averaged across all subjects}

\begin{longtblr}[ caption={Within-subject significant digits averaged across all subjects.},
        label={tab:sig-cortical},]{ colspec={lcc|cc|cc}, width=0.25\linewidth,
        row{even}={white,font=\footnotesize},
        row{odd}={gray9,font=\footnotesize}, rows = {rowsep=0pt}, rowhead=2,
    row{1}={white,font=\bfseries}, row{2}={gray9}} \SetCell[c=1]{c}Region &
    \SetCell[c=2]{c}{cortical thickness }                                 &                                   &
    \SetCell[c=2]{c}{surface area}                                        &
                                                                          & \SetCell[c=2]{c}{cortical volume} &
    \\
                                                                          & lh                                &
    rh                                                                    & lh
                                                                          & rh                                & lh
                                                                          & rh                                                                    \\
    \hline
    bankssts                                                              & $1.65 \pm 0.16$                   &
    $1.69 \pm 0.13$                                                       & $1.15 \pm 0.18$
                                                                          & $1.21 \pm 0.13$                   & $1.08 \pm 0.17$ & $1.14 \pm 0.13$
    \\
    caudalanteriorcingulate                                               & $1.38 \pm 0.14$                   &
    $1.40 \pm 0.14$                                                       & $1.14 \pm 0.22$
                                                                          & $1.19 \pm 0.18$                   & $1.14 \pm 0.24$ & $1.21 \pm 0.20$
    \\
    caudalmiddlefrontal                                                   & $1.77 \pm 0.18$                   &
    $1.77 \pm 0.19$                                                       & $1.40 \pm 0.21$
                                                                          & $1.31 \pm 0.23$                   & $1.40 \pm 0.22$ & $1.30 \pm 0.23$
    \\
    cuneus                                                                & $1.52 \pm 0.19$                   &
    $1.54 \pm 0.19$                                                       & $1.34 \pm 0.14$
                                                                          & $1.33 \pm 0.14$                   & $1.32 \pm 0.14$ & $1.28 \pm 0.15$
    \\
    entorhinal                                                            & $1.22 \pm 0.23$                   &
    $1.22 \pm 0.23$                                                       & $0.82 \pm 0.19$
                                                                          & $0.87 \pm 0.18$                   & $0.80 \pm 0.19$ & $0.81 \pm 0.18$
    \\
    fusiform                                                              & $1.66 \pm 0.17$                   &
    $1.71 \pm 0.16$                                                       & $1.41 \pm 0.18$
                                                                          & $1.43 \pm 0.19$                   & $1.33 \pm 0.18$ & $1.37 \pm 0.20$
    \\
    inferiorparietal                                                      & $1.81 \pm 0.15$                   &
    $1.82 \pm 0.13$                                                       & $1.53 \pm 0.18$
                                                                          & $1.59 \pm 0.20$                   & $1.50 \pm 0.17$ & $1.56 \pm 0.17$
    \\
    inferiortemporal                                                      & $1.66 \pm 0.17$                   &
    $1.70 \pm 0.16$                                                       & $1.37 \pm 0.25$
                                                                          & $1.38 \pm 0.21$                   & $1.37 \pm 0.23$ & $1.41 \pm 0.19$
    \\
    isthmuscingulate                                                      & $1.46 \pm 0.12$                   &
    $1.43 \pm 0.13$                                                       & $1.27 \pm 0.15$
                                                                          & $1.24 \pm 0.15$                   & $1.27 \pm 0.14$ & $1.27 \pm 0.15$
    \\
    lateraloccipital                                                      & $1.75 \pm 0.18$                   &
    $1.77 \pm 0.17$                                                       & $1.58 \pm 0.15$
                                                                          & $1.57 \pm 0.16$                   & $1.49 \pm 0.16$ & $1.50 \pm 0.15$
    \\
    lateralorbitofrontal                                                  & $1.65 \pm 0.17$                   &
    $1.51 \pm 0.15$                                                       & $1.44 \pm 0.23$
                                                                          & $0.95 \pm 0.13$                   & $1.51 \pm 0.16$ & $1.12 \pm 0.14$
    \\
    lingual                                                               & $1.54 \pm 0.22$                   &
    $1.52 \pm 0.21$                                                       & $1.47 \pm 0.18$
                                                                          & $1.46 \pm 0.17$                   & $1.50 \pm 0.18$ & $1.49 \pm 0.18$
    \\
    medialorbitofrontal                                                   & $1.50 \pm 0.15$                   &
    $1.53 \pm 0.15$                                                       & $1.09 \pm 0.16$
                                                                          & $1.15 \pm 0.14$                   & $1.15 \pm 0.17$ & $1.21 \pm 0.13$
    \\
    middletemporal                                                        & $1.74 \pm 0.16$                   &
    $1.81 \pm 0.14$                                                       & $1.42 \pm 0.23$
                                                                          & $1.52 \pm 0.19$                   & $1.44 \pm 0.21$ & $1.55 \pm 0.18$
    \\
    parahippocampal                                                       & $1.54 \pm 0.14$                   &
    $1.56 \pm 0.12$                                                       & $1.13 \pm 0.13$
                                                                          & $1.09 \pm 0.13$                   & $1.11 \pm 0.13$ & $1.07 \pm 0.13$
    \\
    paracentral                                                           & $1.59 \pm 0.22$                   &
    $1.60 \pm 0.22$                                                       & $1.40 \pm 0.17$
                                                                          & $1.40 \pm 0.19$                   & $1.36 \pm 0.18$ & $1.36 \pm 0.20$
    \\
    parsopercularis                                                       & $1.74 \pm 0.17$                   &
    $1.71 \pm 0.16$                                                       & $1.38 \pm 0.19$
                                                                          & $1.30 \pm 0.18$                   & $1.38 \pm 0.19$ & $1.30 \pm 0.20$
    \\
    parsorbitalis                                                         & $1.53 \pm 0.20$                   &
    $1.51 \pm 0.20$                                                       & $1.21 \pm 0.14$
                                                                          & $1.21 \pm 0.18$                   & $1.19 \pm 0.16$ & $1.22 \pm 0.18$
    \\
    parstriangularis                                                      & $1.68 \pm 0.17$                   &
    $1.63 \pm 0.19$                                                       & $1.33 \pm 0.16$
                                                                          & $1.30 \pm 0.22$                   & $1.30 \pm 0.16$ & $1.28 \pm 0.21$
    \\
    pericalcarine                                                         & $1.33 \pm 0.21$                   &
    $1.30 \pm 0.22$                                                       & $1.23 \pm 0.20$
                                                                          & $1.21 \pm 0.22$                   & $1.18 \pm 0.17$ & $1.18 \pm 0.17$
    \\
    postcentral                                                           & $1.84 \pm 0.24$                   &
    $1.81 \pm 0.26$                                                       & $1.68 \pm 0.23$
                                                                          & $1.69 \pm 0.28$                   & $1.64 \pm 0.20$ & $1.63 \pm 0.24$
    \\
    posteriorcingulate                                                    & $1.57 \pm 0.13$                   &
    $1.56 \pm 0.14$                                                       & $1.37 \pm 0.20$
                                                                          & $1.35 \pm 0.21$                   & $1.39 \pm 0.19$ & $1.39 \pm 0.22$
    \\
    precentral                                                            & $1.79 \pm 0.26$                   &
    $1.76 \pm 0.28$                                                       & $1.71 \pm 0.24$
                                                                          & $1.64 \pm 0.27$                   & $1.72 \pm 0.22$ & $1.66 \pm 0.28$
    \\
    precuneus                                                             & $1.83 \pm 0.13$                   &
    $1.84 \pm 0.13$                                                       & $1.65 \pm 0.21$
                                                                          & $1.66 \pm 0.21$                   & $1.61 \pm 0.18$ & $1.62 \pm 0.19$
    \\
    rostralanteriorcingulate                                              & $1.34 \pm 0.14$                   &
    $1.39 \pm 0.15$                                                       & $1.00 \pm 0.16$
                                                                          & $1.07 \pm 0.17$                   & $1.11 \pm 0.19$ & $1.11 \pm 0.18$
    \\
    rostralmiddlefrontal                                                  & $1.77 \pm 0.19$                   &
    $1.74 \pm 0.19$                                                       & $1.44 \pm 0.24$
                                                                          & $1.41 \pm 0.28$                   & $1.49 \pm 0.21$ & $1.48 \pm 0.25$
    \\
    superiorfrontal                                                       & $1.87 \pm 0.17$                   &
    $1.85 \pm 0.18$                                                       & $1.61 \pm 0.23$
                                                                          & $1.56 \pm 0.27$                   & $1.64 \pm 0.21$ & $1.62 \pm 0.25$
    \\
    superiorparietal                                                      & $1.92 \pm 0.18$                   &
    $1.93 \pm 0.17$                                                       & $1.72 \pm 0.24$
                                                                          & $1.65 \pm 0.28$                   & $1.66 \pm 0.22$ & $1.60 \pm 0.26$
    \\
    superiortemporal                                                      & $1.83 \pm 0.17$                   &
    $1.85 \pm 0.15$                                                       & $1.57 \pm 0.22$
                                                                          & $1.58 \pm 0.18$                   & $1.52 \pm 0.21$ & $1.57 \pm 0.18$
    \\
    supramarginal                                                         & $1.83 \pm 0.16$                   &
    $1.85 \pm 0.15$                                                       & $1.57 \pm 0.22$
                                                                          & $1.59 \pm 0.26$                   & $1.56 \pm 0.20$ & $1.56 \pm 0.24$
    \\
    frontalpole                                                           & $1.26 \pm 0.23$                   &
    $1.23 \pm 0.20$                                                       & $0.94 \pm 0.11$
                                                                          & $0.91 \pm 0.11$                   & $0.88 \pm 0.17$ & $0.87 \pm 0.14$
    \\
    temporalpole                                                          & $1.24 \pm 0.26$                   &
    $1.28 \pm 0.25$                                                       & $0.94 \pm 0.16$
                                                                          & $0.99 \pm 0.19$                   & $0.86 \pm 0.20$ & $0.91 \pm 0.22$
    \\
    transversetemporal                                                    & $1.47 \pm 0.20$                   &
    $1.46 \pm 0.18$                                                       & $1.17 \pm 0.13$
                                                                          & $1.13 \pm 0.11$                   & $1.20 \pm 0.15$ & $1.15 \pm 0.13$
    \\
    insula                                                                & $1.47 \pm 0.16$                   &
    $1.42 \pm 0.14$                                                       & $1.13 \pm 0.18$
                                                                          & $1.00 \pm 0.18$                   & $1.29 \pm 0.16$ & $1.19 \pm 0.19$
    \\
\end{longtblr}

\begin{longtblr}[ caption={Within-subject standard-deviation averaged across all subjects for
                cortical metrics.}, label={tab:std-cortical}, ]{
        colspec={lcc|cc|cc}, width=\linewidth,
        row{even}={white,font=\footnotesize},
        row{odd}={gray9,font=\footnotesize}, rows = {rowsep=0pt},
        rowhead=2, row{1}={white,font=\bfseries}, row{2}={gray9}}
    \SetCell[c=1]{c}Region   & \SetCell[c=2]{c}{cortical thickness                                      \\
    (mm)}                    &                                     & \SetCell[c=2]{c}{surface area      \\
    ($\text{mm}^2$)}         &                                     &
    \SetCell[c=2]{c}{cortical volume                                                                    \\ ($\text{mm}^3$)} &
    \\
                             & lh                                  & rh                            & lh
                             & rh                                  & lh                            & rh
    \\
    \hline
    bankssts                 & $0.02 \pm 0.01$                     & $0.02 \pm
    0.01$                    & $\028.65 \pm \015.97$               & $\021.73
    \pm \0\08.68$            & $\077.25 \pm \037.44$               & $\059.87
        \pm \020.45$
    \\
    caudalanteriorcingulate  & $0.04 \pm 0.01$                     & $0.04 \pm
    0.01$                    & $\019.98 \pm \013.83$               & $\021.01
    \pm \014.96$             & $\051.33 \pm \037.32$               & $\051.67
        \pm \041.74$
    \\
    caudalmiddlefrontal      & $0.02 \pm 0.01$                     & $0.02 \pm
    0.01$                    & $\038.58 \pm \036.77$               & $\046.65
    \pm \044.68$             & $104.41 \pm 108.02$                 & $124.11 \pm
    112.10$                                                                                             \\
    cuneus                   & $0.02 \pm 0.01$                     & $0.02 \pm
    0.01$                    & $\028.45 \pm \011.50$               & $\031.25
    \pm \015.67$             & $\060.72 \pm \025.52$               & $\074.77
        \pm \034.16$
    \\
    entorhinal               & $0.08 \pm 0.05$                     & $0.08 \pm
    0.05$                    & $\027.41 \pm \016.67$               & $\022.37
    \pm \011.70$             & $125.48 \pm \071.07$                & $115.94 \pm
    \057.21$                                                                                            \\
    fusiform                 & $0.02 \pm 0.01$                     & $0.02 \pm
    0.01$                    & $\050.70 \pm \025.16$               & $\047.86
    \pm \028.19$             & $182.92 \pm \092.31$                & $170.22 \pm
    103.05$                                                                                             \\
    inferiorparietal         & $0.01 \pm 0.01$                     & $0.01 \pm
    0.01$                    & $\053.01 \pm \029.19$               & $\059.90
    \pm \050.62$             & $145.66 \pm \072.95$                & $159.55 \pm
    110.14$                                                                                             \\
    inferiortemporal         & $0.02 \pm 0.01$                     & $0.02 \pm
    0.01$                    & $\064.73 \pm \042.27$               & $\058.75
    \pm \034.04$             & $198.15 \pm 127.44$                 & $168.38 \pm
    \084.67$                                                                                            \\
    isthmuscingulate         & $0.03 \pm 0.01$                     & $0.03 \pm
    0.01$                    & $\023.74 \pm \011.07$               & $\023.35
    \pm \013.99$             & $\057.43 \pm \029.59$               & $\053.05
        \pm \034.34$
    \\
    lateraloccipital         & $0.02 \pm 0.01$                     & $0.02 \pm
    0.01$                    & $\053.82 \pm \024.63$               & $\056.35
    \pm \028.61$             & $156.83 \pm \066.16$                & $160.98 \pm
    \076.00$                                                                                            \\
    lateralorbitofrontal     & $0.02 \pm 0.01$                     & $0.03 \pm
    0.01$                    & $\043.31 \pm \030.16$               & $117.14 \pm
    \033.75$                 & $\092.60 \pm \056.29$               & $217.89 \pm
    \069.06$                                                                                            \\
    lingual                  & $0.03 \pm 0.01$                     & $0.03 \pm
    0.01$                    & $\044.26 \pm \022.65$               & $\046.73
    \pm \023.96$             & $\089.19 \pm \046.24$               & $\095.82
        \pm \049.65$
    \\
    medialorbitofrontal      & $0.03 \pm 0.01$                     & $0.03 \pm
    0.01$                    & $\066.04 \pm \024.11$               & $\058.06
    \pm \019.00$             & $147.37 \pm \057.84$                & $134.52 \pm
    \042.26$                                                                                            \\
    middletemporal           & $0.02 \pm 0.01$                     & $0.02 \pm
    0.01$                    & $\053.01 \pm \034.97$               & $\044.87
    \pm \028.36$             & $165.49 \pm 108.52$                 & $135.26 \pm
    \077.98$                                                                                            \\
    parahippocampal          & $0.03 \pm 0.01$                     & $0.03 \pm
    0.01$                    & $\019.55 \pm \0\08.42$              & $\020.45
    \pm \0\07.81$            & $\064.22 \pm \025.29$               & $\065.43
        \pm \024.59$
    \\
    paracentral              & $0.03 \pm 0.02$                     & $0.03 \pm
    0.01$                    & $\022.94 \pm \012.98$               & $\026.94
    \pm \019.80$             & $\063.71 \pm \040.74$               & $\073.88
        \pm \056.66$
    \\
    parsopercularis          & $0.02 \pm 0.01$                     & $0.02 \pm
    0.01$                    & $\028.65 \pm \028.77$               & $\029.46
    \pm \026.82$             & $\080.67 \pm \092.87$               & $\082.38
        \pm \089.16$
    \\
    parsorbitalis            & $0.03 \pm 0.02$                     & $0.03 \pm
    0.02$                    & $\017.82 \pm \0\09.77$              & $\021.41
    \pm \010.66$             & $\060.63 \pm \045.20$               & $\068.18
        \pm \036.64$
    \\
    parstriangularis         & $0.02 \pm 0.01$                     & $0.02 \pm
    0.01$                    & $\025.67 \pm \014.65$               & $\034.86
    \pm \037.79$             & $\071.73 \pm \045.49$               & $\096.87
        \pm 102.22$
    \\
    pericalcarine            & $0.03 \pm 0.02$                     & $0.04 \pm
    0.02$                    & $\036.04 \pm \020.18$               & $\042.02
    \pm \024.82$             & $\059.64 \pm \029.98$               & $\068.61
        \pm \034.89$
    \\
    postcentral              & $0.01 \pm 0.02$                     & $0.02 \pm
    0.02$                    & $\043.47 \pm \067.12$               & $\045.98
    \pm \083.10$             & $100.26 \pm 121.35$                 & $104.53 \pm
    156.51$                                                                                             \\
    posteriorcingulate       & $0.02 \pm 0.01$                     & $0.02 \pm
    0.01$                    & $\021.93 \pm \013.05$               & $\024.39
    \pm \019.52$             & $\052.42 \pm \033.33$               & $\056.27
        \pm \052.59$
    \\
    precentral               & $0.02 \pm 0.02$                     & $0.02 \pm
    0.02$                    & $\046.92 \pm \053.54$               & $\057.46
    \pm \070.35$             & $118.04 \pm 157.21$                 & $148.21 \pm
    233.10$                                                                                             \\
    precuneus                & $0.01 \pm 0.01$                     & $0.01 \pm
    0.00$                    & $\038.04 \pm \042.87$               & $\038.95
    \pm \040.96$             & $100.91 \pm 111.15$                 & $102.24 \pm
    \096.62$                                                                                            \\
    rostralanteriorcingulate & $0.05 \pm 0.02$                     & $0.04 \pm
    0.02$                    & $\034.80 \pm \015.03$               & $\022.00
    \pm \010.59$             & $\081.04 \pm \041.59$               & $\061.95
        \pm \033.93$
    \\
    rostralmiddlefrontal     & $0.02 \pm 0.01$                     & $0.02 \pm
    0.01$                    & $\092.87 \pm \096.23$               & $108.40 \pm
    132.97$                  & $213.81 \pm 259.58$                 & $252.00 \pm
    358.20$                                                                                             \\
    superiorfrontal          & $0.01 \pm 0.01$                     & $0.01 \pm
    0.01$                    & $\085.23 \pm \086.47$               & $\098.14
    \pm 120.75$              & $223.91 \pm 234.89$                 & $243.75 \pm
    304.56$                                                                                             \\
    superiorparietal         & $0.01 \pm 0.01$                     & $0.01 \pm
    0.01$                    & $\049.49 \pm \080.81$               & $\062.89
    \pm \096.86$             & $132.77 \pm 207.97$                 & $161.39 \pm
    235.01$                                                                                             \\
    superiortemporal         & $0.02 \pm 0.01$                     & $0.01 \pm
    0.01$                    & $\047.70 \pm \033.64$               & $\041.38
    \pm \023.84$             & $156.30 \pm 101.85$                 & $129.01 \pm
    \078.70$                                                                                            \\
    supramarginal            & $0.01 \pm 0.01$                     & $0.01 \pm
    0.01$                    & $\050.87 \pm \058.82$               & $\050.06
    \pm \083.24$             & $136.23 \pm 168.28$                 & $133.99 \pm
    207.69$                                                                                             \\
    frontalpole              & $0.07 \pm 0.04$                     & $0.07 \pm
    0.04$                    & $\012.99 \pm \0\04.02$              & $\016.42
    \pm \0\04.47$            & $\056.49 \pm \032.17$               & $\067.84
        \pm \028.93$
    \\
    temporalpole             & $0.09 \pm 0.05$                     & $0.08 \pm
    0.05$                    & $\025.08 \pm \010.71$               & $\022.16
    \pm \011.78$             & $154.60 \pm \079.32$                & $138.28 \pm
    \078.33$                                                                                            \\
    transversetemporal       & $0.03 \pm 0.02$                     & $0.03 \pm
    0.02$                    & $\012.73 \pm \0\05.33$              & $\0\09.98
    \pm \0\03.33$            & $\029.55 \pm \012.34$               & $\024.91
        \pm \0\08.79$
    \\
    insula                   & $0.04 \pm 0.02$                     & $0.04 \pm
    0.01$                    & $\073.45 \pm \030.66$               & $\095.70
    \pm \037.63$             & $146.49 \pm \064.11$                & $183.39 \pm
    \081.47$                                                                                            \\
\end{longtblr}

\begin{longtblr}[ caption={Within-subject significant digits averaged across all
                subjects for subcortical volumes.},
        label={tab:sig-std-subcortical-volume},]{ colspec={lc|c},
        row{even}={gray9,font=\footnotesize},
        row{odd}={white,font=\footnotesize}, rows = {rowsep=0pt},
    row{Z}={font=\small}, rowhead=1, row{1}={font=\bfseries}} Region &
    Significant digits                                               & {Standard deviation                        \\ ($\text{mm}^3$)} \\
    \hline
    Left-Thalamus                                                    & $1.42 \pm 0.21$     & $120.08  \pm 69.61$  \\
    Left-Caudate                                                     & $1.57 \pm 0.20$     & $\038.83 \pm 25.11$  \\
    Left-Putamen                                                     & $1.49 \pm 0.22$     & $\065.88 \pm 46.39$  \\
    Left-Pallidum                                                    & $1.25 \pm 0.19$     & $\047.81 \pm 25.09$  \\
    Left-Hippocampus                                                 & $1.48 \pm 0.17$     & $\056.23 \pm 41.03$  \\
    Left-Amygdala                                                    & $1.13 \pm 0.16$     & $\048.71 \pm 20.04$  \\
    Left-Accumbens-area                                              & $0.88 \pm 0.16$     & $\024.20 \pm \08.80$ \\
    Right-Thalamus                                                   & $1.42 \pm 0.20$     & $118.92  \pm 68.76$  \\
    Right-Caudate                                                    & $1.51 \pm 0.24$     & $\049.37 \pm 42.71$  \\
    Right-Putamen                                                    & $1.51 \pm 0.25$     & $\068.07 \pm 70.23$  \\
    Right-Pallidum                                                   & $1.22 \pm 0.19$     & $\049.11 \pm 30.50$  \\
    Right-Hippocampus                                                & $1.55 \pm 0.18$     & $\048.59 \pm 28.98$  \\
    Right-Amygdala                                                   & $1.23 \pm 0.17$     & $\042.21 \pm 18.68$  \\
    Right-Accumbens-area                                             & $0.99 \pm 0.15$     & $\020.50 \pm \07.72$ \\
\end{longtblr}

\begin{table}[h]
    \centering
    \caption{Summary of executions failure and excluded subjects. To standardize
        the sample, we keep 26 repetitions per subject/visits pair.
        Subject/visit pairs with less than 26 repetitions were excluded which is
        12 subjects.}
    \begin{tabular}{l c c}
        \toprule
        \textbf{Stage}     & \textbf{Number of rejected repetitions} &
        \textbf{Total number of repetitions}                                 \\
        \midrule
        Cluster failure    & 1246 (5.80\%)                           & 21488 \\
        FreeSurfer failure & 68 (0.33\%)                             & 21488 \\
        QC failure         & 319 (1.48\%)                            & 21488 \\
        Total              & 1633 (7.60\%)                           & 21488 \\
        \bottomrule
    \end{tabular}
\end{table}

\begin{table}[h!]
    \centering
    \begin{tabular}{c|lccc}
        \toprule
        \textbf{Status} & \textbf{Cohort}             & \textbf{HC}
                        & \textbf{PD-non-MCI}         & \textbf{PD-MCI}
        \\
        \hline
        \multirow{5}{*}{\textbf{\shortstack{Before                                    \\QC}}} & n
                        & 106                         & 181                      & 29 \\
                        & Age (y)                     & $60.6 \pm 10.2   $
                        & $61.7 \pm \09.6$            & $67.7 \pm \07.7$
        \\
                        & Age range                   & $30.6 - 84.3  $
                        & $36.3 - 83.3$               & $49.9 - 80.5$
        \\
                        & Gender (male, \%)           & $58 \; (54.7\%)   $
                        & $119 \; (65.7\%)          $ & $-          $
        \\
                        & Education (y)               & $16.6 \pm \03.3  $
                        & $15.9 \pm \02.9$            & $-          $
        \\
        \hline
        \multirow{5}{*}{\textbf{\shortstack{After                                     \\QC}}} & n
                        & 103                         & 175                      & 27 \\
                        & Age (y)                     & $60.7 \pm 10.3   $
                        & $61.4 \pm \09.5          $  & $67.8 \pm \07.9$
        \\
                        & Age range                   & $30.6 - 84.3  $
                        & $36.3 - 79.9           $    & $49.9 - 80.5$
        \\
                        & Gender (male, \%)           & $57 \; (55.3\%)   $
                        & $114 \; (65.1\%)       $    & $20 \; (74.1\%) $
        \\
                        & Education (y)               & $16.6 \pm \03.3  $
                        & $15.9 \pm \02.9        $    & $15.0 \pm \03.5$
        \\
        \hline
        \multirow{8}{*}{\textbf{\shortstack{After                                     \\MCI\\exclusion}}} & n
                        & $103 $                      & $121                   $ & -- \\
                        & Age (y)                     & $60.7 \pm 10.3   $
                        & $60.7 \pm \09.1        $    & --
        \\
                        & Age range                   & $30.6 - 84.3  $
                        & $39.2 - 78.3           $    & --
        \\
                        & Gender (male, \%)           & $57 \; (55.3\%)   $
                        & $80 \; (66.1\%)        $    & --
        \\
                        & Education (y)               & $16.6 \pm \03.3  $
                        & $16.1 \pm \03.0        $    & --
        \\
                        & UPDRS III OFF baseline      & $-            $
                        & $23.4 \pm 10.1         $    & --
        \\
                        & UPDRS III OFF follow-up     & $-            $
                        & $25.8 \pm 11.1         $    & --
        \\
                        & Duration T2 - T1 (y)        & $\01.4 \pm \00.5 $
                        & $\01.4 \pm \00.7       $    & --
        \\
        \bottomrule
    \end{tabular}
    \vspace{1em}

    \textbf{Abbreviations:} MCI = Mild Cognitive Impairment; UPDRS = Unified
    Parkinson's Disease Rating Scale; PD = Parkinson's disease. Descriptive
    statistics before and after quality control (QC). Values are expressed as
    mean $\pm$ standard deviation. PD-non-MCI longitudinal sample is a subsample
    of the PD-non-MCI original sample that had longitudinal data and disease
    severity scores available.
    \label{tab:cohort_stat_vertical}
\end{table}

\section{Numerical-Anatomical Variability Ratio (\navr)}

\begin{figure}
    \centering
    \includegraphics[width=\linewidth]{figures/screenshot_light.png}
    \caption{Interactive web tool for estimating NAVR and assessing numerical
        variability in neuroimaging studies. Users can input summary statistics
        to obtain NAVR values and visualize the impact of numerical variability
        on effect size estimates. The tool is available at
        \href{https://yohanchatelain.github.io/brain\_render/}{yohanchatelain.github.io/brain\_render}.}
    \label{fig:brain_render_tool}
\end{figure}

\subsection{\navr maps}

Figures \ref{fig:navr_map_area} and \ref{fig:navr_map_volume} show the \navr
maps for cortical surface area and volume, respectively. The maps show the
average \navr values across all subjects for each cortical region. The color
scale indicates the \navr value, with warmer colors indicating higher \navr
values. The maps provide a visual representation of the variability in the
\navr values across different cortical regions, highlighting regions with
higher or lower \navr values.

The NAVR analysis reveals that numerical variability is a pervasive and
systematic source of uncertainty in neuroimaging, with within-subject
differences reaching up to $37-40\%$ of the observed between-subject anatomical
variance in key cortical and subcortical regions (Extended Data
Fig.~\ref{fig:navr_subcortical}, \ref{fig:navr_thickness}). Across anatomical
metrics, numerical uncertainty consistently accounts for a substantial fraction
of biological signal: cortical thickness measurements exhibit a mean NAVR of
0.21 (range: $0.11-0.37$), surface area 0.18 ($0.09-0.42$), and cortical volume
0.17 ($0.09-0.42$). Even subcortical volumes, considered more robust, show
meaningful variability (mean: 0.15, range: $0.01-0.27$). Notably, even the
lowest NAVR values ($0.01-0.11$) indicate that numerical noise is never
negligible, while maxima up to 0.42 suggest that in some regions, it can rival
or exceed half the anatomical variance. The narrow standard deviations across
metrics ($0.06-0.08$) further underscore the consistency of this phenomenon.
These findings carry serious implications: median NAVR values around
$0.16-0.20$ suggest that numerical imprecision can obscure subtle yet
clinically relevant effects or give rise to spurious associations, undermining
the reliability of neuroimaging-based inferences.

\begin{figure}[h]
    \centering
    \begin{subfigure}[b]{\linewidth}
        \centering
        \includegraphics[width=\linewidth]{figures/NAVR_map/NAVR_area_all.png}
        \caption{Cortical surface area}
        \label{fig:navr_map_area}
    \end{subfigure}
    \hfill
    \begin{subfigure}[b]{\linewidth}
        \centering
        \includegraphics[width=\linewidth]{figures/NAVR_map/NAVR_volume_all.png}
        \caption{Cortical volume}
        \label{fig:navr_map_volume}
    \end{subfigure}

    \vspace{0.5cm}
    \centering
    \includegraphics[width=0.8\linewidth]{figures/NAVR_map/jet_colorbar.pdf}

    \caption{Numerical-Anatomical Variability Ratio (\navr) for cortical surface
        area (Fig.~\ref{fig:navr_map_area}) and volume
        (Fig.~\ref{fig:navr_map_volume}) across regions and groups. Higher \navr
        values indicate greater computational uncertainty relative to biological
        variation. The color scale indicates the \navr value, with warmer colors
        indicating higher \navr values.}
    \label{fig:navr_maps}
\end{figure}

\subsection{Consistency results}

\YC{Moved from section~\ref{sec:results}. Rewrite to fit here.}

We instrumented FreeSurfer 7.3.1 with MCA (virtual precision set to mimic
realistic perturbations), and re-analyzed each T1-weighted MRI scan 26 times,
each time with a different random state. For subcortical volumes, statistical
significance ($p < 0.05$) fluctuated notably across MCA repetitions
(Figure~\ref{fig:significance_correlation_subcortical_volume}). For group
comparisons alone, eight subcortical regions alternated between significant and
non-significant outcomes, illustrating how numerical instability can directly
affect clinical interpretation. Overall, 27\% of all statistical comparisons,
including both group differences and clinical correlations, yielded
inconsistent results across numerical states. Similarly, for cortical
thickness, 19\% of the comparisons produced inconsistent results
(Fig.~\ref{fig:significance_correlation_thickness}), suggesting that potential
biomarker relationships might emerge or vanish purely due to computational
noise. This outcome-level variability mirrors the analytical variability
reported in large-scale reproducibility challenges like NARPS, where different
analysis teams reached conflicting conclusions from the same
dataset~\cite{botvinik2020variability}.

Testing for PD progression partial correlations and group differences between
PD patients and healthy controls, 27\% of statistical tests are
non-reproducible for subcortical volumes and 19\% for cortical thickness
(Extended Data Fig.~\ref{fig:significance_correlation_subcortical_volume}
and~\ref{fig:significance_correlation_thickness}). The within-subject
variability due to numerical differences peaked at 37\% and 40\% of the
observed between-subject anatomical variance in critical cortical and
subcortical regions (Fig.~\ref{fig:navr_subcortical}
and~\ref{fig:navr_thickness}). The median NAVR of approximately 0.20 and 0.16
across all cortical regions and subcortical regions underscores that numerical
uncertainty significantly affects typical neuroimaging analyses.

\begin{table}[htbp]
    \centering
    \caption{Proportion of Regions Fluctuating by Analysis Method}
    \label{tab:fluctuating_regions}
    \begin{tabular}{c|rr|rr}
        \hline
        \multirow{2}{*}{\textbf{Number of regions}} & \multicolumn{2}{c|}{\textbf{ANCOVA}} & \multicolumn{2}{c}{\textbf{Partial Correlation}}                                             \\
                                                    & \textbf{Baseline}                    & \textbf{Longitudinal}                            & \textbf{Baseline} & \textbf{Longitudinal} \\
        \hline
        14                                          & 4 (29\%)                             & 3 (21\%)                                         & 4 (29\%)          & 4 (29\%)              \\
        68                                          & 4 (6\%)                              & 12 (18\%)                                        & 19 (28\%)         & 16 (24\%)             \\
        68                                          & 20 (29\%)                            & 41 (60\%)                                        & 4 (6\%)           & 32 (47\%)             \\
        68                                          & 16 (24\%)                            & 25 (37\%)                                        & 11 (16\%)         & 26 (38\%)             \\
        \hline
    \end{tabular}
\end{table}

\begin{table*}[h]
    \centering
    \begin{tblr}[
            caption={ Ansari-Bradley Test Results for Subcortical Structures.
                    * indicates FDR-corrected
                    significance ($p < 0.05$). L/R = Left/Right. Stat = Statistic,
                    p-val = p-value, Partial Corr = Partial Correlation.},
            label={tab:stats-coef-var-subcortical},
        ]
        {
            width = \textwidth,
            colspec = {l | Q[c,m] Q[c,m] | Q[c,m] Q[c,m]},
            row{odd} = {gray9},
            row{even} = {white},
            row{1} = {font=\bfseries, white},
            row{2} = {font=\bfseries, gray9},
            rows = {rowsep=0pt},
            rowhead = 2,
        }
        \hline
        \SetCell[r=2]{m} Region & \SetCell[c=2]{c} ANCOVA &          & \SetCell[c=2]{c} Partial Corr &         \\
        \hline
                                & W                       & p        & W                             & p       \\
        \hline
        L-Thalamus              & 207                     & 1.00     & 463                           & 9.1e-6* \\
        L-Caudate               & 513                     & 1.1e-12* & 480                           & 2.0e-7* \\
        L-Putamen               & 343                     & 0.62     & 437                           & 6.8e-4* \\
        L-Pallidum              & 292                     & 0.99     & 441                           & 3.9e-4* \\
        L-Hippocampus           & 309                     & 0.94     & 384                           & 0.12    \\
        L-Amygdala              & 300                     & 0.97     & 419                           & 6.3e-3* \\
        L-Accumbens             & 226                     & 1.00     & 459                           & 2.0e-5* \\
        R-Thalamus              & 266                     & 1.00     & 487                           & 3.1e-8* \\
        R-Caudate               & 501                     & 3.1e-10* & 496                           & 1.9e-9* \\
        R-Putamen               & 326                     & 0.82     & 419                           & 6.3e-3* \\
        R-Pallidum              & 273                     & 1.00     & 432                           & 1.3e-3* \\
        R-Hippocampus           & 354                     & 0.46     & 462                           & 1.1e-5* \\
        R-Amygdala              & 376                     & 0.19     & 420                           & 5.6e-3* \\
        R-Accumbens             & 216                     & 1.00     & 430                           & 1.7e-3* \\
        \hline
    \end{tblr}
    \caption{ Ansari-Bradley Test Results for Subcortical Structures:
        ANCOVA vs Partial Correlation. * indicates FDR-corrected
        significance ($p < 0.05$). L/R = Left/Right. W = statistic,
        p = p-value, Partial Corr = Partial Correlation.}
    \label{tab:stats-coef-var-subcortical}
\end{table*}

\begin{longtblr}[
        caption={Ansari-Bradley Test Results for Cortical Regions: ANCOVA vs Partial
                Correlation. * indicates FDR-corrected significance ($p < 0.05$). lh/rh =
                left/right hemisphere. ACC = anterior cingulate cortex, MF = middle
                frontal. W = statistic, p = p-value, Partial Corr = Partial
                Correlation.},
        label={tab:stats-coef-var-cortical},
    ]{
        width=\linewidth,
        colspec = {l | *{4}{Q[c,m]} | *{4}{Q[c,m]} | *{4}{Q[c,m]}},
        row{odd} = {gray9},
        row{even} = {white},
        row{1} = {font=\bfseries, white},
        row{2} = {font=\bfseries, gray9},
        row{3} = {font=\bfseries, white},
        rows = {font=\footnotesize, rowsep=0pt},
        rowhead = 3,
    }
    \hline
    \SetCell[r=3]{m} Region   & \SetCell[c=4]{c} Cortical Volume &         &                               &          & \SetCell[c=4]{c} Cortical Thickness &          &                               &         & \SetCell[c=4]{c} Cortical Area &         &                               &          \\
    \hline
                              & \SetCell[c=2]{c} ANCOVA          &         & \SetCell[c=2]{c} Partial Corr &          & \SetCell[c=2]{c} ANCOVA             &          & \SetCell[c=2]{c} Partial Corr &         & \SetCell[c=2]{c} ANCOVA        &         & \SetCell[c=2]{c} Partial Corr &          \\
    \hline
                              & W                                & p       & W                             & p        & W                                   & p        & W                             & p       & W                              & p       & W                             & p        \\
    \hline
    bankssts (lh)             & 426                              & 2.8e-3* & 516                           & 1.5e-13* & 497                                 & 1.4e-9*  & 427                           & 2.5e-3* & 446                            & 1.8e-4* & 498                           & 9.5e-10* \\
    bankssts (rh)             & 382                              & 0.13    & 432                           & 1.3e-3*  & 479                                 & 2.6e-7*  & 457                           & 2.8e-5* & 453                            & 5.8e-5* & 457                           & 2.8e-5*  \\
    caudalACC (lh)            & 400                              & 3.8e-2  & 506                           & 4.0e-11* & 293                                 & 0.98     & 441                           & 3.9e-4* & 374                            & 0.21    & 507                           & 2.6e-11* \\
    caudalACC (rh)            & 240                              & 1.00    & 494                           & 3.7e-9*  & 505                                 & 6.2e-11* & 437                           & 6.8e-4* & 371                            & 0.24    & 477                           & 4.3e-7*  \\
    caudalMF (lh)             & 445                              & 2.1e-4* & 447                           & 1.6e-4*  & 470                                 & 2.1e-6*  & 367                           & 0.29    & 315                            & 0.91    & 469                           & 2.7e-6*  \\
    caudalMF (rh)             & 284                              & 0.99    & 440                           & 4.5e-4*  & 282                                 & 0.99     & 432                           & 1.3e-3* & 360                            & 0.36    & 439                           & 5.1e-4*  \\
    cuneus (lh)               & 389                              & 8.9e-2  & 480                           & 2.0e-7*  & 375                                 & 0.20     & 401                           & 3.4e-2  & 385                            & 0.12    & 488                           & 2.7e-8*  \\
    cuneus (rh)               & 341                              & 0.64    & 485                           & 6.8e-8*  & 362                                 & 0.33     & 409                           & 1.7e-2  & 327                            & 0.81    & 493                           & 4.8e-9*  \\
    entorhinal (lh)           & 375                              & 0.20    & 449                           & 1.1e-4*  & 348                                 & 0.56     & 438                           & 5.6e-4* & 404                            & 2.7e-2  & 463                           & 9.1e-6*  \\
    entorhinal (rh)           & 388                              & 9.7e-2  & 440                           & 4.5e-4*  & 366                                 & 0.30     & 416                           & 8.1e-3* & 376                            & 0.19    & 441                           & 3.9e-4*  \\
    fusiform (lh)             & 357                              & 0.41    & 462                           & 1.1e-5*  & 343                                 & 0.62     & 427                           & 2.5e-3* & 397                            & 4.6e-2  & 481                           & 1.8e-7*  \\
    fusiform (rh)             & 397                              & 4.6e-2  & 461                           & 1.3e-5*  & 386                                 & 0.10     & 438                           & 5.6e-4* & 410                            & 1.6e-2  & 475                           & 6.9e-7*  \\
    inferiorparietal (lh)     & 344                              & 0.61    & 451                           & 8.1e-5*  & 422                                 & 4.5e-3*  & 410                           & 1.6e-2  & 333                            & 0.75    & 475                           & 6.9e-7*  \\
    inferiorparietal (rh)     & 347                              & 0.58    & 453                           & 5.8e-5*  & 424                                 & 3.6e-3*  & 423                           & 4.2e-3* & 318                            & 0.89    & 469                           & 2.7e-6*  \\
    inferiortemporal (lh)     & 390                              & 8.4e-2  & 471                           & 1.8e-6*  & 358                                 & 0.39     & 432                           & 1.3e-3* & 396                            & 5.0e-2  & 495                           & 2.4e-9*  \\
    inferiortemporal (rh)     & 397                              & 4.6e-2  & 466                           & 5.3e-6*  & 394                                 & 5.9e-2   & 427                           & 2.5e-3* & 390                            & 8.4e-2  & 484                           & 9.1e-8*  \\
    isthmuscingulate (lh)     & 398                              & 4.1e-2  & 456                           & 3.4e-5*  & 436                                 & 7.9e-4*  & 426                           & 2.8e-3* & 385                            & 0.12    & 473                           & 1.1e-6*  \\
    isthmuscingulate (rh)     & 400                              & 3.8e-2  & 450                           & 9.7e-5*  & 427                                 & 2.5e-3*  & 430                           & 1.7e-3* & 392                            & 7.1e-2  & 461                           & 1.3e-5*  \\
    lateraloccipital (lh)     & 343                              & 0.62    & 463                           & 9.1e-6*  & 342                                 & 0.63     & 422                           & 4.5e-3* & 329                            & 0.79    & 481                           & 1.8e-7*  \\
    lateraloccipital (rh)     & 327                              & 0.81    & 470                           & 2.1e-6*  & 372                                 & 0.23     & 439                           & 5.1e-4* & 312                            & 0.92    & 488                           & 2.7e-8*  \\
    lateralorbitofrontal (lh) & 364                              & 0.31    & 436                           & 7.9e-4*  & 350                                 & 0.52     & 388                           & 9.7e-2  & 346                            & 0.59    & 444                           & 2.6e-4*  \\
    lateralorbitofrontal (rh) & 341                              & 0.64    & 441                           & 3.9e-4*  & 349                                 & 0.54     & 414                           & 1.0e-2* & 320                            & 0.87    & 449                           & 1.1e-4*  \\
    lingual (lh)              & 355                              & 0.44    & 460                           & 1.5e-5*  & 308                                 & 0.95     & 414                           & 1.0e-2* & 345                            & 0.60    & 474                           & 8.6e-7*  \\
    lingual (rh)              & 364                              & 0.31    & 467                           & 4.2e-6*  & 305                                 & 0.96     & 424                           & 3.6e-3* & 350                            & 0.52    & 484                           & 9.1e-8*  \\
    medialorbitofrontal (lh)  & 359                              & 0.38    & 442                           & 3.3e-4*  & 328                                 & 0.80     & 409                           & 1.7e-2  & 347                            & 0.58    & 452                           & 6.9e-5*  \\
    medialorbitofrontal (rh)  & 384                              & 0.12    & 444                           & 2.6e-4*  & 336                                 & 0.72     & 422                           & 4.5e-3* & 379                            & 0.16    & 453                           & 5.8e-5*  \\
    middletemporal (lh)       & 376                              & 0.19    & 456                           & 3.4e-5*  & 354                                 & 0.46     & 425                           & 3.3e-3* & 370                            & 0.25    & 479                           & 2.6e-7*  \\
    middletemporal (rh)       & 389                              & 8.9e-2  & 453                           & 5.8e-5*  & 373                                 & 0.22     & 435                           & 9.1e-4* & 382                            & 0.13    & 474                           & 8.6e-7*  \\
    parahippocampal (lh)      & 396                              & 5.0e-2  & 455                           & 4.1e-5*  & 380                                 & 0.15     & 439                           & 5.1e-4* & 404                            & 2.7e-2  & 471                           & 1.8e-6*  \\
    parahippocampal (rh)      & 402                              & 3.2e-2  & 446                           & 1.8e-4*  & 393                                 & 6.4e-2   & 426                           & 2.8e-3* & 391                            & 7.8e-2  & 458                           & 2.4e-5*  \\
    paracentral (lh)          & 359                              & 0.38    & 450                           & 9.7e-5*  & 327                                 & 0.81     & 402                           & 3.0e-2  & 344                            & 0.61    & 464                           & 8.1e-6*  \\
    paracentral (rh)          & 371                              & 0.24    & 456                           & 3.4e-5*  & 334                                 & 0.74     & 417                           & 7.8e-3* & 359                            & 0.38    & 469                           & 2.7e-6*  \\
    parsopercularis (lh)      & 380                              & 0.15    & 452                           & 6.9e-5*  & 402                                 & 3.2e-2   & 407                           & 2.1e-2  & 363                            & 0.32    & 468                           & 3.3e-6*  \\
    parsopercularis (rh)      & 366                              & 0.29    & 455                           & 4.1e-5*  & 418                                 & 7.4e-3*  & 426                           & 2.8e-3* & 350                            & 0.52    & 471                           & 1.8e-6*  \\
    parsorbitalis (lh)        & 374                              & 0.21    & 435                           & 9.1e-4*  & 352                                 & 0.49     & 398                           & 4.1e-2  & 359                            & 0.38    & 448                           & 1.3e-4*  \\
    parsorbitalis (rh)        & 382                              & 0.13    & 443                           & 2.9e-4*  & 379                                 & 0.16     & 414                           & 1.0e-2* & 364                            & 0.31    & 456                           & 3.4e-5*  \\
    parstriangularis (lh)     & 371                              & 0.24    & 449                           & 1.1e-4*  & 382                                 & 0.13     & 406                           & 2.3e-2  & 352                            & 0.49    & 462                           & 1.1e-5*  \\
    parstriangularis (rh)     & 371                              & 0.24    & 452                           & 6.9e-5*  & 398                                 & 4.1e-2   & 423                           & 4.2e-3* & 349                            & 0.54    & 465                           & 6.8e-6*  \\
    pericalcarine (lh)        & 367                              & 0.29    & 447                           & 1.6e-4*  & 323                                 & 0.85     & 410                           & 1.6e-2  & 356                            & 0.42    & 463                           & 9.1e-6*  \\
    pericalcarine (rh)        & 364                              & 0.31    & 457                           & 2.8e-5*  & 329                                 & 0.79     & 422                           & 4.5e-3* & 352                            & 0.49    & 472                           & 1.4e-6*  \\
    postcentral (lh)          & 354                              & 0.45    & 453                           & 5.8e-5*  & 341                                 & 0.64     & 407                           & 2.1e-2  & 341                            & 0.64    & 467                           & 4.2e-6*  \\
    postcentral (rh)          & 363                              & 0.32    & 456                           & 3.4e-5*  & 374                                 & 0.21     & 423                           & 4.2e-3* & 339                            & 0.68    & 471                           & 1.8e-6*  \\
    posteriorcingulate (lh)   & 393                              & 6.4e-2  & 461                           & 1.3e-5*  & 421                                 & 5.1e-3*  & 423                           & 4.2e-3* & 378                            & 0.17    & 476                           & 5.6e-7*  \\
    posteriorcingulate (rh)   & 397                              & 4.6e-2  & 458                           & 2.4e-5*  & 419                                 & 6.3e-3*  & 427                           & 2.5e-3* & 384                            & 0.12    & 473                           & 1.1e-6*  \\
    precentral (lh)           & 356                              & 0.42    & 453                           & 5.8e-5*  & 338                                 & 0.69     & 404                           & 2.7e-2  & 339                            & 0.68    & 465                           & 6.8e-6*  \\
    precentral (rh)           & 364                              & 0.31    & 455                           & 4.1e-5*  & 371                                 & 0.24     & 421                           & 5.1e-3* & 345                            & 0.60    & 468                           & 3.3e-6*  \\
    precuneus (lh)            & 378                              & 0.17    & 465                           & 6.8e-6*  & 389                                 & 8.9e-2   & 418                           & 7.4e-3* & 363                            & 0.32    & 478                           & 3.4e-7*  \\
    precuneus (rh)            & 388                              & 9.7e-2  & 466                           & 5.3e-6*  & 408                                 & 1.8e-2   & 426                           & 2.8e-3* & 373                            & 0.22    & 481                           & 1.8e-7*  \\
    rostralACC (lh)           & 374                              & 0.21    & 445                           & 2.1e-4*  & 343                                 & 0.62     & 410                           & 1.6e-2  & 368                            & 0.27    & 456                           & 3.4e-5*  \\
    rostralACC (rh)           & 383                              & 0.12    & 449                           & 1.1e-4*  & 355                                 & 0.44     & 418                           & 7.4e-3* & 377                            & 0.18    & 459                           & 2.0e-5*  \\
    rostralmiddlefrontal (lh) & 368                              & 0.27    & 449                           & 1.1e-4*  & 357                                 & 0.41     & 402                           & 3.0e-2  & 349                            & 0.54    & 461                           & 1.3e-5*  \\
    rostralmiddlefrontal (rh) & 372                              & 0.23    & 453                           & 5.8e-5*  & 373                                 & 0.22     & 417                           & 7.8e-3* & 354                            & 0.45    & 464                           & 8.1e-6*  \\
    superiorfrontal (lh)      & 367                              & 0.29    & 453                           & 5.8e-5*  & 358                                 & 0.39     & 406                           & 2.3e-2  & 349                            & 0.54    & 464                           & 8.1e-6*  \\
    superiorfrontal (rh)      & 375                              & 0.20    & 456                           & 3.4e-5*  & 384                                 & 0.12     & 419                           & 6.3e-3* & 356                            & 0.42    & 467                           & 4.2e-6*  \\
    superiorparietal (lh)     & 370                              & 0.25    & 459                           & 2.0e-5*  & 362                                 & 0.33     & 409                           & 1.7e-2  & 356                            & 0.42    & 474                           & 8.6e-7*  \\
    superiorparietal (rh)     & 375                              & 0.20    & 461                           & 1.3e-5*  & 382                                 & 0.13     & 418                           & 7.4e-3* & 361                            & 0.35    & 476                           & 5.6e-7*  \\
    superiortemporal (lh)     & 385                              & 0.12    & 458                           & 2.4e-5*  & 378                                 & 0.17     & 420                           & 5.6e-3* & 366                            & 0.29    & 471                           & 1.8e-6*  \\
    superiortemporal (rh)     & 394                              & 5.9e-2  & 456                           & 3.4e-5*  & 396                                 & 5.0e-2   & 425                           & 3.3e-3* & 374                            & 0.21    & 470                           & 2.1e-6*  \\
    supramarginal (lh)        & 371                              & 0.24    & 458                           & 2.4e-5*  & 387                                 & 9.5e-2   & 416                           & 8.1e-3* & 356                            & 0.42    & 472                           & 1.4e-6*  \\
    supramarginal (rh)        & 380                              & 0.15    & 462                           & 1.1e-5*  & 406                                 & 2.3e-2   & 424                           & 3.6e-3* & 362                            & 0.33    & 474                           & 8.6e-7*  \\
    frontalpole (lh)          & 376                              & 0.19    & 440                           & 4.5e-4*  & 337                                 & 0.70     & 409                           & 1.7e-2  & 363                            & 0.32    & 452                           & 6.9e-5*  \\
    frontalpole (rh)          & 387                              & 9.5e-2  & 444                           & 2.6e-4*  & 352                                 & 0.49     & 416                           & 8.1e-3* & 374                            & 0.21    & 456                           & 3.4e-5*  \\
    temporalpole (lh)         & 267                              & 1.00    & 424                           & 3.6e-3*  & 377                                 & 0.18     & 425                           & 3.3e-3* & 249                            & 1.00    & 410                           & 1.6e-2   \\
    temporalpole (rh)         & 237                              & 1.00    & 436                           & 7.9e-4*  & 416                                 & 8.6e-3*  & 429                           & 2.0e-3* & 242                            & 1.00    & 404                           & 2.7e-2   \\
    transversetemporal (lh)   & 427                              & 2.5e-3* & 459                           & 2.0e-5*  & 265                                 & 1.00     & 451                           & 8.1e-5* & 427                            & 2.5e-3* & 475                           & 6.9e-7*  \\
    transversetemporal (rh)   & 402                              & 3.2e-2  & 468                           & 3.3e-6*  & 303                                 & 0.96     & 429                           & 2.0e-3* & 417                            & 7.8e-3* & 462                           & 1.1e-5*  \\
    insula (lh)               & 310                              & 0.94    & 496                           & 1.9e-9*  & 329                                 & 0.79     & 449                           & 1.1e-4* & 237                            & 1.00    & 422                           & 4.5e-3*  \\
    insula (rh)               & 292                              & 0.99    & 438                           & 6.0e-4*  & 468                                 & 3.3e-6*  & 420                           & 5.6e-3* & 269                            & 1.00    & 413                           & 1.2e-2*  \\
    \hline
\end{longtblr}

\subsubsection{Consistency of statistical tests}

Figures \ref{fig:navr_consistency_area_plot} and
\ref{fig:consistency_volume_plot} show the consistency of statistical tests for
cortical area and volume, respectively, across all subjects and regions. The
plots show the percentage of subjects for which the statistical test was
significant ($\alpha = 0.05$) for each region. The consistency varies across
regions, with some regions showing higher consistency than others. The red
triangles indicate the IEEE-754 run for reference.

\begin{figure}[h]
    \centering
    \includegraphics[width=\linewidth]{figures/consistency/cortical_area_significance_correlation.pdf}
    \caption{Consistency of statistical tests for cortical area across all
        subjects and regions. The plot shows the percentage of subjects for
        which the statistical test was significant ($\alpha = 0.05$) for each
        region. The consistency varies across regions, with some regions showing
        higher consistency than others.}
    \label{fig:navr_consistency_area_plot}
\end{figure}

\begin{figure}[h]
    \centering
    \includegraphics[width=\linewidth]{figures/consistency/cortical_volume_significance_correlation.pdf}
    \caption{Consistency of statistical tests for cortical volume across all
        subjects and regions. The plot shows the percentage of subjects for
        which the statistical test was significant ($\alpha = 0.05$) for each
        region. The consistency varies across regions, with some regions showing
        higher consistency than others.}
    \label{fig:consistency_volume_plot}
\end{figure}

\subsubsection{Distribution of statistical tests coefficients}

Figures \ref{fig:consistency_area} and \ref{fig:consistency_volume} show the
distribution of partial correlation coefficients for cortical area and volume,
respectively, across all subjects and regions. The red triangles indicate the
IEEE-754 run for reference. The distribution shows the variability in the
coefficients, with some regions exhibiting higher consistency than others.

\begin{figure}
    \centering
    \begin{subfigure}[b]{\linewidth}
        \includegraphics[width=.9\linewidth]{figures/consistency/cortical_thickness_coefficients_distribution-Left.pdf}
        \caption{Left hemisphere}
        \label{fig:consistency_thickness_left}
    \end{subfigure}

    \begin{subfigure}[b]{\linewidth}
        \includegraphics[width=.9\linewidth]{figures/consistency/cortical_thickness_coefficients_distribution-Right.pdf}
        \caption{Right hemisphere}
        \label{fig:consistency_thickness_right}
    \end{subfigure}
    \caption{Distribution of partial correlation coefficients for cortical
        thickness across all subjects and regions. Red triangles indicate the
        IEEE-754 run for reference. }
    \label{fig:consistency_thickness_coefficients}
\end{figure}

\begin{figure}[h]
    \centering
    \begin{subfigure}[b]{\linewidth}
        \includegraphics[width=\linewidth]{figures/consistency/cortical_area_coefficients_distribution-Left.pdf}
        \caption{Left hemisphere}
        \label{fig:consistency_area_left}
    \end{subfigure}
    \hfill
    \begin{subfigure}[b]{\linewidth}
        \includegraphics[width=\linewidth]{figures/consistency/cortical_area_coefficients_distribution-Right.pdf}
        \caption{Right hemisphere}
        \label{fig:consistency_area_right}
    \end{subfigure}
    \caption{ Distribution of partial correlation coefficients for cortical surface area
        across all subjects and regions. Red triangles indicate the IEEE-754 run
        for reference. }
    \label{fig:consistency_area_coefficients}
\end{figure}

\begin{figure}[h]
    \centering
    \begin{subfigure}[b]{\linewidth}
        \includegraphics[width=\linewidth]{figures/consistency/cortical_volume_coefficients_distribution-Left.pdf}
        \caption{Left hemisphere}
        \label{fig:consistency_volume_left}
    \end{subfigure}
    \hfill
    \begin{subfigure}[b]{\linewidth}
        \includegraphics[width=\linewidth]{figures/consistency/cortical_volume_coefficients_distribution-Right.pdf}
        \caption{Right hemisphere}
        \label{fig:consistency_volume_right}
    \end{subfigure}
    \caption{ Distribution of partial correlation coefficients for cortical
        volume across all subjects and regions. Red triangles indicate the
        IEEE-754 run for reference. The distribution shows the variability in
        the coefficients, with some regions exhibiting higher consistency than
        others.}
    \label{fig:consistency_volume_coefficients}
\end{figure}

\subsubsection{Thresholding existing Cohen's d values from the literature}

We applied a thresholding approach to the Cohen's d values reported in the
literature to identify the most relevant findings for our analysis. This
involved setting a minimum effect size threshold, below which results were
considered non-significant or uninformative. The threshold was determined based
on the distribution of Cohen's d values across studies, with a focus on
retaining only those effects that were robust and consistent.

\begin{figure}[h]
    \centering
    \vspace{0.2cm}

    % Header row with column labels
    \begin{minipage}[b]{\linewidth}
        \begin{minipage}[c]{0.05\linewidth}
            % Empty space for alignment with condition labels
        \end{minipage}%
        \begin{minipage}[c]{0.95\linewidth}
            \begin{minipage}[c]{0.47\linewidth}
                \centering\textbf{Unthresholded}
            \end{minipage}%
            \hfill
            \begin{minipage}[c]{0.005\linewidth}
                % Vertical line separator
            \end{minipage}%
            \hfill
            \begin{minipage}[c]{0.47\linewidth}
                \centering\textbf{Thresholded}
            \end{minipage}
        \end{minipage}
    \end{minipage}

    % Horizontal line
    \noindent\rule{\linewidth}{0.5pt}
    \vspace{-1.5cm}

    % 22q11.2 deletion syndrome row
    \begin{minipage}[b]{\linewidth}
        \begin{minipage}[c]{0.05\linewidth}
            \centering\rotatebox{90}{\textbf{22q11.2}} \end{minipage}%
        \begin{minipage}[c]{0.95\linewidth}
            \begin{subfigure}[c]{0.47\linewidth}
                \includegraphics[width=\linewidth]{figures/cohen_d_map/enigma/22q_area_all.png}
                \label{fig:enigma_22q_unthresholded}
            \end{subfigure}%
            \hfill
            \begin{minipage}[c]{0.005\linewidth}
                \centering\rule{0.5pt}{4cm}
            \end{minipage}%
            \hfill
            \begin{subfigure}[c]{0.47\linewidth}
                \includegraphics[width=\linewidth]{figures/cohen_d_map/enigma/22q_area_all_thresholded.png}
                \label{fig:enigma_22q_thresholded}
            \end{subfigure}
        \end{minipage}
    \end{minipage}

    \vspace{-2cm}

    % ADHD row
    \begin{minipage}[b]{\linewidth}
        \begin{minipage}[c]{0.05\linewidth}
            \centering\rotatebox{90}{\textbf{ADHD}} \end{minipage}%
        \begin{minipage}[c]{0.95\linewidth}
            \begin{subfigure}[c]{0.47\linewidth}
                \includegraphics[width=\linewidth]{figures/cohen_d_map/enigma/adhd_area_adult.png}
                \label{fig:enigma_adhd_unthresholded}
            \end{subfigure}%
            \hfill
            \begin{minipage}[c]{0.005\linewidth}
                \centering\rule{0.5pt}{4cm}
            \end{minipage}%
            \hfill
            \begin{subfigure}[c]{0.47\linewidth}
                \includegraphics[width=\linewidth]{figures/cohen_d_map/enigma/adhd_area_adult_thresholded.png}
                \label{fig:enigma_adhd_thresholded}
            \end{subfigure}
        \end{minipage}
    \end{minipage}

    \vspace{-2cm}

    % bipolar disorder row
    \begin{minipage}[b]{\linewidth}
        \begin{minipage}[c]{0.05\linewidth}
            \centering\rotatebox{90}{\textbf{Bipolar}} \end{minipage}%
        \begin{minipage}[c]{0.95\linewidth}
            \begin{subfigure}[c]{0.47\linewidth}
                \includegraphics[width=\linewidth]{figures/cohen_d_map/enigma/bipolar_area_adult.png}
                \label{fig:enigma_bipolar_unthresholded}
            \end{subfigure}%
            \hfill
            \begin{minipage}[c]{0.005\linewidth}
                \centering\rule{0.5pt}{4cm}
            \end{minipage}%
            \hfill
            \begin{subfigure}[c]{0.47\linewidth}
                \includegraphics[width=\linewidth]{figures/cohen_d_map/enigma/bipolar_area_adult_thresholded.png}
                \label{fig:enigma_bipolar_thresholded}
            \end{subfigure}
        \end{minipage}
    \end{minipage}

    \vspace{-2cm}
    % depression
    \begin{minipage}[b]{\linewidth}
        \begin{minipage}[c]{0.05\linewidth}
            \centering\rotatebox{90}{\textbf{Depression}} \end{minipage}%
        \begin{minipage}[c]{0.95\linewidth}
            \begin{subfigure}[c]{0.47\linewidth}
                \includegraphics[width=\linewidth]{figures/cohen_d_map/enigma/depression_area_adult.png}
                \label{fig:enigma_depression_unthresholded}
            \end{subfigure}%
            \hfill
            \begin{minipage}[c]{0.005\linewidth}
                \centering\rule{0.5pt}{4cm}
            \end{minipage}%
            \hfill
            \begin{subfigure}[c]{0.47\linewidth}
                \includegraphics[width=\linewidth]{figures/cohen_d_map/enigma/depression_area_adult_thresholded.png}
                \label{fig:enigma_depression_thresholded}
            \end{subfigure}
        \end{minipage}
    \end{minipage}

    \vspace{-2cm}
    % ocd 
    \begin{minipage}[b]{\linewidth}
        \begin{minipage}[c]{0.05\linewidth}
            \centering\rotatebox{90}{\textbf{OCD}} \end{minipage}%
        \begin{minipage}[c]{0.95\linewidth}
            \begin{subfigure}[c]{0.47\linewidth}
                \includegraphics[width=\linewidth]{figures/cohen_d_map/enigma/ocd_area_adult.png}
                \label{fig:enigma_ocd_unthresholded}
            \end{subfigure}%
            \hfill
            \begin{minipage}[c]{0.005\linewidth}
                \centering\rule{0.5pt}{4cm}
            \end{minipage}%
            \hfill
            \begin{subfigure}[c]{0.47\linewidth}
                \includegraphics[width=\linewidth]{figures/cohen_d_map/enigma/ocd_area_adult_thresholded.png}
                \label{fig:enigma_ocd_thresholded}
            \end{subfigure}
        \end{minipage}
    \end{minipage}

    \vspace{-2cm}
    % schizophrenia
    \begin{minipage}[b]{\linewidth}
        \begin{minipage}[c]{0.05\linewidth}
            \centering\rotatebox{90}{\textbf{Schizophrenia}} \end{minipage}%
        \begin{minipage}[c]{0.95\linewidth}
            \begin{subfigure}[c]{0.47\linewidth}
                \includegraphics[width=\linewidth]{figures/cohen_d_map/enigma/schizophrenia_area_all.png}
                \label{fig:enigma_schizophrenia_unthresholded}
            \end{subfigure}%
            \hfill
            \begin{minipage}[c]{0.005\linewidth}
                \centering\rule{0.5pt}{4cm}
            \end{minipage}%
            \hfill
            \begin{subfigure}[c]{0.47\linewidth}
                \includegraphics[width=\linewidth]{figures/cohen_d_map/enigma/schizophrenia_area_all_thresholded.png}
                \label{fig:enigma_schizophrenia_thresholded}
            \end{subfigure}
        \end{minipage}
    \end{minipage}

    \caption{ENIGMA cortical area Cohen's d maps showing unthresholded effect
        sizes (left) and effect sizes thresholded by the \navr framework (right)
        for different disorders. Black regions indicate areas where Cohen's d
        values fall below the numerical variability threshold, demonstrating
        regions where reported effect sizes may be unreliable due to
        computational uncertainty.}
    \label{fig:navr_enigma_area}
\end{figure}

\end{document}
