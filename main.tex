\documentclass{article}

% Language setting Replace `english' with e.g. `spanish' to change the document
% language
\usepackage[english]{babel}

% Set page size and margins Replace `letterpaper' with `a4paper' for UK/EU
% standard size
\usepackage[letterpaper,top=2cm,bottom=2cm,left=3cm,right=3cm,marginparwidth=1.75cm]{geometry}

% Useful packages
\usepackage{amsmath}
\usepackage[pdftex]{graphicx}
\usepackage[colorlinks=true, allcolors=blue]{hyperref}
\usepackage{todonotes}
\usepackage{longtable}
\usepackage{booktabs}
\usepackage{multirow}
\usepackage{subcaption}
\usepackage{xcolor}
\usepackage{tabularray}
\usepackage{xspace} % For \xspace command
\usepackage{tikz}
\usetikzlibrary{shadings,positioning,overlay-beamer-styles}
\usepackage{xifthen}

\newcommand{\colorbar}[6]% width,height,colors,label min,label max,label step
{   \begin{tikzpicture} \foreach \x [count=\c] in {#3}{ \xdef\numcolo{\c}}
    \pgfmathsetmacro{\piecewidth}{#1/(\numcolo-1)}
    \xdef\lowcolo{} \foreach \x [count=\c] in {#3} {   \ifthenelse{\c = 1} {} {
    \fill[left color=\lowcolo,right color=\x] ({(\c-2)*\piecewidth},0) rectangle
    ({(\c-1)*\piecewidth},#2); } \xdef\lowcolo{\x} } \draw (0,0) rectangle
    (#1,#2);
    \pgfmathsetmacro{\secondlabel}{#4+#6}
    \pgfmathsetmacro{\lastlabel}{#5+0.01}
    \pgfkeys{/pgf/number format/.cd,fixed,precision=2}
    \foreach \x in {#4,\secondlabel,...,\lastlabel} { \draw
    ({(\x-#4)/(#5-#4)*#1},0) -- ++ (0,-0.1) node[below=0.05,font=\tiny]
    {\pgfmathprintnumber{\x}}; }
    \end{tikzpicture}
}

\newcommand{\YC}[1]{\textcolor{red}{YC: #1}}

\newcommand{\0}{\mspace{9mu}}
\newcommand{\navr}[0]{$\nu_{\text{nav}}$\xspace}

\title{Numerical variability in structural MRI measurements on Parkinson's disease}

\author{Yohan Chatelain, Andrzej Sokołowski, Madeleine Sharp, Jean-Baptiste Poline, Tristan Glatard}

\begin{document}

\maketitle

\begin{abstract}
    Numerical variability, arising from floating-point rounding and computational
    effects, is rarely quantified in neuroimaging despite many biomarkers relying on
    subtle morphometric differences. We instrumented FreeSurfer, a widely used
    neuroimaging pipeline, with Monte Carlo Arithmetic to emulate numerical
    differences across computational environments, processing identical MRI scans
    from Parkinson's disease patients and controls multiple times. We introduce the
    Numerical-Anatomical Variability Ratio (NAVR), quantifying numerical variation
    within subjects relative to anatomical differences between subjects. In multiple
    cortical and subcortical regions, numerical variation reached nearly one-third
    of the anatomical signal, altering statistical conclusions about group
    differences and clinical correlations across repeated analyses. Applying
    NAVR-based thresholding to published ENIGMA consortium brain maps revealed that
    numerous small regional effects lie below the numerical noise threshold for
    typical study sizes. Quantifying numerical uncertainty prevents spurious
    biomarker claims and enhances reproducibility in computational neuroscience.
\end{abstract}

\section{Introduction}

Complex neuroimaging pipelines transform raw MRI data into putative biomarkers
such as cortical thickness and subcortical volume. Because disease-related
effects may be subtle, the reliability of these metrics depends critically on
pipeline chosen~\cite{bhagwat2021understanding,botvinik2020variability},
software versions~\cite{sokolowski2024impact,des2023reproducibility}, and
computational
environments~\cite{glatard2012virtual,gronenschild2012effects,des2023reproducibility,vila2024impact}.
However, numerical variability arising from software computations is rarely
quantified.

Floating-point arithmetic ruled by IEEE-754 norm~\cite{markstein2008new} can
cause ostensibly deterministic algorithms to yield subtly different outputs.
Whether such numerical noise significantly impacts biological conclusions
remains unclear. Preliminary studies suggest substantial
effects~\cite{kiar2020comparing,salari2021accurate,des2023reproducibility,chatelain2024numerical},
yet systematic quantification within clinical datasets is lacking.

Parkinson's disease (PD) represents an ideal context to investigate numerical
variability, as the identification of reliable structural biomarkers is highly
desirable for diagnosis, tracking disease progression, and developing targeted
therapies. Structural MRI is particularly attractive for biomarker discovery
due to its non-invasive nature, but subtle anatomical changes associated with
PD pose significant analytical challenges. Consequently, ensuring that
MRI-derived measures are robust to numerical variability is essential to the
successful translation of these biomarkers into clinical practice.

We instrumented FreeSurfer 7.3.1~\cite{fischl2012freesurfer} with Monte Carlo
arithmetic~\cite{parker1997monte}, reprocessing identical MRI scans of PD
patients and healthy controls across multiple perturbed numerical states. To
systematically quantify the impact, we introduce the Numerical-Anatomical
Variability Ratio (NAVR), directly comparing numerical variation within
individuals to anatomical differences between individuals.

We find that in several cortical and subcortical regions, NAVR approaches 0.3,
indicating that numerical noise can represent nearly one-third of the measured
biological variation. Consequently, statistical inferences regarding group
differences and clinical correlations can fluctuate significantly. Applying
NAVR-based thresholding to published ENIGMA~\cite{thompson2014enigma}
consortium brain maps reveals that numerous previously reported small regional
effects likely fall below computational noise thresholds.

Our findings demonstrate that numerical variability is a significant,
quantifiable source of uncertainty in neuroimaging studies. The NAVR framework
we propose offers researchers a practical approach to quantify and report
computational uncertainty, enhancing reproducibility and reliability in
clinical neuroscience.

\section{Impacts of numerical variability}

Numerical instability fundamentally arises from floating-point arithmetic—where
subtle rounding errors accumulate over complex calculations. By repeatedly
processing the same MRI scans under perturbed computational conditions, we
found significant variations directly impacting statistical conclusions. These
fluctuations pose a major concern for interpreting subtle anatomical changes in
PD, underscoring the need for robust numerical precision in biomarker
discovery.

Floating-point rounding errors, although individually small, can collectively
introduce substantial numerical variability. Monte Carlo Arithmetic (MCA)
systematically assesses this variability by perturbing floating-point
operations with random, zero-mean offsets that preserve mathematical
expectations. Repeatedly executing an analysis pipeline with MCA enables us to
estimate the variability of results that could arise from routine differences
across hardware, software libraries, or computational environments.

Parkinson's disease provides an ideal scenario for testing numerical stability,
as structural MRI-derived measurements such as cortical thickness and
subcortical volume hold significant promise as potential biomarkers for
clinical progression due to their non-invasive nature. Furthermore, the
availability of large multi-site datasets (e.g., PPMI) and established
region-of-interest analyses ensure PD is a thoroughly characterized context for
evaluating the impacts of numerical variability. Thus, PD allows us to directly
assess whether numerical noise is sufficiently large to obscure clinically
meaningful anatomical signals.

We instrumented FreeSurfer 7.3.1 with MCA (virtual precision set to 53 bits for
double precision, 24 bits for single precision), and re-analyzed each
T1-weighted MRI scan 26 times. Testing for PD progression partial correlations
and group differences between PD patients and healthy controls, 27\% of
statistical tests are non-reproducible for subcortical volumes and 19\% for
cortical thickness (Extended Data
Fig.~\ref{fig:significance_correlation_subcortical_volume}
and~\ref{fig:significance_correlation_thickness}). The within-subject
variability due to numerical differences peaked at 37\% and 40\% of the
observed between-subject anatomical variance in critical cortical and
subcortical regions (Extended Data Fig.~\ref{fig:navr_subcortical}
and~\ref{fig:navr_thickness}). The median NAVR of approximately 0.20 and 0.16
across all cortical regions and subcortical regions underscores that numerical
uncertainty significantly affects typical neuroimaging analyses.
% NAVR stats
% metric    | mean | std  | median | min  | max
% thickness | 0.21 | 0.06 | 0.20   | 0.11 | 0.37
% area      | 0.18 | 0.08 | 0.16   | 0.09 | 0.42 
% volume    | 0.17 | 0.08 | 0.15   | 0.09 | 0.42
% cortical  | 0.19 | 0.08 | 0.17   | 0.09 | 0.42
% subvolume | 0.15 | 0.07 | 0.16   | 0.01 | 0.27

\subsection{Numerical variability alters statistical inference}
% Comments
% - Extend the results section
% - First result on a page
%   - Writing for clinician audience with no expertise in computer science
%   - Writing for computer science audience with no expertise in neuroimaging  
% - Explain why MCA is important
% - Explain why figure significance_correlation_thickness is important
% - Explain why we choose Parkinson's disease MRI
%   - Explain the context of Parkinson's disease and its impact on neuroimaging research
%   - Cite Andrezj paper on Parkinson's disease MRI
%   - Use paper to replicate a representative of existing results and methodologie used for structural MRI analysis Parkinson's disease
% - refers to NARPS paper for the figures
% - Add more interpretation of the results
% - Get inspiration from the NARPS paper for the writing results section

To test the impact of numerical variability on statistical inference, we
performed group comparisons and partial correlations between regional cortical
thickness and subcortical volumes with clinical measures (UPDRS scores).
Figure~\ref{fig:significance_correlation_subcortical_volume} shows that
statistical significance between Parkinson's disease patients and controls for
subcortical volumes fluctuated notably across MCA replicates. Eight subcortical
regions alternated between significant and non-significant outcomes ($p <
    0.05$), illustrating how numerical instability directly affects clinical
interpretation. In total, 27\% of the comparisons yielded inconsistent results.
Similarly, statistical tests for cortical thickness varied substantially across
numerical states (Fig.~\ref{fig:significance_correlation_thickness}) with 19\%
of the comparisons yielding inconsistent results, suggesting that possible
biomarker relationships might emerge or vanish purely due to numerical noise.
This variability mirrors multi-team analytical differences seen in previous
neuroimaging reproducibility challenges (e.g.,
NARPS~\cite{botvinik2020variability}).

\begin{figure}
    \includegraphics[width=\linewidth]{figures/consistency/subcortical_volume_significance_correlation.pdf}
    \caption{ Proportion of significant tests ($p<0.05$) for subcortical volumes across 26 numerical perturbations.
        measures.\label{fig:significance_correlation_subcortical_volume}}
\end{figure}

\begin{figure}
    \centering
    \includegraphics[width=\linewidth]{figures/consistency/cortical_thickness_significance_correlation.pdf}
    \caption{Proportion of significant tests ($p<0.05$) for cortical thickness across 26 numerical perturbations.
        measures.\label{fig:significance_correlation_thickness}}
    \label{fig:navr_consistency_thickness_plot}
\end{figure}

When looking at the distribution of partial correlation coefficients and
F-statistics from ANCOVA analyses
(Fig.~\ref{fig:statstest_coefficients_distribution}), we observe that the
unperturbed IEEE-754 results (red markers) consistently lie within the range
sampled by MCA. This confirms that the observed variability represents
realistic computational scenarios rather than methodological artifacts. We can
also notice that the spread of the r coefficients and F-statistics is higher
for longitudinal analyses than for baseline analyses, indicating that numerical
variability has a more pronounced impact on longitudinal studies.

\begin{figure}
    \includegraphics[width=\linewidth]{figures/consistency/subcortical_volume_coefficients_distribution.pdf}
    \caption{ Distribution of partial correlation coefficients (r-values) and
        F-statistics from ANCOVA across MCA repetitions for subcortical volume
        measures. Red dots represent the IEEE-754 unperturbed results. The top
        row shows r-values, while the bottom row shows F-values. The left column
        represents baseline analysis, and the right column represents
        longitudinal analysis.\label{fig:statstest_coefficients_distribution}}
\end{figure}

% We first tested the statistical significance of group differences between
% Parkinson's disease (PD) patients and healthy controls (HC), and their
% correlations with clinical measures (UPDRS scores), across the 26 MCA
% repetitions. Similar to the NARPS study~\cite{botvinik2020variability} that
% highlighted the impact of analytical variability in neuroimaging, we observed
% substantial fluctuations in the results due to numerical variability. We used a
% similar layout to NARPS paper to report the results, with the proportion of
% statistically significant tests ($p < 0.05$) varied substantially for both
% cortical thickness (Figure~\ref{fig:significance_correlation_thickness}) and
% subcortical volumes
% (Figure~\ref{fig:significance_correlation_subcortical_volume}). For many brain
% regions, the significance of a finding was inconsistent, appearing in some
% computational runs but not others. Ratios near 0.5 indicate maximal uncertainty,
% where a reported result could be attributed to a lucky roll of the computational
% dice. This variability in statistical significance suggests that the conclusions
% drawn from these analyses are not robust and may be influenced by the specific
% numerical state of the computation. This is specific to brain regions, some
% regions showing high consistency across numerical states, while others exhibit
% substantial variability. This instability is particularly concerning when
% finding biomarkers for the Parkinson's disease progression, as it suggests that
% the biological signals of interest may be obscured by computational noise.
% Sokołowski et al.~\cite{sokolowski2024impact} highlighted similar issues when
% playing with different FreeSurfer versions, showing that the results can vary
% significantly between versions 5, 6 and 7. Biomarkers are crucial for
% understanding disease mechanisms and tracking progression. Having reliable
% biomarkers is essential to better understand the nature of the disease which
% remains not well understood~\cite{bloem2021parkinson}.

% This instability is also reflected in the effect sizes themselves. Partial
% correlation coefficients and F-statistics from ANCOVA analyses showed
% substantial spread around the unperturbed IEEE-754 results (red markers in
% Figure~\ref{fig:statstest_coefficients_distribution} and
% \ref{fig:navr_consistency_thickness}). This demonstrates that numerical
% variability affects not only the binary outcome of statistical significance but
% also the magnitude of the estimated effect, further complicating the
% interpretation of results. We can notice that the unperturbed results (red
% dots) are contained within the distribution of coefficients and F-statistics,
% which indicates that our methodology is sound, but the variability in the
% results highlights the need for caution in interpretation. In particular, when
% interpreting results for possible clinical applications, it is important to
% consider the potential impact of numerical variability on the robustness of
% findings. Yellow dots around red ones can be seen as many potential results
% that could be obtained by running the same analysis under computational
% environments.

\subsection{A framework to quantify the impact of numerical variability}
% Comments:
% - Build a tool that can be broadly applied to any neuroimaging
% - Analytical modeling of sigma_d
% - Explain why having a tool is important
% - Why having a tool fast is important
%   - Measuring numerical variability is a time-consuming process
%   - Analytical modeling of sigma_d allows applying the tool to any
%     neuroimaging papers, existing results.
% - Quality Control impact findings => a tool to find potentially unreliable
%   results
% - To be general, we developped an analytical model
% - Refers to the online tool on yohanchatelain.github.io/brain-render

Quantifying the influence of numerical variability is essential for
establishing the reliability of neuroimaging findings. While anatomical
variability reflects genuine differences across individuals, numerical
variability arises from the instabilities inherent in floating-point
computations and changes in the computational environment. To systematically
assess their relative impact, we introduce the Numerical-Anatomical Variability
Ratio (NAVR), defined as the ratio between numerical uncertainty
($\sigma_{\text{num}}$; Eq.\ref{eq:sigma_num}) and between-subject anatomical
variability ($\sigma_{\text{anat}}$; Eq.\ref{eq:sigma_anat}):
\[
    \nu_{\text{nav}} = \frac{\sigma_{\text{num}}}{\sigma_{\text{anat}}}
\]

NAVR provides a general, interpretable metric to evaluate the robustness of
neuroimaging measurements. Across cortical and subcortical regions, numerical
variability accounted for up to $37-40\%$ of the observed anatomical variance
(Fig.~\ref{fig:navr_subcortical},\ref{fig:navr_thickness}). Mean NAVR values
reached 0.21 for cortical thickness and 0.15 for subcortical volumes,
indicating that numerical imprecision constitutes a nontrivial portion of the
biological signal. Even in regions with low NAVR ($0.01-0.11$), numerical
uncertainty remains non-negligible, while the highest values (up to 0.42)
suggest near-parity with anatomical variability.

\begin{figure}[h]
    \centering
    \begin{subfigure}[b]{\linewidth}
        \centering
        \includegraphics[width=\linewidth]{figures/NAVR_map/NAVR_subcortical_volume_all.png}
        \subcaption{Subcortical volumes}
        \label{fig:navr_subcortical}
    \end{subfigure}

    \begin{subfigure}[b]{\linewidth}
        \centering
        \includegraphics[width=\linewidth]{figures/NAVR_map/NAVR_thickness_all.png}
        \subcaption{Cortical thickness}
        \label{fig:navr_thickness}
    \end{subfigure}

    \begin{subfigure}[b]{\linewidth}
        \centering
        \includegraphics[width=.8\linewidth]{figures/NAVR_map/jet_colorbar.pdf}
    \end{subfigure}

    \caption{Numerical-Anatomical Variability Ratio (\navr) for subcortical
        volumes~\ref{fig:navr_subcortical} and cortical
        thickness~\ref{fig:navr_thickness} across regions and groups. Higher
        \navr values indicate greater computational uncertainty relative to
        biological variation. The color scale indicates the \navr value, with
        warmer colors indicating higher \navr values.}
\end{figure}

To make NAVR broadly applicable, including to previously published studies, we
developed a closed-form analytical model of effect size variability.
Specifically, we linked NAVR to the uncertainty in Cohen's d coefficient ($d$;
Eq.~\ref{eq:cohen_d})—a standard measure of group difference—through a simple
expression:
\[
    \sigma_d = \frac{2}{\sqrt{N}} \text{\navr}
\]
where $N$ is the sample size. This relationship allows researchers to estimate
the numerical uncertainty in effect sizes from summary statistics alone,
without requiring expensive recomputation or perturbed reruns. For instance,
with a typical NAVR of 0.2, suppressing numerical contributions below 0.01 in
$\sigma_d$ would require at least 1500 participants
(Fig.~\ref{fig:sigma_d_contour}). This modeling framework is especially
valuable in settings where reproducibility audits or systematic reviews demand
fast, scalable evaluations.

\begin{figure}
    \includegraphics[width=\linewidth]{figures/sigma_d_contour.pdf}
    \caption{ Relationship between \navr and population sample size  \(N\) for
        predicting the uncertainty in Cohen's d effect size estimation. The
        contour lines represent different \navr values, showing how numerical
        variability scales with sample size. With a typical \navr value of 0.2,
        to maintain reliable effect size estimates $\sigma_d \leq 0.01$, the
        plot suggests to use $N \geq 1500$.\label{fig:sigma_d_contour}}
\end{figure}

Speed is critical because existing methods such as Monte Carlo Arithmetic
(MCA), though accurate, incur prohibitive overhead for complex pipelines like
FreeSurfer. Our analytical approach circumvents this barrier, enabling routine
assessments of numerical robustness at a fraction of the computational cost.
Moreover, NAVR exposes how numerical instability can serve as a proxy for data
quality: regions failing standard QC checks consistently exhibited higher NAVR,
suggesting that this metric could flag unreliable inputs automatically.

To facilitate adoption, we implemented this analytical framework as a public
web tool\footnote{\url{https://yohanchatelain.github.io/brain_render/}},
allowing researchers to evaluate the numerical stability of neuroimaging
findings interactively. By making these capabilities accessible, we provide a
general tool that can be applied retroactively to any neuroimaging
result—regardless of software or study design—thereby improving transparency,
aiding peer review, and supporting more trustworthy scientific inference.

\subsection{Re-evaluating landmark studies reveals widespread potential for unreliable effect sizes}

To assess the broader implications of numerical variability, we applied NAVR to
re-evaluate influential findings from the ENIGMA consortium, which has
substantially shaped our understanding of psychiatric and neurological
disorders. Applying NAVR-based thresholding to ENIGMA's published brain maps
identified multiple regions where reported effect sizes fell below the
computational noise floor, potentially calling into question their reliability.

Figures~\ref{fig:navr_enigma_thickness} and~\ref{fig:navr_enigma_subcortical}
illustrate the impact of applying NAVR thresholds to cortical thickness and
subcortical volume maps from ENIGMA. Regions rendered in black indicate areas
where reported effect sizes were smaller than numerical variability, suggesting
these findings should be interpreted with caution.

This observation highlights potential risks of overestimating small effects and
underscores the importance of systematically accounting for numerical
uncertainty in neuroimaging research. While ENIGMA's primary findings generally
remained robust due to large sample sizes, our analysis indicates that numerous
secondary, smaller-scale effects reported in the literature could be
compromised by numerical instability.

\begin{figure}[h]
    \centering
    \vspace{0.2cm}

    % Header row with column labels
    \begin{minipage}[b]{\linewidth}
        \begin{minipage}[c]{0.05\linewidth}
            % Empty space for alignment with condition labels
        \end{minipage}%
        \begin{minipage}[c]{0.95\linewidth}
            \begin{minipage}[c]{0.47\linewidth}
                \centering\textbf{Unthresholded}
            \end{minipage}%
            \hfill
            \begin{minipage}[c]{0.005\linewidth}
                % Vertical line separator
            \end{minipage}%
            \hfill
            \begin{minipage}[c]{0.47\linewidth}
                \centering\textbf{Thresholded}
            \end{minipage}
        \end{minipage}
    \end{minipage}

    % Horizontal line
    \noindent\rule{\linewidth}{0.5pt}
    \vspace{-1.5cm}

    % 22q11.2 deletion syndrome row
    \begin{minipage}[b]{\linewidth}
        \begin{minipage}[c]{0.05\linewidth}
            \centering\rotatebox{90}{\textbf{22q11.2}} \end{minipage}%
        \begin{minipage}[c]{0.95\linewidth}
            \begin{subfigure}[c]{0.47\linewidth}
                \includegraphics[width=\linewidth]{figures/cohen_d_map/enigma/22q_thickness_all.png}
                \label{fig:enigma_22q_thickness_unthresholded}
            \end{subfigure}%
            \hfill
            \begin{minipage}[c]{0.005\linewidth}
                \centering\rule{0.5pt}{4cm}
            \end{minipage}%
            \hfill
            \begin{subfigure}[c]{0.47\linewidth}
                \includegraphics[width=\linewidth]{figures/cohen_d_map/enigma/22q_thickness_all_thresholded.png}
                \label{fig:enigma_22q_thickness_thresholded}
            \end{subfigure}
        \end{minipage}
    \end{minipage}

    \vspace{-2cm}

    % ADHD row
    \begin{minipage}[b]{\linewidth}
        \begin{minipage}[c]{0.05\linewidth}
            \centering\rotatebox{90}{\textbf{ADHD}} \end{minipage}%
        \begin{minipage}[c]{0.95\linewidth}
            \begin{subfigure}[c]{0.47\linewidth}
                \includegraphics[width=\linewidth]{figures/cohen_d_map/enigma/adhd_thickness_adult.png}
                \label{fig:enigma_adhd_thickness_unthresholded}
            \end{subfigure}%
            \hfill
            \begin{minipage}[c]{0.005\linewidth}
                \centering\rule{0.5pt}{4cm}
            \end{minipage}%
            \hfill
            \begin{subfigure}[c]{0.47\linewidth}
                \includegraphics[width=\linewidth]{figures/cohen_d_map/enigma/adhd_thickness_adult_thresholded.png}
                \label{fig:enigma_adhd_thickness_thresholded}
            \end{subfigure}
        \end{minipage}
    \end{minipage}

    \vspace{-2cm}

    % Autism spectrum disorder row
    \begin{minipage}[b]{\linewidth}
        \begin{minipage}[c]{0.05\linewidth}
            \centering\rotatebox{90}{\textbf{ASD}} \end{minipage}%
        \begin{minipage}[c]{0.95\linewidth}
            \begin{subfigure}[c]{0.47\linewidth}
                \includegraphics[width=\linewidth]{figures/cohen_d_map/enigma/asd_thickness_meta_analysis.png}
                \label{fig:enigma_asd_thickness_unthresholded}
            \end{subfigure}%
            \hfill
            \begin{minipage}[c]{0.005\linewidth}
                \centering\rule{0.5pt}{4cm}
            \end{minipage}%
            \hfill
            \begin{subfigure}[c]{0.47\linewidth}
                \includegraphics[width=\linewidth]{figures/cohen_d_map/enigma/asd_thickness_meta_analysis_thresholded.png}
                \label{fig:enigma_asd_thickness_thresholded}
            \end{subfigure}
        \end{minipage}
    \end{minipage}
    \vspace{-2cm}

    % bipolar disorder row
    \begin{minipage}[b]{\linewidth}
        \begin{minipage}[c]{0.05\linewidth}
            \centering\rotatebox{90}{\textbf{Bipolar}} \end{minipage}%
        \begin{minipage}[c]{0.95\linewidth}
            \begin{subfigure}[c]{0.47\linewidth}
                \includegraphics[width=\linewidth]{figures/cohen_d_map/enigma/bipolar_thickness_adult.png}
                \label{fig:enigma_bipolar_thickness_unthresholded}
            \end{subfigure}%
            \hfill
            \begin{minipage}[c]{0.005\linewidth}
                \centering\rule{0.5pt}{4cm}
            \end{minipage}%
            \hfill
            \begin{subfigure}[c]{0.47\linewidth}
                \includegraphics[width=\linewidth]{figures/cohen_d_map/enigma/bipolar_thickness_adult_thresholded.png}
                \label{fig:enigma_bipolar_thickness_thresholded}
            \end{subfigure}
        \end{minipage}
    \end{minipage}

    \vspace{-2cm}
    % depression
    \begin{minipage}[b]{\linewidth}
        \begin{minipage}[c]{0.05\linewidth}
            \centering\rotatebox{90}{\textbf{Depression}} \end{minipage}%
        \begin{minipage}[c]{0.95\linewidth}
            \begin{subfigure}[c]{0.47\linewidth}
                \includegraphics[width=\linewidth]{figures/cohen_d_map/enigma/depression_thickness_adult.png}
                \label{fig:enigma_depression_thickness_unthresholded}
            \end{subfigure}%
            \hfill
            \begin{minipage}[c]{0.005\linewidth}
                \centering\rule{0.5pt}{4cm}
            \end{minipage}%
            \hfill
            \begin{subfigure}[c]{0.47\linewidth}
                \includegraphics[width=\linewidth]{figures/cohen_d_map/enigma/depression_thickness_adult_thresholded.png}
                \label{fig:enigma_depression_thickness_thresholded}
            \end{subfigure}
        \end{minipage}
    \end{minipage}

    \vspace{-2cm}
    % epilepsy
    \begin{minipage}[b]{\linewidth}
        \begin{minipage}[c]{0.05\linewidth}
            \centering\rotatebox{90}{\textbf{Epilepsy}} \end{minipage}%
        \begin{minipage}[c]{0.95\linewidth}
            \begin{subfigure}[c]{0.47\linewidth}
                \includegraphics[width=\linewidth]{figures/cohen_d_map/enigma/epilepsy_thickness_allepilepsy.png}
                \label{fig:enigma_epilepsy_thickness_unthresholded}
            \end{subfigure}%
            \hfill
            \begin{minipage}[c]{0.005\linewidth}
                \centering\rule{0.5pt}{4cm}
            \end{minipage}%
            \hfill
            \begin{subfigure}[c]{0.47\linewidth}
                \includegraphics[width=\linewidth]{figures/cohen_d_map/enigma/epilepsy_thickness_allepilepsy_thresholded.png}
                \label{fig:enigma_epilepsy_thickness_thresholded}
            \end{subfigure}
        \end{minipage}
    \end{minipage}

    \vspace{-2cm}
    % ocd 
    \begin{minipage}[b]{\linewidth}
        \begin{minipage}[c]{0.05\linewidth}
            \centering\rotatebox{90}{\textbf{OCD}} \end{minipage}%
        \begin{minipage}[c]{0.95\linewidth}
            \begin{subfigure}[c]{0.47\linewidth}
                \includegraphics[width=\linewidth]{figures/cohen_d_map/enigma/ocd_thickness_adult.png}
                \label{fig:enigma_ocd_thickness_unthresholded}
            \end{subfigure}%
            \hfill
            \begin{minipage}[c]{0.005\linewidth}
                \centering\rule{0.5pt}{4cm}
            \end{minipage}%
            \hfill
            \begin{subfigure}[c]{0.47\linewidth}
                \includegraphics[width=\linewidth]{figures/cohen_d_map/enigma/ocd_thickness_adult_thresholded.png}
                \label{fig:enigma_ocd_thickness_thresholded}
            \end{subfigure}
        \end{minipage}
    \end{minipage}

    \vspace{-2cm}
    % schizophrenia
    \begin{minipage}[b]{\linewidth}
        \begin{minipage}[c]{0.05\linewidth}
            \centering\rotatebox{90}{\textbf{Schizophrenia}} \end{minipage}%
        \begin{minipage}[c]{0.95\linewidth}
            \begin{subfigure}[c]{0.47\linewidth}
                \includegraphics[width=\linewidth]{figures/cohen_d_map/enigma/schizophrenia_thickness_all.png}
                \label{fig:enigma_schizophrenia_thickness_unthresholded}
            \end{subfigure}%
            \hfill
            \begin{minipage}[c]{0.005\linewidth}
                \centering\rule{0.5pt}{4cm}
            \end{minipage}%
            \hfill
            \begin{subfigure}[c]{0.47\linewidth}
                \includegraphics[width=\linewidth]{figures/cohen_d_map/enigma/schizophrenia_thickness_all_thresholded.png}
                \label{fig:enigma_schizophrenia_thickness_thresholded}
            \end{subfigure}
        \end{minipage}
    \end{minipage}

    \caption{ENIGMA cortical thickness Cohen's d maps showing unthresholded
        effect sizes (left) and effect sizes thresholded by the \navr framework
        (right) for different disorders. Black regions indicate areas where
        Cohen's d values fall below the numerical variability threshold,
        demonstrating regions where reported effect sizes may be unreliable due
        to computational uncertainty.\label{fig:navr_enigma_thickness}}
\end{figure}

\begin{figure}[h]
    \centering
    \vspace{0.2cm}
    % Header row with column labels
    \begin{minipage}[b]{\linewidth}
        \begin{minipage}[c]{0.05\linewidth}
            % Empty space for alignment with condition labels
        \end{minipage}%
        \begin{minipage}[c]{0.95\linewidth}
            \begin{minipage}[c]{0.47\linewidth}
                \centering\textbf{Unthresholded}
            \end{minipage}%
            \hfill
            \begin{minipage}[c]{0.005\linewidth}
                % Vertical line separator
            \end{minipage}%
            \hfill
            \begin{minipage}[c]{0.47\linewidth}
                \centering\textbf{Thresholded}
            \end{minipage}
        \end{minipage}
    \end{minipage}

    % Horizontal line
    \noindent\rule{\linewidth}{0.5pt}
    \vspace{-1.5cm}

    % 22q11.2 deletion syndrome row
    \begin{minipage}[b]{\linewidth}
        \begin{minipage}[c]{0.05\linewidth}
            \centering\rotatebox{90}{\textbf{22q11.2}} \end{minipage}%
        \begin{minipage}[c]{0.95\linewidth}
            \begin{subfigure}[c]{0.47\linewidth}
                \includegraphics[width=\linewidth]{figures/cohen_d_map/enigma/22q_subcortical_volume_all.png}
                \label{fig:enigma_22q_unthresholded_subcortical}
            \end{subfigure}%
            \hfill
            \begin{minipage}[c]{0.005\linewidth}
                \centering\rule{0.5pt}{4cm}
            \end{minipage}%
            \hfill
            \begin{subfigure}[c]{0.47\linewidth}
                \includegraphics[width=\linewidth]{figures/cohen_d_map/enigma/22q_subcortical_volume_all_thresholded.png}
                \label{fig:enigma_22q_thresholded_subcortical}
            \end{subfigure}
        \end{minipage}
    \end{minipage}

    \vspace{-2cm}

    % ADHD row
    \begin{minipage}[b]{\linewidth}
        \begin{minipage}[c]{0.05\linewidth}
            \centering\rotatebox{90}{\textbf{ADHD}} \end{minipage}%
        \begin{minipage}[c]{0.95\linewidth}
            \begin{subfigure}[c]{0.47\linewidth}
                \includegraphics[width=\linewidth]{figures/cohen_d_map/enigma/adhd_subcortical_volume_adult.png}
                \label{fig:enigma_adhd_unthresholded_subcortical}
            \end{subfigure}%
            \hfill
            \begin{minipage}[c]{0.005\linewidth}
                \centering\rule{0.5pt}{4cm}
            \end{minipage}%
            \hfill
            \begin{subfigure}[c]{0.47\linewidth}
                \includegraphics[width=\linewidth]{figures/cohen_d_map/enigma/adhd_subcortical_volume_adult_thresholded.png}
                \label{fig:enigma_adhd_thresholded_subcortical}
            \end{subfigure}
        \end{minipage}
    \end{minipage}

    \vspace{-2cm}

    % Autism spectrum disorder row
    \begin{minipage}[b]{\linewidth}
        \begin{minipage}[c]{0.05\linewidth}
            \centering\rotatebox{90}{\textbf{ASD}} \end{minipage}%
        \begin{minipage}[c]{0.95\linewidth}
            \begin{subfigure}[c]{0.47\linewidth}
                \includegraphics[width=\linewidth]{figures/cohen_d_map/enigma/asd_subcortical_volume_meta_analysis.png}
                \label{fig:enigma_asd_unthresholded_subcortical}
            \end{subfigure}%
            \hfill
            \begin{minipage}[c]{0.005\linewidth}
                \centering\rule{0.5pt}{4cm}
            \end{minipage}%
            \hfill
            \begin{subfigure}[c]{0.47\linewidth}
                \includegraphics[width=\linewidth]{figures/cohen_d_map/enigma/asd_subcortical_volume_meta_analysis_thresholded.png}
                \label{fig:enigma_asd_thresholded_subcortical}
            \end{subfigure}
        \end{minipage}
    \end{minipage}
    \vspace{-2cm}

    % bipolar disorder row
    \begin{minipage}[b]{\linewidth}
        \begin{minipage}[c]{0.05\linewidth}
            \centering\rotatebox{90}{\textbf{Bipolar}} \end{minipage}%
        \begin{minipage}[c]{0.95\linewidth}
            \begin{subfigure}[c]{0.47\linewidth}
                \includegraphics[width=\linewidth]{figures/cohen_d_map/enigma/bipolar_subcortical_volume_typeII.png}
                \label{fig:enigma_bipolar_unthresholded_subcortical}
            \end{subfigure}%
            \hfill
            \begin{minipage}[c]{0.005\linewidth}
                \centering\rule{0.5pt}{4cm}
            \end{minipage}%
            \hfill
            \begin{subfigure}[c]{0.47\linewidth}
                \includegraphics[width=\linewidth]{figures/cohen_d_map/enigma/bipolar_subcortical_volume_typeII_thresholded.png}
                \label{fig:enigma_bipolar_thresholded_subcortical}
            \end{subfigure}
        \end{minipage}
    \end{minipage}

    \vspace{-2cm}
    % depression
    \begin{minipage}[b]{\linewidth}
        \begin{minipage}[c]{0.05\linewidth}
            \centering\rotatebox{90}{\textbf{Depression}} \end{minipage}%
        \begin{minipage}[c]{0.95\linewidth}
            \begin{subfigure}[c]{0.47\linewidth}
                \includegraphics[width=\linewidth]{figures/cohen_d_map/enigma/depression_subcortical_volume_all.png}
                \label{fig:enigma_depression_unthresholded_subcortical}
            \end{subfigure}%
            \hfill
            \begin{minipage}[c]{0.005\linewidth}
                \centering\rule{0.5pt}{4cm}
            \end{minipage}%
            \hfill
            \begin{subfigure}[c]{0.47\linewidth}
                \includegraphics[width=\linewidth]{figures/cohen_d_map/enigma/depression_subcortical_volume_all_thresholded.png}
                \label{fig:enigma_depression_thresholded_subcortical}
            \end{subfigure}
        \end{minipage}
    \end{minipage}

    \vspace{-2cm}
    % epilepsy
    \begin{minipage}[b]{\linewidth}
        \begin{minipage}[c]{0.05\linewidth}
            \centering\rotatebox{90}{\textbf{Epilepsy}} \end{minipage}%
        \begin{minipage}[c]{0.95\linewidth}
            \begin{subfigure}[c]{0.47\linewidth}
                \includegraphics[width=\linewidth]{figures/cohen_d_map/enigma/epilepsy_subcortical_volume_allepilepsy.png}
                \label{fig:enigma_epilepsy_unthresholded_subcortical}
            \end{subfigure}%
            \hfill
            \begin{minipage}[c]{0.005\linewidth}
                \centering\rule{0.5pt}{4cm}
            \end{minipage}%
            \hfill
            \begin{subfigure}[c]{0.47\linewidth}
                \includegraphics[width=\linewidth]{figures/cohen_d_map/enigma/epilepsy_subcortical_volume_allepilepsy_thresholded.png}
                \label{fig:enigma_epilepsy_thresholded_subcortical}
            \end{subfigure}
        \end{minipage}
    \end{minipage}

    \vspace{-2cm}
    % ocd 
    \begin{minipage}[b]{\linewidth}
        \begin{minipage}[c]{0.05\linewidth}
            \centering\rotatebox{90}{\textbf{OCD}} \end{minipage}%
        \begin{minipage}[c]{0.95\linewidth}
            \begin{subfigure}[c]{0.47\linewidth}
                \includegraphics[width=\linewidth]{figures/cohen_d_map/enigma/ocd_subcortical_volume_adult.png}
                \label{fig:enigma_ocd_unthresholded_subcortical}
            \end{subfigure}%
            \hfill
            \begin{minipage}[c]{0.005\linewidth}
                \centering\rule{0.5pt}{4cm}
            \end{minipage}%
            \hfill
            \begin{subfigure}[c]{0.47\linewidth}
                \includegraphics[width=\linewidth]{figures/cohen_d_map/enigma/ocd_subcortical_volume_adult_thresholded.png}
                \label{fig:enigma_ocd_thresholded_subcortical}
            \end{subfigure}
        \end{minipage}
    \end{minipage}

    \vspace{-2cm}
    % schizophrenia
    \begin{minipage}[b]{\linewidth}
        \begin{minipage}[c]{0.05\linewidth}
            \centering\rotatebox{90}{\textbf{Schizophrenia}} \end{minipage}%
        \begin{minipage}[c]{0.95\linewidth}
            \begin{subfigure}[c]{0.47\linewidth}
                \includegraphics[width=\linewidth]{figures/cohen_d_map/enigma/schizophrenia_subcortical_volume_all.png}
                \label{fig:enigma_schizophrenia_unthresholded_subcortical}
            \end{subfigure}%
            \hfill
            \begin{minipage}[c]{0.005\linewidth}
                \centering\rule{0.5pt}{4cm}
            \end{minipage}%
            \hfill
            \begin{subfigure}[c]{0.47\linewidth}
                \includegraphics[width=\linewidth]{figures/cohen_d_map/enigma/schizophrenia_subcortical_volume_all_thresholded.png}
                \label{fig:enigma_schizophrenia_thresholded_subcortical}
            \end{subfigure}
        \end{minipage}
    \end{minipage}

    \caption{ENIGMA subcortical volume Cohen's d maps showing unthresholded effect sizes (left) and effect sizes thresholded by the \navr framework (right) for different disorders. Black regions indicate areas where Cohen's d values fall below the numerical variability threshold, demonstrating regions where reported effect sizes may be unreliable due to computational uncertainty.}
    \label{fig:navr_enigma_subcortical}
\end{figure}

\section{Discussion}

% Comments:
% 1. Summarize the main results and what do they mean
% 2. Extension beyond FreeSurfer and expectation to generalize the findings to other neuroimaging software
% 3. Discuss the potential sources of numerical variability (minimal local, minimal precision, etc.)

Our systematic perturbation of FreeSurfer revealed that numerical variability
alone can account for up to 30\% of the anatomical variability observed in
structural MRI measurements. This level of uncertainty can significantly impact
statistical outcomes, leading to the appearance or disappearance of clinically
relevant group differences or correlations depending solely on computational
conditions. These findings offer a mechanistic explanation for some of the
reproducibility challenges reported in clinical neuroimaging.

To move beyond identifying the problem, we introduced the Numerical-Anatomical
Variability Ratio (NAVR), a quantitative framework for assessing the relative
magnitude of computational noise. By establishing a theoretical link between
NAVR and the uncertainty in Cohen's d effect sizes, we provide researchers with
a practical tool to assess the robustness of their findings. Our re-analysis of
published ENIGMA results illustrates this utility: while large sample sizes
confer robustness to core findings, many secondary effects fall below the
computational noise floor. This suggests that in smaller exploratory studies,
numerical instability may undermine the reliability of reported effects.

Although our primary analysis focused on FreeSurfer 7.3.1 and Parkinson's
disease, the underlying numerical issues are general. Floating-point arithmetic
is inherently non-associative and sensitive to compiler behavior, hardware
architecture, and thread scheduling. As a result, neuroimaging pipelines—though
deterministic in design—can produce divergent results across computational
environments. Preliminary analyses of FSL~\YC{cite Niusha} and ANTs~\YC{cite
    Mathieu} indicate that such instability is not unique to FreeSurfer, but likely
pervades the field. SPM however seems to be less impacted by numerical
variability. Moreover, our PD cohort was relatively homogeneous in age and
phenotype, potentially reducing anatomical variance and inflating NAVR values.
This highlights the need to apply NAVR across diverse datasets, software
packages, and disease contexts to build a comprehensive understanding of
computational reliability.

Crucially, this instability is not confined to low-level rounding operations.
Image processing workflows involve nonlinear optimization procedures that may
converge to different local minima under small perturbations, resulting in
substantive changes to derived measures. The situation is analogous to deep
learning, where different weight initializations or precision settings can
yield distinct model outcomes. In neuroimaging, such instability means that
even identical inputs can lead to divergent interpretations—raising serious
concerns for both research reproducibility and clinical translation.

NAVR offers a scalable and interpretable metric to quantify this hidden
variability. While floating-point rounding is a major source of instability,
future work should extend this analysis to other contributors, including
algorithmic decisions, preprocessing choices, and data handling practices. A
comprehensive understanding of these factors is essential for developing
numerically robust software tools.

In conclusion, our results demonstrate that computational uncertainty is as
critical as statistical uncertainty in neuroimaging. Incorporating systematic
assessments of numerical variability, through tools like NAVR, is necessary to
ensure the reproducibility and reliability of neuroimaging-based biomarkers.

\section{Methods}

\subsection{Participants}

We used structural MRI data from the Parkinson's Progression Markers Initiative
(PPMI). Participants included 125 Parkinson's disease patients without mild
cognitive impairment (PD-non-MCI) and 106 healthy controls, each providing
longitudinal T1-weighted MRI data across two visits. Patients with mild
cognitive impairment were excluded to reduce confounding influences.

T1-weighted images were drawn from the Parkinson's Progression Markers
Initiative (PPMI) database (www.ppmi-info.org). Inclusion required (i)
diagnosis of idiopathic Parkinson's disease (PD-non-MCI) or healthy control
(HC); (ii) two usable visits separated by $0.9-2.0$ years; and (iii) absence of
other neurological disorders. The final dataset comprised 90 healthy controls
and 118 PD-non-MCI participants (Extended Data Table 1). The study was approved
by the local research ethics boards of all contributing centres, and written
informed consent was obtained from every participant.

Inclusion criteria required: (1) primary PD diagnosis or healthy control
status, (2) availability of two visits with T1-weighted scans, and (3) absence
of other neurological diagnoses. PD severity was assessed using the Unified
Parkinson's Disease Rating Scale (UPDRS). The study received ethics approval
from participating institutions, and all participants provided written informed
consent (Table~\ref{tab:cohort_stat}).

PD and HC groups showed no significant age differences ($p > 0.05$) but
differed in education ($t = -2.05$, $p = 0.04$) and sex distribution ($\chi^2 =
    4.15$, $p = 0.04$). The longitudinal cohort showed no significant demographic
differences between groups (Table~\ref{tab:cohort_stat}).

\begin{table}[h!]
    \centering
    \begin{tabular}{lcc}
        \toprule
        \textbf{Cohort}         & \textbf{HC}        & \textbf{PD-non-MCI} \\
        \hline
        n                       & $90 $              & $118 $              \\
        Age (y)                 & $60.7 \pm 9.7 $    & $61.1 \pm \09.2 $   \\
        Age range               & $30.6 - 79.8 $     & $39.2 - 78.3 $      \\
        Gender (male, \%)       & $48 \; (53.3\%) $  & $77 \; (65.3\%) $
        \\
        Education (y)           & $16.7 \pm \03.3 $  & $16.2 \pm \02.9 $   \\
        UPDRS III OFF baseline  & $- $               & $23.6 \pm 10.3 $    \\
        UPDRS III OFF follow-up & $- $               & $25.6 \pm 11.2 $    \\
        Duration T2 - T1 (y)    & $\01.4 \pm \00.5 $ & $\01.4 \pm \00.6 $  \\
        \bottomrule
    \end{tabular}
    \vspace{1em}

    \caption{\textbf{Abbreviations:} MCI = Mild Cognitive Impairment; UPDRS =
        Unified Parkinson's Disease Rating Scale; PD = Parkinson's disease.
        Values are expressed as mean $\pm$ standard deviation. PD-non-MCI
        longitudinal sample is a subsample of the PD-non-MCI original sample
        that had longitudinal data and disease severity scores available.
        \label{tab:cohort_stat}}
\end{table}

\subsection{Image acquisition and preprocessing}

T1-weighted MRI scans from PPMI were acquired using standardized protocols
(repetition time=2.3 s, echo time=2.98 ms, inversion time=0.9 s, 1 mm isotropic
resolution, number of slices = 192, field of view = 256 mm, and matrix size =
256 $\times$ 256). However, since PPMI is a multisite project there may be
slight differences in the sites' setup. Images underwent standard preprocessing
using FreeSurfer 7.3.1 instrumented with Fuzzy-libm. Each participant's MRI
data were processed 26 times under different numerical perturbations to
quantify numerical variability. Failed runs were discarded, ensuring exactly 26
successful repetitions per subject.

Longitudinal processing followed the standard FreeSurfer
stream~\cite{reuter2012within}: cross-sectional processing of both timepoints,
followed by creation of an unbiased within-subject
template~\cite{reuter2011avoiding} using robust
registration~\cite{reuter2010highly}. Downstream analyses used unperturbed
FreeSurfer to prevent additional numerical perturbations.

\subsection{Numerical Variability Assessment}

We employed Monte Carlo Arithmetic (MCA)~\cite{parker1997monte} to quantify
numerical instability in FreeSurfer computations. MCA introduces controlled
random perturbations into floating-point operations, simulating rounding errors
that occur across different computational environments. This stochastic
approach enables systematic assessment of result stability by measuring
variation across multiple runs of identical analyses.

We used Fuzzy-libm~\cite{salari2021accurate}, which extends MCA to mathematical
library functions (\texttt{exp}, \texttt{log}, \texttt{sin}, \texttt{cos})
through Verificarlo~\cite{denis2016verificarlo}, an LLVM-based compiler.
Virtual precision parameters were set to 53 bits for double precision and 24
bits for single precision to simulate realistic machine-level precision errors.

We processed each visit with FreeSurfer 7.3.1. To sample numerical variability
we compiled FreeSurfer with Fuzzy-libm an implementation of Monte Carlo
arithmetic (MCA) that injects zero-mean rounding noise into every elementary
function call. Virtual precision was set to 53 bits for operations promoted to
double and 24 bits for single precision, thereby preserving IEEE-754
expectations but exposing the variance of alternative execution paths. Each
subject-visit pair was processed 26 times; failed or quality-control-flagged
runs were discarded, and exactly 26 successful runs per pair were retained for
analysis.

\subsubsection{Numerical-Anatomical Variability Ratio (\navr)}

To quantify computational stability relative to anatomical variation, we
developed the Numerical-Anatomical Variability Ratio (\navr). For each brain
region, \navr measures the ratio of measurement uncertainty arising from
computational processes to natural inter-subject anatomical variation:

\[
    \text{\navr} = \frac{\sigma_{\text{num}}}{\sigma_{\text{anat}}}
\]

where $\sigma_{\text{num}}$ represents numerical variability (measurement
precision across MCA repetitions for individual subjects) and
$\sigma_{\text{anat}}$ represents anatomical variability (inter-subject
differences within each repetition).

For each region of interest, measurements from $n$ MCA repetitions across $m$
subject-visit pairs form a data matrix $\mathcal{M}_{n \times m}$, where
element $x_{i,j}$ represents the measurement for subject $j$ in repetition $i$.

Numerical variability quantifies intra-subject measurement consistency:
\begin{equation}
    \sigma^2_{\text{num}} = \frac{1}{m} \sum_{j=1}^{m} \left[ \frac{1}{n-1} \sum_{i=1}^{n} (x_{i,j} - \bar{x}_{\cdot,j})^2 \right]
    \label{eq:sigma_num}
\end{equation}

Anatomical variability captures inter-subject differences:
\begin{equation}
    \sigma^2_{\text{anat}} = \frac{1}{n} \sum_{i=1}^{n} \left[ \frac{1}{m-1} \sum_{j=1}^{m} (x_{i,j} - \bar{x}_{i,\cdot})^2 \right]
    \label{eq:sigma_anat}
\end{equation}

where $\bar{x}_{\cdot,j}$ and $\bar{x}_{i,\cdot}$ denote column and row means,
respectively. Higher \navr values indicate regions where computational
uncertainty approaches or exceeds biological variation, potentially
compromising the detection of true anatomical differences.

\subsubsection{Relationship between \navr~and Effect Size Uncertainty}

We derived the theoretical relationship between \navr~and Cohen's d variability
to quantify how measurement uncertainty affects statistical effect sizes in
group comparisons. For a balanced two-group design with total sample size $N$,
each observation decomposes as $X_{ij} = \mu_i +
    \varepsilon_{ij}^{(\text{anat})} + \varepsilon_{ij}^{(\text{num})}$, where
$\mu_i$ represents the true group mean, $\varepsilon_{ij}^{(\text{anat})} \sim
    \mathcal{N}(0, \sigma_{\text{anat}}^2)$ captures anatomical variation, and
$\varepsilon_{ij}^{(\text{num})} \sim \mathcal{N}(0, \sigma_{\text{num}}^2)$
represents numerical uncertainty.

The standard deviation of Cohen's d attributable to measurement error is:
\begin{equation}
    \sigma_d = \frac{2}{\sqrt{N}} \text{\navr}
\end{equation}

This relationship emerges from error propagation analysis. The difference in
group means has variance $\text{Var}(\bar{X}_1 - \bar{X}_2) =
    4(\sigma_{\text{anat}}^2 + \sigma_{\text{num}}^2)/N$, with the numerical
component contributing $4\sigma_{\text{num}}^2/N$. Since Cohen's d normalizes
by the pooled standard deviation $\sqrt{\sigma_{\text{anat}}^2 +
        \sigma_{\text{num}}^2}$, the measurement error contribution becomes $\sigma_d =
    (2\sigma_{\text{num}}/\sqrt{N})/\sigma_{\text{anat}} = (2/\sqrt{N})
    \text{\navr}$.

This formula indicates that regions with \navr = 0.1 contribute approximately
$0.2/\sqrt{N}$ uncertainty to Cohen's d, while regions with \navr = 1.0
contribute $2/\sqrt{N}$ uncertainty. The relationship provides a direct link
between computational stability (\navr) and statistical reliability in
neuroimaging studies.

\section{Data Availability}
The data that support the findings of this study are available from the
Parkinson's Progression Markers Initiative (PPMI) database
(www.ppmi-info.org/access-data-specimens/download-data), but restrictions apply
to the availability of these data, which were used under license for the
current study, and so are not publicly available. Data are however available
from the authors upon reasonable request and with permission of the PPMI.

\section{Code Availability}

All MCA instrumentation scripts, FreeSurfer build instructions and analysis
notebooks are available at [GitHub URL to be inserted]. Exact commit hashes are
archived on Zenodo (DOI [to be added]) to ensure bit-level reproducibility.

\section{Acknowledgements}

The analyses were conducted on the Virtual Imaging
Platform~\cite{glatard2012virtual}, which utilizes resources provided by the
Biomed virtual organization within the European Grid Infrastructure (EGI). We
extend our gratitude to Sorina Pop from CREATIS, Lyon, France, for her support.

\bibliographystyle{plain}
\bibliography{main}

\clearpage

\appendix

\section{Formula}

\subsection{Significant digits formula}
\label{eq:significant_digits}

We compute the number of significant bits \(\hat{s}\) with probability
\(p_s=0.95\) and confidence \(1-\alpha_s=0.95\) using the
\texttt{significantdigits}
package\footnote{\url{https://github.com/verificarlo/significantdigits}}
(version 0.4.0). \texttt{significantdigits} implements the Centered Normality
Hypothesis approach described in~\cite{sohier2021confidence}:
\[
    \hat{s_i} = -\log_2 \left| \frac{\hat{\sigma_i}}{\hat{\mu_i}} \right| -
    \delta(n, \alpha_s, p_s),
\]
where \(\hat{\sigma_i}\) and \(\hat{\mu_i}\) are the average and standard
deviation over the repetitions, and
\begin{equation}
    \delta(n, \alpha_s, p_s) = \log_2 \left(
    \sqrt{\frac{n-1}{\chi^2_{1-\alpha_s/2}}} \Phi^{-1} \left( \frac{p_s+1}{2}
    \right) \right)
\end{equation}
is a penalty term for estimating \(\hat{s_i}\) with probability \(p_s\) and
confidence level \(1-\alpha_s\) for a sample size \(n\). \(\Phi^{-1}\) is the
inverse cumulative distribution of the standard normal distribution and
\(\chi^2\) is the Chi-2 distribution with \(n\)-1 degrees of freedom.

\subsection{Extended Sørensen-Dice coefficient}
\label{eq:extended_dice}

The extended Sørensen-Dice coefficient is a measure of overlap between multiple
sets, defined as follows:
\[
    \text{Dice}(A_1, A_2, \dots, A_n) = \frac{n \left| \bigcap_{i=1}^{n} A_i \right|}{\sum_{i=1}^{n} \left| A_i \right|}
\].

\subsection{Cohen's d}
\label{eq:cohen_d}

Cohen's d is a measure of effect size that quantifies the difference between
two groups in terms of standard deviations. It is defined as:
\[
    d = \frac{\bar{X}_1 - \bar{X}_2}{\sigma_{pooled}},\; \sigma_{pooled} =
    \sqrt{\frac{(n_1 - 1) s_1^2 + (n_2 - 1) s_2^2}{n_1 + n_2 - 2}}
\]
where \(\bar{X}_1\) and \(\bar{X}_2\) are the means of the two groups, \(s_1\)
and \(s_2\) are the standard deviations, and \(n_1\) and \(n_2\) are the sample
sizes of the two groups. The numerator represents the difference in means,
while the denominator is the pooled standard deviation, which accounts for the
variability within each group.

\section{Cross-sectional Analysis}

As a side result, the cross-sectional analysis measures the impact of numerical
variability in FreeSurfer version 7.3.1 on the PPMI (Parkinson's Progression
Markers Initiative) cohort. This involves comparing the estimation of
structural MRI measures, including cortical and subcortical volumes, cortical
thickness, and surface area. The goal is to assess the stability of these key
metrics and quantify the numerical variability.

FreeSurfer 7.3.1 showed limited numerical precision across all cortical
measures: $1.61 \pm 0.20$ significant digits for cortical thickness, $1.33 \pm
    0.23$ for surface area, and $1.33 \pm 0.23$ for cortical volume
(Figures~\ref{fig:sig_digits_cortical}). Subcortical volumes have a similar
precision with $1.33 \pm 0.22$ significant digits on average
(Figure~\ref{fig:sig_digits_subcortical}). These values indicate measurements
are typically precise to only one decimal place, with some instances showing
complete precision loss. Regional consistency was observed within each metric
type, with cortical thickness showing the highest precision (range: $1.22-1.93$
digits) compared to surface area ($0.82 - 1.72$ digits) and cortical volume
($0.80 - 1.72$ digits). Subcortical volumes exhibited the highest precision
(range: $0.88 - 1.57$ digits), with a mean of $1.33 \pm 0.22$ significant
digits.

To measure the structural overlap, we evaluated using the extended
Sørensen-Dice coefficient: Dice coefficients revealed substantial inter-subject
variability, particularly in temporal pole regions (Figure~\ref{fig:dice}). We
also observed that the Dice coefficient varies across regions, with some
regions showing higher variability than others with cortical volume ($0.00 -
    0.91$) with a mean of $0.75 \pm 0.11$ and subcortical volume ($0.18 - 0.94$)
with a mean of $0.82 \pm 0.08$. Finally, we noticed that subcortical volume
measurements are more stable than cortical volume.

\begin{figure}
    \includegraphics*[width=\linewidth]{figures/dice.pdf}
    \caption{Dice coefficient.\label{fig:dice}}

\end{figure}

\begin{figure}
    \includegraphics*[width=\linewidth]{figures/sig_digits.pdf}
    \caption{Number of significant digits for each cortical region and
        metric.\label{fig:sig_digits_cortical}}
\end{figure}

\begin{figure}
    \includegraphics*[width=\linewidth]{figures/sig_digits_subcortical_volume.pdf}
    \caption{Number of significant digits of subcortical volume for each
        subcortical region.\label{fig:sig_digits_subcortical}}
\end{figure}

\subsection{Within-subject significant digits averaged across all subjects}

\begin{longtblr}[ caption={Within-subject significant digits averaged across all subjects.},
        label={tab:sig-cortical},]{ colspec={lcc|cc|cc}, width=0.25\linewidth,
        row{even}={white,font=\footnotesize},
        row{odd}={gray9,font=\footnotesize}, rows = {rowsep=0pt}, rowhead=2,
    row{1}={white,font=\bfseries}, row{2}={gray9}} \SetCell[c=1]{c}Region &
    \SetCell[c=2]{c}{cortical thickness }                                 &                                   &
    \SetCell[c=2]{c}{surface area}                                        &
                                                                          & \SetCell[c=2]{c}{cortical volume} &
    \\
                                                                          & lh                                &
    rh                                                                    & lh
                                                                          & rh                                & lh
                                                                          & rh                                                                    \\
    \hline
    bankssts                                                              & $1.65 \pm 0.16$                   &
    $1.69 \pm 0.13$                                                       & $1.15 \pm 0.18$
                                                                          & $1.21 \pm 0.13$                   & $1.08 \pm 0.17$ & $1.14 \pm 0.13$
    \\
    caudalanteriorcingulate                                               & $1.38 \pm 0.14$                   &
    $1.40 \pm 0.14$                                                       & $1.14 \pm 0.22$
                                                                          & $1.19 \pm 0.18$                   & $1.14 \pm 0.24$ & $1.21 \pm 0.20$
    \\
    caudalmiddlefrontal                                                   & $1.77 \pm 0.18$                   &
    $1.77 \pm 0.19$                                                       & $1.40 \pm 0.21$
                                                                          & $1.31 \pm 0.23$                   & $1.40 \pm 0.22$ & $1.30 \pm 0.23$
    \\
    cuneus                                                                & $1.52 \pm 0.19$                   &
    $1.54 \pm 0.19$                                                       & $1.34 \pm 0.14$
                                                                          & $1.33 \pm 0.14$                   & $1.32 \pm 0.14$ & $1.28 \pm 0.15$
    \\
    entorhinal                                                            & $1.22 \pm 0.23$                   &
    $1.22 \pm 0.23$                                                       & $0.82 \pm 0.19$
                                                                          & $0.87 \pm 0.18$                   & $0.80 \pm 0.19$ & $0.81 \pm 0.18$
    \\
    fusiform                                                              & $1.66 \pm 0.17$                   &
    $1.71 \pm 0.16$                                                       & $1.41 \pm 0.18$
                                                                          & $1.43 \pm 0.19$                   & $1.33 \pm 0.18$ & $1.37 \pm 0.20$
    \\
    inferiorparietal                                                      & $1.81 \pm 0.15$                   &
    $1.82 \pm 0.13$                                                       & $1.53 \pm 0.18$
                                                                          & $1.59 \pm 0.20$                   & $1.50 \pm 0.17$ & $1.56 \pm 0.17$
    \\
    inferiortemporal                                                      & $1.66 \pm 0.17$                   &
    $1.70 \pm 0.16$                                                       & $1.37 \pm 0.25$
                                                                          & $1.38 \pm 0.21$                   & $1.37 \pm 0.23$ & $1.41 \pm 0.19$
    \\
    isthmuscingulate                                                      & $1.46 \pm 0.12$                   &
    $1.43 \pm 0.13$                                                       & $1.27 \pm 0.15$
                                                                          & $1.24 \pm 0.15$                   & $1.27 \pm 0.14$ & $1.27 \pm 0.15$
    \\
    lateraloccipital                                                      & $1.75 \pm 0.18$                   &
    $1.77 \pm 0.17$                                                       & $1.58 \pm 0.15$
                                                                          & $1.57 \pm 0.16$                   & $1.49 \pm 0.16$ & $1.50 \pm 0.15$
    \\
    lateralorbitofrontal                                                  & $1.65 \pm 0.17$                   &
    $1.51 \pm 0.15$                                                       & $1.44 \pm 0.23$
                                                                          & $0.95 \pm 0.13$                   & $1.51 \pm 0.16$ & $1.12 \pm 0.14$
    \\
    lingual                                                               & $1.54 \pm 0.22$                   &
    $1.52 \pm 0.21$                                                       & $1.47 \pm 0.18$
                                                                          & $1.46 \pm 0.17$                   & $1.50 \pm 0.18$ & $1.49 \pm 0.18$
    \\
    medialorbitofrontal                                                   & $1.50 \pm 0.15$                   &
    $1.53 \pm 0.15$                                                       & $1.09 \pm 0.16$
                                                                          & $1.15 \pm 0.14$                   & $1.15 \pm 0.17$ & $1.21 \pm 0.13$
    \\
    middletemporal                                                        & $1.74 \pm 0.16$                   &
    $1.81 \pm 0.14$                                                       & $1.42 \pm 0.23$
                                                                          & $1.52 \pm 0.19$                   & $1.44 \pm 0.21$ & $1.55 \pm 0.18$
    \\
    parahippocampal                                                       & $1.54 \pm 0.14$                   &
    $1.56 \pm 0.12$                                                       & $1.13 \pm 0.13$
                                                                          & $1.09 \pm 0.13$                   & $1.11 \pm 0.13$ & $1.07 \pm 0.13$
    \\
    paracentral                                                           & $1.59 \pm 0.22$                   &
    $1.60 \pm 0.22$                                                       & $1.40 \pm 0.17$
                                                                          & $1.40 \pm 0.19$                   & $1.36 \pm 0.18$ & $1.36 \pm 0.20$
    \\
    parsopercularis                                                       & $1.74 \pm 0.17$                   &
    $1.71 \pm 0.16$                                                       & $1.38 \pm 0.19$
                                                                          & $1.30 \pm 0.18$                   & $1.38 \pm 0.19$ & $1.30 \pm 0.20$
    \\
    parsorbitalis                                                         & $1.53 \pm 0.20$                   &
    $1.51 \pm 0.20$                                                       & $1.21 \pm 0.14$
                                                                          & $1.21 \pm 0.18$                   & $1.19 \pm 0.16$ & $1.22 \pm 0.18$
    \\
    parstriangularis                                                      & $1.68 \pm 0.17$                   &
    $1.63 \pm 0.19$                                                       & $1.33 \pm 0.16$
                                                                          & $1.30 \pm 0.22$                   & $1.30 \pm 0.16$ & $1.28 \pm 0.21$
    \\
    pericalcarine                                                         & $1.33 \pm 0.21$                   &
    $1.30 \pm 0.22$                                                       & $1.23 \pm 0.20$
                                                                          & $1.21 \pm 0.22$                   & $1.18 \pm 0.17$ & $1.18 \pm 0.17$
    \\
    postcentral                                                           & $1.84 \pm 0.24$                   &
    $1.81 \pm 0.26$                                                       & $1.68 \pm 0.23$
                                                                          & $1.69 \pm 0.28$                   & $1.64 \pm 0.20$ & $1.63 \pm 0.24$
    \\
    posteriorcingulate                                                    & $1.57 \pm 0.13$                   &
    $1.56 \pm 0.14$                                                       & $1.37 \pm 0.20$
                                                                          & $1.35 \pm 0.21$                   & $1.39 \pm 0.19$ & $1.39 \pm 0.22$
    \\
    precentral                                                            & $1.79 \pm 0.26$                   &
    $1.76 \pm 0.28$                                                       & $1.71 \pm 0.24$
                                                                          & $1.64 \pm 0.27$                   & $1.72 \pm 0.22$ & $1.66 \pm 0.28$
    \\
    precuneus                                                             & $1.83 \pm 0.13$                   &
    $1.84 \pm 0.13$                                                       & $1.65 \pm 0.21$
                                                                          & $1.66 \pm 0.21$                   & $1.61 \pm 0.18$ & $1.62 \pm 0.19$
    \\
    rostralanteriorcingulate                                              & $1.34 \pm 0.14$                   &
    $1.39 \pm 0.15$                                                       & $1.00 \pm 0.16$
                                                                          & $1.07 \pm 0.17$                   & $1.11 \pm 0.19$ & $1.11 \pm 0.18$
    \\
    rostralmiddlefrontal                                                  & $1.77 \pm 0.19$                   &
    $1.74 \pm 0.19$                                                       & $1.44 \pm 0.24$
                                                                          & $1.41 \pm 0.28$                   & $1.49 \pm 0.21$ & $1.48 \pm 0.25$
    \\
    superiorfrontal                                                       & $1.87 \pm 0.17$                   &
    $1.85 \pm 0.18$                                                       & $1.61 \pm 0.23$
                                                                          & $1.56 \pm 0.27$                   & $1.64 \pm 0.21$ & $1.62 \pm 0.25$
    \\
    superiorparietal                                                      & $1.92 \pm 0.18$                   &
    $1.93 \pm 0.17$                                                       & $1.72 \pm 0.24$
                                                                          & $1.65 \pm 0.28$                   & $1.66 \pm 0.22$ & $1.60 \pm 0.26$
    \\
    superiortemporal                                                      & $1.83 \pm 0.17$                   &
    $1.85 \pm 0.15$                                                       & $1.57 \pm 0.22$
                                                                          & $1.58 \pm 0.18$                   & $1.52 \pm 0.21$ & $1.57 \pm 0.18$
    \\
    supramarginal                                                         & $1.83 \pm 0.16$                   &
    $1.85 \pm 0.15$                                                       & $1.57 \pm 0.22$
                                                                          & $1.59 \pm 0.26$                   & $1.56 \pm 0.20$ & $1.56 \pm 0.24$
    \\
    frontalpole                                                           & $1.26 \pm 0.23$                   &
    $1.23 \pm 0.20$                                                       & $0.94 \pm 0.11$
                                                                          & $0.91 \pm 0.11$                   & $0.88 \pm 0.17$ & $0.87 \pm 0.14$
    \\
    temporalpole                                                          & $1.24 \pm 0.26$                   &
    $1.28 \pm 0.25$                                                       & $0.94 \pm 0.16$
                                                                          & $0.99 \pm 0.19$                   & $0.86 \pm 0.20$ & $0.91 \pm 0.22$
    \\
    transversetemporal                                                    & $1.47 \pm 0.20$                   &
    $1.46 \pm 0.18$                                                       & $1.17 \pm 0.13$
                                                                          & $1.13 \pm 0.11$                   & $1.20 \pm 0.15$ & $1.15 \pm 0.13$
    \\
    insula                                                                & $1.47 \pm 0.16$                   &
    $1.42 \pm 0.14$                                                       & $1.13 \pm 0.18$
                                                                          & $1.00 \pm 0.18$                   & $1.29 \pm 0.16$ & $1.19 \pm 0.19$
    \\
\end{longtblr}

\begin{longtblr}[ caption={Within-subject standard-deviation averaged across all subjects for
                cortical metrics.}, label={tab:std-cortical}, ]{
        colspec={lcc|cc|cc}, width=\linewidth,
        row{even}={white,font=\footnotesize},
        row{odd}={gray9,font=\footnotesize}, rows = {rowsep=0pt},
        rowhead=2, row{1}={white,font=\bfseries}, row{2}={gray9}}
    \SetCell[c=1]{c}Region   & \SetCell[c=2]{c}{cortical thickness                                      \\
    (mm)}                    &                                     & \SetCell[c=2]{c}{surface area      \\
    ($\text{mm}^2$)}         &                                     &
    \SetCell[c=2]{c}{cortical volume                                                                    \\ ($\text{mm}^3$)} &
    \\
                             & lh                                  & rh                            & lh
                             & rh                                  & lh                            & rh
    \\
    \hline
    bankssts                 & $0.02 \pm 0.01$                     & $0.02 \pm
    0.01$                    & $\028.65 \pm \015.97$               & $\021.73
    \pm \0\08.68$            & $\077.25 \pm \037.44$               & $\059.87
        \pm \020.45$
    \\
    caudalanteriorcingulate  & $0.04 \pm 0.01$                     & $0.04 \pm
    0.01$                    & $\019.98 \pm \013.83$               & $\021.01
    \pm \014.96$             & $\051.33 \pm \037.32$               & $\051.67
        \pm \041.74$
    \\
    caudalmiddlefrontal      & $0.02 \pm 0.01$                     & $0.02 \pm
    0.01$                    & $\038.58 \pm \036.77$               & $\046.65
    \pm \044.68$             & $104.41 \pm 108.02$                 & $124.11 \pm
    112.10$                                                                                             \\
    cuneus                   & $0.02 \pm 0.01$                     & $0.02 \pm
    0.01$                    & $\028.45 \pm \011.50$               & $\031.25
    \pm \015.67$             & $\060.72 \pm \025.52$               & $\074.77
        \pm \034.16$
    \\
    entorhinal               & $0.08 \pm 0.05$                     & $0.08 \pm
    0.05$                    & $\027.41 \pm \016.67$               & $\022.37
    \pm \011.70$             & $125.48 \pm \071.07$                & $115.94 \pm
    \057.21$                                                                                            \\
    fusiform                 & $0.02 \pm 0.01$                     & $0.02 \pm
    0.01$                    & $\050.70 \pm \025.16$               & $\047.86
    \pm \028.19$             & $182.92 \pm \092.31$                & $170.22 \pm
    103.05$                                                                                             \\
    inferiorparietal         & $0.01 \pm 0.01$                     & $0.01 \pm
    0.01$                    & $\053.01 \pm \029.19$               & $\059.90
    \pm \050.62$             & $145.66 \pm \072.95$                & $159.55 \pm
    110.14$                                                                                             \\
    inferiortemporal         & $0.02 \pm 0.01$                     & $0.02 \pm
    0.01$                    & $\064.73 \pm \042.27$               & $\058.75
    \pm \034.04$             & $198.15 \pm 127.44$                 & $168.38 \pm
    \084.67$                                                                                            \\
    isthmuscingulate         & $0.03 \pm 0.01$                     & $0.03 \pm
    0.01$                    & $\023.74 \pm \011.07$               & $\023.35
    \pm \013.99$             & $\057.43 \pm \029.59$               & $\053.05
        \pm \034.34$
    \\
    lateraloccipital         & $0.02 \pm 0.01$                     & $0.02 \pm
    0.01$                    & $\053.82 \pm \024.63$               & $\056.35
    \pm \028.61$             & $156.83 \pm \066.16$                & $160.98 \pm
    \076.00$                                                                                            \\
    lateralorbitofrontal     & $0.02 \pm 0.01$                     & $0.03 \pm
    0.01$                    & $\043.31 \pm \030.16$               & $117.14 \pm
    \033.75$                 & $\092.60 \pm \056.29$               & $217.89 \pm
    \069.06$                                                                                            \\
    lingual                  & $0.03 \pm 0.01$                     & $0.03 \pm
    0.01$                    & $\044.26 \pm \022.65$               & $\046.73
    \pm \023.96$             & $\089.19 \pm \046.24$               & $\095.82
        \pm \049.65$
    \\
    medialorbitofrontal      & $0.03 \pm 0.01$                     & $0.03 \pm
    0.01$                    & $\066.04 \pm \024.11$               & $\058.06
    \pm \019.00$             & $147.37 \pm \057.84$                & $134.52 \pm
    \042.26$                                                                                            \\
    middletemporal           & $0.02 \pm 0.01$                     & $0.02 \pm
    0.01$                    & $\053.01 \pm \034.97$               & $\044.87
    \pm \028.36$             & $165.49 \pm 108.52$                 & $135.26 \pm
    \077.98$                                                                                            \\
    parahippocampal          & $0.03 \pm 0.01$                     & $0.03 \pm
    0.01$                    & $\019.55 \pm \0\08.42$              & $\020.45
    \pm \0\07.81$            & $\064.22 \pm \025.29$               & $\065.43
        \pm \024.59$
    \\
    paracentral              & $0.03 \pm 0.02$                     & $0.03 \pm
    0.01$                    & $\022.94 \pm \012.98$               & $\026.94
    \pm \019.80$             & $\063.71 \pm \040.74$               & $\073.88
        \pm \056.66$
    \\
    parsopercularis          & $0.02 \pm 0.01$                     & $0.02 \pm
    0.01$                    & $\028.65 \pm \028.77$               & $\029.46
    \pm \026.82$             & $\080.67 \pm \092.87$               & $\082.38
        \pm \089.16$
    \\
    parsorbitalis            & $0.03 \pm 0.02$                     & $0.03 \pm
    0.02$                    & $\017.82 \pm \0\09.77$              & $\021.41
    \pm \010.66$             & $\060.63 \pm \045.20$               & $\068.18
        \pm \036.64$
    \\
    parstriangularis         & $0.02 \pm 0.01$                     & $0.02 \pm
    0.01$                    & $\025.67 \pm \014.65$               & $\034.86
    \pm \037.79$             & $\071.73 \pm \045.49$               & $\096.87
        \pm 102.22$
    \\
    pericalcarine            & $0.03 \pm 0.02$                     & $0.04 \pm
    0.02$                    & $\036.04 \pm \020.18$               & $\042.02
    \pm \024.82$             & $\059.64 \pm \029.98$               & $\068.61
        \pm \034.89$
    \\
    postcentral              & $0.01 \pm 0.02$                     & $0.02 \pm
    0.02$                    & $\043.47 \pm \067.12$               & $\045.98
    \pm \083.10$             & $100.26 \pm 121.35$                 & $104.53 \pm
    156.51$                                                                                             \\
    posteriorcingulate       & $0.02 \pm 0.01$                     & $0.02 \pm
    0.01$                    & $\021.93 \pm \013.05$               & $\024.39
    \pm \019.52$             & $\052.42 \pm \033.33$               & $\056.27
        \pm \052.59$
    \\
    precentral               & $0.02 \pm 0.02$                     & $0.02 \pm
    0.02$                    & $\046.92 \pm \053.54$               & $\057.46
    \pm \070.35$             & $118.04 \pm 157.21$                 & $148.21 \pm
    233.10$                                                                                             \\
    precuneus                & $0.01 \pm 0.01$                     & $0.01 \pm
    0.00$                    & $\038.04 \pm \042.87$               & $\038.95
    \pm \040.96$             & $100.91 \pm 111.15$                 & $102.24 \pm
    \096.62$                                                                                            \\
    rostralanteriorcingulate & $0.05 \pm 0.02$                     & $0.04 \pm
    0.02$                    & $\034.80 \pm \015.03$               & $\022.00
    \pm \010.59$             & $\081.04 \pm \041.59$               & $\061.95
        \pm \033.93$
    \\
    rostralmiddlefrontal     & $0.02 \pm 0.01$                     & $0.02 \pm
    0.01$                    & $\092.87 \pm \096.23$               & $108.40 \pm
    132.97$                  & $213.81 \pm 259.58$                 & $252.00 \pm
    358.20$                                                                                             \\
    superiorfrontal          & $0.01 \pm 0.01$                     & $0.01 \pm
    0.01$                    & $\085.23 \pm \086.47$               & $\098.14
    \pm 120.75$              & $223.91 \pm 234.89$                 & $243.75 \pm
    304.56$                                                                                             \\
    superiorparietal         & $0.01 \pm 0.01$                     & $0.01 \pm
    0.01$                    & $\049.49 \pm \080.81$               & $\062.89
    \pm \096.86$             & $132.77 \pm 207.97$                 & $161.39 \pm
    235.01$                                                                                             \\
    superiortemporal         & $0.02 \pm 0.01$                     & $0.01 \pm
    0.01$                    & $\047.70 \pm \033.64$               & $\041.38
    \pm \023.84$             & $156.30 \pm 101.85$                 & $129.01 \pm
    \078.70$                                                                                            \\
    supramarginal            & $0.01 \pm 0.01$                     & $0.01 \pm
    0.01$                    & $\050.87 \pm \058.82$               & $\050.06
    \pm \083.24$             & $136.23 \pm 168.28$                 & $133.99 \pm
    207.69$                                                                                             \\
    frontalpole              & $0.07 \pm 0.04$                     & $0.07 \pm
    0.04$                    & $\012.99 \pm \0\04.02$              & $\016.42
    \pm \0\04.47$            & $\056.49 \pm \032.17$               & $\067.84
        \pm \028.93$
    \\
    temporalpole             & $0.09 \pm 0.05$                     & $0.08 \pm
    0.05$                    & $\025.08 \pm \010.71$               & $\022.16
    \pm \011.78$             & $154.60 \pm \079.32$                & $138.28 \pm
    \078.33$                                                                                            \\
    transversetemporal       & $0.03 \pm 0.02$                     & $0.03 \pm
    0.02$                    & $\012.73 \pm \0\05.33$              & $\0\09.98
    \pm \0\03.33$            & $\029.55 \pm \012.34$               & $\024.91
        \pm \0\08.79$
    \\
    insula                   & $0.04 \pm 0.02$                     & $0.04 \pm
    0.01$                    & $\073.45 \pm \030.66$               & $\095.70
    \pm \037.63$             & $146.49 \pm \064.11$                & $183.39 \pm
    \081.47$                                                                                            \\
\end{longtblr}

\begin{longtblr}[ caption={Within-subject significant digits averaged across all
                subjects for subcortical volumes.},
        label={tab:sig-std-subcortical-volume},]{ colspec={lc|c},
        row{even}={gray9,font=\footnotesize},
        row{odd}={white,font=\footnotesize}, rows = {rowsep=0pt},
    row{Z}={font=\small}, rowhead=1, row{1}={font=\bfseries}} Region &
    Significant digits                                               & {Standard deviation                        \\ ($\text{mm}^3$)} \\
    \hline
    Left-Thalamus                                                    & $1.42 \pm 0.21$     & $120.08  \pm 69.61$  \\
    Left-Caudate                                                     & $1.57 \pm 0.20$     & $\038.83 \pm 25.11$  \\
    Left-Putamen                                                     & $1.49 \pm 0.22$     & $\065.88 \pm 46.39$  \\
    Left-Pallidum                                                    & $1.25 \pm 0.19$     & $\047.81 \pm 25.09$  \\
    Left-Hippocampus                                                 & $1.48 \pm 0.17$     & $\056.23 \pm 41.03$  \\
    Left-Amygdala                                                    & $1.13 \pm 0.16$     & $\048.71 \pm 20.04$  \\
    Left-Accumbens-area                                              & $0.88 \pm 0.16$     & $\024.20 \pm \08.80$ \\
    Right-Thalamus                                                   & $1.42 \pm 0.20$     & $118.92  \pm 68.76$  \\
    Right-Caudate                                                    & $1.51 \pm 0.24$     & $\049.37 \pm 42.71$  \\
    Right-Putamen                                                    & $1.51 \pm 0.25$     & $\068.07 \pm 70.23$  \\
    Right-Pallidum                                                   & $1.22 \pm 0.19$     & $\049.11 \pm 30.50$  \\
    Right-Hippocampus                                                & $1.55 \pm 0.18$     & $\048.59 \pm 28.98$  \\
    Right-Amygdala                                                   & $1.23 \pm 0.17$     & $\042.21 \pm 18.68$  \\
    Right-Accumbens-area                                             & $0.99 \pm 0.15$     & $\020.50 \pm \07.72$ \\
\end{longtblr}

\begin{table}[h]
    \centering
    \caption{Summary of executions failure and excluded subjects. To standardize
        the sample, we keep 26 repetitions per subject/visits pair.
        Subject/visit pairs with less than 26 repetitions were excluded which is
        12 subjects.}
    \begin{tabular}{l c c}
        \toprule
        \textbf{Stage}     & \textbf{Number of rejected repetitions} &
        \textbf{Total number of repetitions}                                 \\
        \midrule
        Cluster failure    & 1246 (5.80\%)                           & 21488 \\
        FreeSurfer failure & 68 (0.33\%)                             & 21488 \\
        QC failure         & 319 (1.48\%)                            & 21488 \\
        Total              & 1633 (7.60\%)                           & 21488 \\
        \bottomrule
    \end{tabular}
\end{table}

\begin{table}[h!]
    \centering
    \begin{tabular}{c|lccc}
        \toprule
        \textbf{Status} & \textbf{Cohort}             & \textbf{HC}
                        & \textbf{PD-non-MCI}         & \textbf{PD-MCI}
        \\
        \hline
        \multirow{5}{*}{\textbf{\shortstack{Before                                    \\QC}}} & n
                        & 106                         & 181                      & 29 \\
                        & Age (y)                     & $60.6 \pm 10.2   $
                        & $61.7 \pm \09.6$            & $67.7 \pm \07.7$
        \\
                        & Age range                   & $30.6 - 84.3  $
                        & $36.3 - 83.3$               & $49.9 - 80.5$
        \\
                        & Gender (male, \%)           & $58 \; (54.7\%)   $
                        & $119 \; (65.7\%)          $ & $-          $
        \\
                        & Education (y)               & $16.6 \pm \03.3  $
                        & $15.9 \pm \02.9$            & $-          $
        \\
        \hline
        \multirow{5}{*}{\textbf{\shortstack{After                                     \\QC}}} & n
                        & 103                         & 175                      & 27 \\
                        & Age (y)                     & $60.7 \pm 10.3   $
                        & $61.4 \pm \09.5          $  & $67.8 \pm \07.9$
        \\
                        & Age range                   & $30.6 - 84.3  $
                        & $36.3 - 79.9           $    & $49.9 - 80.5$
        \\
                        & Gender (male, \%)           & $57 \; (55.3\%)   $
                        & $114 \; (65.1\%)       $    & $20 \; (74.1\%) $
        \\
                        & Education (y)               & $16.6 \pm \03.3  $
                        & $15.9 \pm \02.9        $    & $15.0 \pm \03.5$
        \\
        \hline
        \multirow{8}{*}{\textbf{\shortstack{After                                     \\MCI\\exclusion}}} & n
                        & $103 $                      & $121                   $ & -- \\
                        & Age (y)                     & $60.7 \pm 10.3   $
                        & $60.7 \pm \09.1        $    & --
        \\
                        & Age range                   & $30.6 - 84.3  $
                        & $39.2 - 78.3           $    & --
        \\
                        & Gender (male, \%)           & $57 \; (55.3\%)   $
                        & $80 \; (66.1\%)        $    & --
        \\
                        & Education (y)               & $16.6 \pm \03.3  $
                        & $16.1 \pm \03.0        $    & --
        \\
                        & UPDRS III OFF baseline      & $-            $
                        & $23.4 \pm 10.1         $    & --
        \\
                        & UPDRS III OFF follow-up     & $-            $
                        & $25.8 \pm 11.1         $    & --
        \\
                        & Duration T2 - T1 (y)        & $\01.4 \pm \00.5 $
                        & $\01.4 \pm \00.7       $    & --
        \\
        \bottomrule
    \end{tabular}
    \vspace{1em}

    \textbf{Abbreviations:} MCI = Mild Cognitive Impairment; UPDRS = Unified
    Parkinson's Disease Rating Scale; PD = Parkinson's disease. Descriptive
    statistics before and after quality control (QC). Values are expressed as
    mean $\pm$ standard deviation. PD-non-MCI longitudinal sample is a subsample
    of the PD-non-MCI original sample that had longitudinal data and disease
    severity scores available.
    \label{tab:cohort_stat_vertical}
\end{table}

\section{Numerical-Anatomical Variability Ratio (\navr)}

\subsection{\navr maps}

Figures \ref{fig:navr_map_area} and \ref{fig:navr_map_volume} show the \navr
maps for cortical surface area and volume, respectively. The maps show the
average \navr values across all subjects for each cortical region. The color
scale indicates the \navr value, with warmer colors indicating higher \navr
values. The maps provide a visual representation of the variability in the
\navr values across different cortical regions, highlighting regions with
higher or lower \navr values.

The NAVR analysis reveals that numerical variability is a pervasive and
systematic source of uncertainty in neuroimaging, with within-subject
differences reaching up to $37-40\%$ of the observed between-subject anatomical
variance in key cortical and subcortical regions (Extended Data
Fig.~\ref{fig:navr_subcortical}, \ref{fig:navr_thickness}). Across anatomical
metrics, numerical uncertainty consistently accounts for a substantial fraction
of biological signal: cortical thickness measurements exhibit a mean NAVR of
0.21 (range: $0.11-0.37$), surface area 0.18 ($0.09-0.42$), and cortical volume
0.17 ($0.09-0.42$). Even subcortical volumes, considered more robust, show
meaningful variability (mean: 0.15, range: $0.01-0.27$). Notably, even the
lowest NAVR values ($0.01-0.11$) indicate that numerical noise is never
negligible, while maxima up to 0.42 suggest that in some regions, it can rival
or exceed half the anatomical variance. The narrow standard deviations across
metrics ($0.06-0.08$) further underscore the consistency of this phenomenon.
These findings carry serious implications: median NAVR values around
$0.16-0.20$ suggest that numerical imprecision can obscure subtle yet
clinically relevant effects or give rise to spurious associations, undermining
the reliability of neuroimaging-based inferences.

\begin{figure}[h]
    \centering
    \begin{subfigure}[b]{0.48\linewidth}
        \centering
        \includegraphics[width=\linewidth]{figures/NAVR_map/NAVR_area_all.png}
        \caption{Cortical surface area}
        \label{fig:navr_map_area}
    \end{subfigure}
    \hfill
    \begin{subfigure}[b]{0.48\linewidth}
        \centering
        \includegraphics[width=\linewidth]{figures/NAVR_map/NAVR_volume_all.png}
        \caption{Cortical volume}
        \label{fig:navr_map_volume}
    \end{subfigure}

    \vspace{0.5cm}
    \centering
    \includegraphics[width=0.6\linewidth]{figures/NAVR_map/jet_colorbar.pdf}

    \caption{Numerical-Anatomical Variability Ratio (\navr) for cortical surface
        area (Fig.~\ref{fig:navr_map_area}) and volume
        (Fig.~\ref{fig:navr_map_volume}) across regions and groups. Higher \navr
        values indicate greater computational uncertainty relative to biological
        variation. The color scale indicates the \navr value, with warmer colors
        indicating higher \navr values.}
    \label{fig:navr_maps}
\end{figure}

\subsection{Consistency results}

\subsubsection{Consistency of statistical tests}

Figures \ref{fig:navr_consistency_area_plot} and
\ref{fig:navr_consistency_volume_plot} show the consistency of statistical
tests for cortical area and volume, respectively, across all subjects and
regions. The plots show the percentage of subjects for which the statistical
test was significant ($\alpha = 0.05$) for each region. The consistency varies
across regions, with some regions showing higher consistency than others. The
red triangles indicate the IEEE-754 run for reference.

\begin{figure}[h]
    \centering
    \includegraphics[width=\linewidth]{figures/consistency/cortical_area_significance_correlation.pdf}
    \caption{Consistency of statistical tests for cortical area across all
        subjects and regions. The plot shows the percentage of subjects for
        which the statistical test was significant ($\alpha = 0.05$) for each
        region. The consistency varies across regions, with some regions showing
        higher consistency than others.}
    \label{fig:navr_consistency_area_plot}
\end{figure}

\begin{figure}[h]
    \centering
    \includegraphics[width=\linewidth]{figures/consistency/cortical_volume_significance_correlation.pdf}
    \caption{Consistency of statistical tests for cortical volume across all
        subjects and regions. The plot shows the percentage of subjects for
        which the statistical test was significant ($\alpha = 0.05$) for each
        region. The consistency varies across regions, with some regions showing
        higher consistency than others.}
    \label{fig:navr_consistency_volume_plot}
\end{figure}

\subsubsection{Distribution of statistical tests coefficients}

Figures \ref{fig:navr_consistency_area} and \ref{fig:navr_consistency_volume}
show the distribution of partial correlation coefficients for cortical area and
volume, respectively, across all subjects and regions. The red triangles
indicate the IEEE-754 run for reference. The distribution shows the variability
in the coefficients, with some regions exhibiting higher consistency than
others.

\begin{figure}
    \centering
    \begin{subfigure}[b]{\linewidth}
        \includegraphics[width=\linewidth]{figures/consistency/cortical_thickness_coefficients_distribution-Left.pdf}
        \caption{Left hemisphere}
        \label{fig:navr_consistency_thickness_left}
    \end{subfigure}

    \begin{subfigure}[b]{\linewidth}
        \includegraphics[width=\linewidth]{figures/consistency/cortical_thickness_coefficients_distribution-Right.pdf}
        \caption{Right hemisphere}
        \label{fig:navr_consistency_thickness_right}
    \end{subfigure}
    \caption{Distribution of partial correlation coefficients for cortical
        thickness across all subjects and regions. Red triangles indicate the
            IEEE-754 run for reference. }
    \label{fig:navr_consistency_thickness}
\end{figure}

\begin{figure}[h]
    \centering
    \begin{subfigure}[b]{\linewidth}
        \includegraphics[width=\linewidth]{figures/consistency/cortical_area_coefficients_distribution-Left.pdf}
        \caption{Left hemisphere}
        \label{fig:navr_consistency_area_left}
    \end{subfigure}
    \hfill
    \begin{subfigure}[b]{\linewidth}
        \includegraphics[width=\linewidth]{figures/consistency/cortical_area_coefficients_distribution-Right.pdf}
        \caption{Right hemisphere}
        \label{fig:navr_consistency_area_right}
    \end{subfigure}
    \caption{ Distribution of partial correlation coefficients for cortical area
        across all subjects and regions. Red triangles indicate the IEEE-754 run
        for reference. }
    \label{fig:navr_consistency_area}
\end{figure}

\begin{figure}[h]
    \centering
    \begin{subfigure}[b]{\linewidth}
        \includegraphics[width=\linewidth]{figures/consistency/cortical_volume_coefficients_distribution-Left.pdf}
        \caption{Left hemisphere}
        \label{fig:navr_consistency_volume_left}
    \end{subfigure}
    \hfill
    \begin{subfigure}[b]{\linewidth}
        \includegraphics[width=\linewidth]{figures/consistency/cortical_volume_coefficients_distribution-Right.pdf}
        \caption{Right hemisphere}
        \label{fig:navr_consistency_volume_right}
    \end{subfigure}
    \caption{ Distribution of partial correlation coefficients for cortical
        volume across all subjects and regions. Red triangles indicate the
        IEEE-754 run for reference. The distribution shows the variability in
        the coefficients, with some regions exhibiting higher consistency than
        others.}
    \label{fig:navr_consistency_volume}
\end{figure}

\subsubsection{Thresholding existing Cohen's d values from the literature}

We applied a thresholding approach to the Cohen's d values reported in the
literature to identify the most relevant findings for our analysis. This
involved setting a minimum effect size threshold, below which results were
considered non-significant or uninformative. The threshold was determined based
on the distribution of Cohen's d values across studies, with a focus on
retaining only those effects that were robust and consistent.

\begin{figure}[h]
    \centering
    \vspace{0.2cm}

    % Header row with column labels
    \begin{minipage}[b]{\linewidth}
        \begin{minipage}[c]{0.05\linewidth}
            % Empty space for alignment with condition labels
        \end{minipage}%
        \begin{minipage}[c]{0.95\linewidth}
            \begin{minipage}[c]{0.47\linewidth}
                \centering\textbf{Unthresholded}
            \end{minipage}%
            \hfill
            \begin{minipage}[c]{0.005\linewidth}
                % Vertical line separator
            \end{minipage}%
            \hfill
            \begin{minipage}[c]{0.47\linewidth}
                \centering\textbf{Thresholded}
            \end{minipage}
        \end{minipage}
    \end{minipage}

    % Horizontal line
    \noindent\rule{\linewidth}{0.5pt}
    \vspace{-1.5cm}

    % 22q11.2 deletion syndrome row
    \begin{minipage}[b]{\linewidth}
        \begin{minipage}[c]{0.05\linewidth}
            \centering\rotatebox{90}{\textbf{22q11.2}} \end{minipage}%
        \begin{minipage}[c]{0.95\linewidth}
            \begin{subfigure}[c]{0.47\linewidth}
                \includegraphics[width=\linewidth]{figures/cohen_d_map/enigma/22q_area_all.png}
                \label{fig:enigma_22q_unthresholded}
            \end{subfigure}%
            \hfill
            \begin{minipage}[c]{0.005\linewidth}
                \centering\rule{0.5pt}{4cm}
            \end{minipage}%
            \hfill
            \begin{subfigure}[c]{0.47\linewidth}
                \includegraphics[width=\linewidth]{figures/cohen_d_map/enigma/22q_area_all_thresholded.png}
                \label{fig:enigma_22q_thresholded}
            \end{subfigure}
        \end{minipage}
    \end{minipage}

    \vspace{-2cm}

    % ADHD row
    \begin{minipage}[b]{\linewidth}
        \begin{minipage}[c]{0.05\linewidth}
            \centering\rotatebox{90}{\textbf{ADHD}} \end{minipage}%
        \begin{minipage}[c]{0.95\linewidth}
            \begin{subfigure}[c]{0.47\linewidth}
                \includegraphics[width=\linewidth]{figures/cohen_d_map/enigma/adhd_area_adult.png}
                \label{fig:enigma_adhd_unthresholded}
            \end{subfigure}%
            \hfill
            \begin{minipage}[c]{0.005\linewidth}
                \centering\rule{0.5pt}{4cm}
            \end{minipage}%
            \hfill
            \begin{subfigure}[c]{0.47\linewidth}
                \includegraphics[width=\linewidth]{figures/cohen_d_map/enigma/adhd_area_adult_thresholded.png}
                \label{fig:enigma_adhd_thresholded}
            \end{subfigure}
        \end{minipage}
    \end{minipage}

    \vspace{-2cm}

    % bipolar disorder row
    \begin{minipage}[b]{\linewidth}
        \begin{minipage}[c]{0.05\linewidth}
            \centering\rotatebox{90}{\textbf{Bipolar}} \end{minipage}%
        \begin{minipage}[c]{0.95\linewidth}
            \begin{subfigure}[c]{0.47\linewidth}
                \includegraphics[width=\linewidth]{figures/cohen_d_map/enigma/bipolar_area_adult.png}
                \label{fig:enigma_bipolar_unthresholded}
            \end{subfigure}%
            \hfill
            \begin{minipage}[c]{0.005\linewidth}
                \centering\rule{0.5pt}{4cm}
            \end{minipage}%
            \hfill
            \begin{subfigure}[c]{0.47\linewidth}
                \includegraphics[width=\linewidth]{figures/cohen_d_map/enigma/bipolar_area_adult_thresholded.png}
                \label{fig:enigma_bipolar_thresholded}
            \end{subfigure}
        \end{minipage}
    \end{minipage}

    \vspace{-2cm}
    % depression
    \begin{minipage}[b]{\linewidth}
        \begin{minipage}[c]{0.05\linewidth}
            \centering\rotatebox{90}{\textbf{Depression}} \end{minipage}%
        \begin{minipage}[c]{0.95\linewidth}
            \begin{subfigure}[c]{0.47\linewidth}
                \includegraphics[width=\linewidth]{figures/cohen_d_map/enigma/depression_area_adult.png}
                \label{fig:enigma_depression_unthresholded}
            \end{subfigure}%
            \hfill
            \begin{minipage}[c]{0.005\linewidth}
                \centering\rule{0.5pt}{4cm}
            \end{minipage}%
            \hfill
            \begin{subfigure}[c]{0.47\linewidth}
                \includegraphics[width=\linewidth]{figures/cohen_d_map/enigma/depression_area_adult_thresholded.png}
                \label{fig:enigma_depression_thresholded}
            \end{subfigure}
        \end{minipage}
    \end{minipage}

    \vspace{-2cm}
    % ocd 
    \begin{minipage}[b]{\linewidth}
        \begin{minipage}[c]{0.05\linewidth}
            \centering\rotatebox{90}{\textbf{OCD}} \end{minipage}%
        \begin{minipage}[c]{0.95\linewidth}
            \begin{subfigure}[c]{0.47\linewidth}
                \includegraphics[width=\linewidth]{figures/cohen_d_map/enigma/ocd_area_adult.png}
                \label{fig:enigma_ocd_unthresholded}
            \end{subfigure}%
            \hfill
            \begin{minipage}[c]{0.005\linewidth}
                \centering\rule{0.5pt}{4cm}
            \end{minipage}%
            \hfill
            \begin{subfigure}[c]{0.47\linewidth}
                \includegraphics[width=\linewidth]{figures/cohen_d_map/enigma/ocd_area_adult_thresholded.png}
                \label{fig:enigma_ocd_thresholded}
            \end{subfigure}
        \end{minipage}
    \end{minipage}

    \vspace{-2cm}
    % schizophrenia
    \begin{minipage}[b]{\linewidth}
        \begin{minipage}[c]{0.05\linewidth}
            \centering\rotatebox{90}{\textbf{Schizophrenia}} \end{minipage}%
        \begin{minipage}[c]{0.95\linewidth}
            \begin{subfigure}[c]{0.47\linewidth}
                \includegraphics[width=\linewidth]{figures/cohen_d_map/enigma/schizophrenia_area_all.png}
                \label{fig:enigma_schizophrenia_unthresholded}
            \end{subfigure}%
            \hfill
            \begin{minipage}[c]{0.005\linewidth}
                \centering\rule{0.5pt}{4cm}
            \end{minipage}%
            \hfill
            \begin{subfigure}[c]{0.47\linewidth}
                \includegraphics[width=\linewidth]{figures/cohen_d_map/enigma/schizophrenia_area_all_thresholded.png}
                \label{fig:enigma_schizophrenia_thresholded}
            \end{subfigure}
        \end{minipage}
    \end{minipage}

    \caption{ENIGMA cortical area Cohen's d maps showing unthresholded effect
        sizes (left) and effect sizes thresholded by the \navr framework (right)
        for different disorders. Black regions indicate areas where Cohen's d
        values fall below the numerical variability threshold, demonstrating
        regions where reported effect sizes may be unreliable due to
        computational uncertainty.}
    \label{fig:navr_enigma_area}
\end{figure}

\end{document}
