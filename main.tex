\documentclass{article}

% Language setting Replace `english' with e.g. `spanish' to change the document
% language
\usepackage[english]{babel}

% Set page size and margins Replace `letterpaper' with `a4paper' for UK/EU
% standard size
\usepackage[letterpaper,top=2cm,bottom=2cm,left=3cm,right=3cm,marginparwidth=1.75cm]{geometry}

% Useful packages
\usepackage{amsmath}
\usepackage{mathtools}
\usepackage{amssymb}
\usepackage[pdftex]{graphicx}
\usepackage[colorlinks=true, allcolors=blue]{hyperref}
\usepackage{todonotes}
\usepackage{longtable}
\usepackage{booktabs}
\usepackage{multirow}
\usepackage{subcaption}
\usepackage{xcolor}
\usepackage{tabularray}
\usepackage{xspace} % For \xspace command
\usepackage{tikz}
\usetikzlibrary{shadings,positioning,overlay-beamer-styles}
\usepackage{xifthen}
\usepackage{ulem}
\usepackage{amsthm}
\usepackage{cleveref}

\newcommand{\YC}[1]{\textcolor{red}{YC: #1}}
\newcommand{\TG}[2][]{\ifthenelse{\equal{#1}{done}}{[\textcolor{teal}{$\blacksquare$}]~\textcolor{blue!60}{\textsc{From
Tristan:} #2}}{\textcolor{blue}{\textsc{From Tristan:} #2}}}


\newcommand{\0}{\mspace{9mu}}
\newcommand{\navr}[0]{$\nu_{\mathrm{nav}}$\xspace}
\newcommand{\Var}[0]{\mathrm{Var}}

\title{Numerical variability in structural MRI measures of Parkinson's disease}

\author{Yohan Chatelain, Andrzej Sokołowski, Madeleine Sharp, Jean-Baptiste Poline, Tristan Glatard}

\begin{document}

\maketitle

\begin{abstract}
    Numerical variability is rarely quantified in neuroimaging despite many
    biomarkers relying on subtle morphometric differences across individuals. We
    instrumented FreeSurfer, a widely used neuroimaging pipeline, to simulate
    numerical differences across computational environments, and used it to
    measure numerical variability in MRI analyses of Parkinson's disease
    patients and controls. In multiple cortical and subcortical regions,
    numerical variation reached nearly one-third of the anatomical signal,
    altering statistical conclusions about group differences and clinical
    associations. To address this, we developed a practical tool that estimates
    the Numerical-Anatomical Variability Ratio (NAVR), enabling researchers to
    assess the impact of numerical noise in existing studies. Applying
    NAVR-based thresholding to ENIGMA brain maps showed that many small regional
    effects fall below typical numerical noise thresholds \TG{Note to self:
        re-evaluate this (strong) claim after reading the results} for standard
    study sizes \TG{not sure why study sizes are mentioned here since ENIGMA
        maps were obtained for specific sample sizes}. Our tool provides a practical
    approach to quantify the effect of numerical variability, improving the
    robustness and reproducibility of neuroimaging analyses.
\end{abstract}

\section{Introduction}

% \TG[done]{Rewrite using the following structure:
%     \begin{itemize}
%         \item The reliability of structural MRI measures critically depends on the estimation
%               of sources of analytical variability.
%         \item In particular when effect sizes are moderate, such as in Parkinson's disease.
%         \item Among these sources, numerical variability has been shown to have a measurable
%               impact on analyses across multiple pipeline tools.
%         \item Numerical variability arises from rounding and truncation errors associated
%               with the use of limited-precision numerical formats.
%         \item Due to the complexity of MRI analyses, and in particular their reliance on
%               high-dimensional optimization, such errors propagate and sometimes lead to
%               measurable differences.
%         \item However, numerical variability is understudied, mainly due to practical
%               challenges.
%         \item In particular, the implications of numerical variability on clinical findings
%               is unknown.
%         \item What we have done
%     \end{itemize}
% }

The reliability of MRI measures critically depends on the estimation of all
sources of analytical variability~\cite{carp2012plurality,botvinik2020variability,kennedy2019everything}.
This is particularly important when effect sizes are moderate, as it is the case
in studies of Parkinson's disease (PD)~\cite{he2020progressive}. Among these
sources, numerical variability---differences in computational results arising from
factors like hardware, operating systems, or software library versions---has been
shown to have a measurable impact on analyses across multiple pipeline
tools~\cite{glatard2012virtual,gronenschild2012effects,des2023reproducibility,vila2024impact}.

Numerical variability arises from rounding and truncation errors associated
with the use of limited-precision numerical formats, such as the widespread IEEE-754
standard for floating-point arithmetic~\cite{markstein2008new}. Due to the
complexity of MRI analyses, particularly their reliance on high-dimensional
optimization, such errors can propagate and accumulate, sometimes leading to
measurable differences in final
outputs~\cite{salari2021accurate,kiar2021numerical,chatelain2024numerical,mirhakimi2025numerical}.
However, numerical variability remains understudied, mainly due to the
practical challenges of quantifying its effects. Consequently, the implications
of numerical variability for clinical findings are largely unknown.

The first aim of this study was to quantify the impact of numerical
variability in structural MRI analyses of Parkinson's disease. Building on these
initial observations, we developed an analytical framework and associated tools
to rapidly assess the numerical quality of structural MRI analyses reported in
the published literature. By making numerical-variability evaluation accessible,
our framework and tool enhance transparency, support peer review, and promote
more reliable statistical inference in neuroimaging. Applying this framework to
the PD literature, we provide the first estimates of the numerical quality of
MRI analyses in Parkinson's disease studies. \TG{revise last sentence when results are available.}

% In this study, we systematically investigated the impact of numerical
% variability on structural MRI analyses of Parkinson's disease. We instrumented
% the FreeSurfer tool~\cite{fischl2012freesurfer} with Monte Carlo
% Arithmetic~\cite{parker1997monte} to simulate a range of realistic
% numerical perturbations, and used it to measure how numerical noise affects 
% group comparisons and correlations in the  Parkinson's Progression
% Markers Initiative (PPMI) data. We introduce the
% Numerical-Anatomical Variability Ratio (NAVR) to compare the magnitude of
% numerical noise to that of anatomical variability and demonstrate its use in
% evaluating the robustness of statistical findings, both in our dataset and in
% large-scale meta-analyses from the ENIGMA consortium~\cite{thompson2014enigma}.

\section{Numerical variability alters statistical inference for MRI measures of Parkinson's disease}

By repeatedly processing the same MRI scans under random numerical perturbations
representative of differences across hardware systems, software libraries, or
other computational parameters, we assessed the impact of numerical variability
on conclusions drawn from MRI analyses of Parkinson's disease. We focused on two
common analyses, namely (1) volumetric group differences between PD subjects and
Healthy Controls (HC), and (2) partial correlations between regional volumes and
motor evaluation scores (measured with the MDS-Unified Parkinson's Disease Rating Scale
part 3---UPDRS-III). For both, we conducted a cross-sectional analysis at
baseline and a longitudinal analysis across two time points.

We extracted T1-weighted MRI data from the Parkinson Progression Marker
Initiative (PPMI) dataset~\cite{marek2011parkinson}, selecting participants with
at least two usable visits separated by 0.9-2.0 years, and excluding
participants with Mild Cognitive Impairment (MCI) or other neurological
disorders. The final dataset included  112 PD participants without MCI (PD-non-MCI) and
89 HC (Table~\ref{tab:cohort_stat}).

All images underwent standard pre-processing using FreeSurfer's longitudinal
stream: cross-sectional processing of both timepoints, followed by creation of
an unbiased within-subject template using robust registration. We introduced
numerical noise mimicking realistic perturbations into this pipeline using
Monte Carlo Arithmetic (MCA)~\cite{parker1997monte}, a technique that injects
random, zero-mean perturbations into floating-point operations while perserving
mathematical expectations. We repeated the perturbed analyses,
yielding 26 usable runs, to estimate numerical variability.

For both group and correlation analyses, statistical outcomes varied
substantially across the 26 Monte Carlo Arithmetic (MCA) repetitions (Figures
\ref{fig:significance_correlation_subcortical_volume}-\ref{fig:significance_correlation_thickness}).
In subcortical volumes (14 regions; Figure
\ref{fig:significance_correlation_subcortical_volume}), significance flipped in
27\% of regions, indicating that more than a quarter showed at least one
inconsistent result across repetitions. In cortical thickness (68 regions;
Figure \ref{fig:significance_correlation_thickness}), 21\% of regions were found
unstable. When all cortical and subcortical metrics were considered jointly (872
total comparisons; Appendix Table \ref{tab:fluctuating_regions}), the overall
proportion of inconsistent results reached 27\%. These fluctuations suggest that
apparent biomarker effects may appear or disappear purely due to numerical
noise, echoing the multi-analyst variability reported in NARPS
\cite{botvinik2020variability}. \TG[done]{The numbers in this paragraph are hard
    to follow as the $\%$ refer to different totals. Could this be simplified to one
    main percentage per figure?}

\begin{figure}
    \includegraphics[width=\linewidth]{figures/consistency/subcortical_volume_significance_correlation.pdf}
    \caption{ Proportion of significant tests ($p<0.05$) for subcortical volumes across 26 numerical perturbations.
        measures.\label{fig:significance_correlation_subcortical_volume}}
\end{figure}

\begin{figure}
    \centering
    \includegraphics[width=\linewidth]{figures/consistency/cortical_thickness_significance_correlation.pdf}
    \caption{Proportion of significant tests ($p<0.05$) for cortical thickness across 26 numerical perturbations.
        measures.\label{fig:significance_correlation_thickness}}
    \label{fig:navr_consistency_thickness_plot}
\end{figure}

For subcortical volumes, the distributions of partial-correlation coefficients
($r$) and ANCOVA $F$-statistics
(Figure~\ref{fig:statstest_coefficients_distribution}) show that the unperturbed
IEEE-754 estimates (red markers) consistently fall within the MCA-sampled range.
This supports the validity of the instrumentation rather than indicating a
methodological artifact. The spread of the $r$ coefficients is larger in
longitudinal than in baseline analyses (Ansari-Bradley one-sided test at
$p<0.05$ corrected with Bonferroni; detailed results in Appendix
Table~\ref{tab:stats-coef-var-subcortical}), whereas no significant difference
in spread is detected for $F$-statistics using the same test
(Bonferroni-corrected, $p<0.05$). This indicates that numerical variability has
a more pronounced impact on longitudinal studies than on cross-sectional ones.
An analogous analysis for cortical thickness leads to similar conclusions
(Appendix Figure~\ref{fig:consistency_thickness_coefficients},
Table~\ref{tab:stats-coef-var-cortical}).

\begin{figure}[ht]
    \includegraphics[width=\linewidth]{figures/consistency/subcortical_volume_coefficients_distribution.pdf}
    \caption{ Distribution of partial correlation coefficients (r-values) and
        F-statistics from ANCOVA across MCA repetitions for subcortical volume
        measures. Red dots represent the IEEE-754 unperturbed results. The top
        row shows r-values, while the bottom row shows F-values. The left column
        represents baseline analysis, and the right column represents
        longitudinal analysis.\label{fig:statstest_coefficients_distribution}}
\end{figure}

\section{A framework to quantify the impact of numerical variability}
% Comments:
% - Build a tool that can be broadly applied to any neuroimaging
% - Analytical modeling of sigma_d
% - Explain why having a tool is important
% - Why having a tool fast is important
%   - Measuring numerical variability is a time-consuming process
%   - Analytical modeling of sigma_d allows applying the tool to any
%     neuroimaging papers, existing results.
% - Quality Control impact findings => a tool to find potentially unreliable
%   results
% - To be general, we developped an analytical model
% - Refers to the online tool on yohanchatelain.github.io/brain-render

% \TG[done]{Instead of re-stating the importance of numerical variability in this
%     paragraph, which supposedly should be understood from the end of the
%     introduction, here you could explain the need for a tool to quickly and
%     practically evaluate its impact in a given study, explain and justify your
%     assumptions (e.g., numerical variability is a feature of the pipeline rather
%     than the population---refer to previous works). In doing so, you could explain
%     how running MCA is currently not realistic due to computational requirements
%     (although new architectures may enable it in the coming years), and explain the
%     need for a statistical correction.}

The previous results underscore the importance of quantifying the impact of
numerical variability in MRI analyses, as such variability can substantially
affect statistical inference. However, evaluating numerical variability directly
is highly compute-intensive and impractical to perform routinely. We developed a
practical analytical model of numerical variability in MRI analyses, assuming
that this variability is an inherent property of the computational pipeline and
generalizes across populations. To do so, we derived closed-form approximations
of numerical uncertainty for common statistics and their associated p-values,
which can be readily incorporated into MRI analyses
(Table~\ref{tab:stat_uncertainty}). These formulas rely on the
Numerical-Anatomical Variability Ratio (\navr) that represents the relative
magnitude of numerical variability ($\sigma_{\mathrm{num}}$;
Eq.~\eqref{eq:sigma_num}) to population variability ($\sigma_{\mathrm{anat}}$;
Eq.~\eqref{eq:sigma_anat}):
$$\nu_{\mathrm{nav}} = \frac{\sigma_{\mathrm{num}}}{\sigma_{\mathrm{pop}}}$$
\navr quantifies the intrinsic numerical instability of a neuroimaging pipeline,
allowing comparison across regions and studies. The formulas in
Table~\ref{tab:stat_uncertainty} were obtained by propagating the numerical
variability through the estimators using the delta-method (see
Section~\ref{sec:theoretical_derivations}) and were verified through numerical
simulations.

This analytical framework provides a practical link between the pipeline's
numerical instability (\navr), the study's sample size ($n$), and the
resulting uncertainty in effect sizes and $p$-values. Because the formulas
rely only on the study's summary statistics, they can be used to
assess the potential impact of numerical variability on existing
studies without requiring costly recomputation or access to the original data.

% \TG[done]{I think I would rather present the $\sigma_d$ result first, then explain
%     NAV. When introducing sigma d, you could better explain that numerical
%     variability has to be evaluated in the context of a particular effect size, and its impact will be dependent on sample size.}

\begin{table}[ht]
    \centering
    \begin{tabular}{lll}
        \toprule
        \textbf{Statistic}
                            & \textbf{Numerical standard deviation}
                            & \textbf{Numerical p-value uncertainty}                                                                                                           \\
        \midrule
        Cohen's $d$         & $\sigma_d \approx\nu_{\mathrm{nav}}\frac{2}{\sqrt{n}}$             & \multicolumn{1}{c}{-}                                                       \\
        Two-sample $t$      & $\sigma_t \approx \nu_{\mathrm{nav}}        $                      & $\sigma_{p} \approx 2f_{t,df}(|t|)\nu_{\mathrm{nav}}$                       \\
        Partial correlation & $\sigma_r \geq \nu_{\mathrm{nav}}\sqrt{\frac{(1-r^{2})^{3}}{n-1}}$ & $\sigma_{p} \geq 2f_{t,df}(|t|)\sqrt{\frac{df}{n-1}}\nu_{\mathrm{nav}}$     \\
        ANCOVA              & $\sigma_F \approx 2\sqrt{F}\nu_{\mathrm{nav}}$                     & $\sigma_{p} \approx 2\sqrt{F}f_{\mathcal{F}}(F;1,df_2)\,\nu_{\mathrm{nav}}$ \\
        \bottomrule
    \end{tabular}
    \caption{First-order numerical uncertainty of common statistical tests under
        Monte Carlo Arithmetic perturbations. Cohen's d formula assumes large
        and equal group sizes. $f_{t,df}$ and $f_\mathcal{F}(F;1,df_2)$ denote
        the probability density functions of the Student's $t$-distribution with
        $df$ degrees of freedom and the $\mathcal{F}$-distribution with $(1,
            df_2)$ degrees of freedom, respectively. The $p$-value approximation for
        the partial correlation uses $t=r(df/(1-r^2))^{1/2}$.}
    \label{tab:stat_uncertainty}
\end{table}


We measured $\nu_{nav}$ in the PPMI-extracted dataset used in the previous
section for HC and PD (Figure~\ref{fig:navr}). \TG{Revise the following text
    when Figure 4 is final} Across cortical and subcortical regions, we found
numerical variability accounted for up to 29-45\% of the observed anatomical
variance in baseline analyses among both HC and PD groups. Longitudinal analyses
exhibited substantially higher \navr values with up to 68-89\%. At baseline,
mean \navr values reached 0.19 for HC (0.21 for PD) for cortical thickness, 0.18
(0.19) for subcortical volumes, and 0.43 (0.43) and (0.53) 0.58 for longitudinal
studies, indicating that numerical imprecision constitutes a portion of the
anatomical signal. No significant differences in mean \navr were observed
between HC and PD groups (Bootstrap with 10,000 samples; Appendix
Table~\ref{tab:bootstrap-navr}), confirming that numerical variability is
primarily a property of the computational pipeline rather than the population
being studied. Higher \navr values in longitudinal analyses suggest that
tracking changes over time is more susceptible to numerical noise, likely due to
catastrophic cancellation when computing differences between timepoints.
Catastrophic cancellation occurs when subtracting two nearly equal numbers,
leading to a significant loss of precision by promoting rounding errors. These
values have direct implications: for a typical \navr of 0.2, achieving a
negligible numerical uncertainty on Cohen's d effect size ($\sigma_d \leq 0.01$)
would require a sample size of at least 1,500 participants
(Fig.~\ref{fig:sigma_d_contour}).

\begin{figure}[h]
    \centering
    \includegraphics[width=\linewidth]{figures/NAVR_map/NAVR_thickness_subcortical_volumes.png}
    \caption{Numerical-Anatomical Variability Ratio (\navr) for subcortical volumes (top row in each panel) and cortical
        thickness (bottom row in each panel) in healthy controls (HC) and patients with
        Parkinson's disease (PD). Panels show \navr maps for HC at baseline (a), PD
        at baseline (b), HC longitudinally between follow-up and baseline scans
        (c), and PD longitudinally (d). Higher \navr values indicate greater
        computational uncertainty relative to inter-subject anatomical variability.
        Warmer colors correspond to higher \navr values.}
    \label{fig:navr}
\end{figure}

\begin{figure}
    \includegraphics[width=\linewidth]{figures/sigma_d_contour.pdf}
    \caption{\TG{could you show some of the papers in this figure? } Relationship between \navr and population sample size  \(N\) for
        predicting the uncertainty in Cohen's d effect size estimation. The
        contour lines represent different \navr values, showing how numerical
        variability scales with sample size. With a typical \navr value of 0.2,
        to maintain reliable effect size estimates $\sigma_d \leq 0.01$, the
        plot suggests to use $N \geq 1500$.\label{fig:sigma_d_contour}}
\end{figure}

To facilitate the use of our analytical framework, we developed an interactive,
public web tool available at
\href{https://yohanchatelain.github.io/brain\_render/}{yohanchatelain.github.io/brain\_render}.
(Figure~\ref{fig:brain_render_tool}). By making numerical-variability evaluation
accessible, our framework and accompanying tool help improve transparency,
support peer review, and promote more reliable statistical inference in
neuroimaging.
\begin{figure}[ht]
    \centering
    \includegraphics[width=\linewidth]{figures/screenshot_light.png}
    \caption{Interactive web tool for estimating NAVR and assessing numerical
        variability in neuroimaging studies. Users can input summary statistics
        to obtain NAVR values and visualize the impact of numerical variability
        on effect size estimates. The tool is available at
        \href{https://yohanchatelain.github.io/brain\_render/}{yohanchatelain.github.io/brain\_render}.}
    \label{fig:brain_render_tool}
\end{figure}



\section{Re-evaluating landmark studies reveals widespread potential for unreliable effect sizes \TG{This title is over-emphasized given the results that are currently presented}}

To assess the broader implications of numerical variability, we applied numerical uncertainty propagation (Table~\ref{tab:stat_uncertainty}) to
re-evaluate findings from X articles. Applying \navr-based thresholding to ENIGMA's published brain maps
identified multiple regions where significance of reported p-values flipped.
\TG{Assuming that you're referring to figure 6, this is not so clear. All the
    masked areas are areas with low effect sizes compared to the other ones. On the
    contrary, it seems to me that this is a case where numerical variability
    wouldn't impact the findings much, most likely because N was high. I would
    suggest to include another example, possibly with a lower N, showing a larger
    impact of numerical variability.}

Figures~\ref{fig:navr_enigma_thickness} and~\ref{fig:navr_enigma_subcortical}
illustrate the impact of applying NAVR thresholds to cortical thickness and
subcortical volume maps from ENIGMA. Regions rendered in black indicate areas
where reported effect sizes were smaller than numerical variability, suggesting
these findings should be interpreted with caution. \TG{same comment as in the
    previous figure}

This observation highlights potential risks of overestimating small effects
\TG{I don't think that the results presented show that clearly} and underscores
the importance of systematically accounting for numerical uncertainty in
neuroimaging research. While ENIGMA's primary findings generally remained
robust due to large sample sizes, our analysis indicates that numerous
secondary, smaller-scale effects reported in the literature could be
compromised by numerical instability \TG{I think you should show examples of
    such studies}.

\begin{figure}[h]
    \centering
    \vspace{0.2cm}

    % Header row with column labels
    \begin{minipage}[b]{\linewidth}
        \begin{minipage}[c]{0.05\linewidth}
            % Empty space for alignment with condition labels
        \end{minipage}%
        \begin{minipage}[c]{0.95\linewidth}
            \begin{minipage}[c]{0.47\linewidth}
                \centering\textbf{Unthresholded}
            \end{minipage}%
            \hfill
            \begin{minipage}[c]{0.005\linewidth}
                % Vertical line separator
            \end{minipage}%
            \hfill
            \begin{minipage}[c]{0.47\linewidth}
                \centering\textbf{Thresholded}
            \end{minipage}
        \end{minipage}
    \end{minipage}

    % Horizontal line
    \noindent\rule{\linewidth}{0.5pt}
    \vspace{-1.5cm}

    \begin{minipage}[b]{\linewidth}
        \begin{minipage}[c]{0.05\linewidth}
            \centering\rotatebox{90}{\textbf{L1}}
        \end{minipage}%
        \begin{minipage}[c]{0.95\linewidth}
            \begin{subfigure}[c]{0.47\linewidth}
                \includegraphics[width=\linewidth]{figures/cohen_d_map/Laansma_2021/parkinson/png/parkinson_area_HY3-vs-HY4+5.png}
                \label{fig:Laansma_2021_parkinson_area_HY3_vs_HY45_unthresholded}
            \end{subfigure}%
            \hfill
            \begin{minipage}[c]{0.005\linewidth}
                \centering\rule{0.5pt}{4cm}
            \end{minipage}%
            \hfill
            \begin{subfigure}[c]{0.47\linewidth}
                \includegraphics[width=\linewidth]{figures/cohen_d_map/Laansma_2021/parkinson/png/parkinson_area_HY3-vs-HY4+5_thresholded.png}
                \label{fig:Laansma_2021_parkinson_area_HY3_vs_HY45_thresholded}
            \end{subfigure}
        \end{minipage}
    \end{minipage}

    \vspace{-2cm}

    \begin{minipage}[b]{\linewidth}
        \begin{minipage}[c]{0.05\linewidth}
            \centering\rotatebox{90}{\textbf{L2}} \end{minipage}%
        \begin{minipage}[c]{0.95\linewidth}
            \begin{subfigure}[c]{0.47\linewidth}
                \includegraphics[width=\linewidth]{figures/cohen_d_map/Laansma_2021/parkinson/png/parkinson_area_HY1-vs-HY2.png}
                \label{fig:Laansma_2021_parkinson_area_HY1_vs_HY2_unthresholded}
            \end{subfigure}%
            \hfill
            \begin{minipage}[c]{0.005\linewidth}
                \centering\rule{0.5pt}{4cm}
            \end{minipage}%
            \hfill
            \begin{subfigure}[c]{0.47\linewidth}
                \includegraphics[width=\linewidth]{figures/cohen_d_map/Laansma_2021/parkinson/png/parkinson_area_HY1-vs-HY2_thresholded.png}
                \label{fig:Laansma_2021_parkinson_area_HY1_vs_HY2_thresholded}
            \end{subfigure}
        \end{minipage}
    \end{minipage}

    \vspace{-2cm}

    \begin{minipage}[b]{\linewidth}
        \begin{minipage}[c]{0.05\linewidth}
            \centering\rotatebox{90}{\textbf{L3}} \end{minipage}%
        \begin{minipage}[c]{0.95\linewidth}
            \begin{subfigure}[c]{0.47\linewidth}
                \includegraphics[width=\linewidth]{figures/cohen_d_map/Laansma_2021/parkinson/png/parkinson_area_HY1-vs-PD.png}
                \label{fig:Laansma_2021_parkinson_area_HY1_vs_PD_unthresholded}
            \end{subfigure}%
            \hfill
            \begin{minipage}[c]{0.005\linewidth}
                \centering\rule{0.5pt}{4cm}
            \end{minipage}%
            \hfill
            \begin{subfigure}[c]{0.47\linewidth}
                \includegraphics[width=\linewidth]{figures/cohen_d_map/Laansma_2021/parkinson/png/parkinson_area_HY1-vs-PD_thresholded.png}
                \label{fig:Laansma_2021_parkinson_area_HY1_vs_PD_thresholded}
            \end{subfigure}
        \end{minipage}
    \end{minipage}
    % \vspace{-2cm}

    \caption{\TG{can you add a color bar to the figure?} ENIGMA Parkinson group \TG{wdym?}. LSA* \TG{what's LSA*?} Laansma et
        al.~\cite{laansma2021international}, Cortical Surface Area. L1: HY 3 vs
        HY 4\%5; L2: HY 1 vs HY 2; L3: HY 1 vs PD. Cohen's d maps showing
        unthresholded effect sizes (left) and effect sizes thresholded by the
        \navr framework (right). Black regions indicate areas where Cohen's d
        values fall below the numerical variability threshold, demonstrating
        regions where reported effect sizes may be unreliable due to
        computational uncertainty.\label{fig:navr_laansma}}
\end{figure}

% ---

\begin{figure}[h]
    \centering
    \vspace{0.2cm}

    % Header row with column labels
    \begin{minipage}[b]{\linewidth}
        \begin{minipage}[c]{0.05\linewidth}
            % Empty space for alignment with condition labels
        \end{minipage}%
        \begin{minipage}[c]{0.95\linewidth}
            \begin{minipage}[c]{0.47\linewidth}
                \centering\textbf{Unthresholded}
            \end{minipage}%
            \hfill
            \begin{minipage}[c]{0.005\linewidth}
                % Vertical line separator
            \end{minipage}%
            \hfill
            \begin{minipage}[c]{0.47\linewidth}
                \centering\textbf{Thresholded}
            \end{minipage}
        \end{minipage}
    \end{minipage}

    % Horizontal line
    \noindent\rule{\linewidth}{0.5pt}
    \vspace{-1.5cm}

    % 22q11.2 deletion syndrome row
    \begin{minipage}[b]{\linewidth}
        \begin{minipage}[c]{0.05\linewidth}
            \centering\rotatebox{90}{\textbf{22q11.2}} \end{minipage}%
        \begin{minipage}[c]{0.95\linewidth}
            \begin{subfigure}[c]{0.47\linewidth}
                \includegraphics[width=\linewidth]{figures/cohen_d_map/enigma/22q_thickness_all.png}
                \label{fig:enigma_22q_thickness_unthresholded}
            \end{subfigure}%
            \hfill
            \begin{minipage}[c]{0.005\linewidth}
                \centering\rule{0.5pt}{4cm}
            \end{minipage}%
            \hfill
            \begin{subfigure}[c]{0.47\linewidth}
                \includegraphics[width=\linewidth]{figures/cohen_d_map/enigma/22q_thickness_all_thresholded.png}
                \label{fig:enigma_22q_thickness_thresholded}
            \end{subfigure}
        \end{minipage}
    \end{minipage}

    \vspace{-2cm}

    % ADHD row
    \begin{minipage}[b]{\linewidth}
        \begin{minipage}[c]{0.05\linewidth}
            \centering\rotatebox{90}{\textbf{ADHD}} \end{minipage}%
        \begin{minipage}[c]{0.95\linewidth}
            \begin{subfigure}[c]{0.47\linewidth}
                \includegraphics[width=\linewidth]{figures/cohen_d_map/enigma/adhd_thickness_adult.png}
                \label{fig:enigma_adhd_thickness_unthresholded}
            \end{subfigure}%
            \hfill
            \begin{minipage}[c]{0.005\linewidth}
                \centering\rule{0.5pt}{4cm}
            \end{minipage}%
            \hfill
            \begin{subfigure}[c]{0.47\linewidth}
                \includegraphics[width=\linewidth]{figures/cohen_d_map/enigma/adhd_thickness_adult_thresholded.png}
                \label{fig:enigma_adhd_thickness_thresholded}
            \end{subfigure}
        \end{minipage}
    \end{minipage}

    \vspace{-2cm}

    % Autism spectrum disorder row
    \begin{minipage}[b]{\linewidth}
        \begin{minipage}[c]{0.05\linewidth}
            \centering\rotatebox{90}{\textbf{ASD}} \end{minipage}%
        \begin{minipage}[c]{0.95\linewidth}
            \begin{subfigure}[c]{0.47\linewidth}
                \includegraphics[width=\linewidth]{figures/cohen_d_map/enigma/asd_thickness_meta_analysis.png}
                \label{fig:enigma_asd_thickness_unthresholded}
            \end{subfigure}%
            \hfill
            \begin{minipage}[c]{0.005\linewidth}
                \centering\rule{0.5pt}{4cm}
            \end{minipage}%
            \hfill
            \begin{subfigure}[c]{0.47\linewidth}
                \includegraphics[width=\linewidth]{figures/cohen_d_map/enigma/asd_thickness_meta_analysis_thresholded.png}
                \label{fig:enigma_asd_thickness_thresholded}
            \end{subfigure}
        \end{minipage}
    \end{minipage}
    \vspace{-2cm}

    % bipolar disorder row
    \begin{minipage}[b]{\linewidth}
        \begin{minipage}[c]{0.05\linewidth}
            \centering\rotatebox{90}{\textbf{Bipolar}} \end{minipage}%
        \begin{minipage}[c]{0.95\linewidth}
            \begin{subfigure}[c]{0.47\linewidth}
                \includegraphics[width=\linewidth]{figures/cohen_d_map/enigma/bipolar_thickness_adult.png}
                \label{fig:enigma_bipolar_thickness_unthresholded}
            \end{subfigure}%
            \hfill
            \begin{minipage}[c]{0.005\linewidth}
                \centering\rule{0.5pt}{4cm}
            \end{minipage}%
            \hfill
            \begin{subfigure}[c]{0.47\linewidth}
                \includegraphics[width=\linewidth]{figures/cohen_d_map/enigma/bipolar_thickness_adult_thresholded.png}
                \label{fig:enigma_bipolar_thickness_thresholded}
            \end{subfigure}
        \end{minipage}
    \end{minipage}

    \vspace{-2cm}
    % depression
    \begin{minipage}[b]{\linewidth}
        \begin{minipage}[c]{0.05\linewidth}
            \centering\rotatebox{90}{\textbf{Depression}} \end{minipage}%
        \begin{minipage}[c]{0.95\linewidth}
            \begin{subfigure}[c]{0.47\linewidth}
                \includegraphics[width=\linewidth]{figures/cohen_d_map/enigma/depression_thickness_adult.png}
                \label{fig:enigma_depression_thickness_unthresholded}
            \end{subfigure}%
            \hfill
            \begin{minipage}[c]{0.005\linewidth}
                \centering\rule{0.5pt}{4cm}
            \end{minipage}%
            \hfill
            \begin{subfigure}[c]{0.47\linewidth}
                \includegraphics[width=\linewidth]{figures/cohen_d_map/enigma/depression_thickness_adult_thresholded.png}
                \label{fig:enigma_depression_thickness_thresholded}
            \end{subfigure}
        \end{minipage}
    \end{minipage}

    \vspace{-2cm}
    % epilepsy
    \begin{minipage}[b]{\linewidth}
        \begin{minipage}[c]{0.05\linewidth}
            \centering\rotatebox{90}{\textbf{Epilepsy}} \end{minipage}%
        \begin{minipage}[c]{0.95\linewidth}
            \begin{subfigure}[c]{0.47\linewidth}
                \includegraphics[width=\linewidth]{figures/cohen_d_map/enigma/epilepsy_thickness_allepilepsy.png}
                \label{fig:enigma_epilepsy_thickness_unthresholded}
            \end{subfigure}%
            \hfill
            \begin{minipage}[c]{0.005\linewidth}
                \centering\rule{0.5pt}{4cm}
            \end{minipage}%
            \hfill
            \begin{subfigure}[c]{0.47\linewidth}
                \includegraphics[width=\linewidth]{figures/cohen_d_map/enigma/epilepsy_thickness_allepilepsy_thresholded.png}
                \label{fig:enigma_epilepsy_thickness_thresholded}
            \end{subfigure}
        \end{minipage}
    \end{minipage}

    \vspace{-2cm}
    % ocd 
    \begin{minipage}[b]{\linewidth}
        \begin{minipage}[c]{0.05\linewidth}
            \centering\rotatebox{90}{\textbf{OCD}} \end{minipage}%
        \begin{minipage}[c]{0.95\linewidth}
            \begin{subfigure}[c]{0.47\linewidth}
                \includegraphics[width=\linewidth]{figures/cohen_d_map/enigma/ocd_thickness_adult.png}
                \label{fig:enigma_ocd_thickness_unthresholded}
            \end{subfigure}%
            \hfill
            \begin{minipage}[c]{0.005\linewidth}
                \centering\rule{0.5pt}{4cm}
            \end{minipage}%
            \hfill
            \begin{subfigure}[c]{0.47\linewidth}
                \includegraphics[width=\linewidth]{figures/cohen_d_map/enigma/ocd_thickness_adult_thresholded.png}
                \label{fig:enigma_ocd_thickness_thresholded}
            \end{subfigure}
        \end{minipage}
    \end{minipage}

    \vspace{-2cm}
    % schizophrenia
    \begin{minipage}[b]{\linewidth}
        \begin{minipage}[c]{0.05\linewidth}
            \centering\rotatebox{90}{\textbf{Schizophrenia}} \end{minipage}%
        \begin{minipage}[c]{0.95\linewidth}
            \begin{subfigure}[c]{0.47\linewidth}
                \includegraphics[width=\linewidth]{figures/cohen_d_map/enigma/schizophrenia_thickness_all.png}
                \label{fig:enigma_schizophrenia_thickness_unthresholded}
            \end{subfigure}%
            \hfill
            \begin{minipage}[c]{0.005\linewidth}
                \centering\rule{0.5pt}{4cm}
            \end{minipage}%
            \hfill
            \begin{subfigure}[c]{0.47\linewidth}
                \includegraphics[width=\linewidth]{figures/cohen_d_map/enigma/schizophrenia_thickness_all_thresholded.png}
                \label{fig:enigma_schizophrenia_thickness_thresholded}
            \end{subfigure}
        \end{minipage}
    \end{minipage}

    \caption{ENIGMA cortical thickness Cohen's d maps showing unthresholded
        effect sizes (left) and effect sizes thresholded by the \navr framework
        (right) for different disorders. Black regions indicate areas where
        Cohen's d values fall below the numerical variability threshold,
        demonstrating regions where reported effect sizes may be unreliable due
        to computational uncertainty.\label{fig:navr_enigma_thickness}}
\end{figure}

\begin{figure}[h]
    \centering
    \vspace{0.2cm}
    % Header row with column labels
    \begin{minipage}[b]{\linewidth}
        \begin{minipage}[c]{0.05\linewidth}
            % Empty space for alignment with condition labels
        \end{minipage}%
        \begin{minipage}[c]{0.95\linewidth}
            \begin{minipage}[c]{0.47\linewidth}
                \centering\textbf{Unthresholded}
            \end{minipage}%
            \hfill
            \begin{minipage}[c]{0.005\linewidth}
                % Vertical line separator
            \end{minipage}%
            \hfill
            \begin{minipage}[c]{0.47\linewidth}
                \centering\textbf{Thresholded}
            \end{minipage}
        \end{minipage}
    \end{minipage}

    % Horizontal line
    \noindent\rule{\linewidth}{0.5pt}
    \vspace{-1.5cm}

    % 22q11.2 deletion syndrome row
    \begin{minipage}[b]{\linewidth}
        \begin{minipage}[c]{0.05\linewidth}
            \centering\rotatebox{90}{\textbf{22q11.2}} \end{minipage}%
        \begin{minipage}[c]{0.95\linewidth}
            \begin{subfigure}[c]{0.47\linewidth}
                \includegraphics[width=\linewidth]{figures/cohen_d_map/enigma/22q_subcortical_volume_all.png}
                \label{fig:enigma_22q_unthresholded_subcortical}
            \end{subfigure}%
            \hfill
            \begin{minipage}[c]{0.005\linewidth}
                \centering\rule{0.5pt}{4cm}
            \end{minipage}%
            \hfill
            \begin{subfigure}[c]{0.47\linewidth}
                \includegraphics[width=\linewidth]{figures/cohen_d_map/enigma/22q_subcortical_volume_all_thresholded.png}
                \label{fig:enigma_22q_thresholded_subcortical}
            \end{subfigure}
        \end{minipage}
    \end{minipage}

    \vspace{-2cm}

    % ADHD row
    \begin{minipage}[b]{\linewidth}
        \begin{minipage}[c]{0.05\linewidth}
            \centering\rotatebox{90}{\textbf{ADHD}} \end{minipage}%
        \begin{minipage}[c]{0.95\linewidth}
            \begin{subfigure}[c]{0.47\linewidth}
                \includegraphics[width=\linewidth]{figures/cohen_d_map/enigma/adhd_subcortical_volume_adult.png}
                \label{fig:enigma_adhd_unthresholded_subcortical}
            \end{subfigure}%
            \hfill
            \begin{minipage}[c]{0.005\linewidth}
                \centering\rule{0.5pt}{4cm}
            \end{minipage}%
            \hfill
            \begin{subfigure}[c]{0.47\linewidth}
                \includegraphics[width=\linewidth]{figures/cohen_d_map/enigma/adhd_subcortical_volume_adult_thresholded.png}
                \label{fig:enigma_adhd_thresholded_subcortical}
            \end{subfigure}
        \end{minipage}
    \end{minipage}

    \vspace{-2cm}

    % Autism spectrum disorder row
    \begin{minipage}[b]{\linewidth}
        \begin{minipage}[c]{0.05\linewidth}
            \centering\rotatebox{90}{\textbf{ASD}} \end{minipage}%
        \begin{minipage}[c]{0.95\linewidth}
            \begin{subfigure}[c]{0.47\linewidth}
                \includegraphics[width=\linewidth]{figures/cohen_d_map/enigma/asd_subcortical_volume_meta_analysis.png}
                \label{fig:enigma_asd_unthresholded_subcortical}
            \end{subfigure}%
            \hfill
            \begin{minipage}[c]{0.005\linewidth}
                \centering\rule{0.5pt}{4cm}
            \end{minipage}%
            \hfill
            \begin{subfigure}[c]{0.47\linewidth}
                \includegraphics[width=\linewidth]{figures/cohen_d_map/enigma/asd_subcortical_volume_meta_analysis_thresholded.png}
                \label{fig:enigma_asd_thresholded_subcortical}
            \end{subfigure}
        \end{minipage}
    \end{minipage}
    \vspace{-2cm}

    % bipolar disorder row
    \begin{minipage}[b]{\linewidth}
        \begin{minipage}[c]{0.05\linewidth}
            \centering\rotatebox{90}{\textbf{Bipolar}} \end{minipage}%
        \begin{minipage}[c]{0.95\linewidth}
            \begin{subfigure}[c]{0.47\linewidth}
                \includegraphics[width=\linewidth]{figures/cohen_d_map/enigma/bipolar_subcortical_volume_typeII.png}
                \label{fig:enigma_bipolar_unthresholded_subcortical}
            \end{subfigure}%
            \hfill
            \begin{minipage}[c]{0.005\linewidth}
                \centering\rule{0.5pt}{4cm}
            \end{minipage}%
            \hfill
            \begin{subfigure}[c]{0.47\linewidth}
                \includegraphics[width=\linewidth]{figures/cohen_d_map/enigma/bipolar_subcortical_volume_typeII_thresholded.png}
                \label{fig:enigma_bipolar_thresholded_subcortical}
            \end{subfigure}
        \end{minipage}
    \end{minipage}

    \vspace{-2cm}
    % depression
    \begin{minipage}[b]{\linewidth}
        \begin{minipage}[c]{0.05\linewidth}
            \centering\rotatebox{90}{\textbf{Depression}} \end{minipage}%
        \begin{minipage}[c]{0.95\linewidth}
            \begin{subfigure}[c]{0.47\linewidth}
                \includegraphics[width=\linewidth]{figures/cohen_d_map/enigma/depression_subcortical_volume_all.png}
                \label{fig:enigma_depression_unthresholded_subcortical}
            \end{subfigure}%
            \hfill
            \begin{minipage}[c]{0.005\linewidth}
                \centering\rule{0.5pt}{4cm}
            \end{minipage}%
            \hfill
            \begin{subfigure}[c]{0.47\linewidth}
                \includegraphics[width=\linewidth]{figures/cohen_d_map/enigma/depression_subcortical_volume_all_thresholded.png}
                \label{fig:enigma_depression_thresholded_subcortical}
            \end{subfigure}
        \end{minipage}
    \end{minipage}

    \vspace{-2cm}
    % epilepsy
    \begin{minipage}[b]{\linewidth}
        \begin{minipage}[c]{0.05\linewidth}
            \centering\rotatebox{90}{\textbf{Epilepsy}} \end{minipage}%
        \begin{minipage}[c]{0.95\linewidth}
            \begin{subfigure}[c]{0.47\linewidth}
                \includegraphics[width=\linewidth]{figures/cohen_d_map/enigma/epilepsy_subcortical_volume_allepilepsy.png}
                \label{fig:enigma_epilepsy_unthresholded_subcortical}
            \end{subfigure}%
            \hfill
            \begin{minipage}[c]{0.005\linewidth}
                \centering\rule{0.5pt}{4cm}
            \end{minipage}%
            \hfill
            \begin{subfigure}[c]{0.47\linewidth}
                \includegraphics[width=\linewidth]{figures/cohen_d_map/enigma/epilepsy_subcortical_volume_allepilepsy_thresholded.png}
                \label{fig:enigma_epilepsy_thresholded_subcortical}
            \end{subfigure}
        \end{minipage}
    \end{minipage}

    \vspace{-2cm}
    % ocd 
    \begin{minipage}[b]{\linewidth}
        \begin{minipage}[c]{0.05\linewidth}
            \centering\rotatebox{90}{\textbf{OCD}} \end{minipage}%
        \begin{minipage}[c]{0.95\linewidth}
            \begin{subfigure}[c]{0.47\linewidth}
                \includegraphics[width=\linewidth]{figures/cohen_d_map/enigma/ocd_subcortical_volume_adult.png}
                \label{fig:enigma_ocd_unthresholded_subcortical}
            \end{subfigure}%
            \hfill
            \begin{minipage}[c]{0.005\linewidth}
                \centering\rule{0.5pt}{4cm}
            \end{minipage}%
            \hfill
            \begin{subfigure}[c]{0.47\linewidth}
                \includegraphics[width=\linewidth]{figures/cohen_d_map/enigma/ocd_subcortical_volume_adult_thresholded.png}
                \label{fig:enigma_ocd_thresholded_subcortical}
            \end{subfigure}
        \end{minipage}
    \end{minipage}

    \vspace{-2cm}
    % schizophrenia
    \begin{minipage}[b]{\linewidth}
        \begin{minipage}[c]{0.05\linewidth}
            \centering\rotatebox{90}{\textbf{Schizophrenia}} \end{minipage}%
        \begin{minipage}[c]{0.95\linewidth}
            \begin{subfigure}[c]{0.47\linewidth}
                \includegraphics[width=\linewidth]{figures/cohen_d_map/enigma/schizophrenia_subcortical_volume_all.png}
                \label{fig:enigma_schizophrenia_unthresholded_subcortical}
            \end{subfigure}%
            \hfill
            \begin{minipage}[c]{0.005\linewidth}
                \centering\rule{0.5pt}{4cm}
            \end{minipage}%
            \hfill
            \begin{subfigure}[c]{0.47\linewidth}
                \includegraphics[width=\linewidth]{figures/cohen_d_map/enigma/schizophrenia_subcortical_volume_all_thresholded.png}
                \label{fig:enigma_schizophrenia_thresholded_subcortical}
            \end{subfigure}
        \end{minipage}
    \end{minipage}

    \caption{ ENIGMA subcortical volume Cohen's d maps showing unthresholded
        effect sizes (left) and effect sizes thresholded by the \navr framework
        (right) for different disorders. Black regions indicate areas where
        Cohen's d values fall below the numerical variability threshold,
        demonstrating regions where reported effect sizes may be unreliable due
        to computational uncertainty.}
    \label{fig:navr_enigma_subcortical}
\end{figure}

\section{Discussion}

% Comments:
% 1. Summarize the main results and what do they mean
% 2. Extension beyond FreeSurfer and expectation to generalize the findings to other neuroimaging software
% 3. Discuss the potential sources of numerical variability (minimal local, minimal precision, etc.)

% [] Mention using neuroimaging as biomarker (n=1 scenario), personalized medicine,

Our systematic perturbation of FreeSurfer revealed that numerical variability
alone can account for up to 30\% \TG{check value, can we report the average
    $\nu_{nav}$ observed in PD participants?} of the anatomical variability observed
in structural MRI measurements. This level of uncertainty can significantly
impact statistical outcomes in Parkinson's disease analyses, leading to the
appearance or disappearance of group differences or correlations depending
solely on computational conditions. These findings offer a mechanistic
explanation for some of the reproducibility challenges reported in clinical
neuroimaging.

To facilitate numerical evaluations in previous and future studies, we
introduced the Numerical-Anatomical Variability Ratio (NAVR), a quantitative
framework for assessing the relative magnitude of computational noise. By
establishing a theoretical link between NAVR and the uncertainty in common
statistics, we provided a practical tool to assess the numerical robustness of MRI measures.
Our re-analysis of published ENIGMA results illustrates this utility: while
large sample sizes confer robustness to core findings, many secondary effects
fall below the computational noise floor. This suggests that in exploratory
studies with lower sample size, numerical instability may undermine the
reliability of reported effects. \TG{revise this part when results are available}

Although our primary analysis focused on FreeSurfer 7.3.1 and Parkinson's
disease, the underlying numerical issues are more general.
Previous analyses of FSL~\cite{mirhakimi2025numerical} and ANTs~\YC{cite
    Mathieu} indicate that such instability is not unique to FreeSurfer but likely
pervades the field. SPM however seems to be less impacted by numerical
variability, possibly due to the use of Bayesian optimization~\cite{mirhakimi2025numerical}.

Traditional image processing methods rely on nonlinear optimization
procedures that can converge to different local minima under small
perturbations, resulting in substantive changes to derived measures. To address
this problem and decrease the execution time, the neuroimaging field is
increasingly shifting toward deep learning models. The latest FreeSurfer
release (v8), for example, now incorporates Deep Learning (DL) models such as
FastSurfer~\cite{henschel2020fastsurfer} and
Synthmorph~\cite{hoffmann2021synthmorph} to replace its classical segmentation
and registration steps. However, this shift does not eliminate the problem of
instability but rather reframes it. While these DL models have been shown
stable during the inference stage~\cite{pepe2023numerical}, their training
process is subject to its own sources of variability. Factors like weight
initialization and floating-point precision can cause different training runs
to yield distinct models with varying performance, and their quantification
remains an open question in neuroimaging. This phenomenon is analogous to the
local minima issue in classical optimization. Ultimately, whether arising from
classical optimization or DL training, such instability means that even
identical inputs can lead to divergent interpretations, raising critical
concerns for research reproducibility and clinical translation. \TG{Reword paragraph to refer to Ines' latest results on FastSurfer training}

To address the generalizability of our findings, we considered the
characteristics of our data cohorts. A potential limitation is that our
Parkinson's Disease cohort was relatively homogeneous in age and phenotype,
which could reduce anatomical variance and consequently inflate NAVR values.
However, direct statistical analyses revealed no significant differences in
numerical variability between the PD and healthy control groups (see
Supplementary Fig. X). This key finding supports the idea that the pipeline's
instability is a consistent factor and that our results are likely to generalize
across these populations.\TG[done]{I would moderate this statement a little bit.
    I think it would be interesting the measure the NAV in different pipelines and
    datasets, but I don't think there's a very strong need for that.} Measuring the
NAVR across more diverse datasets, software packages, and disease contexts would
nevertheless be a valuable step. Such work would help build a more comprehensive
map of computational reliability across the entire neuroimaging landscape.

NAVR provides a scalable, interpretable metric to quantify hidden numerical
variability. Although floating-point rounding is a dominant source of
instability, future work should broaden this analysis to other contributors,
including algorithmic choices, preprocessing decisions, and data handling
practices. A comprehensive understanding of these factors is essential for
developing numerically robust software. Our results show that computational
uncertainty is as critical as statistical uncertainty in neuroimaging;
systematic assessments of numerical variability, exemplified by NAVR, are
therefore necessary to ensure the reproducibility and reliability of
neuroimaging-based biomarkers. Extending this quantification to the
deep-learning training stage is equally important, given the field's central
role in modern neuroimaging, and would support more robust and interpretable
models. Likewise, evaluating numerical variability in the classical
optimization schemes used in non-linear registration is a key milestone, as
these traditional tools often provide the reference for training deep-learning
models. Advancing both lines of analysis will benefit conventional and
learning-based approaches alike. \TG[done]{I would merge this paragraph with
    the previous one, and highlight specific next steps regarding numerical
    variability in neuroimaging.}

\section{Methods}

\TG{include a summary of your methods here.}

\TG{Overall the methods are quite brief, you should add more details so that
    people get a better sense of what you did, see detailed suggestions in the text}

\subsection{Participants}

Structural MRI data were obtained from the Parkinson's Progression Markers
Initiative (PPMI; \url{www.ppmi-info.org}). The
study included 201 participants: 112 individuals diagnosed with Parkinson's
disease without mild cognitive impairment (PD-non-MCI) and 89 healthy controls
(HC). All participants had two usable T1-weighted MRI scans acquired
approximately $1.4 \pm 0.5$ years apart ($0.9$-$2.0$ years). Patients with mild
cognitive impairment were excluded to minimize confounding effects of cognitive
decline.

Inclusion criteria were: (i) diagnosis of idiopathic Parkinson's disease
(PD-non-MCI) or healthy control status; (ii) availability of two high-quality
T1-weighted scans at distinct visits; and (iii) absence of other neurological or
psychiatric conditions. PD severity was quantified using the Unified Parkinson's
Disease Rating Scale part III (UPDRS-III) in the OFF medication state at both
baseline and follow-up visits.

All procedures were approved by the research ethics boards of participating PPMI
sites, and written informed consent was obtained from all participants in
accordance with the Declaration of Helsinki. The PD and HC groups did not differ
significantly in age, education, or sex distribution ($p > 0.05$;
Table~\ref{tab:cohort_stat}).

\begin{table}[h!]
    \centering
    \begin{tabular}{lcc}
        \toprule
        \textbf{Cohort}              & \textbf{HC}     & \textbf{PD-non-MCI} \\
        \hline
        $n$                          & $89$            & $112$               \\
        Age (years)                  & $60.7 \pm 9.7$  & $60.6 \pm 8.9$      \\
        Age range                    & $30.6$ - $79.8$ & $39.2$ - $78.3$     \\
        Gender (male,\%)             & $47$ (52.8\%)   & $74$ (66.1\%)       \\
        Education (years)            & $16.7 \pm 3.4$  & $16.0 \pm 3.1$      \\
        UPDRS-III OFF baseline       & $-$             & $23.3 \pm 10.0$     \\
        UPDRS-III OFF follow-up      & $-$             & $25.6 \pm 11.2$     \\
        Inter-visit interval (years) & $1.4 \pm 0.5$   & $1.4 \pm 0.6$       \\
        \bottomrule
    \end{tabular}
    \vspace{1em}
    \caption{\textbf{Participant characteristics.} Values represent mean $\pm$
        standard deviation. PD = Parkinson's disease; MCI = mild cognitive impairment;
        UPDRS = Unified Parkinson's Disease Rating Scale. The PD-non-MCI longitudinal
        subset corresponds to participants with available follow-up MRI and  and disease
        severity scores available. \label{tab:cohort_stat}}
\end{table}

\subsection{Image acquisition and preprocessing}

T1-weighted MRI scans were obtained from the Parkinson's Progression Markers
Initiative (PPMI), acquired using standardized 3D MPRAGE protocols across sites
(repetition time = 2.3 s, echo time = 2.98 ms, inversion time = 0.9 s, voxel
size = 1 mm isotropic, 192 sagittal slices, field of view = 256 mm, matrix size
= 256 $\times$ 256). Minor variations in acquisition parameters may exist across
scanners and sites due to PPMI's multisite design.

Structural images were processed using FreeSurfer 7.3.1 instrumented with
Fuzzy-libm, a modified math library that introduces stochastic perturbations
into floating-point operations to assess numerical stability (Section~\label{sec:numerical_variability_assessment}).
Each participant's MRI was processed 26 times under independent perturbations to
estimate numerical variability. Runs that failed \TG{defined
    failed: failed due to technical reasons or failed QC?} due to processing errors or
quality-control (QC) issues were excluded, ensuring exactly 26 valid outputs per
participant. \TG{you should mention QC too}

Longitudinal reconstruction followed the standard FreeSurfer
pipeline~\cite{reuter2012within}, consisting of independent cross-sectional
processing for each timepoint, followed by creation of an unbiased
within-subject template~\cite{reuter2011avoiding} using robust
registration~\cite{reuter2010highly}. All downstream statistical analyses were
performed on unperturbed FreeSurfer outputs to avoid introducing additional
numerical variability.

\subsection{Numerical Variability Assessment}
\label{sec:numerical_variability_assessment}
We employed Monte Carlo Arithmetic (MCA)~\cite{parker1997monte} to quantify
numerical instability in FreeSurfer computations. MCA introduces controlled
random perturbations into floating-point operations, simulating rounding errors
that occur across different computational environments. This stochastic
approach enables systematic assessment of result stability by measuring
variation across multiple runs of identical analyses.

\TG{I think you should give more details about MCA, including the equations and also your improvements to fuzzy-libm}

We used Fuzzy-libm~\cite{salari2021accurate}, which extends MCA to mathematical
library functions (\texttt{exp}, \texttt{log}, \texttt{sin}, \texttt{cos})
through Verificarlo~\cite{denis2016verificarlo}, an LLVM-based compiler.
Virtual precision parameters were set to 53 bits for double precision and 24
bits for single precision to simulate realistic machine-level precision errors.

We processed each visit with FreeSurfer 7.3.1. To sample numerical variability
we compiled FreeSurfer \TG{did you really compile Freesurfer?} with Fuzzy-libm
an implementation of Monte Carlo arithmetic (MCA) that injects zero-mean
rounding noise into every elementary function call. Virtual precision was set
to 53 bits for operations promoted to double and 24 bits for single precision,
\TG{there's quite some redundancy with the previous paragraph} thereby
preserving IEEE-754 expectations but exposing the variance of alternative
execution paths. Each subject-visit pair was processed 26 times; failed or
quality-control-flagged runs were discarded, and exactly 26 successful runs per
pair were retained for analysis.

\subsubsection{Numerical-Anatomical Variability Ratio (\navr)}

To quantify computational stability relative to anatomical variation, we
introduce the Numerical-Anatomical Variability Ratio (\navr). For each brain
region, \navr measures the ratio of measurement uncertainty arising from
computational processes to natural inter-subject anatomical variation:

\[
    \text{\navr} = \frac{\sigma_{\mathrm{num}}}{\sigma_{\mathrm{anat}}}
\]

where $\sigma_{\mathrm{num}}$ represents numerical variability (measurement
precision across MCA repetitions for individual subjects) and
$\sigma_{\mathrm{anat}}$ represents anatomical variability (inter-subject
differences within each repetition).
For each region of interest, measurements from $k$ MCA repetitions across $n$
subject-visit pairs form a data matrix $\mathcal{M}_{k \times n}$ with entries
$x_i^{(r)}$, where $i=1,\dots,n$ indexes subject-visits and $r=1,\dots,k$ indexes repetitions.
Let
\[
    \bar x_i=\frac{1}{k}\sum_{r=1}^k x_i^{(r)}, \qquad
    \bar x^{(r)}=\frac{1}{n}\sum_{i=1}^n x_i^{(r)}.
\]

\textbf{Numerical variability (within-subject, across repetitions):}
\begin{equation}
    \sigma^2_{\mathrm{num}}
    = \frac{1}{n}\sum_{i=1}^{n}\left[
    \frac{1}{k-1}\sum_{r=1}^{k}\bigl(x_i^{(r)}-\bar x_i\bigr)^2
    \right].
    \label{eq:sigma_num}
\end{equation}

\textbf{Anatomical variability (within-repetition, across subjects):}
\begin{equation}
    \sigma^2_{\mathrm{anat}}
    = \frac{1}{k}\sum_{r=1}^{k}\left[
    \frac{1}{n-1}\sum_{i=1}^{n}\bigl(x_i^{(r)}-\bar x^{(r)}\bigr)^2
    \right].
    \label{eq:sigma_anat}
\end{equation}
where $\bar{x}_i$ and $\bar{x}^{(r)}$ denote column and row means,
respectively. Higher \navr values indicate regions where computational
uncertainty approaches or exceeds biological variation, potentially
compromising the detection of true anatomical differences.\TG{didn't you pool
    sigma anat across PD and HC?}



\subsubsection{Relationship between \navr~and downstream statistical test uncertainty}
\label{sec:theoretical_derivations}

\TG[done]{The remainder of this paragraph is quite informal, could you tidy up the
    equations to start from the definition of Cohen's d and clearly derive its
    standard deviation? You should also clarify the assumptions that are made.}

To establish a quantitative link between a method's computational reproducibility
and the reliability of group-level statistical inferences, we derived analytical
expressions connecting numerical variability to the uncertainty of commonly used
statistical tests (Cohen's~$d$, $t$-tests, partial correlation, and ANCOVA).
Our goal is to characterize how numerical noise propagates through the analytical
pipeline to produce uncertainty in the reported statistics.

For each Monte Carlo Arithmetic (MCA) repetition $r$, we denote by
$\tilde{\mathbf{x}}^{(r)} = (\tilde{x}_1^{(r)}, \dots, \tilde{x}_N^{(r)})^\top$
the vector of perturbed measurements across $N$ subjects.
Each perturbed observation is modeled as the sum of the subject's
true underlying biological value and a numerical error term:
\begin{equation*}
    \tilde{x}_i^{(r)} = x_i + \varepsilon_i^{(r)},
    \qquad
    \mathbb{E}[\varepsilon_i^{(r)}] = 0,
    \qquad
    \mathrm{Cov}[\boldsymbol{\varepsilon}^{(r)}] = \Sigma_{\mathrm{num}}.
    \label{eq:model_noise}
\end{equation*}
Here, $\mathbf{x} = (x_1, \dots, x_N)^\top$ represents the fixed,
biological measurements, while $\boldsymbol{\varepsilon}^{(r)}$
captures the random numerical perturbations introduced during computation.

\vspace{0.4em}
\noindent\textbf{Assumptions.}
To isolate the contribution of numerical variability, we make three simplifying assumptions:
\begin{enumerate}

    \item \textbf{Numerical error model.}
          The numerical perturbations are modeled as independent, zero-mean Gaussian
          random variables:
          $\varepsilon_i^{(r)} \sim \mathcal{N}(0,\,\sigma_{\mathrm{num},i}^2),
              \boldsymbol{\varepsilon}^{(r)} \sim
              \mathcal{N}\!\left(\mathbf{0},\,\Sigma_{\mathrm{num}}\right)$,
          where, under homoscedasticity,
          $\Sigma_{\mathrm{num}} = \sigma_{\mathrm{num}}^2 I_N$.

    \item \textbf{Anatomical variability.}
          The between-subject (biological) variance
          $\sigma_{\mathrm{anat}}^2 = \Var(\{x_i\})$
          dominates the numerical noise, i.e.\
          $\sigma_{\mathrm{num},i} \ll \sigma_{\mathrm{anat}}$.
          This implies that the pooled empirical standard deviation
          of the observed data can be approximated by the biological one,
          $s_p \approx \sigma_{\mathrm{anat}}$.

    \item \textbf{Baseline (null-hypothesis) scenario.} \YC{Do we need this assumption?
              It seems we can derive the same results without it.}
          We condition on the observed biological measurements $\mathbf{x}$
          and quantify variability only from numerical perturbations
          $\boldsymbol{\varepsilon}^{(r)}$. This isolates finite-precision effects from sampling
          variation; the resulting expressions remain accurate near the null
          and for small effect sizes. In this setting, the biological
          values $\mathbf{x} = (x_1, \dots, x_N)^\top$ are treated as fixed,
          and all randomness arises from numerical perturbations
          $\boldsymbol{\varepsilon}^{(r)}$.

\end{enumerate}

Under these assumptions, each downstream statistic
$ds = f(\tilde{\mathbf{x}})$ can be linearized around the
baseline $\mathbf{x}$ as
$ds(\tilde{\mathbf{x}}) \approx ds(\mathbf{x}) +
    \nabla_{\!\mathbf{x}}f(\mathbf{x})^\top \boldsymbol{\varepsilon}$,
allowing the numerical variance to be expressed through the delta method as
\begin{equation}
    \Var_{\mathrm{num}}[ds]
    \approx
    \nabla_{\!\mathbf{x}} f(\mathbf{x})^\top
    \Sigma_{\mathrm{num}}
    \nabla_{\!\mathbf{x}} f(\mathbf{x})
    = \sigma_{\mathrm{num}}^2 \|\nabla_{\!\mathbf{x}} f(\mathbf{x})\|_2^2.
    \label{eq:var-ds}
\end{equation}



Table~\ref{tab:stat_uncertainty} summarizes the derived expressions for the
numerical standard deviation of several common statistics.

\paragraph{Cohen's \textit{d}}

Cohen's effect size quantifies the standardized difference between two sample means.
For two independent groups $G_1$ and $G_2$ with sample sizes $n_1$ and $n_2$
($df = n_1 + n_2 - 2$), we define:
\begin{equation}
    d = \frac{\Delta}{s_p}
    = \frac{\bar{x}_1 - \bar{x}_2}{s_p},
    \qquad
    s_p = \sqrt{\frac{(n_1 - 1)s_1^2 + (n_2 - 1)s_2^2}{df}}.
    \label{eq:cohen-d-def}
\end{equation}

The variance of $d$ across Monte Carlo Arithmetic (MCA) repetitions, conditional
on the fixed dataset $\mathbf{x}$, is given by:
\begin{equation*}
    \Var_{\mathrm{num}}[d]
    = \Var\!\left[d(\tilde{\mathbf{x}})\,\big|\,\mathbf{x}\right].
    \label{eq:var-rep-def}
\end{equation*}
Applying the multivariate delta method (Eq.~\ref{eq:var-ds}) around the baseline $\mathbf{x}_0 = \mathbf{x}$, we obtain:
\begin{equation}
    \Var_{\mathrm{num}}[d]
    \approx
    \sigma_{\mathrm{num}}^2
    \sum_{i=1}^{n}
    \left(\frac{\partial d}{\partial x_i}\right)^{\!2}.
    \label{eq:var-rep-approx}
\end{equation}
For an observation $x_i \in G_g$ ($g \in \{1,2\}$), the chain rule gives:
\begin{equation*}
    \frac{\partial d}{\partial x_i} = \frac{1}{s_p}\frac{\partial \Delta}{\partial x_i} - \frac{\Delta}{s_p^2}\frac{\partial s_p}{\partial x_i}, \quad
    \frac{\partial \Delta}{\partial x_i} = \pm \frac{1}{n_g}, \quad
    \frac{\partial s_p}{\partial x_i} = \frac{x_i - \bar{x}_g}{df s_p}
    \label{eq:d-deriv-chain}
\end{equation*}
where the sign in $\partial \Delta/\partial x_i$ is positive for $g=1$ and negative for $g=2$.
Substituting back into the expression for $\partial d/\partial x_i$, we have:
\begin{equation*}
    \frac{\partial d}{\partial x_i} = \pm\frac{1}{n_g s_p} - \frac{\Delta(x_i - \bar{x}_g)}{df s_p^3}
    \Rightarrow {\left(\frac{\partial d}{\partial x_i}\right)}^2 = \frac{1}{n_g^2 s_p^2} \pm \frac{2\Delta(x_i - \bar{x}_g)}{n_g df s_p^4} + \frac{\Delta^2(x_i - \bar{x}_g)^2}{df^2 s_p^6}
\end{equation*}
and summing over all $i$ in group $G_g$:
\begin{equation*}
    \sum_{i\in G_g}\left(\frac{\partial d}{\partial x_i}\right)^2 =
    \frac{1}{n_g s_p^2} \pm \frac{2\Delta}{n_g df s_p^4}\sum_{i\in G_g}(x_i - \bar{x}_g) + \frac{\Delta^2}{df^2 s_p^6}\sum_{i\in G_g}(x_i - \bar{x}_g)^2 = \frac{1}{n_g s_p^2} + \frac{\Delta^2}{df^2 s_p^6}\left((n_g - 1)s_g^2 \right)
\end{equation*}
since $\sum_{i\in G_g}(X_i - \bar{X}_g) = 0$, so finally summing over both groups:
\begin{align*}
    \sum_{i = 1}^{n}\left(\frac{\partial d}{\partial X_i}\right)^2
     & = \frac{1}{s_p^2}\left(\frac{1}{n_1} + \frac{1}{n_2}\right) +
    \frac{\Delta^2}{df^2 s_p^6}\left((n_1 - 1)s_1^2 + (n_2 - 1)s_2^2 \right)                                         \\
     & = \frac{1}{s_p^2}\left(\frac{1}{n_1} + \frac{1}{n_2} +
    \frac{1}{df}\frac{\Delta^2}{s_p^2}\frac{1}{s_p^2}\frac{\left((n_1 - 1)s_1^2 + (n_2 - 1)s_2^2 \right)}{df}\right) \\
     & = \frac{1}{s_p^2}\left(\frac{1}{n_1} + \frac{1}{n_2} + \frac{d^2}{df} \right).
\end{align*}
Finally, assuming $s_p \approx \sigma_{\mathrm{anat}}$, the anatomical (between-subject)
variance, Eq.~\eqref{eq:var-rep-approx} becomes:
\begin{equation}
    \Var_{\mathrm{num}}[d]
    \approx
    \frac{\sigma_{\mathrm{num}}^2}{\sigma_{\mathrm{anat}}^2}
    \left(\frac{1}{n_1}+\frac{1}{n_2}+\frac{d^2}{df}\right).
    \label{eq:var-rep-simplified}
\end{equation}
For balanced groups ($n_1 = n_2 = n/2$) and large $n$, the $d^2/df$ term is negligible:
\begin{align*}
    \Var_{\mathrm{num}}[d]   & \approx \frac{4}{n} \frac{\sigma_{\mathrm{num}}^2}{\sigma_{\mathrm{anat}}^2}   = \frac{4}{n} \nu_{\mathrm{nav}}^2 \\
    \sigma_{\mathrm{num}}[d] & \approx \frac{2}{\sqrt{n}}\,\nu_{\mathrm{nav}}
    \label{eq:var-rep-final-simplified}
\end{align*}
where $\nu_{\mathrm{nav}} = \sigma_{\mathrm{num}} / \sigma_{\mathrm{anat}}$
is the numerical–anatomical variability ratio.
This expression quantitatively links the numerical uncertainty captured by
$\nu_{\mathrm{nav}}$ to the variability of Cohen's~$d$ effect size,
providing a practical measure of the stability of statistical inferences
under finite-precision arithmetic.

\paragraph{Two-sample \textit{t}-test statistic}

The pooled two-sample $t$ statistic quantifies the standardized difference between
two group means:
\begin{equation*}
    t =
    \frac{\bar{x}_1 - \bar{x}_2}{s_p\sqrt{\tfrac{1}{n_1} + \tfrac{1}{n_2}}}
    =
    \frac{d}{\sqrt{\tfrac{1}{n_1} + \tfrac{1}{n_2}}},
    \label{eq:t-stat-def}
\end{equation*}
where $d$ is Cohen's~$d$ defined in Eq.~\eqref{eq:cohen-d-def}.
Defining
\(
\omega_n = \tfrac{1}{n_1} + \tfrac{1}{n_2},
\)
the $t$ statistic can be expressed as
\(t = d / \sqrt{\omega_n}\).
From Eq.~\eqref{eq:var-rep-simplified}, the variance of $d$ due to numerical
perturbations propagates to the variance of $t$ as:
\begin{equation}
    \begin{aligned}
        \Var_{\mathrm{num}}[t]
         & =
        \Var\!\left[\frac{d}{\sqrt{\omega_n}}\right]
        \approx
        \frac{1}{\omega_n}
        \nu_{\mathrm{nav}}^{2}
        \left(\omega_n + \frac{d^{2}}{df}\right) \\[3pt]
         & =
        \nu_{\mathrm{nav}}^{2}
        \left(1 + \frac{d^{2}}{df\,\omega_n}\right).
        \label{eq:t-var-approx}
    \end{aligned}
\end{equation}

\vspace{0.5em}
\noindent
To analyze the correction term, consider
\begin{equation*}
    df\,\omega_n = (n_1 + n_2 - 2)
    \left(\frac{1}{n_1} + \frac{1}{n_2}\right)
    = (n_1 + n_2 - 2)\frac{n_1 + n_2}{n_1 n_2}
    \label{eq:df-omega-analysis}
\end{equation*}
When $n_1 \gg n_2$,
\(
df\,\omega_n \approx \tfrac{n_1 - 2}{n_2};
\)
symmetrically, when $n_2 \gg n_1$,
\(
df\,\omega_n \approx \tfrac{n_2 - 2}{n_1}.
\)
In both cases, $1/(df\,\omega_n) \to 0$, so the correction term
\(\tfrac{d^2}{df\,\omega_n}\) vanishes for unbalanced groups.
When the groups are balanced ($n_1 = n_2 = n/2$),
$df \omega_n = 4\!\left(1 - \tfrac{2}{n}\right) \to 4$ as $n \to \infty$,
so that $\tfrac{d^2}{df\,\omega_n} \to d^2 / 4$.
Substituting these results into Eq.~\eqref{eq:t-var-approx} gives:
\begin{equation*}
    \Var_{\mathrm{num}}[t]
    \approx
    \nu_{\mathrm{nav}}^{2}\,(1 + \epsilon),
    \label{eq:t-var-final}
\end{equation*}
where $\epsilon$ tends to $0$ for strongly unbalanced groups and to $d^{2}/4$
for large balanced samples.
For small effect sizes ($d^2 \ll 4$) and unbalanced groups, the correction term
$\epsilon$ is negligible, yielding the simplified expression:
\begin{align*}
    \Var_{\mathrm{num}}[t]   & \approx \nu_{\mathrm{nav}}^{2} \notag \\
    \sigma_{\mathrm{num}}[t] & \approx \nu_{\mathrm{nav}}.
    \label{eq:t-var-simplified}
\end{align*}

\vspace{0.5em}
\noindent
The uncertainty of the corresponding $p$-values can then be derived. Let $X$ be the
random variable with $\mathbb{E}_{\mathrm{num}}[X]=t_0$ and
$\Var_{\mathrm{num}}[X]=\nu_{\mathrm{nav}}^{2}$. Let $f_t$, $F_t$ be the
probability density and cumulative distribution functions of the Student
$t$-distribution with $df$ degrees of freedom. Applying the delta method to the
two-sided $p$-value $p(X) = 2\left(1 - F_t(|X|)\right)$ gives:
\begin{align*}
    \Var_{\mathrm{num}}[p(X)] & = \Var_{\mathrm{num}}\left[2(1 - F_t(|X|))\right]                             \notag     \\
                              & \approx {\left(-2f_t(|t_0|)\,\mathrm{sign}(t_0)\right)}^{2}\Var_{\mathrm{num}}[X] \notag \\
                              & \approx 4{\left(f_t(|t_0|)\right)}^{2}\nu_{\mathrm{nav}}^{2},
\end{align*}
which gives the standard deviation of the numerical uncertainty in the
$p$-value:
\begin{equation*}
    \sigma_{\mathrm{num}}[p(X)] = 2f_t(|t_0|)\,\nu_{\mathrm{nav}}.
    \label{eq:p-std}
\end{equation*}

\paragraph{ANCOVA group effect.}

Analysis of covariance (ANCOVA) evaluates group differences using a general
linear model:
\begin{equation*}
    \mathbf{y} = \mathbf{X}\boldsymbol{\beta} + \boldsymbol{\varepsilon},
    \qquad
    \boldsymbol{\varepsilon} \sim \mathcal{N}(\mathbf{0},\, \sigma^2 I),
    \label{eq:ancova-model}
\end{equation*}
where $\mathbf{y}$ is the vector of measurements across subjects, and
$\mathbf{X}$ includes an intercept, diagnostic group (PD vs.\ HC), and
covariates (e.g., age and sex). The adjusted group difference is expressed as
the one-degree-of-freedom contrast $c^{\top}\boldsymbol{\beta}$, with contrast
vector $c = [0,\, 1,\, 0,\, 0]^{\top}$. Let
$\widehat{\boldsymbol{\beta}} = (\mathbf{X}^{\top}\mathbf{X})^{-1}
    \mathbf{X}^{\top}\mathbf{y}$ denote the ordinary least-squares (OLS) estimator
and $\widehat{\sigma}_{\mathrm{res}}^{2} = SS_{\mathrm{res}} / df_2$ the
residual mean square, where $df_2 = n - \mathrm{rank}(\mathbf{X})$.
The sum of squares associated with the group effect is:
\begin{equation*}
    SS_{\mathrm{group}}
    =
    \frac{(c^{\top}\widehat{\boldsymbol{\beta}})^{2}}
    {c^{\top}(\mathbf{X}^{\top}\mathbf{X})^{-1}c},
    \qquad df_1 = 1.
\end{equation*}
The corresponding ANCOVA $F$ statistic is given by:
\begin{equation*}
    F
    =
    \frac{MS_{\mathrm{group}}}{MS_{\mathrm{res}}}
    =
    \frac{(c^{\top}\widehat{\boldsymbol{\beta}})^{2}}
    {\widehat{\sigma}_{\mathrm{res}}^{2}\;
    c^{\top}(\mathbf{X}^{\top}\mathbf{X})^{-1}c},
    \qquad
    F \sim \mathcal{F}(df_1 = 1,\, df_2).
    \label{eq:ancova-F}
\end{equation*}
Significance is evaluated using the upper-tail of the central $\mathcal{F}$ distribution under
the null hypothesis:
\[
    p(X) = 1 - F_{\mathcal{F}}\left(X;df_1,df_2\right),
\]
with $X \sim \mathcal{F}(df_1, df_2)$ and $F_{\mathcal{F}}$ the cumulative distribution
function of the $F$ distribution with $(df_1, df_2)$ degrees of freedom. For
$df_1 = 1$, the ANCOVA $F$ statistic is equivalent to the two-sample $t$-test
through $F = t^{2}$ (see~\cite[p.~403]{johnson1995continuous}).
Then the uncertainty in the $F$ statistic due to numerical noise follows directly from the uncertainty of $t$:
\begin{align*}
    \Var_{\mathrm{num}}[F]
     & = \Var_{\mathrm{num}}[t^{2}]
    \approx (2t)^{2}\Var_{\mathrm{num}}[t]
    = 4t^{2}\nu_{\mathrm{nav}}^{2}
    = 4F\nu_{\mathrm{nav}}^{2},
\end{align*}
yielding
\begin{equation}
    \sigma_{\mathrm{num}}[F] = 2\sqrt{F}\,\nu_{\mathrm{nav}}.
    \label{eq:ancova-F-sigma}
\end{equation}

\vspace{0.4em}
\noindent
The uncertainty in the corresponding $p$-values can be obtained by the delta
method. Let $X$ be a random variable with $\mathbb{E}_{\mathrm{num}}[X]=F_0$ and
$\Var_{\mathrm{num}}[X]=4F_0\nu_{\mathrm{nav}}^{2}$  and $f_{\mathcal{F}}$, $F_{\mathcal{F}}$ the probability
density and cumulative distribution functions. Applying the delta-method to the
upper-tail $p$-value is $p(X) = 1 - F_{\mathcal{F}}(X)$ yields:
\begin{align*}
    \Var_{\mathrm{num}}[p(X)] & = \Var_{\mathrm{num}}[1 - F_{\mathcal{F}}(X)]           \\
                              & \approx f_{\mathcal{F}}(F_0)^{2}\Var_{\mathrm{num}}[X]  \\
                              & = 4F_0{f_{\mathcal{F}}(F_0)}^{2}\nu_{\mathrm{nav}}^{2},
\end{align*}
so that
\begin{equation}
    \sigma_{\mathrm{num}}[p(X)] = 2\sqrt{F_0}\,f_{\mathcal{F}}(F_0)\,\nu_{\mathrm{nav}}.
    \label{eq:ancova-p-sigma}
\end{equation}

Equations~\eqref{eq:ancova-F-sigma} and~\eqref{eq:ancova-p-sigma} show that
numerical imprecision introduces a variance in the estimated $F$ statistic and
its corresponding $p$-value that scales linearly with the numerical-anatomical
variability ratio $\nu_{\mathrm{nav}}$, and proportionally to $\sqrt{F}$ for
the group effect magnitude.

\paragraph{Partial correlation.}

Partial correlation measures the association between two variables $(x,y)$ while
controlling for the influence of one or more additional variables $z$.
In our analysis, this corresponds to quantifying the relationship between
regional brain measurements and UPDRS-III motor scores, controlling for age and sex.
The sample partial correlation is defined as:
\begin{equation*}
    r_{xy,z}
    =
    \frac{r_{xy} - r_{xz}r_{yz}}
    {\sqrt{(1 - r_{xz}^{2})(1 - r_{yz}^{2})}},
    \label{eq:partial-r-def}
\end{equation*}
where $r_{xy}$ denotes the Pearson correlation between variables $x$ and $y$,
\[
    r_{xy} = \frac{s_{xy}}{s_x s_y}.
\]
To simplify notation, we set
$a = r_{xy}$, $b = r_{xz}$, and $c = r_{yz}$ so that
\[
    R(a,b,c)
    = \frac{a - bc}{\sqrt{(1 - b^{2})(1 - c^{2})}}
    = \frac{a - bc}{D},
    \qquad
    D = \sqrt{(1 - b^{2})(1 - c^{2})}.
\]
Applying the delta method (Eq.~\ref{eq:var-ds}) to the partial correlation, we have:
\begin{equation}
    \Var_{\mathrm{num}}[R] \approx \sigma_{\mathrm{num}}^{2} \sum_{i=1}^{n} \left(\frac{\partial R}{\partial x_i}\right)^{2},
    \label{eq:delta-partial}
\end{equation}

Assuming only $x$ is affected by numerical perturbations while $y$ and $z$ are
fixed, the gradient with respect to each observation $x_i$ is:
\[
    \frac{\partial R}{\partial x_i}
    =
    \frac{\partial R}{\partial a}\frac{\partial a}{\partial x_i}
    +
    \frac{\partial R}{\partial b}\frac{\partial b}{\partial x_i}.
\]
The first-order partial derivatives are:
\[
    \frac{\partial R}{\partial a} = \frac{1}{D}, \qquad
    \frac{\partial R}{\partial b} = \frac{(1 - c^{2})(ab - c)}{D^{3}}.
\]
and the derivatives of the correlations with respect to $x_i$ are (see
Eq~\ref{eq:pearson_derivative}):
\[
    \frac{\partial a}{\partial x_i}
    =
    \frac{(v_i - a\,u_i)}{(n-1)s_x} = \frac{\alpha_i}{(n-1)s_x},
    \qquad
    \frac{\partial b}{\partial x_i}
    = \frac{(w_i - b\,u_i)}{(n-1)s_x} = \frac{\beta_i}{(n-1)s_x}.
\]
where $u_i = (x_i - \bar{x})/s_x$, $v_i = (y_i - \bar{y})/s_y$, and $w_i = (z_i - \bar{z})/s_z$ are standardized and centered observations of
$x$, $y$, and $z$ respectively.
Then $\partial R / \partial x_i$ becomes:
\begin{align*}
    \frac{\partial R}{\partial x_i}
     & =
    \frac{1}{(n-1) s_x}
    \left[
        \frac{\alpha_i}{D} + \beta_i
        \frac{(1 - c^{2})(ab - c)}{D^{3}}
        \right]
\end{align*}
thus $(\partial R / \partial x_i)^{2}$ is:
\begin{align*}
    \left(\frac{\partial R}{\partial x_i}\right)^{2}
     & =
    \frac{1}{(n-1)^{2}s_{x}^{2}}
    \left[
        \frac{\alpha_i^{2}}{D^{2}}
        +
        2\frac{\alpha_i\beta_i(1 - c^{2})(ab - c)}{D^{4}}
        +
        \frac{\beta_i^{2}(1 - c^{2})^{2}(ab - c)^{2}}{D^{6}}
    \right]                           \\
     & =
    \frac{1}{(n-1)^{2}s_{x}^{2}}
    \left[
        \frac{\alpha_i^{2}}{(1 - b^{2})(1 - c^{2})}
        +
        2\frac{\alpha_i\beta_i(1 - c^{2})(ab - c)}{(1 - b^{2})^{2}(1 - c^{2})^{2}}
        +
        \frac{\beta_i^{2}(1 - c^{2})^{2}(ab - c)^{2}}{(1 - b^{2})^{3}(1 - c^{2})^{3}}
    \right]                           \\
     & = \frac{1}{(n-1)^{2}s_{x}^{2}}
    \left[
        \frac{\alpha_i^{2}}{(1 - b^{2})(1 - c^{2})}
        +
        2\frac{\alpha_i\beta_i(ab - c)}{(1 - b^{2})^{2}(1 - c^{2})}
        +
        \frac{\beta_i^{2}(ab - c)^{2}}{(1 - b^{2})^{3}(1 - c^{2})}
        \right].
\end{align*}
Using the correlation identities
$\sum\alpha_i^2 = (n-1)(1 - a^{2})$,
$\sum\beta_i^2 = (n-1)(1 - b^{2})$, and
$\sum\alpha_i\beta_i = (n-1)(ab - c)$ (see proof in
Appendix~\ref{sec:correlation_identities}),
we sum over all $i$ to obtain:
\begin{align*}
    \sum_{i=1}^{n}
    \left(\frac{\partial R}{\partial x_i}\right)^{2}
     & =
    \frac{1}{(n-1)s_{x}^{2}}
    \left[
        \frac{(1 - a^{2})}{(1 - b^{2})(1 - c^{2})}
        +
        2\frac{(ab - c)^{2}}{(1 - b^{2})^{2}(1 - c^{2})}
        +
        \frac{(ab - c)^{2}}{(1 - b^{2})^{2}(1 - c^{2})}
    \right] \\
     & =
    \frac{1}{(n-1)s_{x}^{2}}
    \left[
        \frac{(1 - a^{2})}{(1 - b^{2})(1 - c^{2})}
        +
        \frac{3(ab - c)^{2}}{(1 - b^{2})^{2}(1 - c^{2})}
    \right] \\
     & =
    \frac{1}{(n-1)s_{x}^{2}}
    \left[
    \frac{(1 - a^{2})(1+3r_{yz,x}^{2})}{(1 - b^{2})(1 - c^{2})}
    \right].
\end{align*}
Substituting back into Eq.~\eqref{eq:delta-partial} with $s_x^2 \simeq \sigma_{\mathrm{anat}}^2$ gives:
\begin{equation}
    \Var_{\mathrm{num}}[R]
    \approx
    \frac{\nu_{\mathrm{nav}}^{2}}{(n-1)}
    \left[
    \frac{(1 - a^{2})(1+3r_{yz,x}^{2})}{(1 - b^{2})(1 - c^{2})}
    \right].
    \label{eq:partial-var}
\end{equation}
Since $a,b,c$ are rarely not reported in practice, we further simplify
this expression by deriving the lower bound:
\begin{equation}
    (1 - R^{2})^{3} \le \frac{(1 - a^{2})}{(1 - b^{2})(1 - c^{2})},
    \label{eq:partial-corr-bound}
\end{equation}
First, note that the squared partial correlation is:
\[
    1 - R^{2}
    =
    \frac{(1-b^{2})(1-c^{2})-(a-bc)^{2}}{(1-b^{2})(1-c^{2})}
    =
    \frac{\Delta}{(1-b^{2})(1-c^{2})},
\]
where
\[
    \Delta
    =
    (1-b^{2})(1-c^{2})-(a-bc)^{2}
    =
    (1-a^{2})(1-b^{2})-(ac-b)^{2}
    =
    (1-a^{2})(1-c^{2})-(ab-c)^{2}.
\]
Each equality above follows from expanding both sides.
Because every squared term is nonnegative, we obtain the three inequalities
\begin{equation}
    \label{eq:three-ineq}
    \Delta \le (1-a^{2})(1-b^{2}),
    \qquad
    \Delta \le (1-b^{2})(1-c^{2}),
    \qquad
    \Delta \le (1-c^{2})(1-a^{2}).
\end{equation}
Multiplying the first and third inequalities in~\eqref{eq:three-ineq} gives
\[
    \Delta^{2} \le (1-a^{2})^{2}(1-b^{2})(1-c^{2}),
\]
and multiplying also by the middle one yields
\[
    \Delta^{3} \le (1-a^{2})^{2}(1-b^{2})^{2}(1-c^{2})^{2}.
\]
Since $0\le 1-a^{2}\le 1$, we have $(1-a^{2})^{2}\le (1-a^{2})$, so that
\[
    \Delta^{3} \le (1-a^{2})(1-b^{2})^{2}(1-c^{2})^{2}.
\]
Dividing both sides by $(1-b^{2})^{3}(1-c^{2})^{3}$ and substituting
$\Delta=(1-b^{2})(1-c^{2})(1-R^{2})$ yields
\[
    (1-R^{2})^{3}
    =
    \frac{\Delta^{3}}{(1-b^{2})^{3}(1-c^{2})^{3}}
    \le
    \frac{1-a^{2}}{(1-b^{2})(1-c^{2})}.
\]
This establishes the claimed bound~\eqref{eq:partial-corr-bound}.
Since $3r_{yz,x}^2+1 \geq 1$ it follows immediately that
\[
    (1 - R^{2})^{3} \leq
    \frac{(1-a^{2})}{(1-b^{2})(1-c^{2})} \leq
    \frac{(1-a^{2})(1+3r_{yz,x}^{2})}{(1-b^{2})(1-c^{2})}
\]
So, substituting into Eq.~\eqref{eq:partial-var} gives the lower bound:
\begin{equation}
    \nu_{\mathrm{nav}}^{2}\frac{(1 - R^{2})^{3}}{n-1}
    \lesssim
    \Var_{\mathrm{num}}[R].
    \label{eq:partial-var-bound}
\end{equation}
Taking the square root yields the standard deviation:
\[
    \nu_{\mathrm{nav}}
    \sqrt{\frac{(1 - r_{xy,z}^{2})^{3}}{n - 1}}
    \lesssim
    \sigma_{\mathrm{num}}[R].
\]
The two-sided significance of a partial correlation is computed from the
$t$-statistic
\[
    t
    =
    R\sqrt{\frac{df}{1-R^{2}}},
    \qquad
    df = n-k-2,
\]
where $k$ is the number of controlling variables. Let $X$ be the random variable
with $\mathbb{E}_{\mathrm{num}}[X]=R_0$, $t_0^2=R_0(df/(1-R_0^2))$ with
$\Var_{\mathrm{num}}[X]$ bounded by~\Cref{eq:partial-var-bound}. Let $f_t$,
$F_t$ be the probability density and cumulative distribution functions of the
Student $t$-distribution with $df$ degrees of freedom. Applying the delta method
to the two-sided $p$-value $p(X) = 2\left(1 - F_t(|X|)\right)$ gives:
\begin{equation}
    \Var_{\mathrm{num}}[p(X)]
    \approx
    \left(\frac{\partial p}{\partial t}\frac{\partial t}{\partial R}\right)^{2}
    \Var_{\mathrm{num}}[X],
    \label{eq:partial-p-var}
\end{equation}
with the partial derivatives given by:
\begin{equation}
    \frac{\partial p}{\partial t} = -2 f_t(|t|)\,\mathrm{sign}(t),\qquad
    \frac{\partial t}{\partial R} = \sqrt{\frac{df}{(1-R^{2})^{3}}}.
    \label{eq:partial-t-deriv}
\end{equation}
Combining equations \eqref{eq:partial-p-var} and \eqref{eq:partial-t-deriv} yields
\[
    \Var_{\mathrm{num}}[p(X)]
    \geq
    4f_t(|t_0|)^{2}\frac{df}{(1-R_0^{2})^{3}}\Var_{\mathrm{num}}[X].
\]
Using \Cref{eq:partial-var-bound} to bound $\Var_{\mathrm{num}}[R]$,
the dependence on $(1-R^{2})^{3}$ cancels, leading to:
\[
    \sigma_{\mathrm{num}}[p(X)]
    \geq
    2f_t(|t_0|)\sqrt{\frac{df}{n-1}}\nu_{\mathrm{nav}}.
\]

\section{Data Availability}
The data that support the findings of this study are available from the
Parkinson's Progression Markers Initiative (PPMI) database
(www.ppmi-info.org/access-data-specimens/download-data), but restrictions apply
to the availability of these data, which were used under license for the
current study, and so are not publicly available. Data are however available
from the authors upon reasonable request and with permission of the PPMI.

\section{Code Availability}

All MCA instrumentation scripts, FreeSurfer build instructions and analysis
notebooks are available at [GitHub URL to be inserted]. Exact commit hashes are
archived on Zenodo (DOI [to be added]) to ensure bit-level reproducibility.

\section{Acknowledgements}

The analyses were conducted on the Virtual Imaging
Platform~\cite{glatard2012virtual}, which utilizes resources provided by the
Biomed virtual organization within the European Grid Infrastructure (EGI). We
extend our gratitude to Sorina Pop from CREATIS, Lyon, France, for her support.
\TG{acknowledge MJFF project LivingPark}

\bibliographystyle{plain}
\bibliography{main}

\clearpage

\appendix

\section{Formula}

\subsection{Significant digits formula}
\label{eq:significant_digits}

We compute the number of significant bits \(\hat{s}\) with probability
\(p_s=0.95\) and confidence \(1-\alpha_s=0.95\) using the
\texttt{significantdigits}
package\footnote{\url{https://github.com/verificarlo/significantdigits}}
(version 0.4.0). \texttt{significantdigits} implements the Centered Normality
Hypothesis approach described in~\cite{sohier2021confidence}:
\[
    \hat{s_i} = -\log_2 \left| \frac{\hat{\sigma_i}}{\hat{\mu_i}} \right| -
    \delta(n, \alpha_s, p_s),
\]
where \(\hat{\sigma_i}\) and \(\hat{\mu_i}\) are the average and standard
deviation over the repetitions, and
\begin{equation}
    \delta(n, \alpha_s, p_s) = \log_2 \left(
    \sqrt{\frac{n-1}{\chi^2_{1-\alpha_s/2}}} \Phi^{-1} \left( \frac{p_s+1}{2}
    \right) \right)
\end{equation}
is a penalty term for estimating \(\hat{s_i}\) with probability \(p_s\) and
confidence level \(1-\alpha_s\) for a sample size \(n\). \(\Phi^{-1}\) is the
inverse cumulative distribution of the standard normal distribution and
\(\chi^2\) is the Chi-2 distribution with \(n\)-1 degrees of freedom.

\subsection{Extended Sørensen-Dice coefficient}
\label{eq:extended_dice}

The extended Sørensen-Dice coefficient is a measure of overlap between multiple
sets, defined as follows:
\[
    \text{Dice}(A_1, A_2, \dots, A_n) = \frac{n \left| \bigcap_{i=1}^{n} A_i \right|}{\sum_{i=1}^{n} \left| A_i \right|}
\].

\subsection{Partial derivatives of sample statistics}

We derive below the partial derivatives of common sample statistics for a dataset
\(x = \{x_1, x_2, \ldots, x_n\}\)
with respect to an individual observation \(x_i\), where \(n\) denotes the sample size.
The Kronecker delta \(\delta_{ij}\) equals 1 when \(i = j\) and 0 otherwise.

\subsubsection{Sample Mean}
\label{sec:sample_mean_derivation}

The partial derivative of the sample mean with respect to \(x_i\) is constant:
\begin{equation}
    \frac{\partial \overline{x}}{\partial x_i} = \frac{1}{n}.
    \label{eq:sample_mean_derivative}
\end{equation}

\begin{proof}
    The sample mean is defined as
    \[
        \overline{x} = \frac{1}{n} \sum_{j=1}^{n} x_j.
    \]
    Taking the partial derivative with respect to \(x_i\) gives
    \[
        \frac{\partial \overline{x}}{\partial x_i}
        = \frac{\partial}{\partial x_i} \left( \frac{1}{n} \sum_{j=1}^{n} x_j \right)
        = \frac{1}{n} \sum_{j=1}^{n} \frac{\partial x_j}{\partial x_i}
        = \frac{1}{n} \sum_{j=1}^{n} \delta_{ij}
        = \frac{1}{n}.
    \]
\end{proof}

\subsubsection{Sample Variance}
\label{sec:sample_variance_derivation}

The partial derivative of the sample variance with respect to \(x_i\) is
\begin{equation}
    \frac{\partial s^2}{\partial x_i} = \frac{2(x_i - \overline{x})}{n-1}.
    \label{eq:sample_variance_derivative}
\end{equation}

\begin{proof}
    The sample variance is defined as
    \[
        s^2 = \frac{1}{n-1} \sum_{j=1}^{n} {(x_j - \overline{x})}^2.
    \]
    Differentiating with respect to \(x_i\) yields
    \begin{align*}
        \frac{\partial s^2}{\partial x_i}
         & = \frac{1}{n-1} \sum_{j=1}^{n}
        \frac{\partial}{\partial x_i} (x_j - \overline{x})^2                           \\
         & = \frac{2}{n-1} \sum_{j=1}^{n}
        (x_j - \overline{x}) \frac{\partial (x_j - \overline{x})}{\partial x_i}        \\
         & = \frac{2}{n-1} \sum_{j=1}^{n}
        (x_j - \overline{x}) \left( \delta_{ij} - \frac{1}{n} \right)                  \\
         & = \frac{2}{n-1} \left[ (x_i - \overline{x})\!\left( 1 - \frac{1}{n} \right)
        - \frac{1}{n} \sum_{\substack{j=1                                              \\ j \ne i}}^{n} (x_j - \overline{x}) \right].
    \end{align*}
    Since \(\sum_{j=1}^{n} (x_j - \overline{x}) = 0\) then \(\sum_{\substack{j=1 \\ j \ne i}}^{n} (x_j - \overline{x}) = - (x_i - \overline{x})\) so the second term simplifies, giving
    \[
        \frac{\partial s^2}{\partial x_i}
        = \frac{2(x_i - \overline{x})}{n-1}.
    \]
\end{proof}

\subsubsection{Sample Standard Deviation}
\label{sec:sample_std_derivation}

The partial derivative of the sample standard deviation with respect to \(x_i\) is
\begin{equation}
    \frac{\partial s}{\partial x_i} = \frac{x_i - \overline{x}}{(n-1)s}.
    \label{eq:sample_std_derivative}
\end{equation}


\begin{proof}
    Given that \(s = \sqrt{s^2}\), the derivative follows directly from the chain rule:
    \begin{align*}
        \frac{\partial s^2}{\partial x_i}  & = \frac{2(x_i - \overline{x})}{n-1}, \\
        2s \frac{\partial s}{\partial x_i} & = \frac{2(x_i - \overline{x})}{n-1}, \\
        \frac{\partial s}{\partial x_i}    & = \frac{x_i - \overline{x}}{(n-1)s}.
    \end{align*}
\end{proof}

\subsubsection{Pooled Standard Deviation}
\label{sec:pooled_std_derivation}

The pooled standard deviation is a weighted average of the variances
of two groups \(|G_1|=n_1\) and \(|G_2|=n_2\) with \(df=n_1+n_2-2\). Its partial derivative with respect to \(x_i\), \(i \in G_g\) is given by
\begin{equation}
    \frac{\partial s_p}{\partial x_i} = \frac{x_i - \overline{x}_g}{df s_p}.
\end{equation}

\begin{proof}
    Let \(x\) be partitioned into two groups \(G_1\) and \(G_2\) with sizes \(n_1\)
    and \(n_2\), \(\overline{x}_1\), \(\overline{x}_2\) be the sample means and
    \(s_1\), \(s_2\) be the sample standard deviations of groups \(G_1\) and
    \(G_2\), respectively. Let \(df=n_1+n_2-2\) then the pooled standard deviation \(s_p\) is defined as
    \[
        s_p = \sqrt{\frac{(n_1 - 1)s_1^2 + (n_2 - 1)s_2^2}{df}}.
    \]
    Differentiating \(s_p\) with respect to \(x_i\) in group \(G_g\) gives:
    \begin{align*}
        \frac{\partial s_p}{\partial x_i} & = \frac{\partial}{\partial x_i} {\left[\frac{(n_1 - 1)s_1^2 + (n_2 - 1)s_2^2}{df}\right]}^{\frac{1}{2}}   \\
                                          & = \frac{1}{2 s_p} \frac{1}{df} \frac{\partial}{\partial x_i} \left[(n_1 - 1)s_1^2 + (n_2 - 1)s_2^2\right] \\
                                          & = \frac{1}{df}\frac{1}{2 s_p}2(x_i - \bar{x}_g)                                                           \\
        \frac{\partial s_p}{\partial x_i} & = \frac{x_i - \bar{x}_g}{df s_p}
    \end{align*}
\end{proof}

\subsubsection{Sample Covariance}
\label{sec:sample_covariance_derivation}

The partial derivative of the sample covariance with respect to \(x_i\) is
\begin{equation}
    \frac{\partial s_{xy}}{\partial x_i} = \frac{y_i - \overline{y}}{(n-1)}.
    \label{eq:sample_covariance_derivative}
\end{equation}
\begin{proof}
    The sample covariance between two variables \(x\) and \(y\) is defined as
    \[
        s_{xy} = \frac{1}{n-1} \sum_{j=1}^{n} (x_j - \overline{x})(y_j - \overline{y}).
    \]
    Taking the partial derivative with respect to \(x_i\) gives
    \begin{align*}
        \frac{\partial s_{xy}}{\partial x_i}
         & = \frac{1}{n-1} \sum_{j=1}^{n}
        \frac{\partial}{\partial x_i} \left[ (x_j - \overline{x})(y_j - \overline{y}) \right] \\
         & = \frac{1}{n-1} \sum_{j=1}^{n}
        (y_j - \overline{y}) \frac{\partial (x_j - \overline{x})}{\partial x_i}               \\
         & = \frac{1}{n-1} \sum_{j=1}^{n}
        (y_j - \overline{y}) \left( \delta_{ij} - \frac{1}{n} \right)                         \\
         & = \frac{1}{n-1} \left[ (y_i - \overline{y})\!\left( 1 - \frac{1}{n} \right)
        - \frac{1}{n} \sum_{\substack{j=1                                                     \\ j \ne i}}^{n} (y_j - \overline{y}) \right].
    \end{align*}
    Since \(\sum_{j=1}^{n} (y_j - \overline{y}) = 0\) then \(\sum_{\substack{j=1 \\ j \ne i}}^{n} (y_j - \overline{y}) = - (y_i - \overline{y})\) so the second term simplifies, giving
    \[
        \frac{\partial s_{xy}}{\partial x_i}
        = \frac{y_i - \overline{y}}{(n-1)s_x}.
    \]
\end{proof}

\subsubsection{Pearson correlation coefficient}
\label{sec:pearson_derivation}
The partial derivative of \(r_{x,y}\) with respect to an individual
observation \(x_i\) is given by
\begin{equation}
    \frac{\partial r_{x,y}}{\partial x_i} = \frac{1}{(n-1)s_x } \left( \frac{y_i - \overline{y}}{s_y} - \frac{r_{x,y}}{s_x} \frac{x_i - \overline{x}}{s_x} \right).
    \label{eq:pearson_derivative}
\end{equation}

\begin{proof}
    The Pearson correlation coefficient \(r\) between two variables \(x\) and \(y\)
    is defined as
    \[
        r_{x,y} = \frac{s_{xy}}{s_x s_y},
    \]
    using the quotient rule, we differentiate \(r(x,y)\) with respect to \(x_i\):
    \begin{align*}
        \frac{\partial r_{x,y}}{\partial x_i} & = \frac{1}{s_x^2 s_y^2}\left[\frac{\partial s_{xy}}{\partial x_i} \cdot s_x s_y - s_{xy} \cdot \frac{\partial s_x s_y}{\partial x_i}\right] \\
                                              & = \frac{1}{s_x s_y} \frac{\partial s_{xy}}{\partial x_i} - \frac{s_{xy}}{s_x^2 s_y} \frac{\partial s_x}{\partial x_i}                       \\
                                              & = \frac{1}{s_x s_y} \frac{\partial s_{xy}}{\partial x_i} - \frac{r_{x,y}}{s_x} \frac{\partial s_x}{\partial x_i}.
    \end{align*}
    Substituting the partial derivatives of the sample covariance (Eq.~\ref{eq:sample_covariance_derivative}) and standard deviation (Eq.~\ref{eq:sample_std_derivative}) we obtain
    \begin{align*}
        \frac{\partial r_{x,y}}{\partial x_i} & = \frac{1}{s_x s_y} \cdot \frac{y_i - \overline{y}}{(n-1)} - \frac{r_{x,y}}{s_x} \cdot \frac{x_i - \overline{x}}{(n-1)s_x} \\
                                              & = \frac{1}{s_x s_y} \cdot \frac{y_i - \overline{y}}{(n-1)} - \frac{r_{x,y}}{s_x^2} \cdot \frac{x_i - \overline{x}}{(n-1)}  \\
                                              & = \frac{1}{(n-1)s_x} \left( \frac{y_i - \overline{y}}{s_y} - \frac{r_{x,y}}{s_x} \frac{x_i - \overline{x}}{s_x} \right).
    \end{align*}
\end{proof}

\subsubsection{Correlation identities}
\label{sec:correlation_identities}
Let \(\tilde{x}_i = (x_i - \overline{x})/s_x\), \(\tilde{y}_i = (y_i - \overline{y})/s_y\) and \(\tilde{z}_i = (z_i - \overline{z})/s_z\) be the standardized variables.
The following identities hold:
\begin{align}
    \sum_{i=1}^{n} \left( \tilde{x}_i - r_{xy} \tilde{y}_i \right)^{2}                                            & = (n-1)(1 - r_{xy}^2)          \\
    \sum_{i=1}^{n} \left( \tilde{y}_i - r_{xy} \tilde{x}_i \right)\left( \tilde{z}_i - r_{xz} \tilde{x}_i \right) & = (n-1)(r_{xy}r_{xz} - r_{yz})
    \label{eq:correlation_identity}
\end{align}

\begin{proof}
    Let note that \((n-1)s_x = \sum_{i=1}^{n} (x_i - \overline{x})^2\) and \((n-1)s_{xy} = \sum_{i=1}^{n} (x_i - \overline{x})(y_i - \overline{y})\) then by using the definitions of standardized variables, we have for the first identity:
    \begin{align*}
        \sum_{i=1}^{n} \left( \tilde{x}_i - r_{xy} \tilde{y}_i \right)^{2} & = \sum_{i=1}^{n} \left( \tilde{x}_i^2 - 2r_{xy} \tilde{x}_i \tilde{y}_i + r_{xy}^2 \tilde{y}_i^2 \right)                                                                                                  \\
                                                                           & = \sum_{i=1}^{n} \left[ \frac{(x_i - \overline{x})^2}{s_x^2} - 2r_{xy} \frac{(x_i - \overline{x})(y_i - \overline{y})}{s_x s_y} + r_{xy}^2 \frac{(y_i - \overline{y})^2}{s_y^2} \right]                   \\
                                                                           & = \frac{1}{s_x^2} \sum_{i=1}^{n} (x_i - \overline{x})^2 - \frac{2r_{xy}}{s_x s_y} \sum_{i=1}^{n} (x_i - \overline{x})(y_i - \overline{y}) +  \frac{r_{xy}^2}{s_y^2} \sum_{i=1}^{n} (y_i - \overline{y})^2 \\
                                                                           & = (n-1) - 2r_{xy} (n-1) r_{xy} + r_{xy}^2 (n-1)                                                                                                                                                           \\
                                                                           & = (n-1)(1 - 2r_{xy}^2 + r_{xy}^2)                                                                                                                                                                         \\
                                                                           & = (n-1)(1 - r_{xy}^2).
    \end{align*}
    and for the second identity:
    \begin{align*}
        \sum_{i=1}^{n} \left( \tilde{y}_i - r_{xy} \tilde{x}_i \right)\left( \tilde{z}_i - r_{xz} \tilde{x}_i \right)
         & = \sum_{i=1}^{n} \left( \tilde{y}_i \tilde{z}_i - r_{xz} \tilde{y}_i \tilde{x}_i - r_{xy} \tilde{x}_i \tilde{z}_i + r_{xy} r_{xz} \tilde{x}_i^2 \right) \\
         & = (n-1) r_{yz} - r_{xz} (n-1) r_{xy} - r_{xy} (n-1) r_{xz} + r_{xy} r_{xz} (n-1)                                                                        \\
         & = (n-1)(r_{yz} - 2 r_{xy} r_{xz} + r_{xy} r_{xz})                                                                                                       \\
         & = (n-1)(r_{xy} r_{xz} - r_{yz}).
    \end{align*}
\end{proof}

\section{Cross-sectional Analysis}

As a side result, the cross-sectional analysis measures the impact of numerical
variability in FreeSurfer version 7.3.1 on the PPMI (Parkinson's Progression
Markers Initiative) cohort. This involves comparing the estimation of
structural MRI measures, including cortical and subcortical volumes, cortical
thickness, and surface area. The goal is to assess the stability of these key
metrics and quantify the numerical variability.

FreeSurfer 7.3.1 showed limited numerical precision across all cortical
measures: $1.61 \pm 0.20$ significant digits for cortical thickness, $1.33 \pm
    0.23$ for surface area, and $1.33 \pm 0.23$ for cortical volume
(Figures~\ref{fig:sig_digits_cortical}). Subcortical volumes have a similar
precision with $1.33 \pm 0.22$ significant digits on average
(Figure~\ref{fig:sig_digits_subcortical}). These values indicate measurements
are typically precise to only one decimal place, with some instances showing
complete precision loss. Regional consistency was observed within each metric
type, with cortical thickness showing the highest precision (range: $1.22-1.93$
digits) compared to surface area ($0.82 - 1.72$ digits) and cortical volume
($0.80 - 1.72$ digits). Subcortical volumes exhibited the highest precision
(range: $0.88 - 1.57$ digits), with a mean of $1.33 \pm 0.22$ significant
digits.

To measure the structural overlap, we evaluated using the extended
Sørensen-Dice coefficient: Dice coefficients revealed substantial inter-subject
variability, particularly in temporal pole regions (Figure~\ref{fig:dice}). We
also observed that the Dice coefficient varies across regions, with some
regions showing higher variability than others with cortical volume ($0.00 -
    0.91$) with a mean of $0.75 \pm 0.11$ and subcortical volume ($0.18 - 0.94$)
with a mean of $0.82 \pm 0.08$. Finally, we noticed that subcortical volume
measurements are more stable than cortical volume.

\begin{figure}
    \includegraphics*[width=\linewidth]{figures/dice.pdf}
    \caption{Dice coefficient.\label{fig:dice}}

\end{figure}

\begin{figure}
    \includegraphics*[width=\linewidth]{figures/sig_digits.pdf}
    \caption{Number of significant digits for each cortical region and
        metric.\label{fig:sig_digits_cortical}}
\end{figure}

\begin{figure}
    \includegraphics*[width=\linewidth]{figures/sig_digits_subcortical_volume.pdf}
    \caption{Number of significant digits of subcortical volume for each
        subcortical region.\label{fig:sig_digits_subcortical}}
\end{figure}

\subsection{Within-subject significant digits averaged across all subjects}

\begin{longtblr}[ caption={Within-subject significant digits averaged across all subjects.},
        label={tab:sig-cortical},]{ colspec={lcc|cc|cc}, width=0.25\linewidth,
        row{even}={white,font=\footnotesize},
        row{odd}={gray9,font=\footnotesize}, rows = {rowsep=0pt}, rowhead=2,
    row{1}={white,font=\bfseries}, row{2}={gray9}} \SetCell[c=1]{c}Region &
    \SetCell[c=2]{c}{cortical thickness }                                 &                                   &
    \SetCell[c=2]{c}{surface area}                                        &
                                                                          & \SetCell[c=2]{c}{cortical volume} &
    \\
                                                                          & lh                                &
    rh                                                                    & lh
                                                                          & rh                                & lh
                                                                          & rh                                                                    \\
    \hline
    bankssts                                                              & $1.65 \pm 0.16$                   &
    $1.69 \pm 0.13$                                                       & $1.15 \pm 0.18$
                                                                          & $1.21 \pm 0.13$                   & $1.08 \pm 0.17$ & $1.14 \pm 0.13$
    \\
    caudalanteriorcingulate                                               & $1.38 \pm 0.14$                   &
    $1.40 \pm 0.14$                                                       & $1.14 \pm 0.22$
                                                                          & $1.19 \pm 0.18$                   & $1.14 \pm 0.24$ & $1.21 \pm 0.20$
    \\
    caudalmiddlefrontal                                                   & $1.77 \pm 0.18$                   &
    $1.77 \pm 0.19$                                                       & $1.40 \pm 0.21$
                                                                          & $1.31 \pm 0.23$                   & $1.40 \pm 0.22$ & $1.30 \pm 0.23$
    \\
    cuneus                                                                & $1.52 \pm 0.19$                   &
    $1.54 \pm 0.19$                                                       & $1.34 \pm 0.14$
                                                                          & $1.33 \pm 0.14$                   & $1.32 \pm 0.14$ & $1.28 \pm 0.15$
    \\
    entorhinal                                                            & $1.22 \pm 0.23$                   &
    $1.22 \pm 0.23$                                                       & $0.82 \pm 0.19$
                                                                          & $0.87 \pm 0.18$                   & $0.80 \pm 0.19$ & $0.81 \pm 0.18$
    \\
    fusiform                                                              & $1.66 \pm 0.17$                   &
    $1.71 \pm 0.16$                                                       & $1.41 \pm 0.18$
                                                                          & $1.43 \pm 0.19$                   & $1.33 \pm 0.18$ & $1.37 \pm 0.20$
    \\
    inferiorparietal                                                      & $1.81 \pm 0.15$                   &
    $1.82 \pm 0.13$                                                       & $1.53 \pm 0.18$
                                                                          & $1.59 \pm 0.20$                   & $1.50 \pm 0.17$ & $1.56 \pm 0.17$
    \\
    inferiortemporal                                                      & $1.66 \pm 0.17$                   &
    $1.70 \pm 0.16$                                                       & $1.37 \pm 0.25$
                                                                          & $1.38 \pm 0.21$                   & $1.37 \pm 0.23$ & $1.41 \pm 0.19$
    \\
    isthmuscingulate                                                      & $1.46 \pm 0.12$                   &
    $1.43 \pm 0.13$                                                       & $1.27 \pm 0.15$
                                                                          & $1.24 \pm 0.15$                   & $1.27 \pm 0.14$ & $1.27 \pm 0.15$
    \\
    lateraloccipital                                                      & $1.75 \pm 0.18$                   &
    $1.77 \pm 0.17$                                                       & $1.58 \pm 0.15$
                                                                          & $1.57 \pm 0.16$                   & $1.49 \pm 0.16$ & $1.50 \pm 0.15$
    \\
    lateralorbitofrontal                                                  & $1.65 \pm 0.17$                   &
    $1.51 \pm 0.15$                                                       & $1.44 \pm 0.23$
                                                                          & $0.95 \pm 0.13$                   & $1.51 \pm 0.16$ & $1.12 \pm 0.14$
    \\
    lingual                                                               & $1.54 \pm 0.22$                   &
    $1.52 \pm 0.21$                                                       & $1.47 \pm 0.18$
                                                                          & $1.46 \pm 0.17$                   & $1.50 \pm 0.18$ & $1.49 \pm 0.18$
    \\
    medialorbitofrontal                                                   & $1.50 \pm 0.15$                   &
    $1.53 \pm 0.15$                                                       & $1.09 \pm 0.16$
                                                                          & $1.15 \pm 0.14$                   & $1.15 \pm 0.17$ & $1.21 \pm 0.13$
    \\
    middletemporal                                                        & $1.74 \pm 0.16$                   &
    $1.81 \pm 0.14$                                                       & $1.42 \pm 0.23$
                                                                          & $1.52 \pm 0.19$                   & $1.44 \pm 0.21$ & $1.55 \pm 0.18$
    \\
    parahippocampal                                                       & $1.54 \pm 0.14$                   &
    $1.56 \pm 0.12$                                                       & $1.13 \pm 0.13$
                                                                          & $1.09 \pm 0.13$                   & $1.11 \pm 0.13$ & $1.07 \pm 0.13$
    \\
    paracentral                                                           & $1.59 \pm 0.22$                   &
    $1.60 \pm 0.22$                                                       & $1.40 \pm 0.17$
                                                                          & $1.40 \pm 0.19$                   & $1.36 \pm 0.18$ & $1.36 \pm 0.20$
    \\
    parsopercularis                                                       & $1.74 \pm 0.17$                   &
    $1.71 \pm 0.16$                                                       & $1.38 \pm 0.19$
                                                                          & $1.30 \pm 0.18$                   & $1.38 \pm 0.19$ & $1.30 \pm 0.20$
    \\
    parsorbitalis                                                         & $1.53 \pm 0.20$                   &
    $1.51 \pm 0.20$                                                       & $1.21 \pm 0.14$
                                                                          & $1.21 \pm 0.18$                   & $1.19 \pm 0.16$ & $1.22 \pm 0.18$
    \\
    parstriangularis                                                      & $1.68 \pm 0.17$                   &
    $1.63 \pm 0.19$                                                       & $1.33 \pm 0.16$
                                                                          & $1.30 \pm 0.22$                   & $1.30 \pm 0.16$ & $1.28 \pm 0.21$
    \\
    pericalcarine                                                         & $1.33 \pm 0.21$                   &
    $1.30 \pm 0.22$                                                       & $1.23 \pm 0.20$
                                                                          & $1.21 \pm 0.22$                   & $1.18 \pm 0.17$ & $1.18 \pm 0.17$
    \\
    postcentral                                                           & $1.84 \pm 0.24$                   &
    $1.81 \pm 0.26$                                                       & $1.68 \pm 0.23$
                                                                          & $1.69 \pm 0.28$                   & $1.64 \pm 0.20$ & $1.63 \pm 0.24$
    \\
    posteriorcingulate                                                    & $1.57 \pm 0.13$                   &
    $1.56 \pm 0.14$                                                       & $1.37 \pm 0.20$
                                                                          & $1.35 \pm 0.21$                   & $1.39 \pm 0.19$ & $1.39 \pm 0.22$
    \\
    precentral                                                            & $1.79 \pm 0.26$                   &
    $1.76 \pm 0.28$                                                       & $1.71 \pm 0.24$
                                                                          & $1.64 \pm 0.27$                   & $1.72 \pm 0.22$ & $1.66 \pm 0.28$
    \\
    precuneus                                                             & $1.83 \pm 0.13$                   &
    $1.84 \pm 0.13$                                                       & $1.65 \pm 0.21$
                                                                          & $1.66 \pm 0.21$                   & $1.61 \pm 0.18$ & $1.62 \pm 0.19$
    \\
    rostralanteriorcingulate                                              & $1.34 \pm 0.14$                   &
    $1.39 \pm 0.15$                                                       & $1.00 \pm 0.16$
                                                                          & $1.07 \pm 0.17$                   & $1.11 \pm 0.19$ & $1.11 \pm 0.18$
    \\
    rostralmiddlefrontal                                                  & $1.77 \pm 0.19$                   &
    $1.74 \pm 0.19$                                                       & $1.44 \pm 0.24$
                                                                          & $1.41 \pm 0.28$                   & $1.49 \pm 0.21$ & $1.48 \pm 0.25$
    \\
    superiorfrontal                                                       & $1.87 \pm 0.17$                   &
    $1.85 \pm 0.18$                                                       & $1.61 \pm 0.23$
                                                                          & $1.56 \pm 0.27$                   & $1.64 \pm 0.21$ & $1.62 \pm 0.25$
    \\
    superiorparietal                                                      & $1.92 \pm 0.18$                   &
    $1.93 \pm 0.17$                                                       & $1.72 \pm 0.24$
                                                                          & $1.65 \pm 0.28$                   & $1.66 \pm 0.22$ & $1.60 \pm 0.26$
    \\
    superiortemporal                                                      & $1.83 \pm 0.17$                   &
    $1.85 \pm 0.15$                                                       & $1.57 \pm 0.22$
                                                                          & $1.58 \pm 0.18$                   & $1.52 \pm 0.21$ & $1.57 \pm 0.18$
    \\
    supramarginal                                                         & $1.83 \pm 0.16$                   &
    $1.85 \pm 0.15$                                                       & $1.57 \pm 0.22$
                                                                          & $1.59 \pm 0.26$                   & $1.56 \pm 0.20$ & $1.56 \pm 0.24$
    \\
    frontalpole                                                           & $1.26 \pm 0.23$                   &
    $1.23 \pm 0.20$                                                       & $0.94 \pm 0.11$
                                                                          & $0.91 \pm 0.11$                   & $0.88 \pm 0.17$ & $0.87 \pm 0.14$
    \\
    temporalpole                                                          & $1.24 \pm 0.26$                   &
    $1.28 \pm 0.25$                                                       & $0.94 \pm 0.16$
                                                                          & $0.99 \pm 0.19$                   & $0.86 \pm 0.20$ & $0.91 \pm 0.22$
    \\
    transversetemporal                                                    & $1.47 \pm 0.20$                   &
    $1.46 \pm 0.18$                                                       & $1.17 \pm 0.13$
                                                                          & $1.13 \pm 0.11$                   & $1.20 \pm 0.15$ & $1.15 \pm 0.13$
    \\
    insula                                                                & $1.47 \pm 0.16$                   &
    $1.42 \pm 0.14$                                                       & $1.13 \pm 0.18$
                                                                          & $1.00 \pm 0.18$                   & $1.29 \pm 0.16$ & $1.19 \pm 0.19$
    \\
\end{longtblr}

\begin{longtblr}[ caption={Within-subject standard-deviation averaged across all subjects for
                cortical metrics.}, label={tab:std-cortical}, ]{
        colspec={lcc|cc|cc}, width=\linewidth,
        row{even}={white,font=\footnotesize},
        row{odd}={gray9,font=\footnotesize}, rows = {rowsep=0pt},
        rowhead=2, row{1}={white,font=\bfseries}, row{2}={gray9}}
    \SetCell[c=1]{c}Region   & \SetCell[c=2]{c}{cortical thickness                                      \\
    (mm)}                    &                                     & \SetCell[c=2]{c}{surface area      \\
    ($\text{mm}^2$)}         &                                     &
    \SetCell[c=2]{c}{cortical volume                                                                    \\ ($\text{mm}^3$)} &
    \\
                             & lh                                  & rh                            & lh
                             & rh                                  & lh                            & rh
    \\
    \hline
    bankssts                 & $0.02 \pm 0.01$                     & $0.02 \pm
    0.01$                    & $\028.65 \pm \015.97$               & $\021.73
    \pm \0\08.68$            & $\077.25 \pm \037.44$               & $\059.87
        \pm \020.45$
    \\
    caudalanteriorcingulate  & $0.04 \pm 0.01$                     & $0.04 \pm
    0.01$                    & $\019.98 \pm \013.83$               & $\021.01
    \pm \014.96$             & $\051.33 \pm \037.32$               & $\051.67
        \pm \041.74$
    \\
    caudalmiddlefrontal      & $0.02 \pm 0.01$                     & $0.02 \pm
    0.01$                    & $\038.58 \pm \036.77$               & $\046.65
    \pm \044.68$             & $104.41 \pm 108.02$                 & $124.11 \pm
    112.10$                                                                                             \\
    cuneus                   & $0.02 \pm 0.01$                     & $0.02 \pm
    0.01$                    & $\028.45 \pm \011.50$               & $\031.25
    \pm \015.67$             & $\060.72 \pm \025.52$               & $\074.77
        \pm \034.16$
    \\
    entorhinal               & $0.08 \pm 0.05$                     & $0.08 \pm
    0.05$                    & $\027.41 \pm \016.67$               & $\022.37
    \pm \011.70$             & $125.48 \pm \071.07$                & $115.94 \pm
    \057.21$                                                                                            \\
    fusiform                 & $0.02 \pm 0.01$                     & $0.02 \pm
    0.01$                    & $\050.70 \pm \025.16$               & $\047.86
    \pm \028.19$             & $182.92 \pm \092.31$                & $170.22 \pm
    103.05$                                                                                             \\
    inferiorparietal         & $0.01 \pm 0.01$                     & $0.01 \pm
    0.01$                    & $\053.01 \pm \029.19$               & $\059.90
    \pm \050.62$             & $145.66 \pm \072.95$                & $159.55 \pm
    110.14$                                                                                             \\
    inferiortemporal         & $0.02 \pm 0.01$                     & $0.02 \pm
    0.01$                    & $\064.73 \pm \042.27$               & $\058.75
    \pm \034.04$             & $198.15 \pm 127.44$                 & $168.38 \pm
    \084.67$                                                                                            \\
    isthmuscingulate         & $0.03 \pm 0.01$                     & $0.03 \pm
    0.01$                    & $\023.74 \pm \011.07$               & $\023.35
    \pm \013.99$             & $\057.43 \pm \029.59$               & $\053.05
        \pm \034.34$
    \\
    lateraloccipital         & $0.02 \pm 0.01$                     & $0.02 \pm
    0.01$                    & $\053.82 \pm \024.63$               & $\056.35
    \pm \028.61$             & $156.83 \pm \066.16$                & $160.98 \pm
    \076.00$                                                                                            \\
    lateralorbitofrontal     & $0.02 \pm 0.01$                     & $0.03 \pm
    0.01$                    & $\043.31 \pm \030.16$               & $117.14 \pm
    \033.75$                 & $\092.60 \pm \056.29$               & $217.89 \pm
    \069.06$                                                                                            \\
    lingual                  & $0.03 \pm 0.01$                     & $0.03 \pm
    0.01$                    & $\044.26 \pm \022.65$               & $\046.73
    \pm \023.96$             & $\089.19 \pm \046.24$               & $\095.82
        \pm \049.65$
    \\
    medialorbitofrontal      & $0.03 \pm 0.01$                     & $0.03 \pm
    0.01$                    & $\066.04 \pm \024.11$               & $\058.06
    \pm \019.00$             & $147.37 \pm \057.84$                & $134.52 \pm
    \042.26$                                                                                            \\
    middletemporal           & $0.02 \pm 0.01$                     & $0.02 \pm
    0.01$                    & $\053.01 \pm \034.97$               & $\044.87
    \pm \028.36$             & $165.49 \pm 108.52$                 & $135.26 \pm
    \077.98$                                                                                            \\
    parahippocampal          & $0.03 \pm 0.01$                     & $0.03 \pm
    0.01$                    & $\019.55 \pm \0\08.42$              & $\020.45
    \pm \0\07.81$            & $\064.22 \pm \025.29$               & $\065.43
        \pm \024.59$
    \\
    paracentral              & $0.03 \pm 0.02$                     & $0.03 \pm
    0.01$                    & $\022.94 \pm \012.98$               & $\026.94
    \pm \019.80$             & $\063.71 \pm \040.74$               & $\073.88
        \pm \056.66$
    \\
    parsopercularis          & $0.02 \pm 0.01$                     & $0.02 \pm
    0.01$                    & $\028.65 \pm \028.77$               & $\029.46
    \pm \026.82$             & $\080.67 \pm \092.87$               & $\082.38
        \pm \089.16$
    \\
    parsorbitalis            & $0.03 \pm 0.02$                     & $0.03 \pm
    0.02$                    & $\017.82 \pm \0\09.77$              & $\021.41
    \pm \010.66$             & $\060.63 \pm \045.20$               & $\068.18
        \pm \036.64$
    \\
    parstriangularis         & $0.02 \pm 0.01$                     & $0.02 \pm
    0.01$                    & $\025.67 \pm \014.65$               & $\034.86
    \pm \037.79$             & $\071.73 \pm \045.49$               & $\096.87
        \pm 102.22$
    \\
    pericalcarine            & $0.03 \pm 0.02$                     & $0.04 \pm
    0.02$                    & $\036.04 \pm \020.18$               & $\042.02
    \pm \024.82$             & $\059.64 \pm \029.98$               & $\068.61
        \pm \034.89$
    \\
    postcentral              & $0.01 \pm 0.02$                     & $0.02 \pm
    0.02$                    & $\043.47 \pm \067.12$               & $\045.98
    \pm \083.10$             & $100.26 \pm 121.35$                 & $104.53 \pm
    156.51$                                                                                             \\
    posteriorcingulate       & $0.02 \pm 0.01$                     & $0.02 \pm
    0.01$                    & $\021.93 \pm \013.05$               & $\024.39
    \pm \019.52$             & $\052.42 \pm \033.33$               & $\056.27
        \pm \052.59$
    \\
    precentral               & $0.02 \pm 0.02$                     & $0.02 \pm
    0.02$                    & $\046.92 \pm \053.54$               & $\057.46
    \pm \070.35$             & $118.04 \pm 157.21$                 & $148.21 \pm
    233.10$                                                                                             \\
    precuneus                & $0.01 \pm 0.01$                     & $0.01 \pm
    0.00$                    & $\038.04 \pm \042.87$               & $\038.95
    \pm \040.96$             & $100.91 \pm 111.15$                 & $102.24 \pm
    \096.62$                                                                                            \\
    rostralanteriorcingulate & $0.05 \pm 0.02$                     & $0.04 \pm
    0.02$                    & $\034.80 \pm \015.03$               & $\022.00
    \pm \010.59$             & $\081.04 \pm \041.59$               & $\061.95
        \pm \033.93$
    \\
    rostralmiddlefrontal     & $0.02 \pm 0.01$                     & $0.02 \pm
    0.01$                    & $\092.87 \pm \096.23$               & $108.40 \pm
    132.97$                  & $213.81 \pm 259.58$                 & $252.00 \pm
    358.20$                                                                                             \\
    superiorfrontal          & $0.01 \pm 0.01$                     & $0.01 \pm
    0.01$                    & $\085.23 \pm \086.47$               & $\098.14
    \pm 120.75$              & $223.91 \pm 234.89$                 & $243.75 \pm
    304.56$                                                                                             \\
    superiorparietal         & $0.01 \pm 0.01$                     & $0.01 \pm
    0.01$                    & $\049.49 \pm \080.81$               & $\062.89
    \pm \096.86$             & $132.77 \pm 207.97$                 & $161.39 \pm
    235.01$                                                                                             \\
    superiortemporal         & $0.02 \pm 0.01$                     & $0.01 \pm
    0.01$                    & $\047.70 \pm \033.64$               & $\041.38
    \pm \023.84$             & $156.30 \pm 101.85$                 & $129.01 \pm
    \078.70$                                                                                            \\
    supramarginal            & $0.01 \pm 0.01$                     & $0.01 \pm
    0.01$                    & $\050.87 \pm \058.82$               & $\050.06
    \pm \083.24$             & $136.23 \pm 168.28$                 & $133.99 \pm
    207.69$                                                                                             \\
    frontalpole              & $0.07 \pm 0.04$                     & $0.07 \pm
    0.04$                    & $\012.99 \pm \0\04.02$              & $\016.42
    \pm \0\04.47$            & $\056.49 \pm \032.17$               & $\067.84
        \pm \028.93$
    \\
    temporalpole             & $0.09 \pm 0.05$                     & $0.08 \pm
    0.05$                    & $\025.08 \pm \010.71$               & $\022.16
    \pm \011.78$             & $154.60 \pm \079.32$                & $138.28 \pm
    \078.33$                                                                                            \\
    transversetemporal       & $0.03 \pm 0.02$                     & $0.03 \pm
    0.02$                    & $\012.73 \pm \0\05.33$              & $\0\09.98
    \pm \0\03.33$            & $\029.55 \pm \012.34$               & $\024.91
        \pm \0\08.79$
    \\
    insula                   & $0.04 \pm 0.02$                     & $0.04 \pm
    0.01$                    & $\073.45 \pm \030.66$               & $\095.70
    \pm \037.63$             & $146.49 \pm \064.11$                & $183.39 \pm
    \081.47$                                                                                            \\
\end{longtblr}

\begin{longtblr}[ caption={Within-subject significant digits averaged across all
                subjects for subcortical volumes.},
        label={tab:sig-std-subcortical-volume},]{ colspec={lc|c},
        row{even}={gray9,font=\footnotesize},
        row{odd}={white,font=\footnotesize}, rows = {rowsep=0pt},
    row{Z}={font=\small}, rowhead=1, row{1}={font=\bfseries}} Region &
    Significant digits                                               & {Standard deviation                        \\ ($\text{mm}^3$)} \\
    \hline
    Left-Thalamus                                                    & $1.42 \pm 0.21$     & $120.08  \pm 69.61$  \\
    Left-Caudate                                                     & $1.57 \pm 0.20$     & $\038.83 \pm 25.11$  \\
    Left-Putamen                                                     & $1.49 \pm 0.22$     & $\065.88 \pm 46.39$  \\
    Left-Pallidum                                                    & $1.25 \pm 0.19$     & $\047.81 \pm 25.09$  \\
    Left-Hippocampus                                                 & $1.48 \pm 0.17$     & $\056.23 \pm 41.03$  \\
    Left-Amygdala                                                    & $1.13 \pm 0.16$     & $\048.71 \pm 20.04$  \\
    Left-Accumbens-area                                              & $0.88 \pm 0.16$     & $\024.20 \pm \08.80$ \\
    Right-Thalamus                                                   & $1.42 \pm 0.20$     & $118.92  \pm 68.76$  \\
    Right-Caudate                                                    & $1.51 \pm 0.24$     & $\049.37 \pm 42.71$  \\
    Right-Putamen                                                    & $1.51 \pm 0.25$     & $\068.07 \pm 70.23$  \\
    Right-Pallidum                                                   & $1.22 \pm 0.19$     & $\049.11 \pm 30.50$  \\
    Right-Hippocampus                                                & $1.55 \pm 0.18$     & $\048.59 \pm 28.98$  \\
    Right-Amygdala                                                   & $1.23 \pm 0.17$     & $\042.21 \pm 18.68$  \\
    Right-Accumbens-area                                             & $0.99 \pm 0.15$     & $\020.50 \pm \07.72$ \\
\end{longtblr}

\begin{table}[h]
    \centering
    \caption{Summary of executions failure and excluded subjects. To standardize
        the sample, we keep 26 repetitions per subject/visits pair.
        Subject/visit pairs with less than 26 repetitions were excluded which is
        12 subjects.}
    \begin{tabular}{l c c}
        \toprule
        \textbf{Stage}     & \textbf{Number of rejected repetitions} &
        \textbf{Total number of repetitions}                                 \\
        \midrule
        Cluster failure    & 1246 (5.80\%)                           & 21488 \\
        FreeSurfer failure & 68 (0.33\%)                             & 21488 \\
        QC failure         & 319 (1.48\%)                            & 21488 \\
        Total              & 1633 (7.60\%)                           & 21488 \\
        \bottomrule
    \end{tabular}
\end{table}

\begin{table}[h!]
    \centering
    \begin{tabular}{c|lccc}
        \toprule
        \textbf{Status} & \textbf{Cohort}             & \textbf{HC}
                        & \textbf{PD-non-MCI}         & \textbf{PD-MCI}
        \\
        \hline
        \multirow{5}{*}{\textbf{\shortstack{Before                                    \\QC}}} & n
                        & 106                         & 181                      & 29 \\
                        & Age (y)                     & $60.6 \pm 10.2   $
                        & $61.7 \pm \09.6$            & $67.7 \pm \07.7$
        \\
                        & Age range                   & $30.6 - 84.3  $
                        & $36.3 - 83.3$               & $49.9 - 80.5$
        \\
                        & Gender (male, \%)           & $58 \; (54.7\%)   $
                        & $119 \; (65.7\%)          $ & $-          $
        \\
                        & Education (y)               & $16.6 \pm \03.3  $
                        & $15.9 \pm \02.9$            & $-          $
        \\
        \hline
        \multirow{5}{*}{\textbf{\shortstack{After                                     \\QC}}} & n
                        & 103                         & 175                      & 27 \\
                        & Age (y)                     & $60.7 \pm 10.3   $
                        & $61.4 \pm \09.5          $  & $67.8 \pm \07.9$
        \\
                        & Age range                   & $30.6 - 84.3  $
                        & $36.3 - 79.9           $    & $49.9 - 80.5$
        \\
                        & Gender (male, \%)           & $57 \; (55.3\%)   $
                        & $114 \; (65.1\%)       $    & $20 \; (74.1\%) $
        \\
                        & Education (y)               & $16.6 \pm \03.3  $
                        & $15.9 \pm \02.9        $    & $15.0 \pm \03.5$
        \\
        \hline
        \multirow{8}{*}{\textbf{\shortstack{After                                     \\MCI\\exclusion}}} & n
                        & $103 $                      & $121                   $ & -- \\
                        & Age (y)                     & $60.7 \pm 10.3   $
                        & $60.7 \pm \09.1        $    & --
        \\
                        & Age range                   & $30.6 - 84.3  $
                        & $39.2 - 78.3           $    & --
        \\
                        & Gender (male, \%)           & $57 \; (55.3\%)   $
                        & $80 \; (66.1\%)        $    & --
        \\
                        & Education (y)               & $16.6 \pm \03.3  $
                        & $16.1 \pm \03.0        $    & --
        \\
                        & UPDRS III OFF baseline      & $-            $
                        & $23.4 \pm 10.1         $    & --
        \\
                        & UPDRS III OFF follow-up     & $-            $
                        & $25.8 \pm 11.1         $    & --
        \\
                        & Duration T2 - T1 (y)        & $\01.4 \pm \00.5 $
                        & $\01.4 \pm \00.7       $    & --
        \\
        \bottomrule
    \end{tabular}
    \vspace{1em}

    \textbf{Abbreviations:} MCI = Mild Cognitive Impairment; UPDRS = Unified
    Parkinson's Disease Rating Scale; PD = Parkinson's disease. Descriptive
    statistics before and after quality control (QC). Values are expressed as
    mean $\pm$ standard deviation. PD-non-MCI longitudinal sample is a subsample
    of the PD-non-MCI original sample that had longitudinal data and disease
    severity scores available.
    \label{tab:cohort_stat_vertical}
\end{table}

\section{Numerical-Anatomical Variability Ratio (\navr)}

\subsection{\navr maps}

Figure~\ref{fig:navr_maps_area_volume} shows the \navr
maps for cortical surface area and volume, for each region. The color
scale indicates the \navr value, with warmer colors indicating higher \navr
values. The maps provide a visual representation of the variability in the
\navr values across different cortical regions, highlighting regions with
higher or lower \navr values.

\begin{figure}[t]
    \centering

    \includegraphics[width=\textwidth]{figures/NAVR_map/NAVR_area_volume.png}

    \caption{Numerical-Anatomical Variability Ratio (\navr) for cortical surface (top row in each panel)
        area and cortical volume (bottom row) in HC and PD. Panels show \navr{} maps at baseline for PD (a) and
        (HC), longitudinally for HC (c) and PD (d). Higher \navr{} values
        indicate greater computational uncertainty relative to inter-subject
        anatomical variability. Warmer colors denote higher \navr{} values.}
    \label{fig:navr_maps_area_volume}
\end{figure}


\subsection{Consistency results}

\YC{Moved from section~\ref{sec:results}. Rewrite to fit here.}

We instrumented FreeSurfer 7.3.1 with MCA (virtual precision set to mimic
realistic perturbations), and re-analyzed each T1-weighted MRI scan 26 times,
each time with a different random state. For subcortical volumes, statistical
significance ($p < 0.05$) fluctuated notably across MCA repetitions
(Figure~\ref{fig:significance_correlation_subcortical_volume}). For group
comparisons alone, eight subcortical regions alternated between significant and
non-significant outcomes, illustrating how numerical instability can directly
affect clinical interpretation. Overall, 27\% of all statistical comparisons,
including both group differences and clinical correlations, yielded
inconsistent results across numerical states. Similarly, for cortical
thickness, 19\% of the comparisons produced inconsistent results
(Fig.~\ref{fig:significance_correlation_thickness}), suggesting that potential
biomarker relationships might emerge or vanish purely due to computational
noise. This outcome-level variability mirrors the analytical variability
reported in large-scale reproducibility challenges like NARPS, where different
analysis teams reached conflicting conclusions from the same
dataset~\cite{botvinik2020variability}.

Testing for PD progression partial correlations and group differences between
PD patients and healthy controls, 27\% of statistical tests are
non-reproducible for subcortical volumes and 19\% for cortical thickness
(Extended Data Fig.~\ref{fig:significance_correlation_subcortical_volume}
and~\ref{fig:significance_correlation_thickness}). The within-subject
variability due to numerical differences peaked at 37\% and 40\% of the
observed between-subject anatomical variance in critical cortical and
subcortical regions (Fig.~\ref{fig:navr_subcortical}
and~\ref{fig:navr_thickness}). The median NAVR of approximately 0.20 and 0.16
across all cortical regions and subcortical regions underscores that numerical
uncertainty significantly affects typical neuroimaging analyses.

\begin{table}[ht]
    \centering
    \small
    \setlength{\tabcolsep}{6pt}
    \renewcommand{\arraystretch}{1.15}
    \caption{Number and percentage of regions showing significance instability across 26 MCA repetitions, for ANCOVA and partial correlation analyses at baseline and longitudinal levels.}
    \label{tab:fluctuating_regions}
    \begin{tabular}{lcccc}
        \hline
        \multirow{2}{*}{\textbf{Metric}}    &
        \multicolumn{2}{c}{\textbf{ANCOVA}} &
        \multicolumn{2}{c}{\textbf{Partial correlation}}                                                                            \\
        \cline{2-5}
                                            & \textbf{Baseline} & \textbf{Longitudinal} & \textbf{Baseline} & \textbf{Longitudinal} \\
        \hline
        Cortical Area (68 regions)          & 18 (27\%)         & 36 (53\%)             & 4 (6\%)           & 26 (38\%)             \\
        Cortical Thickness (68 regions)     & 11 (16\%)         & 9 (13\%)              & 19 (28\%)         & 17 (25\%)             \\
        Cortical Volume (68 regions)        & 14 (20\%)         & 20 (29\%)             & 8 (12\%)          & 36 (53\%)             \\
        Subcortical Volume (14 regions)     & 4 (29\%)          & 2 (14\%)              & 4 (29\%)          & 5 (36\%)              \\
        \hline
    \end{tabular}
\end{table}

% Updated Ansari-Bradley Test Results Tables

% Table 1: Subcortical Structures
\begin{table}[h]
    \centering
    \centering
    \caption{Ansari–Bradley Test Results for Subcortical Structures.
        * indicates FDR-corrected significance ($p<0.05$). L/R = Left/Right.  W = statistic, p = p-value.}
    \label{tab:stats-coef-var-subcortical}
    \begin{tblr}[
        ]
        {
            width = \textwidth,
            colspec = {l | Q[c,m] Q[c,m] | Q[c,m] Q[c,m]},
            row{odd} = {gray9},
            row{even} = {white},
            row{1} = {font=\bfseries, white},
            row{2} = {font=\bfseries, gray9},
            rows = {rowsep=0pt},
            rowhead = 2,
        }
        \hline
        \SetCell[r=2]{m} Region & \SetCell[c=2]{c} ANCOVA &          & \SetCell[c=2]{c} Partial Corr &         \\
        \hline
                                & W                       & p        & W                             & p       \\
        \hline
        L-Thalamus              & 248                     & 1.00     & 465                           & 6.1e-6* \\
        L-Caudate               & 501                     & 3.1e-10* & 459                           & 2.0e-5* \\
        L-Putamen               & 327                     & 0.81     & 491                           & 9.6e-9* \\
        L-Pallidum              & 264                     & 1.00     & 449                           & 1.1e-4* \\
        L-Hippocampus           & 314                     & 0.91     & 428                           & 2.2e-3* \\
        L-Amygdala              & 261                     & 1.00     & 476                           & 5.5e-7* \\
        L-Accumbens             & 265                     & 1.00     & 442                           & 3.3e-4* \\
        R-Thalamus              & 281                     & 1.00     & 441                           & 3.9e-4* \\
        R-Caudate               & 485                     & 5.5e-8*  & 478                           & 3.4e-7* \\
        R-Putamen               & 294                     & 0.98     & 489                           & 1.7e-8* \\
        R-Pallidum              & 212                     & 1.00     & 469                           & 2.7e-6* \\
        R-Hippocampus           & 335                     & 0.73     & 493                           & 5.1e-9* \\
        R-Amygdala              & 316                     & 0.90     & 408                           & 1.9e-2* \\
        R-Accumbens             & 214                     & 1.00     & 418                           & 7.0e-3* \\
        \hline
    \end{tblr}
\end{table}

% Table 2: Cortical Regions
\begin{longtblr}[
        caption={Ansari-Bradley Test Results for Cortical Regions: ANCOVA vs Partial
                Correlation. * indicates FDR-corrected significance ($p < 0.05$). lh/rh =
                left/right hemisphere. ACC = anterior cingulate cortex, MF = middle
                frontal. W = statistic, p = p-value.},
        label={tab:stats-coef-var-cortical},
    ]{
        width=\linewidth,
        colspec = {l | *{4}{Q[c,m]} | *{4}{Q[c,m]} | *{4}{Q[c,m]}},
        row{odd} = {gray9},
        row{even} = {white},
        row{1} = {font=\bfseries, white},
        row{2} = {font=\bfseries, gray9},
        row{3} = {font=\bfseries, white},
        rows = {font=\footnotesize, rowsep=0pt},
        rowhead = 3,
    }
    \hline
    \SetCell[r=3]{m} Region   & \SetCell[c=4]{c} Cortical Volume &          &                               &          & \SetCell[c=4]{c} Cortical Thickness &          &                               &         & \SetCell[c=4]{c} Cortical Area &          &                               &          \\
    \hline
                              & \SetCell[c=2]{c} ANCOVA          &          & \SetCell[c=2]{c} Partial Corr &          & \SetCell[c=2]{c} ANCOVA             &          & \SetCell[c=2]{c} Partial Corr &         & \SetCell[c=2]{c} ANCOVA        &          & \SetCell[c=2]{c} Partial Corr &          \\
    \hline
                              & W                                & p        & W                             & p        & W                                   & p        & W                             & p       & W                              & p        & W                             & p        \\
    \hline
    bankssts (lh)             & 447                              & 1.6e-4*  & 486                           & 4.1e-8*  & 454                                 & 4.8e-5*  & 433                           & 1.2e-3* & 455                            & 4.1e-5*  & 481                           & 1.6e-7*  \\
    bankssts (rh)             & 403                              & 2.9e-2   & 466                           & 5.0e-6*  & 469                                 & 2.7e-6*  & 449                           & 1.1e-4* & 447                            & 1.6e-4*  & 461                           & 1.3e-5*  \\
    caudalACC (lh)            & 446                              & 1.8e-4*  & 487                           & 3.1e-8*  & 322                                 & 0.86     & 475                           & 6.9e-7* & 350                            & 0.52     & 480                           & 2.0e-7*  \\
    caudalACC (rh)            & 327                              & 0.81     & 513                           & 1.1e-12* & 501                                 & 3.1e-10* & 453                           & 5.8e-5* & 304                            & 0.96     & 500                           & 4.6e-10* \\
    caudalMF (lh)             & 437                              & 6.8e-4*  & 420                           & 5.6e-3*  & 479                                 & 2.6e-7*  & 413                           & 1.2e-2  & 362                            & 0.35     & 461                           & 1.3e-5*  \\
    caudalMF (rh)             & 439                              & 5.2e-4*  & 479                           & 2.6e-7*  & 352                                 & 0.49     & 445                           & 2.1e-4* & 390                            & 8.0e-2   & 489                           & 1.7e-8*  \\
    cuneus (lh)               & 441                              & 3.9e-4*  & 471                           & 1.7e-6*  & 507                                 & 2.6e-11* & 454                           & 4.8e-5* & 486                            & 4.1e-8*  & 483                           & 9.4e-8*  \\
    cuneus (rh)               & 387                              & 9.8e-2   & 481                           & 1.6e-7*  & 474                                 & 8.7e-7*  & 461                           & 1.3e-5* & 497                            & 1.4e-9*  & 487                           & 3.1e-8*  \\
    entorhinal (lh)           & 359                              & 0.39     & 409                           & 1.7e-2   & 297                                 & 0.98     & 396                           & 5.2e-2  & 474                            & 8.7e-7*  & 434                           & 1.0e-3*  \\
    entorhinal (rh)           & 372                              & 0.23     & 428                           & 2.2e-3*  & 246                                 & 1.00     & 396                           & 5.2e-2  & 505                            & 6.2e-11* & 455                           & 4.1e-5*  \\
    fusiform (lh)             & 477                              & 4.3e-7*  & 447                           & 1.6e-4*  & 421                                 & 5.0e-3*  & 419                           & 6.3e-3* & 421                            & 5.0e-3*  & 455                           & 4.1e-5*  \\
    fusiform (rh)             & 457                              & 2.8e-5*  & 460                           & 1.6e-5*  & 423                                 & 4.0e-3*  & 437                           & 6.8e-4* & 475                            & 6.9e-7*  & 454                           & 4.8e-5*  \\
    inferiorparietal (lh)     & 477                              & 4.3e-7*  & 503                           & 1.4e-10* & 347                                 & 0.56     & 458                           & 2.4e-5* & 487                            & 3.1e-8*  & 471                           & 1.7e-6*  \\
    inferiorparietal (rh)     & 369                              & 0.26     & 427                           & 2.5e-3*  & 460                                 & 1.6e-5*  & 383                           & 0.13    & 409                            & 1.7e-2   & 455                           & 4.1e-5*  \\
    inferiortemporal (lh)     & 437                              & 6.8e-4*  & 444                           & 2.5e-4*  & 356                                 & 0.44     & 461                           & 1.3e-5* & 472                            & 1.4e-6*  & 444                           & 2.5e-4*  \\
    inferiortemporal (rh)     & 335                              & 0.73     & 406                           & 2.3e-2   & 326                                 & 0.82     & 457                           & 2.8e-5* & 375                            & 0.20     & 456                           & 3.4e-5*  \\
    isthmuscingulate (lh)     & 432                              & 1.3e-3*  & 504                           & 9.5e-11* & 515                                 & 3.1e-13* & 462                           & 1.1e-5* & 413                            & 1.2e-2   & 505                           & 6.2e-11* \\
    isthmuscingulate (rh)     & 494                              & 3.7e-9*  & 469                           & 2.7e-6*  & 489                                 & 1.7e-8*  & 455                           & 4.1e-5* & 513                            & 1.1e-12* & 474                           & 8.7e-7*  \\
    lateraloccipital (lh)     & 494                              & 3.7e-9*  & 470                           & 2.1e-6*  & 430                                 & 1.7e-3*  & 440                           & 4.5e-4* & 475                            & 6.9e-7*  & 460                           & 1.6e-5*  \\
    lateraloccipital (rh)     & 472                              & 1.4e-6*  & 454                           & 4.8e-5*  & 498                                 & 9.5e-10* & 412                           & 1.3e-2  & 471                            & 1.7e-6*  & 461                           & 1.3e-5*  \\
    lateralorbitofrontal (lh) & 265                              & 1.00     & 414                           & 1.1e-2   & 184                                 & 1.00     & 464                           & 7.5e-6* & 294                            & 0.98     & 440                           & 4.5e-4*  \\
    lateralorbitofrontal (rh) & 271                              & 1.00     & 478                           & 3.4e-7*  & 422                                 & 4.5e-3*  & 429                           & 2.0e-3* & 258                            & 1.00     & 468                           & 3.3e-6*  \\
    lingual (lh)              & 250                              & 1.00     & 463                           & 9.1e-6*  & 494                                 & 3.7e-9*  & 468                           & 3.3e-6* & 453                            & 5.8e-5*  & 497                           & 1.4e-9*  \\
    lingual (rh)              & 435                              & 9.0e-4*  & 460                           & 1.6e-5*  & 480                                 & 2.0e-7*  & 467                           & 4.1e-6* & 477                            & 4.3e-7*  & 459                           & 2.0e-5*  \\
    medialorbitofrontal (lh)  & 459                              & 2.0e-5*  & 467                           & 4.1e-6*  & 479                                 & 2.6e-7*  & 410                           & 1.6e-2  & 396                            & 5.2e-2   & 469                           & 2.7e-6*  \\
    medialorbitofrontal (rh)  & 395                              & 5.6e-2   & 484                           & 7.2e-8*  & 300                                 & 0.97     & 430                           & 1.7e-3* & 405                            & 2.5e-2   & 459                           & 2.0e-5*  \\
    middletemporal (lh)       & 488                              & 2.3e-8*  & 487                           & 3.1e-8*  & 252                                 & 1.00     & 419                           & 6.3e-3* & 457                            & 2.8e-5*  & 464                           & 7.5e-6*  \\
    middletemporal (rh)       & 352                              & 0.49     & 463                           & 9.1e-6*  & 396                                 & 5.2e-2   & 428                           & 2.2e-3* & 427                            & 2.5e-3*  & 450                           & 9.5e-5*  \\
    parahippocampal (lh)      & 502                              & 2.1e-10* & 476                           & 5.5e-7*  & 462                                 & 1.1e-5*  & 437                           & 6.8e-4* & 425                            & 3.2e-3*  & 463                           & 9.1e-6*  \\
    parahippocampal (rh)      & 473                              & 1.1e-6*  & 468                           & 3.3e-6*  & 349                                 & 0.54     & 458                           & 2.4e-5* & 415                            & 9.6e-3*  & 459                           & 2.0e-5*  \\
    paracentral (lh)          & 314                              & 0.91     & 424                           & 3.6e-3*  & 204                                 & 1.00     & 425                           & 3.2e-3* & 491                            & 9.6e-9*  & 438                           & 6.0e-4*  \\
    paracentral (rh)          & 331                              & 0.77     & 484                           & 7.2e-8*  & 280                                 & 1.00     & 441                           & 3.9e-4* & 330                            & 0.78     & 485                           & 5.5e-8*  \\
    parsopercularis (lh)      & 362                              & 0.35     & 461                           & 1.3e-5*  & 494                                 & 3.7e-9*  & 418                           & 7.0e-3* & 459                            & 2.0e-5*  & 472                           & 1.4e-6*  \\
    parsopercularis (rh)      & 388                              & 9.1e-2   & 482                           & 1.2e-7*  & 305                                 & 0.96     & 477                           & 4.3e-7* & 447                            & 1.6e-4*  & 466                           & 5.0e-6*  \\
    parsorbitalis (lh)        & 293                              & 0.98     & 490                           & 1.3e-8*  & 230                                 & 1.00     & 462                           & 1.1e-5* & 419                            & 6.3e-3*  & 462                           & 1.1e-5*  \\
    parsorbitalis (rh)        & 295                              & 0.98     & 455                           & 4.1e-5*  & 336                                 & 0.71     & 427                           & 2.5e-3* & 269                            & 1.00     & 471                           & 1.7e-6*  \\
    parstriangularis (lh)     & 339                              & 0.68     & 507                           & 2.6e-11* & 401                                 & 3.5e-2   & 419                           & 6.3e-3* & 419                            & 6.3e-3*  & 504                           & 9.5e-11* \\
    parstriangularis (rh)     & 404                              & 2.7e-2   & 455                           & 4.1e-5*  & 288                                 & 0.99     & 394                           & 6.0e-2  & 362                            & 0.35     & 463                           & 9.1e-6*  \\
    pericalcarine (lh)        & 442                              & 3.3e-4*  & 465                           & 6.1e-6*  & 441                                 & 3.9e-4*  & 470                           & 2.1e-6* & 460                            & 1.6e-5*  & 509                           & 9.9e-12* \\
    pericalcarine (rh)        & 390                              & 8.0e-2   & 449                           & 1.1e-4*  & 480                                 & 2.0e-7*  & 451                           & 8.1e-5* & 443                            & 2.9e-4*  & 492                           & 7.0e-9*  \\
    postcentral (lh)          & 329                              & 0.79     & 460                           & 1.6e-5*  & 331                                 & 0.77     & 423                           & 4.0e-3* & 372                            & 0.23     & 477                           & 4.3e-7*  \\
    postcentral (rh)          & 350                              & 0.52     & 449                           & 1.1e-4*  & 269                                 & 1.00     & 426                           & 2.8e-3* & 514                            & 6.1e-13* & 387                           & 9.8e-2   \\
    posteriorcingulate (lh)   & 317                              & 0.90     & 488                           & 2.3e-8*  & 487                                 & 3.1e-8*  & 436                           & 7.9e-4* & 328                            & 0.80     & 465                           & 6.1e-6*  \\
    posteriorcingulate (rh)   & 315                              & 0.91     & 478                           & 3.4e-7*  & 480                                 & 2.0e-7*  & 450                           & 9.5e-5* & 356                            & 0.44     & 485                           & 5.5e-8*  \\
    precentral (lh)           & 284                              & 0.99     & 457                           & 2.8e-5*  & 239                                 & 1.00     & 372                           & 0.23    & 444                            & 2.5e-4*  & 424                           & 3.6e-3*  \\
    precentral (rh)           & 325                              & 0.83     & 420                           & 5.6e-3*  & 312                                 & 0.93     & 387                           & 9.8e-2  & 470                            & 2.1e-6*  & 488                           & 2.3e-8*  \\
    precuneus (lh)            & 240                              & 1.00     & 408                           & 1.9e-2   & 251                                 & 1.00     & 483                           & 9.4e-8* & 410                            & 1.6e-2   & 430                           & 1.7e-3*  \\
    precuneus (rh)            & 288                              & 0.99     & 414                           & 1.1e-2   & 244                                 & 1.00     & 465                           & 6.1e-6* & 363                            & 0.34     & 457                           & 2.8e-5*  \\
    rostralACC (lh)           & 269                              & 1.00     & 436                           & 7.9e-4*  & 505                                 & 6.2e-11* & 467                           & 4.1e-6* & 353                            & 0.48     & 460                           & 1.6e-5*  \\
    rostralACC (rh)           & 275                              & 1.00     & 501                           & 3.1e-10* & 481                                 & 1.6e-7*  & 448                           & 1.3e-4* & 275                            & 1.00     & 448                           & 1.3e-4*  \\
    rostralmiddlefrontal (lh) & 370                              & 0.25     & 482                           & 1.2e-7*  & 276                                 & 1.00     & 430                           & 1.7e-3* & 283                            & 0.99     & 489                           & 1.7e-8*  \\
    rostralmiddlefrontal (rh) & 318                              & 0.89     & 418                           & 7.0e-3*  & 319                                 & 0.88     & 431                           & 1.5e-3* & 219                            & 1.00     & 442                           & 3.3e-4*  \\
    superiorfrontal (lh)      & 370                              & 0.25     & 443                           & 2.9e-4*  & 456                                 & 3.4e-5*  & 464                           & 7.5e-6* & 481                            & 1.6e-7*  & 435                           & 9.0e-4*  \\
    superiorfrontal (rh)      & 392                              & 6.9e-2   & 432                           & 1.3e-3*  & 271                                 & 1.00     & 467                           & 4.1e-6* & 376                            & 0.19     & 480                           & 2.0e-7*  \\
    superiorparietal (lh)     & 369                              & 0.26     & 333                           & 0.75     & 369                                 & 0.26     & 440                           & 4.5e-4* & 489                            & 1.7e-8*  & 386                           & 0.10     \\
    superiorparietal (rh)     & 506                              & 4.0e-11* & 430                           & 1.7e-3*  & 437                                 & 6.8e-4*  & 463                           & 9.1e-6* & 456                            & 3.4e-5*  & 430                           & 1.7e-3*  \\
    superiortemporal (lh)     & 413                              & 1.2e-2   & 484                           & 7.2e-8*  & 427                                 & 2.5e-3*  & 441                           & 3.9e-4* & 446                            & 1.8e-4*  & 481                           & 1.6e-7*  \\
    superiortemporal (rh)     & 401                              & 3.5e-2   & 452                           & 6.8e-5*  & 364                                 & 0.32     & 469                           & 2.7e-6* & 425                            & 3.2e-3*  & 470                           & 2.1e-6*  \\
    supramarginal (lh)        & 466                              & 5.0e-6*  & 466                           & 5.0e-6*  & 274                                 & 1.00     & 488                           & 2.3e-8* & 509                            & 9.9e-12* & 481                           & 1.6e-7*  \\
    supramarginal (rh)        & 394                              & 6.0e-2   & 422                           & 4.5e-3*  & 483                                 & 9.4e-8*  & 412                           & 1.3e-2  & 376                            & 0.19     & 432                           & 1.3e-3*  \\
    frontalpole (lh)          & 385                              & 0.11     & 445                           & 2.1e-4*  & 319                                 & 0.88     & 453                           & 5.8e-5* & 347                            & 0.56     & 428                           & 2.2e-3*  \\
    frontalpole (rh)          & 284                              & 0.99     & 410                           & 1.6e-2   & 391                                 & 7.5e-2   & 451                           & 8.1e-5* & 218                            & 1.00     & 399                           & 4.1e-2   \\
    temporalpole (lh)         & 356                              & 0.44     & 379                           & 0.16     & 308                                 & 0.94     & 384                           & 0.12    & 385                            & 0.11     & 436                           & 7.9e-4*  \\
    temporalpole (rh)         & 252                              & 1.00     & 344                           & 0.61     & 421                                 & 5.0e-3*  & 403                           & 2.9e-2  & 225                            & 1.00     & 396                           & 5.2e-2   \\
    transversetemporal (lh)   & 406                              & 2.3e-2   & 467                           & 4.1e-6*  & 367                                 & 0.29     & 421                           & 5.0e-3* & 445                            & 2.1e-4*  & 482                           & 1.2e-7*  \\
    transversetemporal (rh)   & 324                              & 0.84     & 417                           & 7.8e-3*  & 477                                 & 4.3e-7*  & 442                           & 3.3e-4* & 347                            & 0.56     & 435                           & 9.0e-4*  \\
    insula (lh)               & 249                              & 1.00     & 470                           & 2.1e-6*  & 220                                 & 1.00     & 457                           & 2.8e-5* & 263                            & 1.00     & 438                           & 6.0e-4*  \\
    insula (rh)               & 267                              & 1.00     & 444                           & 2.5e-4*  & 401                                 & 3.5e-2   & 439                           & 5.2e-4* & 235                            & 1.00     & 410                           & 1.6e-2   \\
    \hline
\end{longtblr}

\begin{table}[t]
    \centering
    \caption{Bootstrap comparison of $\nu_{\text{nav}}$ between HC and PD.
        Values are observed differences (PD$-$HC). 95\% CIs from percentile bootstrap.}
    \label{tab:bootstrap-navr}
    \begin{tabular}{lcccc}
        \toprule
        \textbf{Metric}    & {\textbf{Observed} $\Delta$} & \textbf{95\% CI}    & {\textbf{p-value}} & \textbf{Study} \\
        \midrule
        area               & 0.005                        & {[}-0.002, 0.012{]} & 0.504              & baseline       \\
        thickness          & 0.026                        & {[}-0.019, 0.033{]} & 0.485              & baseline       \\
        volume             & 0.012                        & {[} 0.003, 0.021{]} & 0.499              & baseline       \\
        subcortical volume & -0.014                       & {[} 0.004, 0.025{]} & 0.490              & baseline       \\
        \midrule
        area               & 0.021                        & {[}-0.002, 0.042{]} & 0.496              & longitudinal   \\
        thickness          & 0.002                        & {[}-0.014, 0.017{]} & 0.798              & longitudinal   \\
        volume             & 0.025                        & {[} 0.002, 0.048{]} & 0.492              & longitudinal   \\
        subcortical volume & -0.029                       & {[}-0.050,-0.015{]} & 0.500              & longitudinal   \\
        \bottomrule
    \end{tabular}
\end{table}

\subsubsection{Consistency of statistical tests}

Figures \ref{fig:navr_consistency_area_plot} and
\ref{fig:consistency_volume_plot} show the consistency of statistical tests for
cortical area and volume, respectively, across all subjects and regions. The
plots show the percentage of subjects for which the statistical test was
significant ($\alpha = 0.05$) for each region. The consistency varies across
regions, with some regions showing higher consistency than others. The red
triangles indicate the IEEE-754 run for reference.

\begin{figure}[h]
    \centering
    \includegraphics[width=\linewidth]{figures/consistency/cortical_area_significance_correlation.pdf}
    \caption{Consistency of statistical tests for cortical area across all
        subjects and regions. The plot shows the percentage of subjects for
        which the statistical test was significant ($\alpha = 0.05$) for each
        region. The consistency varies across regions, with some regions showing
        higher consistency than others.}
    \label{fig:navr_consistency_area_plot}
\end{figure}

\begin{figure}[h]
    \centering
    \includegraphics[width=\linewidth]{figures/consistency/cortical_volume_significance_correlation.pdf}
    \caption{Consistency of statistical tests for cortical volume across all
        subjects and regions. The plot shows the percentage of subjects for
        which the statistical test was significant ($\alpha = 0.05$) for each
        region. The consistency varies across regions, with some regions showing
        higher consistency than others.}
    \label{fig:consistency_volume_plot}
\end{figure}

\subsubsection{Distribution of statistical tests coefficients}

Figures \ref{fig:consistency_area} and \ref{fig:consistency_volume} show the
distribution of partial correlation coefficients for cortical area and volume,
respectively, across all subjects and regions. The red triangles indicate the
IEEE-754 run for reference. The distribution shows the variability in the
coefficients, with some regions exhibiting higher consistency than others.

\begin{figure}
    \centering
    \begin{subfigure}[b]{\linewidth}
        \includegraphics[width=.9\linewidth]{figures/consistency/cortical_thickness_coefficients_distribution-Left.pdf}
        \caption{Left hemisphere}
        \label{fig:consistency_thickness_left}
    \end{subfigure}

    \begin{subfigure}[b]{\linewidth}
        \includegraphics[width=.9\linewidth]{figures/consistency/cortical_thickness_coefficients_distribution-Right.pdf}
        \caption{Right hemisphere}
        \label{fig:consistency_thickness_right}
    \end{subfigure}
    \caption{Distribution of partial correlation coefficients for cortical
        thickness across all subjects and regions. Red triangles indicate the
        IEEE-754 run for reference. }
    \label{fig:consistency_thickness_coefficients}
\end{figure}

\begin{figure}[h]
    \centering
    \begin{subfigure}[b]{\linewidth}
        \includegraphics[width=\linewidth]{figures/consistency/cortical_area_coefficients_distribution-Left.pdf}
        \caption{Left hemisphere}
        \label{fig:consistency_area_left}
    \end{subfigure}
    \hfill
    \begin{subfigure}[b]{\linewidth}
        \includegraphics[width=\linewidth]{figures/consistency/cortical_area_coefficients_distribution-Right.pdf}
        \caption{Right hemisphere}
        \label{fig:consistency_area_right}
    \end{subfigure}
    \caption{ Distribution of partial correlation coefficients for cortical surface area
        across all subjects and regions. Red triangles indicate the IEEE-754 run
        for reference. }
    \label{fig:consistency_area_coefficients}
\end{figure}

\begin{figure}[h]
    \centering
    \begin{subfigure}[b]{\linewidth}
        \includegraphics[width=\linewidth]{figures/consistency/cortical_volume_coefficients_distribution-Left.pdf}
        \caption{Left hemisphere}
        \label{fig:consistency_volume_left}
    \end{subfigure}
    \hfill
    \begin{subfigure}[b]{\linewidth}
        \includegraphics[width=\linewidth]{figures/consistency/cortical_volume_coefficients_distribution-Right.pdf}
        \caption{Right hemisphere}
        \label{fig:consistency_volume_right}
    \end{subfigure}
    \caption{ Distribution of partial correlation coefficients for cortical
        volume across all subjects and regions. Red triangles indicate the
        IEEE-754 run for reference. The distribution shows the variability in
        the coefficients, with some regions exhibiting higher consistency than
        others.}
    \label{fig:consistency_volume_coefficients}
\end{figure}

\subsubsection{Thresholding existing Cohen's d values from the literature}

We applied a thresholding approach to the Cohen's d values reported in the
literature to identify the most relevant findings for our analysis. This
involved setting a minimum effect size threshold, below which results were
considered non-significant or uninformative. The threshold was determined based
on the distribution of Cohen's d values across studies, with a focus on
retaining only those effects that were robust and consistent.

\begin{figure}[h]
    \centering
    \vspace{0.2cm}

    % Header row with column labels
    \begin{minipage}[b]{\linewidth}
        \begin{minipage}[c]{0.05\linewidth}
            % Empty space for alignment with condition labels
        \end{minipage}%
        \begin{minipage}[c]{0.95\linewidth}
            \begin{minipage}[c]{0.47\linewidth}
                \centering\textbf{Unthresholded}
            \end{minipage}%
            \hfill
            \begin{minipage}[c]{0.005\linewidth}
                % Vertical line separator
            \end{minipage}%
            \hfill
            \begin{minipage}[c]{0.47\linewidth}
                \centering\textbf{Thresholded}
            \end{minipage}
        \end{minipage}
    \end{minipage}

    % Horizontal line
    \noindent\rule{\linewidth}{0.5pt}
    \vspace{-1.5cm}

    % 22q11.2 deletion syndrome row
    \begin{minipage}[b]{\linewidth}
        \begin{minipage}[c]{0.05\linewidth}
            \centering\rotatebox{90}{\textbf{22q11.2}} \end{minipage}%
        \begin{minipage}[c]{0.95\linewidth}
            \begin{subfigure}[c]{0.47\linewidth}
                \includegraphics[width=\linewidth]{figures/cohen_d_map/enigma/22q_area_all.png}
                \label{fig:enigma_22q_unthresholded}
            \end{subfigure}%
            \hfill
            \begin{minipage}[c]{0.005\linewidth}
                \centering\rule{0.5pt}{4cm}
            \end{minipage}%
            \hfill
            \begin{subfigure}[c]{0.47\linewidth}
                \includegraphics[width=\linewidth]{figures/cohen_d_map/enigma/22q_area_all_thresholded.png}
                \label{fig:enigma_22q_thresholded}
            \end{subfigure}
        \end{minipage}
    \end{minipage}

    \vspace{-2cm}

    % ADHD row
    \begin{minipage}[b]{\linewidth}
        \begin{minipage}[c]{0.05\linewidth}
            \centering\rotatebox{90}{\textbf{ADHD}} \end{minipage}%
        \begin{minipage}[c]{0.95\linewidth}
            \begin{subfigure}[c]{0.47\linewidth}
                \includegraphics[width=\linewidth]{figures/cohen_d_map/enigma/adhd_area_adult.png}
                \label{fig:enigma_adhd_unthresholded}
            \end{subfigure}%
            \hfill
            \begin{minipage}[c]{0.005\linewidth}
                \centering\rule{0.5pt}{4cm}
            \end{minipage}%
            \hfill
            \begin{subfigure}[c]{0.47\linewidth}
                \includegraphics[width=\linewidth]{figures/cohen_d_map/enigma/adhd_area_adult_thresholded.png}
                \label{fig:enigma_adhd_thresholded}
            \end{subfigure}
        \end{minipage}
    \end{minipage}

    \vspace{-2cm}

    % bipolar disorder row
    \begin{minipage}[b]{\linewidth}
        \begin{minipage}[c]{0.05\linewidth}
            \centering\rotatebox{90}{\textbf{Bipolar}} \end{minipage}%
        \begin{minipage}[c]{0.95\linewidth}
            \begin{subfigure}[c]{0.47\linewidth}
                \includegraphics[width=\linewidth]{figures/cohen_d_map/enigma/bipolar_area_adult.png}
                \label{fig:enigma_bipolar_unthresholded}
            \end{subfigure}%
            \hfill
            \begin{minipage}[c]{0.005\linewidth}
                \centering\rule{0.5pt}{4cm}
            \end{minipage}%
            \hfill
            \begin{subfigure}[c]{0.47\linewidth}
                \includegraphics[width=\linewidth]{figures/cohen_d_map/enigma/bipolar_area_adult_thresholded.png}
                \label{fig:enigma_bipolar_thresholded}
            \end{subfigure}
        \end{minipage}
    \end{minipage}

    \vspace{-2cm}
    % depression
    \begin{minipage}[b]{\linewidth}
        \begin{minipage}[c]{0.05\linewidth}
            \centering\rotatebox{90}{\textbf{Depression}} \end{minipage}%
        \begin{minipage}[c]{0.95\linewidth}
            \begin{subfigure}[c]{0.47\linewidth}
                \includegraphics[width=\linewidth]{figures/cohen_d_map/enigma/depression_area_adult.png}
                \label{fig:enigma_depression_unthresholded}
            \end{subfigure}%
            \hfill
            \begin{minipage}[c]{0.005\linewidth}
                \centering\rule{0.5pt}{4cm}
            \end{minipage}%
            \hfill
            \begin{subfigure}[c]{0.47\linewidth}
                \includegraphics[width=\linewidth]{figures/cohen_d_map/enigma/depression_area_adult_thresholded.png}
                \label{fig:enigma_depression_thresholded}
            \end{subfigure}
        \end{minipage}
    \end{minipage}

    \vspace{-2cm}
    % ocd 
    \begin{minipage}[b]{\linewidth}
        \begin{minipage}[c]{0.05\linewidth}
            \centering\rotatebox{90}{\textbf{OCD}} \end{minipage}%
        \begin{minipage}[c]{0.95\linewidth}
            \begin{subfigure}[c]{0.47\linewidth}
                \includegraphics[width=\linewidth]{figures/cohen_d_map/enigma/ocd_area_adult.png}
                \label{fig:enigma_ocd_unthresholded}
            \end{subfigure}%
            \hfill
            \begin{minipage}[c]{0.005\linewidth}
                \centering\rule{0.5pt}{4cm}
            \end{minipage}%
            \hfill
            \begin{subfigure}[c]{0.47\linewidth}
                \includegraphics[width=\linewidth]{figures/cohen_d_map/enigma/ocd_area_adult_thresholded.png}
                \label{fig:enigma_ocd_thresholded}
            \end{subfigure}
        \end{minipage}
    \end{minipage}

    \vspace{-2cm}
    % schizophrenia
    \begin{minipage}[b]{\linewidth}
        \begin{minipage}[c]{0.05\linewidth}
            \centering\rotatebox{90}{\textbf{Schizophrenia}} \end{minipage}%
        \begin{minipage}[c]{0.95\linewidth}
            \begin{subfigure}[c]{0.47\linewidth}
                \includegraphics[width=\linewidth]{figures/cohen_d_map/enigma/schizophrenia_area_all.png}
                \label{fig:enigma_schizophrenia_unthresholded}
            \end{subfigure}%
            \hfill
            \begin{minipage}[c]{0.005\linewidth}
                \centering\rule{0.5pt}{4cm}
            \end{minipage}%
            \hfill
            \begin{subfigure}[c]{0.47\linewidth}
                \includegraphics[width=\linewidth]{figures/cohen_d_map/enigma/schizophrenia_area_all_thresholded.png}
                \label{fig:enigma_schizophrenia_thresholded}
            \end{subfigure}
        \end{minipage}
    \end{minipage}

    \caption{ENIGMA cortical area Cohen's d maps showing unthresholded effect
        sizes (left) and effect sizes thresholded by the \navr framework (right)
        for different disorders. Black regions indicate areas where Cohen's d
        values fall below the numerical variability threshold, demonstrating
        regions where reported effect sizes may be unreliable due to
        computational uncertainty.}
    \label{fig:navr_enigma_area}
\end{figure}

\end{document}
