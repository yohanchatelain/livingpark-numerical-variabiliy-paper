\documentclass{article}

% Language setting Replace `english' with e.g. `spanish' to change the document
% language
\usepackage[english]{babel}

% Set page size and margins Replace `letterpaper' with `a4paper' for UK/EU
% standard size
\usepackage[letterpaper,top=2cm,bottom=2cm,left=3cm,right=3cm,marginparwidth=1.75cm]{geometry}

% Useful packages
\usepackage{amsmath}
\usepackage{graphicx}
\usepackage[colorlinks=true, allcolors=blue]{hyperref}
\usepackage{todonotes}
\usepackage{longtable}
\usepackage{booktabs}
\usepackage{multirow}
\usepackage{subcaption}
\usepackage{xcolor}
\usepackage{tabularray}
\usepackage{booktabs} % For \toprule, \midrule, \bottomrule

\newcommand{\0}{\mspace{9mu}}

\title{Numerical Variability in FreeSurfer: Implications for Parkinson's
    Disease Research}

\author{Yohan Chatelain, Andrzej Sokołowski, Jean-Baptiste Poline, Madeleine Sharp, Tristan Glatard}

\begin{document}

\maketitle

\begin{abstract}
    Neuroimaging studies, particularly those involving structural MRI analysis,
    have faced challenges in reproducibility due to software variability. This
    study investigates the impact of numerical instability in FreeSurfer 7.3.1
    on structural MRI measures and their relationships with clinical outcomes in
    Parkinson's disease (PD). Utilizing Monte Carlo Arithmetic (MCA), we
    evaluate the extent of numerical variability in FreeSurfer's estimation of
    brain structure metrics, including cortical thickness, surface area,
    cortical and subcortical volumes. Our findings suggest that numerical
    variability affects the precision of these metrics, with implications for
    both cross-sectional and longitudinal analyses. Specifically, the
    variability influences the detection of group differences between PD
    patients and healthy controls, as well as correlations between MRI-derived
    measures and disease severity. These results highlight the need for greater
    caution in interpreting structural MRI metrics in clinical research and
    underscore the importance of addressing numerical variability to improve the
    reliability of neuroimaging outcomes in PD studies.
\end{abstract}

\section{Introduction}

Neuroimaging reproducibility has emerged as a critical challenge in neuroscience research. While inter-software variability is well-documented~\cite{botvinik2020variability,gronenschild2012effects,bhagwat2021understanding}, within-version numerical variability—small output variations from identical software runs—remains underexplored despite potentially significant clinical implications.

Numerical variability arises from computational factors including floating-point precision, parallel processing, and random initializations. In Parkinson's disease (PD) research, where MRI-derived metrics like cortical thickness and subcortical volumes serve as potential biomarkers, such variability could obscure subtle disease-related changes and compromise statistical reliability.

Previous studies have demonstrated substantial between-version differences in FreeSurfer outputs~\cite{haddad2023multisite}, but the impact of computational uncertainty within single software versions on clinical associations remains unclear. This gap is particularly concerning for PD research, where establishing reliable brain-behavior relationships is essential for developing neuroimaging biomarkers.

Here, we investigate numerical variability in FreeSurfer 7.3.1 using Monte Carlo Arithmetic to simulate realistic computational perturbations. We introduce the Numerical-Anatomical Variability Ratio (NAVR) to quantify computational uncertainty relative to biological variation and derive its theoretical relationship to statistical effect sizes. Using longitudinal data from the Parkinson's Progression Markers Initiative, we assess how numerical precision affects group comparisons and clinical correlations in PD research.

\subsection{Motivation}

Despite promising associations between MRI-derived metrics and PD severity, no neuroimaging biomarkers are widely accepted for clinical diagnosis or monitoring. Measurement variability across studies undermines reliability and generalizability, hindering translation to clinical practice.

While between-software and between-version variability have been extensively studied, within-version numerical variability remains underexplored. This computational uncertainty could significantly impact PD research by: (1) masking subtle disease-related changes essential for early detection, (2) compromising statistical power for detecting group differences and clinical correlations, and (3) reducing reproducibility across studies using identical analysis pipelines.

Understanding numerical variability is crucial for advancing MRI as a diagnostic tool in PD, where reliable quantification of brain-behavior relationships is essential for developing clinically meaningful biomarkers.

\subsection{Objectives}

This study investigates numerical variability impacts in FreeSurfer 7.3.1 on PD neuroimaging outcomes. Specific aims include: (1) quantifying how computational uncertainty affects group difference detection between PD patients and healthy controls, (2) assessing numerical variability effects on brain-behavior correlations with clinical measures (UPDRS scores), and (3) developing the NAVR framework to predict statistical reliability from computational precision.

Our findings will inform strategies for mitigating numerical variability effects and enhancing reproducibility in clinical neuroimaging studies.

\section{Results}

PD and HC groups showed no significant age differences ($p > 0.05$) but differed in education ($t = -2.05$, $p = 0.04$) and sex distribution ($\chi^2 = 4.15$, $p = 0.04$). The longitudinal cohort showed no significant demographic differences between groups (Table~\ref{tab:cohort_stat}).

\subsection{Statistical consistency varies with numerical precision}

Statistical significance proportions across 26 MCA repetitions varied substantially for subcortical volumes (Figure~\ref{fig:significance_correlation}). Ratios near 0.5 indicated maximal uncertainty, while values approaching 0 or 1 suggested consistent results across computational variations.

Effect size distributions showed notable variability across MCA repetitions (Figure~\ref{fig:statstest_coefficients_distribution}). Partial correlation coefficients and F-statistics from ANCOVA analyses demonstrated spread around standard IEEE-754 results (red markers), indicating that numerical precision affects both statistical significance and effect size estimation. 

\begin{figure}
\includegraphics[width=\linewidth]{figures/consistency/subcortical_volume_significance_correlation.pdf}
\caption{Proportion of statistically significant tests ($p < 0.05$) across the 26 MCA repetitions for subcortical volume measures.\label{fig:significance_correlation}}
\end{figure}

\begin{figure}
\includegraphics[width=\linewidth]{figures/consistency/subcortical_volume_coefficients_distribution.pdf}
\caption{Distribution of partial correlation coefficients (r-values) and F-statistics from ANCOVA across MCA repetitions for subcortical volume measures. Red dots represent the IEEE results. The top row shows r-values, while the bottom row shows F-values. The left column represents baseline analysis, and the right column represents longitudinal analysis.\label{fig:statstest_coefficients_distribution}}
\end{figure}

\subsection{NAVR reveals region-specific numerical instabilities}

NAVR values varied substantially across brain regions, with some showing computational uncertainty comparable to or exceeding biological variation (Figure~\ref{fig:navr_subcortical}). Regions with high NAVR values indicate areas where numerical precision limitations may compromise the detection of true anatomical differences.

\begin{figure}
    \includegraphics[width=\linewidth]{figures/NAVR_map/NAVR_subcortical_volume_all.png}
        \caption{Numerical-Anatomical Variability Ratio (NAVR) for subcortical volumes across regions and groups. Higher NAVR values indicate greater computational uncertainty relative to biological variation.\label{fig:navr_subcortical}}
\end{figure}

\subsection{Cross-sectional numerical precision analysis}

FreeSurfer 7.3.1 showed limited numerical precision across all cortical measures: $1.49 \pm 0.27$ significant digits for thickness, $1.03 \pm 0.27$ for surface area, and $1.00 \pm 0.28$ for volume (Figures~\ref{fig:sig_digits_cortical}, \ref{fig:sig_digits_subcortical}). These values indicate measurements are typically precise to only one decimal place, with some instances showing complete precision loss.

Regional consistency was observed within each metric type, with thickness showing the highest precision (range: 1.05-1.75 digits) compared to area (0.58-1.35 digits) and volume (0.52-1.36 digits). Dice coefficients revealed substantial inter-subject variability, particularly in vessel regions (Figure~\ref{fig:dice}).
 
\begin{figure}
    \includegraphics*[width=\linewidth]{figures/dice.pdf}
    \caption{Dice coefficient.\label{fig:dice}}

\end{figure}

\begin{figure}
    \includegraphics*[width=\linewidth]{figures/sig_digits.pdf}
    \caption{Number of significant digits for each cortical region and metric.\label{fig:sig_digits_cortical}}
\end{figure}

\begin{figure}
    \includegraphics*[width=\linewidth]{figures/sig_digits_subcortical_volume.pdf}
    \caption{Number of significant digits of subcortical volume for each ROI.\label{fig:sig_digits_subcortical}}
\end{figure}

% WhiteSurfArea & 2.0 +/- 0.34 & 1.97 +/- 0.31

\begin{itemize}
    \item Presentation of the basic numerical summary of the collected data for both HC
          and PD groups.
\end{itemize}

\subsection{Longitudinal Analysis}

\subsubsection{Subcortical volume analysis}

\begin{enumerate}
    \item The partial correlation between the UPDRS-III scores and the subcortical volume
          at baseline.
    \item The partial correlation between the rate of change in UPDRS-III scores and the
          rate of change in subcortical volume longitudinally.
    \item The group differences in the baseline subcortical volume between PD patients
          and healthy controls.
    \item The group differences in the rate of change in subcortical volume
          longitudinally between PD patients and healthy controls.
\end{enumerate}



\subsubsection{Vertex-wise analysis}

\begin{enumerate}
    \item The correlation between the UPDRS-III scores and the cortical thickness at
          baseline.
    \item The correlation between the rate of change in UPDRS-III scores and the rate of
          change in cortical thickness longitudinally.
    \item The group differences in the baseline cortical thickness between PD patients
          and healthy controls.
    \item The group differences in the rate of change in cortical thickness
          longitudinally between PD patients and healthy controls.
\end{enumerate}

To assess statistical significance, we employed permutation tests. Permutation
test provides a non-parametric test to evaluate whether a given measure
significantly deviates from the population. Variability introduced by MCA
reveals the numerical instability of FreeSurfer 7.3.1 shows that the
significance of cortical regions fluctuated across repetitions.

Specifically, only two regions were found to be statistically significant in
two repetitions out of 26, suggesting a lack of statistical power. The surface
sizes of significant regions varied considerably, with examples such as
$1005.95 \pm 39.34 mm^2$ and $996.83 \pm 42.91 mm^2$.

\paragraph{Group Analysis}

No clusters were found to be statistically significant using the permutation
test (1000 permutations).

\paragraph{Correlation Analysis}

The following regions were identified using a permutation test (1000
permutations) for the correlation analysis. Each region was significant in only
one MCA repetition, further demonstrating a lack of replicability.

\begin{table}[h]
    \centering
    \caption{Significant regions identified using a permutation test (1000 permutations) for the correlation analysis.}
    \begin{tabular}{lllllll}
        \toprule
        \textbf{Region}     & \textbf{Size ($mm^2$)} & \textbf{MNI X} & \textbf{MNI Y} & \textbf{MNI Z} & \textbf{Max} & \textbf{Frequency} \\
        \midrule
        L inferior temporal & 3337.13                & -56.6          & -43.6          & -18.1          & 2.8572       & 1/25               \\
        R lingual           & 1150.84                & 23.2           & -61.3          & $\00.4$        & 4.6403       & 1/25               \\
        R parstriangularis  & 3265.87                & 44.4           & 35.8           & $\03.7$        & 2.8129       & 1/25               \\
        \bottomrule
    \end{tabular}
\end{table}

\section{Discussion}

Our analysis reveals significant numerical instability in FreeSurfer 7.3.1, with cortical measurements showing limited precision (1-1.5 significant digits) that substantially impacts statistical reliability in neuroimaging studies. These precision limitations pose particular challenges for detecting subtle disease-related changes in conditions like Parkinson's disease.

The absence of significant baseline differences between PD and HC groups, combined with inconsistent cluster detection (only 1/26 clusters reproduced across repetitions), demonstrates how numerical variability can compromise reproducibility. The NAVR framework quantifies this relationship, showing that computational uncertainty approaches or exceeds biological variation in many brain regions.

Statistical test consistency varied markedly across MCA repetitions, with methodological choices (Z-test vs. permutation test) further influencing outcome reliability. Effect size distributions showed substantial spread around standard IEEE-754 results, indicating that numerical precision directly affects both significance testing and effect size estimation.

Importantly, inter-subject variability exceeded intra-subject variability, suggesting that FreeSurfer maintains relative consistency across different individuals despite numerical limitations. This supports continued use while highlighting the need for improved computational precision in future neuroimaging software development.


\section{Methods}

\subsection{Numerical variability assessment}

We employed Monte Carlo Arithmetic (MCA)~\cite{parker1997monte} to quantify numerical instability in FreeSurfer computations. MCA introduces controlled random perturbations into floating-point operations, simulating rounding errors that occur across different computational environments. This stochastic approach enables systematic assessment of result stability by measuring variation across multiple runs of identical analyses.

We used Fuzzy-libm~\cite{salari2021accurate}, which extends MCA to mathematical library functions (\texttt{exp}, \texttt{log}, \texttt{sin}, \texttt{cos}) through Verificarlo~\cite{denis2016verificarlo}, an LLVM-based compiler. Virtual precision parameters were set to 53 bits for double precision and 24 bits for single precision to simulate realistic machine-level precision errors.

\subsection{Participants}

We analyzed data from the Parkinson's Progression Markers Initiative (PPMI), a multi-site longitudinal study. From 316 initial participants, we selected 125 Parkinson's disease patients without mild cognitive impairment (PD-non-MCI) and 106 healthy controls (HC) with complete longitudinal T1-weighted MRI data. PD-MCI patients were excluded to avoid confounding effects of cognitive impairment.

Inclusion criteria required: (1) primary PD diagnosis or healthy control status, (2) availability of two visits with T1-weighted scans, and (3) absence of other neurological diagnoses. PD severity was assessed using the Unified Parkinson's Disease Rating Scale (UPDRS). The study received ethics approval from participating institutions, and all participants provided written informed consent (Table~\ref{tab:cohort_stat}).

\begin{table}[h!]
    \centering
    \begin{tabular}{lcc}
        \toprule
        \textbf{Cohort}         & \textbf{HC}        & \textbf{PD-non-MCI} \\
        \hline
        n                       & $103 $             & $121 $              \\
        Age (y)                 & $60.7 \pm 10.3 $   & $60.7 \pm \09.1 $   \\
        Age
        range                   & $30.6 - 84.3 $     & $39.2 - 78.3 $      \\
        Gender (male, \%)       & $57 \; (55.3\%)
        $                       & $80 \; (66.1\%) $                        \\
        Education (y)           & $16.6 \pm \03.3 $  & $16.1 \pm \03.0 $   \\
        UPDRS III OFF baseline  & $- $               & $23.4 \pm 10.1 $    \\
        UPDRS III OFF follow-up & $- $               & $25.8 \pm 11.1 $    \\
        Duration T2 - T1 (y)    & $\01.4 \pm \00.5 $ & $\01.4 \pm \00.7 $  \\ \bottomrule
    \end{tabular}
    \vspace{1em}

    \caption{\textbf{Abbreviations:} MCI = Mild Cognitive Impairment; UPDRS = Unified Parkinson's Disease Rating Scale; PD = Parkinson's disease. Values are expressed as mean $\pm$ standard deviation. PD-non-MCI longitudinal sample is a subsample of the PD-non-MCI original sample that had longitudinal data and disease severity scores available.
        \label{tab:cohort_stat}}
\end{table}

\subsection{Image acquisition and preprocessing}

T1-weighted MRI images were acquired using standardized PPMI parameters: TR/TE/TI = 2.3/2.98/0.9 s, 1mm isotropic resolution, 192 slices, 256×256 matrix. Minor site-specific variations existed due to the multi-site design.

We processed images using FreeSurfer 7.3.1 instrumented with Fuzzy-libm to introduce controlled numerical perturbations. Each participant underwent 34 \texttt{recon-all} executions, extracting cortical thickness, surface area, and volumes. After quality control and exclusion of failed runs, we randomly selected 25 successful repetitions per subject to ensure balanced datasets for statistical analysis.

Longitudinal processing followed the standard FreeSurfer stream~\cite{reuter2012within}: cross-sectional processing of both timepoints, followed by creation of an unbiased within-subject template~\cite{reuter2011avoiding} using robust registration~\cite{reuter2010highly}. Downstream analyses used unperturbed FreeSurfer to prevent additional numerical perturbations.

\subsection{Numerical Variability Assessment}

We assessed FreeSurfer 7.3.1 numerical stability in cross-sectional and longitudinal contexts using the Numerical-Anatomical Variability Ratio (NAVR) and its relationship to statistical effect sizes.

\subsubsection{Numerical-Anatomical Variability Ratio (NAVR)}

To quantify computational stability relative to biological variation, we developed the Numerical-Anatomical Variability Ratio (NAVR). For each brain region, NAVR measures the ratio of measurement uncertainty arising from computational processes to natural inter-subject anatomical variation:

\[
\text{NAVR} = \frac{\sigma_{\text{num}}}{\sigma_{\text{anat}}}
\]

where $\sigma_{\text{num}}$ represents numerical variability (measurement precision across MCA repetitions for individual subjects) and $\sigma_{\text{anat}}$ represents anatomical variability (inter-subject differences within each repetition).

For each region of interest, measurements from $n$ MCA repetitions across $m$ subject-visit pairs form a data matrix $\mathcal{M}_{n \times m}$, where element $x_{i,j}$ represents the measurement for subject $j$ in repetition $i$.

Numerical variability quantifies intra-subject measurement consistency:
\begin{equation}
\sigma^2_{\text{num}} = \frac{1}{m} \sum_{j=1}^{m} \left[ \frac{1}{n-1} \sum_{i=1}^{n} (x_{i,j} - \bar{x}_{\cdot,j})^2 \right]
\end{equation}

Anatomical variability captures inter-subject differences:
\begin{equation}
\sigma^2_{\text{anat}} = \frac{1}{n} \sum_{i=1}^{n} \left[ \frac{1}{m-1} \sum_{j=1}^{m} (x_{i,j} - \bar{x}_{i,\cdot})^2 \right]
\end{equation}

where $\bar{x}_{\cdot,j}$ and $\bar{x}_{i,\cdot}$ denote column and row means, respectively. Higher NAVR values indicate regions where computational uncertainty approaches or exceeds biological variation, potentially compromising the detection of true anatomical differences.

\subsubsection{Relationship between NAVR and Effect Size Uncertainty}

We derived the theoretical relationship between NAVR and Cohen's d variability to quantify how measurement uncertainty affects statistical effect sizes in group comparisons.

For a balanced two-group design with total sample size $N$, each observation decomposes as $X_{ij} = \mu_i + \varepsilon_{ij}^{(\text{anat})} + \varepsilon_{ij}^{(\text{num})}$, where $\mu_i$ represents the true group mean, $\varepsilon_{ij}^{(\text{anat})} \sim \mathcal{N}(0, \sigma_{\text{anat}}^2)$ captures anatomical variation, and $\varepsilon_{ij}^{(\text{num})} \sim \mathcal{N}(0, \sigma_{\text{num}}^2)$ represents numerical uncertainty.

The standard deviation of Cohen's d attributable to measurement error is:
\begin{equation}
\sigma_d = \frac{2}{\sqrt{N}} \cdot \text{NAVR}
\end{equation}

This relationship emerges from error propagation analysis. The difference in group means has variance $\text{Var}(\bar{X}_1 - \bar{X}_2) = 4(\sigma_{\text{anat}}^2 + \sigma_{\text{num}}^2)/N$, with the numerical component contributing $4\sigma_{\text{num}}^2/N$. Since Cohen's d normalizes by the pooled standard deviation $\sqrt{\sigma_{\text{anat}}^2 + \sigma_{\text{num}}^2}$, the measurement error contribution becomes $\sigma_d = (2\sigma_{\text{num}}/\sqrt{N})/\sigma_{\text{anat}} = (2/\sqrt{N}) \cdot \text{NAVR}$.

This formula indicates that regions with NAVR = 0.1 contribute approximately $0.2/\sqrt{N}$ uncertainty to Cohen's d, while regions with NAVR = 1.0 contribute $2/\sqrt{N}$ uncertainty. The relationship provides a direct link between computational stability (NAVR) and statistical reliability in neuroimaging studies.



\subsubsection{Cross-sectional Analysis}

We extracted cortical and subcortical volumes, thickness, and surface areas from 26 MCA repetitions using the Destrieux 2009 atlas. Numerical precision was quantified using significant digits analysis, while measurement variability was assessed through standard deviations across repetitions.

Group comparisons between HC and PD used two-sample t-tests. Structural overlap was evaluated using the extended Sørensen-Dice coefficient:
\[
    \text{Dice}(A_1, A_2, \dots, A_n) = \frac{n \left| \bigcap_{i=1}^{n} A_i \right|}{\sum_{i=1}^{n} \left| A_i \right|}
\]

\subsubsection{Longitudinal Analysis}

Longitudinal analyses examined correlations between brain metrics and UPDRS scores, plus group differences between PD and HC across two timepoints. 

For subcortical regions, we assessed baseline volume-UPDRS correlations and group differences using partial correlation and ANCOVA. Longitudinal changes were analyzed using volume change rates: $(\text{volume}_2 - \text{volume}_1) / \text{volume}_1 \times 100$.

Vertex-wise cortical analyses examined thickness-UPDRS correlations and group differences at baseline and longitudinally. Thickness change rates were calculated as $(\text{thickness}_2 - \text{thickness}_1) / (\text{time}_2 - \text{time}_1)$ (mm/year). All analyses included age and sex covariates, with additional time-between-visits adjustment for longitudinal models. Cluster-wise permutation testing used $p < 0.05$ threshold, reporting cluster frequency across 26 MCA repetitions.


\section{Conclusions}

This study demonstrates significant numerical limitations in FreeSurfer 7.3.1, with cortical measurements exhibiting only 1-1.5 significant digits of precision. These computational constraints substantially impact statistical reliability and reproducibility in neuroimaging research, particularly for detecting subtle disease-related changes.

Our NAVR framework quantifies the relationship between computational uncertainty and biological variation, revealing that numerical instability approaches or exceeds anatomical variability in many brain regions. This finding has direct implications for statistical power, as demonstrated by inconsistent cluster detection (only 1/26 clusters reproduced) and variable effect sizes across identical analyses.

While inter-subject variability exceeded intra-subject variability—supporting relative consistency across individuals—the absence of significant PD-HC differences and weak clinical correlations highlight how numerical limitations can mask true biological signals. The theoretical relationship between NAVR and Cohen's d uncertainty provides a framework for predicting statistical reliability based on computational precision.

These findings emphasize the critical need for improved numerical precision in neuroimaging software. Future developments should prioritize computational stability to enhance the detection of subtle neurological changes and improve reproducibility across studies. The NAVR framework offers a practical tool for assessing and comparing the numerical reliability of neuroimaging methodologies.

\section{Acknowledgements}

The analyses were conducted on the Virtual Imaging
Platform~\cite{glatard2012virtual}, which utilizes resources provided by the
Biomed virtual organization within the European Grid Infrastructure (EGI). We
extend our gratitude to Sorina Pop from CREATIS, Lyon, France, for her support.

\bibliographystyle{alpha}
\bibliography{main}

\clearpage

\appendix

\section{Formula}

\subsection{Significant digits formula}

We compute the number of significant bits \(\hat{s}\) with probability
\(p_s=0.95\) and confidence \(1-\alpha_s=0.95\) using the \emph{Significant
    Digits}
package\footnote{\url{https://github.com/verificarlo/significantdigits}}
(version 0.2.0). \emph{Significant Digits} implements the Centered Normality
Hypothesis approach described in~\cite{sohier2021confidence}:
\[
    \hat{s_i} = -\log_2 \left| \frac{\hat{\sigma_i}}{\hat{\mu_i}} \right| -
    \delta(n, \alpha_s, p_s),
\]
where \(\hat{\sigma_i}\) and \(\hat{\mu_i}\) are the average and standard
deviation over the repetitions, and
\begin{equation}
    \delta(n, \alpha_s, p_s) = \log_2 \left(
    \sqrt{\frac{n-1}{\chi^2_{1-\alpha_s/2}}} \Phi^{-1} \left( \frac{p_s+1}{2}
    \right) \right)
\end{equation}
is a penalty term for estimating \(\hat{s_i}\) with probability \(p_s\) and
confidence level \(1-\alpha_s\) for a sample size \(n\). \(\Phi^{-1}\) is the
inverse cumulative distribution of the standard normal distribution and
\(\chi^2\) is the Chi-2 distribution with \(n\)-1 degrees of freedom.

\section{Cross-sectional Analysis}

As a side result, the cross-sectional analysis measures the impact of numerical
variability in FreeSurfer version 7.3.1 on the PPMI (Parkinson's Progression
Markers Initiative) cohort. This will involve comparing the estimation of
structural MRI measures, including cortical and subcortical volumes, cortical
thickness, and surface area. The goal is to assess the stability of these key
metrics and determine how numerical variability may affect their reliability in
clinical research.

\subsection{Significant digits average across all subjects and regions}

\begin{longtblr}[
        caption={Significant digits average across all subjects and regions.},
        label={tab:sig-cortical},
    ]{
        colspec={lcc|cc|cc}, width=0.25\linewidth,
        row{even}={white,font=\footnotesize},
        row{odd}={gray9,font=\footnotesize},
        rows = {rowsep=0pt},
        rowhead=2,
        row{1}={white,font=\bfseries},
        row{2}={gray9}
    }
    \SetCell[c=1]{c}Region    & \SetCell[c=2]{c}{cortical thickness } &                 & \SetCell[c=2]{c}{surface area} &                 & \SetCell[c=2]{c}{cortical volume} &                 \\
                              & lh                                    & rh              & lh                             & rh              & lh                                & rh              \\
    \hline
    G Ins lg and S cent ins   & $1.16 \pm 0.19$                       & $1.14 \pm 0.19$ & $0.85 \pm 0.12$                & $0.72 \pm 0.17$ & $0.88 \pm 0.13$                   & $0.84 \pm 0.14$ \\
    G and S cingul-Ant        & $1.58 \pm 0.19$                       & $1.61 \pm 0.18$ & $1.13 \pm 0.20$                & $1.15 \pm 0.21$ & $1.11 \pm 0.21$                   & $1.12 \pm 0.19$ \\
    G and S cingul-Mid-Ant    & $1.57 \pm 0.17$                       & $1.59 \pm 0.17$ & $1.05 \pm 0.20$                & $1.08 \pm 0.21$ & $1.02 \pm 0.19$                   & $1.05 \pm 0.20$ \\
    G and S cingul-Mid-Post   & $1.65 \pm 0.17$                       & $1.62 \pm 0.18$ & $1.23 \pm 0.20$                & $1.18 \pm 0.23$ & $1.20 \pm 0.21$                   & $1.14 \pm 0.24$ \\
    G and S frontomargin      & $1.42 \pm 0.21$                       & $1.30 \pm 0.20$ & $1.07 \pm 0.21$                & $0.92 \pm 0.16$ & $1.00 \pm 0.19$                   & $0.86 \pm 0.16$ \\
    G and S occipital inf     & $1.50 \pm 0.18$                       & $1.50 \pm 0.17$ & $1.07 \pm 0.16$                & $1.05 \pm 0.15$ & $1.03 \pm 0.16$                   & $1.04 \pm 0.17$ \\
    G and S paracentral       & $1.45 \pm 0.23$                       & $1.45 \pm 0.24$ & $1.12 \pm 0.15$                & $1.16 \pm 0.19$ & $1.05 \pm 0.17$                   & $1.09 \pm 0.21$ \\
    G and S subcentral        & $1.62 \pm 0.18$                       & $1.62 \pm 0.18$ & $1.11 \pm 0.14$                & $1.13 \pm 0.19$ & $1.10 \pm 0.15$                   & $1.14 \pm 0.18$ \\
    G and S transv frontopol  & $1.30 \pm 0.25$                       & $1.34 \pm 0.22$ & $0.91 \pm 0.16$                & $0.91 \pm 0.15$ & $0.87 \pm 0.19$                   & $0.92 \pm 0.17$ \\
    G cingul-Post-dorsal      & $1.51 \pm 0.17$                       & $1.48 \pm 0.18$ & $0.98 \pm 0.14$                & $0.95 \pm 0.17$ & $0.99 \pm 0.16$                   & $0.97 \pm 0.19$ \\
    G cingul-Post-ventral     & $1.20 \pm 0.16$                       & $1.28 \pm 0.18$ & $0.77 \pm 0.13$                & $0.87 \pm 0.15$ & $0.75 \pm 0.14$                   & $0.90 \pm 0.16$ \\
    G cuneus                  & $1.41 \pm 0.22$                       & $1.43 \pm 0.22$ & $1.24 \pm 0.17$                & $1.22 \pm 0.19$ & $1.19 \pm 0.18$                   & $1.16 \pm 0.19$ \\
    G front inf-Opercular     & $1.66 \pm 0.22$                       & $1.65 \pm 0.21$ & $1.09 \pm 0.17$                & $1.09 \pm 0.20$ & $1.11 \pm 0.18$                   & $1.12 \pm 0.21$ \\
    G front inf-Orbital       & $1.41 \pm 0.24$                       & $1.31 \pm 0.24$ & $0.78 \pm 0.15$                & $0.67 \pm 0.16$ & $0.81 \pm 0.17$                   & $0.67 \pm 0.17$ \\
    G front inf-Triangul      & $1.56 \pm 0.23$                       & $1.44 \pm 0.24$ & $1.03 \pm 0.17$                & $0.90 \pm 0.22$ & $1.06 \pm 0.19$                   & $0.92 \pm 0.23$ \\
    G front middle            & $1.70 \pm 0.23$                       & $1.62 \pm 0.23$ & $1.17 \pm 0.23$                & $0.97 \pm 0.22$ & $1.21 \pm 0.24$                   & $0.99 \pm 0.23$ \\
    G front sup               & $1.78 \pm 0.23$                       & $1.73 \pm 0.23$ & $1.37 \pm 0.20$                & $1.28 \pm 0.22$ & $1.39 \pm 0.21$                   & $1.33 \pm 0.22$ \\
    G insular short           & $1.21 \pm 0.23$                       & $1.06 \pm 0.18$ & $0.89 \pm 0.13$                & $0.65 \pm 0.14$ & $1.00 \pm 0.16$                   & $0.89 \pm 0.14$ \\
    G oc-temp lat-fusifor     & $1.58 \pm 0.19$                       & $1.58 \pm 0.20$ & $1.15 \pm 0.17$                & $1.13 \pm 0.18$ & $1.17 \pm 0.18$                   & $1.15 \pm 0.19$ \\
    G oc-temp med-Lingual     & $1.37 \pm 0.23$                       & $1.35 \pm 0.23$ & $1.25 \pm 0.18$                & $1.26 \pm 0.18$ & $1.25 \pm 0.19$                   & $1.26 \pm 0.20$ \\
    G oc-temp med-Parahip     & $1.41 \pm 0.22$                       & $1.35 \pm 0.21$ & $1.02 \pm 0.20$                & $0.88 \pm 0.22$ & $1.02 \pm 0.21$                   & $0.96 \pm 0.19$ \\
    G occipital middle        & $1.64 \pm 0.18$                       & $1.66 \pm 0.18$ & $1.01 \pm 0.17$                & $1.10 \pm 0.18$ & $1.14 \pm 0.18$                   & $1.15 \pm 0.19$ \\
    G occipital sup           & $1.55 \pm 0.19$                       & $1.57 \pm 0.20$ & $1.17 \pm 0.13$                & $1.10 \pm 0.15$ & $1.15 \pm 0.15$                   & $1.09 \pm 0.16$ \\
    G orbital                 & $1.53 \pm 0.24$                       & $1.43 \pm 0.21$ & $1.27 \pm 0.18$                & $1.08 \pm 0.16$ & $1.28 \pm 0.18$                   & $1.22 \pm 0.16$ \\
    G pariet inf-Angular      & $1.69 \pm 0.21$                       & $1.71 \pm 0.18$ & $1.11 \pm 0.19$                & $1.05 \pm 0.21$ & $1.10 \pm 0.20$                   & $1.06 \pm 0.21$ \\
    G pariet inf-Supramar     & $1.71 \pm 0.21$                       & $1.74 \pm 0.19$ & $1.15 \pm 0.20$                & $1.19 \pm 0.26$ & $1.16 \pm 0.21$                   & $1.20 \pm 0.27$ \\
    G parietal sup            & $1.73 \pm 0.22$                       & $1.68 \pm 0.22$ & $1.16 \pm 0.22$                & $1.06 \pm 0.24$ & $1.21 \pm 0.23$                   & $1.09 \pm 0.25$ \\
    G postcentral             & $1.62 \pm 0.26$                       & $1.57 \pm 0.30$ & $1.21 \pm 0.21$                & $1.23 \pm 0.27$ & $1.23 \pm 0.21$                   & $1.26 \pm 0.26$ \\
    G precentral              & $1.56 \pm 0.30$                       & $1.51 \pm 0.35$ & $1.26 \pm 0.18$                & $1.25 \pm 0.23$ & $1.29 \pm 0.22$                   & $1.26 \pm 0.28$ \\
    G precuneus               & $1.70 \pm 0.19$                       & $1.70 \pm 0.21$ & $1.22 \pm 0.19$                & $1.19 \pm 0.21$ & $1.24 \pm 0.19$                   & $1.20 \pm 0.23$ \\
    G rectus                  & $1.35 \pm 0.21$                       & $1.29 \pm 0.22$ & $0.96 \pm 0.15$                & $0.96 \pm 0.15$ & $0.95 \pm 0.15$                   & $0.96 \pm 0.16$ \\
    G subcallosal             & $1.07 \pm 0.13$                       & $1.02 \pm 0.13$ & $0.55 \pm 0.15$                & $0.61 \pm 0.11$ & $0.57 \pm 0.15$                   & $0.67 \pm 0.13$ \\
    G temp sup-G T transv     & $1.42 \pm 0.23$                       & $1.41 \pm 0.21$ & $0.97 \pm 0.16$                & $0.88 \pm 0.12$ & $1.00 \pm 0.18$                   & $0.93 \pm 0.13$ \\
    G temp sup-Lateral        & $1.63 \pm 0.23$                       & $1.66 \pm 0.22$ & $1.02 \pm 0.19$                & $1.26 \pm 0.16$ & $1.22 \pm 0.20$                   & $1.28 \pm 0.18$ \\
    G temp sup-Plan polar     & $1.31 \pm 0.23$                       & $1.22 \pm 0.21$ & $0.81 \pm 0.17$                & $0.69 \pm 0.17$ & $0.81 \pm 0.19$                   & $0.79 \pm 0.16$ \\
    G temp sup-Plan tempo     & $1.61 \pm 0.20$                       & $1.62 \pm 0.19$ & $1.07 \pm 0.18$                & $1.06 \pm 0.15$ & $1.03 \pm 0.18$                   & $1.07 \pm 0.14$ \\
    G temporal inf            & $1.54 \pm 0.20$                       & $1.56 \pm 0.21$ & $1.08 \pm 0.20$                & $1.15 \pm 0.18$ & $1.09 \pm 0.22$                   & $1.19 \pm 0.19$ \\
    G temporal middle         & $1.64 \pm 0.19$                       & $1.72 \pm 0.18$ & $1.15 \pm 0.19$                & $1.27 \pm 0.17$ & $1.21 \pm 0.20$                   & $1.32 \pm 0.18$ \\
    Lat Fis-ant-Horizont      & $1.33 \pm 0.20$                       & $1.39 \pm 0.20$ & $0.74 \pm 0.20$                & $0.84 \pm 0.19$ & $0.65 \pm 0.19$                   & $0.76 \pm 0.19$ \\
    Lat Fis-ant-Vertical      & $1.32 \pm 0.21$                       & $1.22 \pm 0.23$ & $0.66 \pm 0.21$                & $0.58 \pm 0.25$ & $0.58 \pm 0.21$                   & $0.49 \pm 0.23$ \\
    Lat Fis-post              & $1.62 \pm 0.18$                       & $1.67 \pm 0.15$ & $1.09 \pm 0.14$                & $1.09 \pm 0.14$ & $1.00 \pm 0.14$                   & $1.02 \pm 0.14$ \\
    Pole occipital            & $1.42 \pm 0.21$                       & $1.46 \pm 0.22$ & $1.14 \pm 0.14$                & $1.23 \pm 0.16$ & $1.06 \pm 0.18$                   & $1.14 \pm 0.17$ \\
    Pole temporal             & $1.34 \pm 0.27$                       & $1.42 \pm 0.24$ & $1.06 \pm 0.21$                & $1.11 \pm 0.20$ & $1.00 \pm 0.24$                   & $1.08 \pm 0.23$ \\
    S calcarine               & $1.45 \pm 0.21$                       & $1.42 \pm 0.21$ & $1.25 \pm 0.22$                & $1.21 \pm 0.21$ & $1.21 \pm 0.18$                   & $1.17 \pm 0.17$ \\
    S central                 & $1.60 \pm 0.25$                       & $1.61 \pm 0.24$ & $1.37 \pm 0.25$                & $1.33 \pm 0.30$ & $1.32 \pm 0.21$                   & $1.28 \pm 0.26$ \\
    S cingul-Marginalis       & $1.66 \pm 0.20$                       & $1.65 \pm 0.21$ & $1.18 \pm 0.18$                & $1.19 \pm 0.22$ & $1.07 \pm 0.18$                   & $1.06 \pm 0.22$ \\
    S circular insula ant     & $1.38 \pm 0.22$                       & $1.37 \pm 0.22$ & $0.98 \pm 0.16$                & $0.96 \pm 0.18$ & $0.86 \pm 0.15$                   & $0.84 \pm 0.16$ \\
    S circular insula inf     & $1.45 \pm 0.17$                       & $1.50 \pm 0.17$ & $1.09 \pm 0.18$                & $1.09 \pm 0.17$ & $0.98 \pm 0.15$                   & $0.98 \pm 0.15$ \\
    S circular insula sup     & $1.58 \pm 0.18$                       & $1.54 \pm 0.17$ & $1.12 \pm 0.13$                & $1.02 \pm 0.15$ & $1.01 \pm 0.13$                   & $0.94 \pm 0.14$ \\
    S collat transv ant       & $1.33 \pm 0.20$                       & $1.40 \pm 0.19$ & $0.93 \pm 0.19$                & $0.96 \pm 0.21$ & $0.83 \pm 0.19$                   & $0.85 \pm 0.19$ \\
    S collat transv post      & $1.31 \pm 0.19$                       & $1.30 \pm 0.18$ & $0.81 \pm 0.17$                & $0.73 \pm 0.16$ & $0.72 \pm 0.18$                   & $0.65 \pm 0.17$ \\
    S front inf               & $1.61 \pm 0.19$                       & $1.58 \pm 0.20$ & $1.13 \pm 0.24$                & $1.06 \pm 0.25$ & $1.06 \pm 0.23$                   & $0.99 \pm 0.25$ \\
    S front middle            & $1.46 \pm 0.19$                       & $1.41 \pm 0.20$ & $0.90 \pm 0.23$                & $0.85 \pm 0.22$ & $0.81 \pm 0.24$                   & $0.70 \pm 0.23$ \\
    S front sup               & $1.68 \pm 0.19$                       & $1.63 \pm 0.19$ & $1.09 \pm 0.22$                & $0.96 \pm 0.25$ & $1.01 \pm 0.21$                   & $0.87 \pm 0.24$ \\
    S interm prim-Jensen      & $1.22 \pm 0.28$                       & $1.33 \pm 0.21$ & $0.61 \pm 0.30$                & $0.65 \pm 0.23$ & $0.53 \pm 0.29$                   & $0.57 \pm 0.22$ \\
    S intrapariet and P trans & $1.72 \pm 0.18$                       & $1.71 \pm 0.19$ & $1.14 \pm 0.22$                & $1.10 \pm 0.25$ & $1.11 \pm 0.21$                   & $1.03 \pm 0.23$ \\
    S oc-temp lat             & $1.41 \pm 0.19$                       & $1.47 \pm 0.18$ & $0.83 \pm 0.17$                & $0.96 \pm 0.19$ & $0.73 \pm 0.17$                   & $0.86 \pm 0.19$ \\
    S oc-temp med and Lingual & $1.58 \pm 0.17$                       & $1.61 \pm 0.17$ & $1.15 \pm 0.19$                & $1.16 \pm 0.19$ & $1.08 \pm 0.18$                   & $1.10 \pm 0.17$ \\
    S oc middle and Lunatus   & $1.49 \pm 0.19$                       & $1.47 \pm 0.19$ & $1.00 \pm 0.20$                & $0.96 \pm 0.21$ & $0.91 \pm 0.19$                   & $0.87 \pm 0.20$ \\
    S oc sup and transversal  & $1.60 \pm 0.17$                       & $1.59 \pm 0.18$ & $1.17 \pm 0.18$                & $1.12 \pm 0.20$ & $1.10 \pm 0.18$                   & $1.06 \pm 0.18$ \\
    S occipital ant           & $1.40 \pm 0.19$                       & $1.47 \pm 0.17$ & $0.67 \pm 0.15$                & $0.85 \pm 0.15$ & $0.60 \pm 0.16$                   & $0.76 \pm 0.15$ \\
    S orbital-H Shaped        & $1.50 \pm 0.18$                       & $1.48 \pm 0.19$ & $1.16 \pm 0.26$                & $1.14 \pm 0.28$ & $1.14 \pm 0.23$                   & $1.14 \pm 0.22$ \\
    S orbital lateral         & $1.23 \pm 0.20$                       & $1.21 \pm 0.19$ & $0.69 \pm 0.19$                & $0.65 \pm 0.20$ & $0.59 \pm 0.19$                   & $0.56 \pm 0.18$ \\
    S orbital med-olfact      & $1.32 \pm 0.17$                       & $1.20 \pm 0.17$ & $0.92 \pm 0.17$                & $0.60 \pm 0.13$ & $0.85 \pm 0.14$                   & $0.63 \pm 0.13$ \\
    S parieto occipital       & $1.70 \pm 0.19$                       & $1.69 \pm 0.21$ & $1.32 \pm 0.19$                & $1.33 \pm 0.19$ & $1.24 \pm 0.18$                   & $1.27 \pm 0.19$ \\
    S pericallosal            & $1.21 \pm 0.13$                       & $1.24 \pm 0.14$ & $0.98 \pm 0.25$                & $1.04 \pm 0.26$ & $0.86 \pm 0.21$                   & $0.92 \pm 0.21$ \\
    S postcentral             & $1.74 \pm 0.20$                       & $1.73 \pm 0.22$ & $1.20 \pm 0.27$                & $1.20 \pm 0.31$ & $1.13 \pm 0.25$                   & $1.13 \pm 0.27$ \\
    S precentral-inf-part     & $1.64 \pm 0.22$                       & $1.62 \pm 0.28$ & $1.21 \pm 0.24$                & $1.19 \pm 0.26$ & $1.13 \pm 0.24$                   & $1.10 \pm 0.29$ \\
    S precentral-sup-part     & $1.57 \pm 0.24$                       & $1.51 \pm 0.31$ & $1.09 \pm 0.22$                & $1.09 \pm 0.23$ & $1.03 \pm 0.21$                   & $1.00 \pm 0.24$ \\
    S suborbital              & $1.27 \pm 0.16$                       & $1.05 \pm 0.20$ & $0.74 \pm 0.16$                & $0.52 \pm 0.22$ & $0.61 \pm 0.15$                   & $0.43 \pm 0.20$ \\
    S subparietal             & $1.53 \pm 0.17$                       & $1.50 \pm 0.17$ & $1.04 \pm 0.21$                & $0.98 \pm 0.23$ & $0.95 \pm 0.20$                   & $0.88 \pm 0.21$ \\
    S temporal inf            & $1.45 \pm 0.16$                       & $1.46 \pm 0.15$ & $0.86 \pm 0.17$                & $0.94 \pm 0.17$ & $0.76 \pm 0.16$                   & $0.83 \pm 0.16$ \\
    S temporal sup            & $1.73 \pm 0.17$                       & $1.75 \pm 0.15$ & $1.29 \pm 0.22$                & $1.33 \pm 0.19$ & $1.23 \pm 0.20$                   & $1.24 \pm 0.18$ \\
    S temporal transverse     & $1.32 \pm 0.19$                       & $1.30 \pm 0.18$ & $0.82 \pm 0.19$                & $0.77 \pm 0.19$ & $0.73 \pm 0.18$                   & $0.68 \pm 0.18$ \\
\end{longtblr}

\begin{longtblr}[
        caption={Standard-deviation average across all subjects and regions.},
        label={tab:std-cortical},
    ]{
        colspec={lcc|cc|cc}, width=\linewidth,
        row{even}={white,font=\footnotesize},
        row{odd}={gray9,font=\footnotesize},
        rows = {rowsep=0pt},
        rowhead=2,
        row{1}={white,font=\bfseries},
        row{2}={gray9}
    }
    \SetCell[c=1]{c}Region    & \SetCell[c=2]{c}{cortical thickness                                                                                                               \\ (mm)} &                 & \SetCell[c=2]{c}{surface area \\ ($\text{mm}^2$)} &                    & \SetCell[c=2]{c}{cortical volume \\ ($\text{mm}^3$)} &                     \\
                              & lh                                  & rh              & lh                 & rh                   & lh                    & rh                    \\
    \hline
    G Ins lg and S cent ins   & $0.09 \pm 0.04$                     & $0.10 \pm 0.05$ & $26.80 \pm 12.53$  & $34.88 \pm 19.32$    & $\081.42 \pm \048.62$ & $\088.35 \pm \046.33$ \\
    G and S cingul-Ant        & $0.03 \pm 0.02$                     & $0.03 \pm 0.02$ & $51.54 \pm 39.32$  & $63.44 \pm 42.94$    & $145.54  \pm 108.64$  & $173.81  \pm \099.91$ \\
    G and S cingul-Mid-Ant    & $0.03 \pm 0.02$                     & $0.03 \pm 0.02$ & $34.57 \pm 24.10$  & $35.63 \pm 22.48$    & $\089.14 \pm \052.30$ & $\095.86 \pm \057.77$ \\
    G and S cingul-Mid-Post   & $0.02 \pm 0.02$                     & $0.02 \pm 0.02$ & $24.28 \pm 19.46$  & $30.54 \pm 27.37$    & $\065.07 \pm \053.03$ & $\085.59 \pm \075.73$ \\
    G and S frontomargin      & $0.04 \pm 0.03$                     & $0.05 \pm 0.03$ & $31.83 \pm 21.28$  & $31.61 \pm 13.38$    & $\094.56 \pm \064.13$ & $\097.65 \pm \046.48$ \\
    G and S occipital inf     & $0.03 \pm 0.02$                     & $0.03 \pm 0.02$ & $39.02 \pm 19.81$  & $32.57 \pm 15.92$    & $119.36  \pm \056.10$ & $\099.07 \pm \046.18$ \\
    G and S paracentral       & $0.03 \pm 0.02$                     & $0.04 \pm 0.02$ & $33.65 \pm 17.96$  & $29.67 \pm 26.32$    & $109.16  \pm \066.28$ & $\094.71 \pm \072.49$ \\
    G and S subcentral        & $0.03 \pm 0.02$                     & $0.03 \pm 0.02$ & $33.87 \pm 19.69$  & $31.97 \pm 21.67$    & $104.20  \pm \061.20$ & $\089.71 \pm \060.11$ \\
    G and S transv frontopol  & $0.06 \pm 0.05$                     & $0.05 \pm 0.03$ & $28.12 \pm 12.80$  & $40.99 \pm 16.84$    & $104.12  \pm \065.82$ & $134.29  \pm \065.41$ \\
    G cingul-Post-dorsal      & $0.04 \pm 0.03$                     & $0.04 \pm 0.04$ & $19.23 \pm 14.88$  & $17.77 \pm 12.57$    & $\064.32 \pm \044.09$ & $\059.57 \pm \035.92$ \\
    G cingul-Post-ventral     & $0.07 \pm 0.05$                     & $0.06 \pm 0.05$ & $15.66 \pm 17.15$  & $12.99 \pm 22.56$    & $\050.17 \pm \027.85$ & $\038.86 \pm \036.83$ \\
    G cuneus                  & $0.03 \pm 0.02$                     & $0.03 \pm 0.02$ & $35.26 \pm 30.33$  & $40.96 \pm 42.52$    & $\084.71 \pm \074.31$ & $101.9   \pm \097.02$ \\
    G front inf-Opercular     & $0.03 \pm 0.02$                     & $0.03 \pm 0.02$ & $32.65 \pm 25.00$  & $35.08 \pm 34.37$    & $105.47  \pm \078.45$ & $108.38  \pm \098.3$  \\
    G front inf-Orbital       & $0.05 \pm 0.04$                     & $0.06 \pm 0.04$ & $21.20 \pm 11.01$  & $25.25 \pm 10.37$    & $\071.05 \pm \046.15$ & $\086.33 \pm \038.23$ \\
    G front inf-Triangul      & $0.03 \pm 0.03$                     & $0.04 \pm 0.03$ & $30.29 \pm 19.17$  & $37.08 \pm 32.23$    & $\091.18 \pm \065.35$ & $116.92  \pm 104.8$   \\
    G front middle            & $0.02 \pm 0.02$                     & $0.03 \pm 0.02$ & $92.58 \pm 75.11$  & $126.22 \pm 77.76\0$ & $273.22  \pm 234.35$  & $403.59  \pm 245.74$  \\
    G front sup               & $0.02 \pm 0.02$                     & $0.02 \pm 0.02$ & $97.83 \pm 73.49$  & $113.70 \pm 97.83\0$ & $299.95  \pm 206.52$  & $339.13  \pm 278.46$  \\
    G insular short           & $0.09 \pm 0.05$                     & $0.11 \pm 0.05$ & $29.04 \pm 15.99$  & $47.90 \pm 23.20$    & $\098.87 \pm \058.14$ & $112.37  \pm \056.78$ \\
    G oc-temp lat-fusifor     & $0.03 \pm 0.02$                     & $0.03 \pm 0.03$ & $37.79 \pm 21.25$  & $41.78 \pm 26.17$    & $128.12  \pm \071.17$ & $146.71  \pm \094.52$ \\
    G oc-temp med-Lingual     & $0.04 \pm 0.02$                     & $0.04 \pm 0.02$ & $51.64 \pm 39.25$  & $48.19 \pm 36.79$    & $122.30  \pm \098.61$ & $118.59  \pm \098.07$ \\
    G oc-temp med-Parahip     & $0.05 \pm 0.03$                     & $0.06 \pm 0.03$ & $37.22 \pm 36.52$  & $49.76 \pm 69.91$    & $143.85  \pm 100.75$  & $158.91  \pm 121.6$   \\
    G occipital middle        & $0.02 \pm 0.02$                     & $0.02 \pm 0.02$ & $47.90 \pm 22.84$  & $52.98 \pm 30.88$    & $138.60  \pm \079.51$ & $148.24  \pm \095.59$ \\
    G occipital sup           & $0.03 \pm 0.02$                     & $0.03 \pm 0.02$ & $26.48 \pm 14.00$  & $34.90 \pm 21.23$    & $\068.90 \pm \041.73$ & $\091.73 \pm \053.70$ \\
    G orbital                 & $0.03 \pm 0.02$                     & $0.04 \pm 0.02$ & $39.63 \pm 27.95$  & $63.63 \pm 27.59$    & $126.67  \pm \098.09$ & $162.43  \pm \078.39$ \\
    G pariet inf-Angular      & $0.02 \pm 0.02$                     & $0.02 \pm 0.01$ & $57.33 \pm 29.55$  & $77.98 \pm 47.01$    & $179.81  \pm \096.06$ & $236.38  \pm 142.78$  \\
    G pariet inf-Supramar     & $0.02 \pm 0.02$                     & $0.02 \pm 0.01$ & $62.02 \pm 46.17$  & $54.59 \pm 60.03$    & $192.54  \pm 148.64$  & $170.62  \pm 190.83$  \\
    G parietal sup            & $0.02 \pm 0.01$                     & $0.02 \pm 0.01$ & $62.42 \pm 52.63$  & $65.13 \pm 50.00$    & $161.05  \pm 166.21$  & $177.73  \pm 159.43$  \\
    G postcentral             & $0.03 \pm 0.03$                     & $0.03 \pm 0.03$ & $42.75 \pm 36.99$  & $39.75 \pm 47.18$    & $106.60  \pm \082.38$ & $\096.85 \pm 114.98$  \\
    G precentral              & $0.04 \pm 0.04$                     & $0.04 \pm 0.05$ & $42.85 \pm 36.80$  & $47.68 \pm 45.38$    & $138.71  \pm 137.38$  & $164.16  \pm 191.76$  \\
    G precuneus               & $0.02 \pm 0.02$                     & $0.02 \pm 0.02$ & $49.44 \pm 43.95$  & $49.64 \pm 40.76$    & $145.37  \pm 137.81$  & $149.86  \pm 128.30$  \\
    G rectus                  & $0.05 \pm 0.03$                     & $0.06 \pm 0.04$ & $32.20 \pm 15.60$  & $27.50 \pm 12.79$    & $106.76  \pm \051.68$ & $\091.10 \pm \049.31$ \\
    G subcallosal             & $0.08 \pm 0.02$                     & $0.09 \pm 0.03$ & $51.68 \pm 17.71$  & $34.17 \pm 12.84$    & $110.85  \pm \037.44$ & $\073.96 \pm \031.21$ \\
    G temp sup-G T transv     & $0.04 \pm 0.04$                     & $0.04 \pm 0.03$ & $15.35 \pm \08.69$ & $13.73 \pm \05.19$   & $\042.31 \pm \028.07$ & $\037.23 \pm \014.97$ \\
    G temp sup-Lateral        & $0.03 \pm 0.02$                     & $0.03 \pm 0.02$ & $38.43 \pm 27.47$  & $29.34 \pm 15.20$    & $147.99  \pm \097.89$ & $111.92  \pm \061.68$ \\
    G temp sup-Plan polar     & $0.07 \pm 0.05$                     & $0.08 \pm 0.04$ & $32.33 \pm 18.14$  & $47.67 \pm 23.04$    & $119.94  \pm \070.10$ & $124.65  \pm \058.27$ \\
    G temp sup-Plan tempo     & $0.03 \pm 0.02$                     & $0.03 \pm 0.02$ & $24.87 \pm 16.10$  & $20.93 \pm \09.79$   & $\071.54 \pm \039.15$ & $\055.07 \pm \022.84$ \\
    G temporal inf            & $0.03 \pm 0.02$                     & $0.03 \pm 0.02$ & $63.81 \pm 45.99$  & $52.37 \pm 28.88$    & $230.73  \pm 163.99$  & $180.27  \pm 100.37$  \\
    G temporal middle         & $0.03 \pm 0.02$                     & $0.02 \pm 0.01$ & $54.58 \pm 37.80$  & $46.74 \pm 27.32$    & $184.08  \pm 135.35$  & $155.87  \pm \083.42$ \\
    Lat Fis-ant-Horizont      & $0.04 \pm 0.03$                     & $0.04 \pm 0.02$ & $15.78 \pm 07.02$  & $16.00 \pm \09.18$   & $\037.09 \pm \018.17$ & $\037.24 \pm \022.79$ \\
    Lat Fis-ant-Vertical      & $0.05 \pm 0.03$                     & $0.06 \pm 0.06$ & $19.09 \pm \08.70$ & $16.93 \pm \07.9$    & $\047.83 \pm \023.08$ & $\042.18 \pm \019.39$ \\
    Lat Fis-post              & $0.02 \pm 0.02$                     & $0.02 \pm 0.02$ & $24.88 \pm 13.15$  & $30.93 \pm 13.87$    & $\056.61 \pm \030.98$ & $\071.03 \pm \031.70$ \\
    Pole occipital            & $0.03 \pm 0.02$                     & $0.03 \pm 0.02$ & $40.70 \pm 18.96$  & $56.64 \pm 36.50$    & $113.61  \pm 66.05$   & $157.76  \pm \095.97$ \\
    Pole temporal             & $0.06 \pm 0.04$                     & $0.05 \pm 0.03$ & $46.23 \pm 30.07$  & $43.02 \pm 26.78$    & $251.05  \pm 187.99$  & $215.93  \pm 147.79$  \\
    S calcarine               & $0.03 \pm 0.02$                     & $0.03 \pm 0.02$ & $44.93 \pm 39.69$  & $47.41 \pm 45.06$    & $\077.13 \pm \067.45$ & $\083.23 \pm \065.81$ \\
    S central                 & $0.02 \pm 0.02$                     & $0.02 \pm 0.02$ & $45.58 \pm 48.37$  & $52.54 \pm 69.94$    & $\080.26 \pm \074.65$ & $\092.02 \pm 109.44$  \\
    S cingul-Marginalis       & $0.02 \pm 0.02$                     & $0.02 \pm 0.02$ & $21.30 \pm 17.52$  & $25.74 \pm 22.14$    & $\053.01 \pm \041.86$ & $\071.65 \pm \058.25$ \\
    S circular insula ant     & $0.05 \pm 0.04$                     & $0.05 \pm 0.04$ & $15.81 \pm 10.25$  & $18.35 \pm 12.25$    & $\046.26 \pm \034.95$ & $\053.18 \pm \033.38$ \\
    S circular insula inf     & $0.04 \pm 0.03$                     & $0.03 \pm 0.02$ & $33.02 \pm 19.40$  & $27.57 \pm 16.37$    & $\092.16 \pm \047.87$ & $\073.13 \pm \035.95$ \\
    S circular insula sup     & $0.03 \pm 0.03$                     & $0.03 \pm 0.02$ & $35.16 \pm 16.49$  & $36.41 \pm 20.50$    & $\091.74 \pm \044.20$ & $\089.54 \pm \047.94$ \\
    S collat transv ant       & $0.05 \pm 0.04$                     & $0.04 \pm 0.03$ & $31.84 \pm 18.98$  & $31.78 \pm 17.89$    & $\095.76 \pm \052.02$ & $\097.81 \pm \051.0$  \\
    S collat transv post      & $0.04 \pm 0.03$                     & $0.04 \pm 0.02$ & $22.36 \pm \09.36$ & $24.44 \pm 10.34$    & $\049.00 \pm \022.33$ & $\052.38 \pm \023.3$  \\
    S front inf               & $0.02 \pm 0.02$                     & $0.02 \pm 0.02$ & $50.05 \pm 48.10$  & $55.04 \pm 53.15$    & $122.15  \pm 115.86$  & $137.13  \pm 139.79$  \\
    S front middle            & $0.03 \pm 0.02$                     & $0.03 \pm 0.02$ & $51.70 \pm 33.07$  & $91.25 \pm 60.17$    & $137.58  \pm \098.08$ & $282.75  \pm 181.72$  \\
    S front sup               & $0.02 \pm 0.01$                     & $0.02 \pm 0.02$ & $70.65 \pm 53.57$  & $88.18 \pm 73.29$    & $189.44  \pm 133.17$  & $242.99  \pm 183.7$   \\
    S interm prim-Jensen      & $0.07 \pm 0.09$                     & $0.05 \pm 0.04$ & $22.91 \pm 15.67$  & $26.85 \pm 14.81$    & $\055.28 \pm \035.58$ & $\062.20 \pm \032.78$ \\
    S intrapariet and P trans & $0.02 \pm 0.02$                     & $0.02 \pm 0.01$ & $65.99 \pm 57.12$  & $83.26 \pm 77.54$    & $133.09  \pm 111.32$  & $186.66  \pm 155.46$  \\
    S oc-temp lat             & $0.04 \pm 0.03$                     & $0.03 \pm 0.02$ & $37.74 \pm 16.79$  & $33.31 \pm 16.25$    & $101.88  \pm \046.87$ & $\090.78 \pm \043.89$ \\
    S oc-temp med and Lingual & $0.02 \pm 0.02$                     & $0.02 \pm 0.02$ & $43.42 \pm 29.10$  & $40.87 \pm 26.30$    & $105.48  \pm \057.26$ & $\093.32 \pm \050.19$ \\
    S oc middle and Lunatus   & $0.03 \pm 0.02$                     & $0.03 \pm 0.02$ & $28.69 \pm 15.64$  & $29.71 \pm 19.00$    & $\061.89 \pm \033.18$ & $\065.79 \pm \040.53$ \\
    S oc sup and transversal  & $0.02 \pm 0.02$                     & $0.02 \pm 0.02$ & $24.28 \pm 14.75$  & $30.11 \pm 20.51$    & $\054.26 \pm \032.40$ & $\065.35 \pm \038.76$ \\
    S occipital ant           & $0.04 \pm 0.02$                     & $0.03 \pm 0.02$ & $40.53 \pm 15.92$  & $29.35 \pm 13.03$    & $\098.81 \pm \042.71$ & $\076.07 \pm \035.41$ \\
    S orbital-H Shaped        & $0.03 \pm 0.02$                     & $0.03 \pm 0.02$ & $30.77 \pm 27.50$  & $30.63 \pm 24.64$    & $\075.75 \pm \060.88$ & $\069.25 \pm \045.60$ \\
    S orbital lateral         & $0.05 \pm 0.03$                     & $0.05 \pm 0.03$ & $20.80 \pm 10.56$  & $28.47 \pm 13.86$    & $\050.60 \pm \026.90$ & $\066.88 \pm \032.94$ \\
    S orbital med-olfact      & $0.05 \pm 0.03$                     & $0.06 \pm 0.03$ & $22.67 \pm \09.39$ & $52.03 \pm 15.39$    & $\051.77 \pm \021.69$ & $\096.61 \pm \038.19$ \\
    S parieto occipital       & $0.02 \pm 0.02$                     & $0.02 \pm 0.03$ & $29.51 \pm 23.78$  & $32.84 \pm 34.50$    & $\068.41 \pm \046.21$ & $\072.54 \pm \068.14$ \\
    S pericallosal            & $0.04 \pm 0.02$                     & $0.04 \pm 0.02$ & $44.29 \pm 80.37$  & $44.16 \pm 60.70$    & $\078.45 \pm 149.00$  & $\074.66 \pm \086.88$ \\
    S postcentral             & $0.02 \pm 0.02$                     & $0.02 \pm 0.02$ & $59.33 \pm 63.12$  & $57.23 \pm 76.08$    & $132.12  \pm 121.05$  & $118.64  \pm 135.52$  \\
    S precentral-inf-part     & $0.02 \pm 0.03$                     & $0.03 \pm 0.04$ & $32.56 \pm 41.72$  & $35.05 \pm 38.33$    & $\082.93 \pm 113.93$  & $\095.11 \pm 120.79$  \\
    S precentral-sup-part     & $0.03 \pm 0.04$                     & $0.04 \pm 0.05$ & $32.51 \pm 24.59$  & $33.81 \pm 26.76$    & $\074.56 \pm \059.33$ & $\081.98 \pm \069.50$ \\
    S suborbital              & $0.05 \pm 0.03$                     & $0.09 \pm 0.08$ & $37.26 \pm 17.06$  & $27.92 \pm 11.25$    & $101.77  \pm \051.24$ & $\071.72 \pm \029.02$ \\
    S subparietal             & $0.03 \pm 0.03$                     & $0.03 \pm 0.03$ & $29.36 \pm 27.35$  & $38.95 \pm 35.41$    & $\070.32 \pm \055.27$ & $101.75  \pm \074.08$ \\
    S temporal inf            & $0.03 \pm 0.02$                     & $0.03 \pm 0.02$ & $60.38 \pm 30.96$  & $41.64 \pm 17.77$    & $156.80  \pm \075.30$ & $106.77  \pm \042.99$ \\
    S temporal sup            & $0.02 \pm 0.02$                     & $0.02 \pm 0.02$ & $86.27 \pm 60.78$  & $83.72 \pm 54.81$    & $217.81  \pm 141.29$  & $220.72  \pm 129.73$  \\
    S temporal transverse     & $0.04 \pm 0.03$                     & $0.05 \pm 0.03$ & $17.51 \pm 9.28$   & $14.56 \pm \06.46$   & $\040.39 \pm \021.69$ & $\036.01 \pm \017.13$

\end{longtblr}

\begin{longtblr}[
        caption={Significant digits average across all subjects and regions.},
        label={tab:sig-subcortical-volume},
    ]{
        colspec={lc|lc},
        row{even}={gray9,font=\footnotesize},
        row{odd}={white,font=\footnotesize},
        rows = {rowsep=0pt},
        row{Z}={font=\small},
        rowhead=1,
        rowfoot=1,
        row{1}={font=\bfseries}
    }
    Region                       & Subcortical volume & Region                        & Subcortical volume \\
    \hline
    3rd-Ventricle                & $1.44 \pm 0.20$    & CC Anterior                   & $1.34 \pm 0.36$    \\
    4th-Ventricle                & $1.32 \pm 0.20$    & CC Central                    & $1.08 \pm 0.38$    \\
    5th-Ventricle                & $13.51 \pm 4.98\0$ & CC Mid Anterior               & $1.20 \pm 0.39$    \\
    Brain-Stem                   & $1.70 \pm 0.21$    & CC Mid Posterior              & $1.13 \pm 0.36$    \\
    CSF                          & $1.14 \pm 0.21$    & CC Posterior                  & $1.38 \pm 0.47$    \\
    Left-Accumbens-area          & $0.87 \pm 0.17$    & Right-Accumbens-area          & $0.98 \pm 0.16$    \\
    Left-Amygdala                & $1.12 \pm 0.16$    & Right-Amygdala                & $1.22 \pm 0.18$    \\
    Left-Caudate                 & $1.56 \pm 0.21$    & Right-Caudate                 & $1.50 \pm 0.26$    \\
    Left-Cerebellum-Cortex       & $1.88 \pm 0.23$    & Right-Cerebellum-Cortex       & $1.87 \pm 0.25$    \\
    Left-Cerebellum-White-Matter & $1.24 \pm 0.24$    & Right-Cerebellum-White-Matter & $1.25 \pm 0.28$    \\
    Left-Hippocampus             & $1.47 \pm 0.18$    & Right-Hippocampus             & $1.53 \pm 0.19$    \\
    Left-Inf-Lat-Vent            & $0.82 \pm 0.23$    & Right-Inf-Lat-Vent            & $0.88 \pm 0.27$    \\
    Left-Lateral-Ventricle       & $1.88 \pm 0.25$    & Right-Lateral-Ventricle       & $1.83 \pm 0.28$    \\
    Left-Pallidum                & $1.24 \pm 0.20$    & Right-Pallidum                & $1.21 \pm 0.21$    \\
    Left-Putamen                 & $1.47 \pm 0.23$    & Right-Putamen                 & $1.49 \pm 0.28$    \\
    Left-Thalamus                & $1.41 \pm 0.22$    & Right-Thalamus                & $1.42 \pm 0.22$    \\
    Left-VentralDC               & $1.40 \pm 0.16$    & Right-VentralDC               & $1.39 \pm 0.15$    \\
    Left-WM-hypointensities      & $15.22 \pm 0.00\0$ & Right-WM-hypointensities      & $15.22 \pm 0.00\0$ \\
    Left-choroid-plexus          & $0.81 \pm 0.16$    & Right-choroid-plexus          & $0.83 \pm 0.16$    \\
    Left-non-WM-hypointensities  & $15.22 \pm 0.00\0$ & Right-non-WM-hypointensities  & $15.22 \pm 0.00\0$ \\
    Left-vessel                  & $0.46 \pm 0.63$    & Right-vessel                  & $0.23 \pm 0.89$    \\
    Optic-Chiasm                 & $0.76 \pm 0.23$    &                               &                    \\
\end{longtblr}

\begin{longtblr}[
        caption={Standard-deviation average across all subjects and regions.},
        label={tab:std-subcortical-volume},
    ]{
        colspec={lc|lc},
        row{even}={gray9,font=\footnotesize},
        row{odd}={white,font=\footnotesize},
        rows = {rowsep=0pt},
        row{Z}={font=\small},
        rowhead=1,
        row{1}={font=\bfseries}
    }
    Region                       & {Subcortical volume                                                        \\ ($\text{mm}^3$)} & Region                        & {Subcortical volume \\ ($\text{mm}^3$)} \\
    \hline
    3rd-Ventricle                & $25.18 \pm 34.65$   & CC Anterior                   & $23.67 \pm 29.16$    \\
    4th-Ventricle                & $38.47 \pm 23.49$   & CC Central                    & $23.89 \pm 23.90$    \\
    5th-Ventricle                & $0.09 \pm  0.43$    & CC Mid Anterior               & $19.86 \pm 24.04$    \\
    Brain-Stem                   & $187.47 \pm 103.98$ & CC Mid Posterior              & $21.69 \pm 29.83$    \\
    CSF                          & $38.22   \pm 49.09$ & CC Posterior                  & $31.59 \pm 53.10$    \\
    Left-Accumbens-area          & $24.50 \pm 9.50$    & Right-Accumbens-area          & $20.89   \pm 8.86$   \\
    Left-Amygdala                & $49.73 \pm 20.47$   & Right-Amygdala                & $43.96   \pm 24.00$  \\
    Left-Caudate                 & $40.85 \pm 28.91$   & Right-Caudate                 & $51.88   \pm 45.72$  \\
    Left-Cerebellum-Cortex       & $313.86 \pm 227.78$ & Right-Cerebellum-Cortex       & $338.10  \pm 271.17$ \\
    Left-Cerebellum-White-Matter & $380.83 \pm 224.68$ & Right-Cerebellum-White-Matter & $379.10  \pm 329.94$ \\
    Left-Hippocampus             & $57.14 \pm 40.36$   & Right-Hippocampus             & $51.70   \pm 39.33$  \\
    Left-Inf-Lat-Vent            & $31.66 \pm 16.10$   & Right-Inf-Lat-Vent            & $28.75   \pm 15.28$  \\
    Left-Lateral-Ventricle       & $88.18 \pm 185.84$  & Right-Lateral-Ventricle       & $121.41  \pm 578.96$ \\
    Left-Pallidum                & $49.21 \pm 30.03$   & Right-Pallidum                & $52.09   \pm 37.88$  \\
    Left-Putamen                 & $69.42 \pm 51.72$   & Right-Putamen                 & $73.75   \pm 96.90$  \\
    Left-Thalamus                & $124.64 \pm 81.18$  & Right-Thalamus                & $124.52  \pm 111.74$ \\
    Left-VentralDC               & $65.09 \pm 30.33$   & Right-VentralDC               & $65.07   \pm 27.96$  \\
    Left-WM-hypointensities      & $15.22 \pm 0.00$    & Right-WM-hypointensities      & $0.00    \pm 0.00$   \\
    Left-choroid-plexus          & $43.74 \pm 22.99$   & Right-choroid-plexus          & $44.79   \pm 25.13$  \\
    Left-non-WM-hypointensities  & $0.00 \pm 0.00$     & Right-non-WM-hypointensities  & $0.00    \pm 0.00$   \\
    Left-vessel                  & $6.42 \pm 4.91$     & Right-vessel                  & $7.92    \pm 5.38$   \\
    Optic-Chiasm                 & $13.11 \pm 6.56\0$  &                               &                      \\
\end{longtblr}

\section{Longitudinal Analysis}

\subsection{Vertex-wise Analysis}
\subsubsection{Z-Test Results}

\subsubsection{Group analysis}
The following table summarizes significant regions identified in the group
analysis using a Z-test (\(\alpha = 0.05\)). Each region was statistically
significant in only one MCA repetition, highlighting a lack of replicability.

\begin{table}[h]
    \centering
    \caption{Significant regions identified using a Z-test (\(\alpha = 0.05\)) for the group analysis.}
    \begin{tabular}{l c c c c c c}
        \toprule
        \textbf{Region}       & \textbf{Size ($mm^2$)} & \textbf{MNI X} & \textbf{MNI Y} & \textbf{MNI Z} & \textbf{Max} & \textbf{Frequency} \\
        \midrule
        \textbf{Baseline}     &                        &                &                &                &              &                    \\
        R postcentral         & 844.98                 & 23.8           & -36.4          & 57.2           & -2.964       & 1 / 25             \\
        \midrule
        \textbf{Longitudinal} &                        &                &                &                &              &                    \\
        L inferior parietal   & 3337.65                & -56.6          & -43.6          & -18.1          & 2.8572       & 1 / 25             \\
        R parstriangularis    & 3265.87                & 44.4           & 35.8           & 3.7            & 2.8129       & 1 / 25             \\
        \bottomrule
    \end{tabular}
\end{table}

\subsubsection{Correlation analysis}
The following table presents significant regions identified in the correlation
analysis using a Z-test (\(\alpha = 0.05\)). Notably, only two regions were
found to be significant in two different MCA repetitions, with variability in
their estimated size and MNI coordinates.

\begin{table}[h]
    \centering
    \caption{Significant regions identified using a Z-test (\(\alpha = 0.05\)) for the correlation analysis.}
    \begin{tabular}{lllllll}
        \toprule
        \textbf{Region}     & \textbf{Size ($mm^2$)}   & \textbf{MNI X}    & \textbf{MNI Y}  & \textbf{MNI Z}   & \textbf{Max}       & \textbf{Frequency} \\
        \midrule
        \textbf{Baseline}   &                          &                   &                 &                  &                    &                    \\
        L postcentral       & $1006.20 \pm 39.27     $ & $ -37.0         $ & $-33.8/-34    $ & $62.5/61.7     $ & $-4.7429/-3.7576 $ & 2/25               \\
        L precentral        & $1011.32               $ & $ -18.2         $ & $-19.8        $ & $70.1          $ & $-4.2896         $ & 1/25               \\
        R lateral occipital & $\0997.20 \pm 43.08  $   & $ \014.2/20.7 $   & $-99.8/-96.5  $ & $\01.9/-13.1 $   & $-2.9967/-3.1263 $ & 2/25               \\
        R pericalcarine     & $2085.19               $ & $ \010.8      $   & $-74.9        $ & $\04.4       $   & $-3.000          $ & 1/25               \\
        \bottomrule
    \end{tabular}
\end{table}

\begin{table}[h]
    \centering
    \caption{Summary of executions failure and excluded subjects. To standardize the sample, we keep 25 repetitions per subject/visits pair.
        Subject/visit pairs with less than 25 repetitions were excluded which is 12 subjects.}
    \begin{tabular}{l c c}
        \toprule
        \textbf{Stage}     & \textbf{Number of rejected repetitions} & \textbf{Total number of repetitions} \\
        \midrule
        Cluster failure    & 1246 (5.80\%)                           & 21488                                \\
        FreeSurfer failure & 68 (0.33\%)                             & 21488                                \\
        QC failure         & 319 (1.48\%)                            & 21488                                \\
        Total              & 1633 (7.60\%)                           & 21488                                \\
        \bottomrule
    \end{tabular}
\end{table}

\begin{table}[h!]
    \centering
    \begin{tabular}{c|lccc}
        \toprule
        \textbf{Status} & \textbf{Cohort}         & \textbf{HC}         & \textbf{PD-non-MCI}        & \textbf{PD-MCI}   \\
        \hline
        \multirow{5}{*}{\textbf{\shortstack{Before                                                                       \\QC}}}
                        & n                       & 106                 & 181                        & 29                \\
                        & Age (y)                 & $60.6 \pm 10.2   $  & $61.7 \pm \09.6$           & $67.7 \pm \07.7$  \\
                        & Age range               & $30.6 - 84.3  $     & $36.3 - 83.3$              & $49.9 - 80.5$     \\
                        & Gender (male, \%)       & $58 \; (54.7\%)   $ & $119 \; (65.7\%)$          & $-          $     \\
                        & Education (y)           & $16.6 \pm \03.3  $  & $15.9 \pm \02.9$           & $-          $     \\
        \hline
        \multirow{5}{*}{\textbf{\shortstack{After                                                                        \\QC}}}
                        & n                       & 103                 & 175                        & 27                \\
                        & Age (y)                 & $60.7 \pm 10.3   $  & $61.4 \pm \09.5          $ & $67.8 \pm \07.9$  \\
                        & Age range               & $30.6 - 84.3  $     & $36.3 - 79.9           $   & $49.9 - 80.5$     \\
                        & Gender (male, \%)       & $57 \; (55.3\%)   $ & $114 \; (65.1\%)       $   & $20 \; (74.1\%) $ \\
                        & Education (y)           & $16.6 \pm \03.3  $  & $15.9 \pm \02.9        $   & $15.0 \pm \03.5$  \\
        \hline
        \multirow{8}{*}{\textbf{\shortstack{After                                                                        \\MCI\\exclusion}}}
                        & n                       & $103           $    & $121                   $   & --                \\
                        & Age (y)                 & $60.7 \pm 10.3   $  & $60.7 \pm \09.1        $   & --                \\
                        & Age range               & $30.6 - 84.3  $     & $39.2 - 78.3           $   & --                \\
                        & Gender (male, \%)       & $57 \; (55.3\%)   $ & $80 \; (66.1\%)        $   & --                \\
                        & Education (y)           & $16.6 \pm \03.3  $  & $16.1 \pm \03.0        $   & --                \\
                        & UPDRS III OFF baseline  & $-            $     & $23.4 \pm 10.1         $   & --                \\
                        & UPDRS III OFF follow-up & $-            $     & $25.8 \pm 11.1         $   & --                \\
                        & Duration T2 - T1 (y)    & $\01.4 \pm \00.5 $  & $\01.4 \pm \00.7       $   & --                \\
        \bottomrule
    \end{tabular}
    \vspace{1em}

    \textbf{Abbreviations:} MCI = Mild Cognitive Impairment; UPDRS = Unified Parkinson's Disease Rating Scale; PD = Parkinson's disease. Descriptive statistics before and after quality control (QC). Values are expressed as mean $\pm$ standard deviation. PD-non-MCI longitudinal sample is a subsample of the PD-non-MCI original sample that had longitudinal data and disease severity scores available.
    \label{tab:cohort_stat_vertical}
\end{table}

\section{NAVR}

\subsection{NAVR maps}

\begin{figure}[h]
    \includegraphics[width=\linewidth]{figures/NAVR_map/NAVR_thickness_all.png}
    \caption{NAVR maps for cortical thickness. The maps show the average NAVR values across all subjects for each cortical region. The color scale indicates the NAVR value, with warmer colors indicating higher NAVR values.}
    \label{fig:NAVR_map_thickness}
\end{figure}

\begin{figure}[h]
    \includegraphics[width=\linewidth]{figures/NAVR_map/NAVR_area_all.png}
    \caption{NAVR maps for cortical surface area. The maps show the average NAVR values across all subjects for each cortical region. The color scale indicates the NAVR value, with warmer colors indicating higher NAVR values.}
    \label{fig:NAVR_map_area}
\end{figure}

\begin{figure}[h]
    \includegraphics[width=\linewidth]{figures/NAVR_map/NAVR_volume_all.png}
    \caption{NAVR maps for cortical volume. The maps show the average NAVR values across all subjects for each cortical region. The color scale indicates the NAVR value, with warmer colors indicating higher NAVR values.}
    \label{fig:NAVR_map_volume}
\end{figure}

\subsection{Consistency results}

\begin{figure}[h]
    \centering
    \begin{subfigure}[b]{\textwidth}
        \includegraphics[width=\textwidth]{figures/consistency/cortical_thickness_coefficients_distribution-Left.pdf}
        \caption{Left hemisphere}
        \label{fig:NAVR_consistency_thickness_left}
    \end{subfigure}
    
    \begin{subfigure}[b]{\textwidth}
        \includegraphics[width=\textwidth]{figures/consistency/cortical_thickness_coefficients_distribution-Right.pdf}
        \caption{Right hemisphere}
        \label{fig:NAVR_consistency_thickness_right}
    \end{subfigure}
    \caption{Distribution of partial correlation coefficients for cortical thickness across all subjects and regions. Red triangles indicate the IEEE-754 run for reference. The distribution shows the variability in the coefficients, with some regions exhibiting higher consistency than others.}
    \label{fig:NAVR_consistency_thickness}
\end{figure}

\begin{figure}[h]
    \centering
    \begin{subfigure}[b]{\textwidth}
        \includegraphics[width=\textwidth]{figures/consistency/cortical_area_coefficients_distribution-Left.pdf}
        \caption{Left hemisphere}
        \label{fig:NAVR_consistency_area_left}
    \end{subfigure}
    \hfill
    \begin{subfigure}[b]{\textwidth}
        \includegraphics[width=\textwidth]{figures/consistency/cortical_area_coefficients_distribution-Right.pdf}
        \caption{Right hemisphere}
        \label{fig:NAVR_consistency_area_right}
    \end{subfigure}
    \caption{Distribution of partial correlation coefficients for cortical area across all subjects and regions. Red triangles indicate the IEEE-754 run for reference. The distribution shows the variability in the coefficients, with some regions exhibiting higher consistency than others.}
    \label{fig:NAVR_consistency_area}
\end{figure}

\begin{figure}[h]
    \centering
    \begin{subfigure}[b]{\textwidth}
        \includegraphics[width=\textwidth]{figures/consistency/cortical_volume_coefficients_distribution-Left.pdf}
        \caption{Left hemisphere}
        \label{fig:NAVR_consistency_volume_left}
    \end{subfigure}
    \hfill
    \begin{subfigure}[b]{\textwidth}
        \includegraphics[width=\textwidth]{figures/consistency/cortical_volume_coefficients_distribution-Right.pdf}
        \caption{Right hemisphere}
        \label{fig:NAVR_consistency_volume_right}
    \end{subfigure}
    \caption{Distribution of partial correlation coefficients for cortical volume across all subjects and regions. Red triangles indicate the IEEE-754 run for reference. The distribution shows the variability in the coefficients, with some regions exhibiting higher consistency than others.}
    \label{fig:NAVR_consistency_volume}
\end{figure}



\end{document}

% 2nd table describing the raw -> removed subjects 
% main table describing 
% 
% which part of the code is responsible for the instabilities
% include that in the discussion 
%
%
